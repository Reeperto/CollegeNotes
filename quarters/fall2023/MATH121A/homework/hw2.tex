\documentclass[12pt,titlepage]{extarticle}
% Document Layout and Font
\usepackage{subfiles}
\usepackage[margin=2cm, headheight=15pt]{geometry}
\usepackage{fancyhdr}
\usepackage{enumitem}	
\usepackage{wrapfig}
\usepackage{float}
\usepackage{multicol}

\usepackage[p,osf]{scholax}

\renewcommand*\contentsname{Table of Contents}
\renewcommand{\headrulewidth}{0pt}
\pagestyle{fancy}
\fancyhf{}
\fancyfoot[R]{$\thepage$}
\setlength{\parindent}{0cm}
\setlength{\headheight}{17pt}
\hfuzz=9pt

% Figures
\usepackage{svg}

% Utility Management
\usepackage{color}
\usepackage{colortbl}
\usepackage{xcolor}
\usepackage{xpatch}
\usepackage{xparse}

\definecolor{gBlue}{HTML}{7daea3}
\definecolor{gOrange}{HTML}{e78a4e}
\definecolor{gGreen}{HTML}{a9b665}
\definecolor{gPurple}{HTML}{d3869b}

\definecolor{links}{HTML}{1c73a5}
\definecolor{bar}{HTML}{584AA8}

% Math Packages
\usepackage{mathtools, amsmath, amsthm, thmtools, amssymb, physics}
\usepackage[scaled=1.075,ncf,vvarbb]{newtxmath}

\newcommand\B{\mathbb{C}}
\newcommand\C{\mathbb{C}}
\newcommand\R{\mathbb{R}}
\newcommand\Q{\mathbb{Q}}
\newcommand\N{\mathbb{N}}
\newcommand\Z{\mathbb{Z}}

\DeclareMathOperator{\lcm}{lcm}

% Probability Theory
\newcommand\Prob[1]{\mathbb{P}\qty(#1)}
\newcommand\Var[1]{\text{Var}\qty(#1)}
\newcommand\Exp[1]{\mathbb{E}\qty[#1]}

% Analysis
\newcommand\ball[1]{\B\qty(#1)}
\newcommand\conj[1]{\overline{#1}}
\DeclareMathOperator{\Arg}{Arg}
\DeclareMathOperator{\cis}{cis}

% Linear Algebra
\DeclareMathOperator{\dom}{dom}
\DeclareMathOperator{\range}{range}
\DeclareMathOperator{\spann}{span}
\DeclareMathOperator{\nullity}{nullity}

% TIKZ
\usepackage{tikz}
\usepackage{pgfplots}
\usetikzlibrary{arrows.meta}
\usetikzlibrary{math}
\usetikzlibrary{cd}

% Boxes and Theorems
\usepackage[most]{tcolorbox}
\tcbuselibrary{skins}
\tcbuselibrary{breakable}
\tcbuselibrary{theorems}

\newtheoremstyle{default}{0pt}{0pt}{}{}{\bfseries}{\normalfont.}{0.5em}{}
\theoremstyle{default}

\renewcommand*{\proofname}{\textit{\textbf{Proof.}}}
\renewcommand*{\qedsymbol}{$\blacksquare$}
\tcolorboxenvironment{proof}{
	breakable,
	coltitle = black,
	colback = white,
	frame hidden,
	boxrule = 0pt,
	boxsep = 0pt,
	borderline west={3pt}{0pt}{bar},
	% borderline west={3pt}{0pt}{gPurple},
	sharp corners = all,
	enhanced,
}

\newtheorem{theorem}{Theorem}[section]{\bfseries}{}
\tcolorboxenvironment{theorem}{
	breakable,
	enhanced,
	boxrule = 0pt,
	frame hidden,
	coltitle = black,
	colback = blue!7,
	% colback = gBlue!30,
	left = 0.5em,
	sharp corners = all,
}

\newtheorem{corollary}{Corollary}[section]{\bfseries}{}
\tcolorboxenvironment{corollary}{
	breakable,
	enhanced,
	boxrule = 0pt,
	frame hidden,
	coltitle = black,
	colback = white!0,
	left = 0.5em,
	sharp corners = all,
}

\newtheorem{lemma}{Lemma}[section]{\bfseries}{}
\tcolorboxenvironment{lemma}{
	breakable,
	enhanced,
	boxrule = 0pt,
	frame hidden,
	coltitle = black,
	colback = green!7,
	left = 0.5em,
	sharp corners = all,
}

\newtheorem{definition}{Definition}[section]{\bfseries}{}
\tcolorboxenvironment{definition}{
	breakable,
	coltitle = black,
	colback = white,
	frame hidden,
	boxsep = 0pt,
	boxrule = 0pt,
	borderline west = {3pt}{0pt}{orange},
	% borderline west = {3pt}{0pt}{gOrange},
	sharp corners = all,
	enhanced,
}

\newtheorem{example}{Example}[section]{\bfseries}{}
\tcolorboxenvironment{example}{
	% title = \textbf{Example},
	% detach title,
	% before upper = {\tcbtitle\quad},
	breakable,
	coltitle = black,
	colback = white,
	frame hidden,
	boxrule = 0pt,
	boxsep = 0pt,
	borderline west={3pt}{0pt}{green!70!black},
	% borderline west={3pt}{0pt}{gGreen},
	sharp corners = all,
	enhanced,
}

\newtheoremstyle{remark}{0pt}{4pt}{}{}{\bfseries\itshape}{\normalfont.}{0.5em}{}
\theoremstyle{remark}
\newtheorem*{remark}{Remark}


% TColorBoxes
\newtcolorbox{week}{
	colback = black,
	coltext = white,
	fontupper = {\large\bfseries},
	width = 1.2\paperwidth,
	size = fbox,
	halign upper = center,
	center
}

\newcommand{\banner}[2]{
    \pagebreak
    \begin{week}
   		\section*{#1}
    \end{week}
    \addcontentsline{toc}{section}{#1}
    \addtocounter{section}{1}
    \setcounter{subsection}{0}
}

% Hyperref
\usepackage{hyperref}
\hypersetup{
	colorlinks=true,
	linktoc=all,
	linkcolor=links,
	bookmarksopen=true
}

% Error Handling
\PackageWarningNoLine{ExtSizes}{It is better to use one of the extsizes 
                          classes,^^J if you can}


\def\homeworknumber{2}
\fancyhead[R]{\textbf{Math 140A: Homework \#\homeworknumber}}
\fancyhead[L]{Eli Griffiths}
\renewcommand{\headrulewidth}{1pt}
\setlength\parindent{0pt}


% Sec 1.4:  9, 10, 12, 15.
% Sec 1.5:  1, 8, 11, 19, 20.
% Sec 1.6:  2, 4, 9, 24

\begin{document}

\subsection*{1.4.9}
Let $\mqty(a & b \\ c & d) \in M_{2\times 2}(\mathbb{F})$ with $a,b,c,d \in \mathbb{F}$. Note that it can be written as $a \mqty(1 & 0 \\ 0 & 1) + b \mqty(0 & 1 \\ 0 & 0) + c \mqty(0 & 0 \\ 1 & 0) + d \mqty(0 & 0 \\ 0 & 1)$. Therefore $M_{2\times 2}(\mathbb{F})$ is generated by those matrices.

\subsection*{1.4.10}
Consider a linear combination of the matrices with coefficients $a,b,c \in \mathbb{F}$
\[
	a\mqty(1 & 0 \\ 0 & 0) + b\mqty(0 & 0 \\ 0 & 1) + c\mqty(0 & 1 \\ 1 & 0) = \mqty(a & c \\ c & b)
\]
Notice the linear combination is a symmetric matrix. Therefore $\spann\qty{M_1, M_2, M_3}$ is the set of all symmetric $2 \times 2$ matrices.

\subsection*{1.4.12}
\begin{proof}
	Let $V$ be a vector space and $W \subset V$.
	\begin{enumerate}
		\item[$\Rightarrow)$]
			Assume that $W$ is a subspace of $V$. This means that $W$ is a subset of $V$ and is closed under linear combinations. This means that the set of all linear combinations of vectors in $W$ stay within $W$. Equivalently, $\spann\qty{W} = W$.
		\item[$\Leftarrow)$]
			Assume that $\spann\qty{W} = W$. Since $W \subset V$, by theorem 1.5 (the span of subset of a vector space is a subspace) the span of $W$ is a subspace of $V$.
	\end{enumerate}
\end{proof}

\subsection*{1.4.15}
\begin{proof}
	Let $V$ be a vector space and $S_1, S_2 \subset V$. Let $v \in \spann\qty{S_1 \cap S_2}$ such that $v = c_1 x_1 + c_2 x_2 + \ldots + c_n x_n$ where $x_i \in S_1 \cap S_2$. Therefore $x_i \in S_1$ and $x_i \in S_2$. This means that the linear combination is in both $\spann\qty{S_1}$ and $\spann\qty{S_2}$. Therefore $v \in \spann\qty{S_1}\cap\spann\qty{S_2}$.
\end{proof}

An example where both are equal is $S_1 = S_2 = \mathbb{R}$. An example where they are not equal is $S_1 = \qty{0, 1}$ and $S_2 = \qty{0, 2}$.

\subsection*{1.5.1}
\begin{enumerate}[label=\alph*)]
	\item False
	\item True
	\item False
	\item False
	\item True
	\item True
\end{enumerate}

\subsection*{1.5.8}
\subsubsection*{Part A}
\begin{proof}
	Let $S = \qty{(1,1,0), (1, 0, 1), (0,1,1)} \subset \mathbb{R}^3$. Assume towards contradiction that $S$ is linearly dependent. Then there exists $a,b,c \in \mathbb{R} \neq 0$ such that 
    \[
        a(1,1,0) + b(1,0,1) + c(0,1,1) = 0
    \]
    This means that
    \begin{align*}
        a + b &= 0 \\
        a + c &= 0 \\
        b + c &= 0
    \end{align*}
    However, $a + b = 0 \implies a = -b$ and therefore $a + c = -b + c \implies b = c$. Since $b+c = 0$, $b + b = 0 \implies b = 0$, contradicting the assumption that $S$ is dependent. Therefore $S$ is independent.
\end{proof}

\subsubsection*{Part B}
\begin{proof}
	Let $S = \qty{(1,1,0), (1, 0, 1), (0,1,1)} \subset \mathbb{F}^3$ where $\mathbb{F}$ has characteristic 2. Note then that
    \[
        (1, 1, 0) + (0,1,1) = (1,0,1)
    \]
    Therefore $S$ is linearly dependent.
\end{proof}

\subsection*{1.5.11}
Since there are two scalar options for each vector ($0$ or $1$), there are $2^n$ linear combination choices and therefore there are $2^n$ vectors in the span.

\subsection*{1.5.19}
\begin{proof}
	Let $\qty{A_1, A_2, \ldots, A_k}$ be a subset of $M_{n\times n}(\mathbb{F})$ and assume that is linearly independent. Therefore $c_1 A_1 + c_2 A_2 + \ldots + c_k A_k = 0$ only when $c_i = 0$. Note that
	\begin{align*}
		c_1 A_1 + c_2 A_2 + \ldots c_k A_k &= 0 \\
		(c_1 A_1 + c_2 A_2 + \ldots c_k A_k)^\transp &= 0^\transp \\
		(c_1 A_1)^\transp + (c_2 A_2)^\transp + \ldots (c_k A_k)^\transp &= 0 \\
		c_1 A_1^\transp + c_2 A_2^\transp + \ldots c_k A_k^\transp &= 0 \\
	\end{align*}
	Therefore the linear combination of the transposes is also equal to zero only when $c_i = 0$, meaning the set of transposes is a linearly independent subset.
\end{proof}

\subsection*{1.5.20}
\begin{proof}
	Let $f,g \in \mathcal{F}(\mathbb{R}, \mathbb{R})$ where $f(t) = e^{rt}$ and $g(t) = e^{st}$ with $r \neq s$. Assume towards contradiction that they are linearly dependent. That is $c_1 e^{rt} + c_2 e^{st} = 0$ with $c_1, c_2 \neq 0$. Then
	\begin{align*}
		c_1 e^{rt} + c_2 e^{st} &= 0 \\
		c_1 + c_2 e^{(s-r)t} &= 0 \\
		c_2 e^{(s-r)t} &= -c_1 \\
		e^{(s-r)t} &= -\frac{c_1}{c_2}
	\end{align*}
	However, this implies that $e^{(s-r)t}$ is a constant function which it is not. Therefore $c_1, c_2 = 0$.
\end{proof}


\subsection*{1.6.2}
% b and e are not bases
All are bases except for $b$ and $e$.

\subsection*{1.6.4}
The polynomials do not generate the space as there are only $3$ independent vectors which cannot generate a $4$ dimensional space.

\subsection*{1.6.9}
\[
    \mqty(a_1 \\ a_2 \\ a_3 \\ a_4) = a_1 u_1 + (a_2 - a_1) u_2 + (a_3 - a_2) u_3 + (a_4 - a_3) u_4
\]


\subsection*{1.6.24}
\begin{proof}
	Let $f \in \mathcal{P}_n (\mathbb{R})$ such that $f$ has degree $n$. The question of if any $g$ in $\mathcal{P}_n(\mathbb{R})$ can be written as
	\[
		g(x) = c_0 f(x) + c_1 f'(x) + \ldots + c_n f^{(n)}(x)
	\]
	is the same as the question does $\spann\qty{f, f', \ldots, f^{(n)}} = \mathcal{P}_n(\mathbb{R})$. Note that the set $\qty{f, f', \ldots, f^{(n)}}$ is linearly independent since each entry is of a different order. Since the dimension of $\mathcal{P}_n(\mathbb{R})$ is $n+1$ and $\qty{f, f', \ldots, f^{(n)}}$ is a set of $n+1$ linearly independent vectors in $\mathcal{P}_n(\mathbb{R})$, $\qty{f, f', \ldots, f^{(n)}}$ forms a basis and therefore $\spann\qty{f, f', \ldots, f^{(n)}} = \mathcal{P}_n(\mathbb{R})$. This implies that any function $g \in \mathcal{P}_n(\mathbb{R})$ can be represented as
	\[
		g(x) = c_0 f(x) + c_1 f'(x) + \ldots + c_n f^{(n)}(x)
	.\]
\end{proof}

\end{document}
