\documentclass[12pt,titlepage]{extarticle}
% Document Layout and Font
\usepackage{subfiles}
\usepackage[margin=2cm, headheight=15pt]{geometry}
\usepackage{fancyhdr}
\usepackage{enumitem}	
\usepackage{wrapfig}
\usepackage{float}
\usepackage{multicol}

\usepackage[p,osf]{scholax}

\renewcommand*\contentsname{Table of Contents}
\renewcommand{\headrulewidth}{0pt}
\pagestyle{fancy}
\fancyhf{}
\fancyfoot[R]{$\thepage$}
\setlength{\parindent}{0cm}
\setlength{\headheight}{17pt}
\hfuzz=9pt

% Figures
\usepackage{svg}

% Utility Management
\usepackage{color}
\usepackage{colortbl}
\usepackage{xcolor}
\usepackage{xpatch}
\usepackage{xparse}

\definecolor{gBlue}{HTML}{7daea3}
\definecolor{gOrange}{HTML}{e78a4e}
\definecolor{gGreen}{HTML}{a9b665}
\definecolor{gPurple}{HTML}{d3869b}

\definecolor{links}{HTML}{1c73a5}
\definecolor{bar}{HTML}{584AA8}

% Math Packages
\usepackage{mathtools, amsmath, amsthm, thmtools, amssymb, physics}
\usepackage[scaled=1.075,ncf,vvarbb]{newtxmath}

\newcommand\B{\mathbb{C}}
\newcommand\C{\mathbb{C}}
\newcommand\R{\mathbb{R}}
\newcommand\Q{\mathbb{Q}}
\newcommand\N{\mathbb{N}}
\newcommand\Z{\mathbb{Z}}

\DeclareMathOperator{\lcm}{lcm}

% Probability Theory
\newcommand\Prob[1]{\mathbb{P}\qty(#1)}
\newcommand\Var[1]{\text{Var}\qty(#1)}
\newcommand\Exp[1]{\mathbb{E}\qty[#1]}

% Analysis
\newcommand\ball[1]{\B\qty(#1)}
\newcommand\conj[1]{\overline{#1}}
\DeclareMathOperator{\Arg}{Arg}
\DeclareMathOperator{\cis}{cis}

% Linear Algebra
\DeclareMathOperator{\dom}{dom}
\DeclareMathOperator{\range}{range}
\DeclareMathOperator{\spann}{span}
\DeclareMathOperator{\nullity}{nullity}

% TIKZ
\usepackage{tikz}
\usepackage{pgfplots}
\usetikzlibrary{arrows.meta}
\usetikzlibrary{math}
\usetikzlibrary{cd}

% Boxes and Theorems
\usepackage[most]{tcolorbox}
\tcbuselibrary{skins}
\tcbuselibrary{breakable}
\tcbuselibrary{theorems}

\newtheoremstyle{default}{0pt}{0pt}{}{}{\bfseries}{\normalfont.}{0.5em}{}
\theoremstyle{default}

\renewcommand*{\proofname}{\textit{\textbf{Proof.}}}
\renewcommand*{\qedsymbol}{$\blacksquare$}
\tcolorboxenvironment{proof}{
	breakable,
	coltitle = black,
	colback = white,
	frame hidden,
	boxrule = 0pt,
	boxsep = 0pt,
	borderline west={3pt}{0pt}{bar},
	% borderline west={3pt}{0pt}{gPurple},
	sharp corners = all,
	enhanced,
}

\newtheorem{theorem}{Theorem}[section]{\bfseries}{}
\tcolorboxenvironment{theorem}{
	breakable,
	enhanced,
	boxrule = 0pt,
	frame hidden,
	coltitle = black,
	colback = blue!7,
	% colback = gBlue!30,
	left = 0.5em,
	sharp corners = all,
}

\newtheorem{corollary}{Corollary}[section]{\bfseries}{}
\tcolorboxenvironment{corollary}{
	breakable,
	enhanced,
	boxrule = 0pt,
	frame hidden,
	coltitle = black,
	colback = white!0,
	left = 0.5em,
	sharp corners = all,
}

\newtheorem{lemma}{Lemma}[section]{\bfseries}{}
\tcolorboxenvironment{lemma}{
	breakable,
	enhanced,
	boxrule = 0pt,
	frame hidden,
	coltitle = black,
	colback = green!7,
	left = 0.5em,
	sharp corners = all,
}

\newtheorem{definition}{Definition}[section]{\bfseries}{}
\tcolorboxenvironment{definition}{
	breakable,
	coltitle = black,
	colback = white,
	frame hidden,
	boxsep = 0pt,
	boxrule = 0pt,
	borderline west = {3pt}{0pt}{orange},
	% borderline west = {3pt}{0pt}{gOrange},
	sharp corners = all,
	enhanced,
}

\newtheorem{example}{Example}[section]{\bfseries}{}
\tcolorboxenvironment{example}{
	% title = \textbf{Example},
	% detach title,
	% before upper = {\tcbtitle\quad},
	breakable,
	coltitle = black,
	colback = white,
	frame hidden,
	boxrule = 0pt,
	boxsep = 0pt,
	borderline west={3pt}{0pt}{green!70!black},
	% borderline west={3pt}{0pt}{gGreen},
	sharp corners = all,
	enhanced,
}

\newtheoremstyle{remark}{0pt}{4pt}{}{}{\bfseries\itshape}{\normalfont.}{0.5em}{}
\theoremstyle{remark}
\newtheorem*{remark}{Remark}


% TColorBoxes
\newtcolorbox{week}{
	colback = black,
	coltext = white,
	fontupper = {\large\bfseries},
	width = 1.2\paperwidth,
	size = fbox,
	halign upper = center,
	center
}

\newcommand{\banner}[2]{
    \pagebreak
    \begin{week}
   		\section*{#1}
    \end{week}
    \addcontentsline{toc}{section}{#1}
    \addtocounter{section}{1}
    \setcounter{subsection}{0}
}

% Hyperref
\usepackage{hyperref}
\hypersetup{
	colorlinks=true,
	linktoc=all,
	linkcolor=links,
	bookmarksopen=true
}

% Error Handling
\PackageWarningNoLine{ExtSizes}{It is better to use one of the extsizes 
                          classes,^^J if you can}


\def\homeworknumber{8}
\fancyhead[R]{\textbf{Math 140A: Homework \#\homeworknumber}}
\fancyhead[L]{Eli Griffiths}
\renewcommand{\headrulewidth}{1pt}
\setlength\parindent{0pt}


% 15.1, 15.2b, 15.3, 15.5, 15.7a, 17.1, 17.2, 17.5, 17.6

\begin{document}

\subsection*{15.1}
\subsubsection*{Part A}
The series converges by the alternating series test since $\frac{1}{n} > \frac{1}{n+1}$ and $\lim\frac{1}{n} = 0$.

\subsubsection*{Part B}
The series diverges by the ratio test since $\qty|\frac{a_{n+1}}{a_n}| = \frac{n+1}{2} \to \infty$.

\subsection*{15.2}
\subsubsection*{Part B}
Since $-1 < \sin(\frac{\pi n}{7}) < 1$ for all $n$, $|\sin(\frac{\pi n}{7})| < 1$. Since it is also periodic, it is possible to choose $1 > r > \max\qty{\sin(\frac{\pi n}{7}) : n = 1,\ldots,14}$. Note then that $|\sin(\frac{\pi n}{7})|^n < r^n$. Since $0 \leq r < 1$, by the comparison test it follows that the original series converges absolutely and hence converges.

\subsection*{15.3}
\begin{proof}
    Suppose that $p > 1$. Then
    \begin{align*}
        \int_{2}^n \frac{1}{x (\log x)^p} \dd x &= \int_{\log 2}^{\log n} \frac{1}{u^p} \dd u \\
                                                &= -\frac{1}{p-1}\qty(\frac{1}{(\log p)^{p-1}} - \frac{1}{(\log 2)^{p-1}}) \\
                                                &= \frac{1}{p-1}\qty(\frac{1}{(\log 2)^{p-1}} - \frac{1}{(\log p)^{p-1}}) \xrightarrow{n \to \infty} \frac{1}{p-1}\cdot \frac{1}{(\log 2)^{p-1}}
    \end{align*}
    Therefore the interval converges and therefore by the integral test the series converges. If $p = 1$, then
    \begin{align*}
        \int_{2}^n \frac{1}{x \log x} &= \int_{\log 2}^{\log n} \frac{1}{u} \dd u = \log(\log n) - \log (\log 2) \xrightarrow{n \to \infty} \infty
    \end{align*}
    Hence the series diverges by the integral test. Assume that $0 < p < 1$. Then by the first case of $p > 1$,
    \begin{align*}
        \int_{2}^n \frac{1}{x (\log x)^p} \dd x &= \frac{1}{p-1}\qty(\frac{1}{(\log 2)^{p-1}} - \frac{1}{(\log p)^{p-1}}) \\
                                                &= \frac{1}{1 - p}\qty((\log n)^{1-p} - (\log 2)^{1-p}) \xrightarrow{n \to \infty} \infty
    \end{align*}
    Therefore the series diverges by the integral test. If $p \leq 0$, then the series terms do not converge to $0$ and therefore the series does not converge.
\end{proof}


\subsection*{15.5}
It wouldn't be useful to use the comparison test as it would require using an exponent larger than $p$ to compare with, which is a part of the result that is trying to be proven.

\subsection*{15.7}
\subsubsection*{Part A}
\begin{proof}
    Let $(a_n)$ be a sequence and assume that it is decreasing and that $\sum a_n$ converges. Note that this means $a_n > 0$ for all $n$ and that $a_n \to 0$. Let $\epsilon > 0$. Since $\sum a_n$ converges, by the Cauchy criterion there exists $M \in \mathbb{N}$ such that for $n > M$
    \[
        a_{M + 1} + \ldots a_n < \frac{\epsilon}{2}
    .\]
    Since $(a_n)$ converges, there is some $P \in \mathbb{N}$ such that $n \geq P$ implies $a_n < \frac{\epsilon}{2M}$. Let $N = \max\qty{M, P}$. Then for $n > N$
    \begin{align*}
        n \cdot a_n &= \underbrace{a_n + \ldots + a_n}_{n \text{ times}} \\
                    &\leq \underbrace{a_P + \ldots + a_P}_{m \text{ times}} + \underbrace{a_{M + 1} + \ldots + a_n}_{n - m \text{ times}} \\
                    &< M a_P + \frac{\epsilon}{2} \\
                    &< M \cdot \frac{\epsilon}{2M} + \frac{\epsilon}{2} = \epsilon
    \end{align*}
    Therefore $\lim n a_n = 0$.
\end{proof}

\subsection*{17.1}
\subsubsection*{Part A}
\begin{align*}
    \dom(f+g) &\implies (-\infty, 4] \\
    \dom(fg) &\implies (-\infty, 4] \\
    \dom(f \circ g) &\implies [-2, 2] \\
    \dom(g \circ f) &= (-\infty, 4]
\end{align*}

\subsubsection*{Part B}
\begin{align*}
    (f\circ g)(0) &= 2 \\
    (g \circ f)(0) &= 4 \\
    (f\circ g)(1) &= \sqrt{3} \\
    (g \circ f)(1) &= 3 \\
    (f\circ g)(2) &= 0 \\
    (g \circ f)(2) &= 2 \\
\end{align*}

\subsubsection*{Part C}
The functions are not equal.

\subsubsection*{Part D}
Only $(g\circ f)(3)$ is meaningful as the $x$ value is in its domain.

\subsection*{17.2}
\subsubsection*{Part A}
\begin{align*}
    (f+g)(x) &= \begin{cases}
        x^2 & x < 0 \\
        4 + x^2 & x \geq 0
    \end{cases} \\
    (fg)(x) &= \begin{cases}
        0 & x < 0 \\
        4x^2 & x \geq 0
    \end{cases} \\
    (f \circ g)(x) &= 4, \forall x \in \mathbb{R} \\
    (g\circ f)(x) &= \begin{cases}
        0 & x < 0 \\
        16 & x \geq 0
    \end{cases}
\end{align*}

\subsubsection*{Part B}
Only $g, fg$ and $f \circ g$ are continuous.

\subsection*{17.5}
\begin{proof}
    Consider the real valued function $f(x) = x^m$ on all of $\mathbb{R}$ where $m \in \mathbb{N}$. Let $(x_n)$ be a sequence in $\mathbb{R}$ that converges to $x_0 \in \mathbb{R}$. Note then that $\lim f(x_n) = \lim (x_n)^m = \qty(\lim x_n)^m = x_0^m = f(x_0)$. Therefore $f(x)$ is continuous at $x_0$ and hence on all of $\mathbb{R}$.
\end{proof}

\subsection*{17.6}
\begin{proof}
    Let $p(x)$ and $q(x)$ be real polynomials and consider the domain $D = \qty{x \in \mathbb{R} : q(x) \neq 0}$. Let $(x_n)$ be a sequence in $D$ that converges to $x_0 \in D$. Note then that
    \[
        \lim \frac{p(x_n)}{q(x_n)} \xRightarrow{q(x) \neq 0} \frac{\lim p(x_n)}{\lim q(x_n)} = \frac{p(x_0)}{q(x_0)}
    .\]
    Therefore the rational function $\frac{p}{q}$ on the domain $D$ is continuous.
\end{proof}


\end{document}
