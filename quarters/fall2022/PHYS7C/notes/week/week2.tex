\documentclass{standalone}

\begin{document}

\subsection{Applications of Newtons Laws}\label{newtonsapplications}

\textbf{How to deal with multiple objects:} 

\noindent
As long as there is no relative motion between multiple objects, you can treat them as a single object. Otherwise, construct a free body diagram for all objects in the system.

\subsubsection{Equilibrium}\label{equilibrium}

\begin{align*}
			&	\text{In Equilibrium} & \text{Not In Equilibrium} & \\
			&	\sum \vec{F} = \vec{0} & \sum \vec{F} = \vec{0} &
\end{align*}

When an object is in equilibrium, it's velocity remain constant (and therefore its trajectory remains constant and linear).

\subsubsection{Friction Forces}\label{friction}

\begin{align*}
				f^{riction}_k &= \mu_k n \\
				f^{riction}_s &\le \mu_s n 
.\end{align*}

In general $\mu_k < \mu_s$, but both coefficients of friction can hold values larger than 1. (\textit{Note: both coefficients are dimensionless; no units}). 

\end{document}
