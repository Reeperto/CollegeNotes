\documentclass[12pt]{extarticle}

\usepackage{multicol}

% Document Layout and Font
\usepackage{subfiles}
\usepackage[margin=2cm, headheight=15pt]{geometry}
\usepackage{fancyhdr}
\usepackage{enumitem}	
\usepackage{wrapfig}
\usepackage{float}
\usepackage{multicol}

\usepackage[p,osf]{scholax}

\renewcommand*\contentsname{Table of Contents}
\renewcommand{\headrulewidth}{0pt}
\pagestyle{fancy}
\fancyhf{}
\fancyfoot[R]{$\thepage$}
\setlength{\parindent}{0cm}
\setlength{\headheight}{17pt}
\hfuzz=9pt

% Figures
\usepackage{svg}

% Utility Management
\usepackage{color}
\usepackage{colortbl}
\usepackage{xcolor}
\usepackage{xpatch}
\usepackage{xparse}

\definecolor{gBlue}{HTML}{7daea3}
\definecolor{gOrange}{HTML}{e78a4e}
\definecolor{gGreen}{HTML}{a9b665}
\definecolor{gPurple}{HTML}{d3869b}

\definecolor{links}{HTML}{1c73a5}
\definecolor{bar}{HTML}{584AA8}

% Math Packages
\usepackage{mathtools, amsmath, amsthm, thmtools, amssymb, physics}
\usepackage[scaled=1.075,ncf,vvarbb]{newtxmath}

\newcommand\B{\mathbb{C}}
\newcommand\C{\mathbb{C}}
\newcommand\R{\mathbb{R}}
\newcommand\Q{\mathbb{Q}}
\newcommand\N{\mathbb{N}}
\newcommand\Z{\mathbb{Z}}

\DeclareMathOperator{\lcm}{lcm}

% Probability Theory
\newcommand\Prob[1]{\mathbb{P}\qty(#1)}
\newcommand\Var[1]{\text{Var}\qty(#1)}
\newcommand\Exp[1]{\mathbb{E}\qty[#1]}

% Analysis
\newcommand\ball[1]{\B\qty(#1)}
\newcommand\conj[1]{\overline{#1}}
\DeclareMathOperator{\Arg}{Arg}
\DeclareMathOperator{\cis}{cis}

% Linear Algebra
\DeclareMathOperator{\dom}{dom}
\DeclareMathOperator{\range}{range}
\DeclareMathOperator{\spann}{span}
\DeclareMathOperator{\nullity}{nullity}

% TIKZ
\usepackage{tikz}
\usepackage{pgfplots}
\usetikzlibrary{arrows.meta}
\usetikzlibrary{math}
\usetikzlibrary{cd}

% Boxes and Theorems
\usepackage[most]{tcolorbox}
\tcbuselibrary{skins}
\tcbuselibrary{breakable}
\tcbuselibrary{theorems}

\newtheoremstyle{default}{0pt}{0pt}{}{}{\bfseries}{\normalfont.}{0.5em}{}
\theoremstyle{default}

\renewcommand*{\proofname}{\textit{\textbf{Proof.}}}
\renewcommand*{\qedsymbol}{$\blacksquare$}
\tcolorboxenvironment{proof}{
	breakable,
	coltitle = black,
	colback = white,
	frame hidden,
	boxrule = 0pt,
	boxsep = 0pt,
	borderline west={3pt}{0pt}{bar},
	% borderline west={3pt}{0pt}{gPurple},
	sharp corners = all,
	enhanced,
}

\newtheorem{theorem}{Theorem}[section]{\bfseries}{}
\tcolorboxenvironment{theorem}{
	breakable,
	enhanced,
	boxrule = 0pt,
	frame hidden,
	coltitle = black,
	colback = blue!7,
	% colback = gBlue!30,
	left = 0.5em,
	sharp corners = all,
}

\newtheorem{corollary}{Corollary}[section]{\bfseries}{}
\tcolorboxenvironment{corollary}{
	breakable,
	enhanced,
	boxrule = 0pt,
	frame hidden,
	coltitle = black,
	colback = white!0,
	left = 0.5em,
	sharp corners = all,
}

\newtheorem{lemma}{Lemma}[section]{\bfseries}{}
\tcolorboxenvironment{lemma}{
	breakable,
	enhanced,
	boxrule = 0pt,
	frame hidden,
	coltitle = black,
	colback = green!7,
	left = 0.5em,
	sharp corners = all,
}

\newtheorem{definition}{Definition}[section]{\bfseries}{}
\tcolorboxenvironment{definition}{
	breakable,
	coltitle = black,
	colback = white,
	frame hidden,
	boxsep = 0pt,
	boxrule = 0pt,
	borderline west = {3pt}{0pt}{orange},
	% borderline west = {3pt}{0pt}{gOrange},
	sharp corners = all,
	enhanced,
}

\newtheorem{example}{Example}[section]{\bfseries}{}
\tcolorboxenvironment{example}{
	% title = \textbf{Example},
	% detach title,
	% before upper = {\tcbtitle\quad},
	breakable,
	coltitle = black,
	colback = white,
	frame hidden,
	boxrule = 0pt,
	boxsep = 0pt,
	borderline west={3pt}{0pt}{green!70!black},
	% borderline west={3pt}{0pt}{gGreen},
	sharp corners = all,
	enhanced,
}

\newtheoremstyle{remark}{0pt}{4pt}{}{}{\bfseries\itshape}{\normalfont.}{0.5em}{}
\theoremstyle{remark}
\newtheorem*{remark}{Remark}


% TColorBoxes
\newtcolorbox{week}{
	colback = black,
	coltext = white,
	fontupper = {\large\bfseries},
	width = 1.2\paperwidth,
	size = fbox,
	halign upper = center,
	center
}

\newcommand{\banner}[2]{
    \pagebreak
    \begin{week}
   		\section*{#1}
    \end{week}
    \addcontentsline{toc}{section}{#1}
    \addtocounter{section}{1}
    \setcounter{subsection}{0}
}

% Hyperref
\usepackage{hyperref}
\hypersetup{
	colorlinks=true,
	linktoc=all,
	linkcolor=links,
	bookmarksopen=true
}

% Error Handling
\PackageWarningNoLine{ExtSizes}{It is better to use one of the extsizes 
                          classes,^^J if you can}


\newcommand{\powerset}[1]{\mathcal{P}(#1)}
\fancyhead[R]{\textbf{Homework \#8}}
\renewcommand{\headrulewidth}{1pt}
\setlength\parindent{0pt}

\usetikzlibrary{arrows.meta}

\begin{document}

 % 7.3.1ab - , 7.3.4 - , 7.3.6 - , 7.3.10, 7.4.3 -

\section*{Problem 7.3.1}

A relation $\mathcal{R}$ is antisymmetric if $((x, y) \in \mathcal{R}) \land ((y, x) \in \mathcal{R}) \Rightarrow x = y$. Give examples of relations $\mathcal{R}$ on $A = \qty{1, 2, 3}$ having the stated property.

\begin{enumerate}
	\item[(a)] $\mathcal{R}$ is both symmetric and antisymmetric. 
	\item[(b)] $\mathcal{R}$ is neither symmetric nor antisymmetric.
\end{enumerate}

\subsection*{Solution}
\begin{multicols}{2}
	\centering
	\subsection*{Part A}
	\begin{align*}
		\mathcal{R} &= \qty{(1,1)} \\
		\mathcal{R} &= \qty{(3,3)} \\
		\mathcal{R} &= \qty{(1,1), (2,2), (3,3)}
	.\end{align*}

	\columnbreak

	\centering
	\subsection*{Part B}
	\begin{align*}
		\mathcal{R} &= \qty{(1,2), (2,3), (2,1)} \\
		\mathcal{R} &= \qty{(1,3), (1,2), (3,1)} \\
		\mathcal{R} &= \qty{(1,2), (3,1), (1,3)}
	.\end{align*}
\end{multicols}

\section*{Problem 7.3.4}
\begin{enumerate}
	\item[(a)] Let $\sim$ be the relation defined on $\mathbb{Z}$ by $a \sim b \Longleftrightarrow a + b$ is even. Show that $\sim$ is an equivalence relation and determine the distinct equivalence classes.
	\item[(b)] Suppose that 'even' is replaced by 'odd' in part (a). Which of the properties reflexive, symmetric, transitive does $\sim$ posses?
\end{enumerate}

\subsection*{Solution}
\subsection*{Part A}

\begin{proof}
	Proceed to show that $\sim$ is an equivalence relation on $\mathbb{Z}$. \\

	\qquad\begin{minipage}{\dimexpr\textwidth-2cm}
		(Reflexivity)\quad Let $a \in \mathbb{Z}$. Then $a + a = 2a$, which by definition is even. Therefore since $a + a$ is even it follows that $a \sim a$, hence $\sim$ is reflexive.
	\end{minipage} \\ \\

	\qquad\begin{minipage}{\dimexpr\textwidth-2cm}
		(Symmetry)\quad Let $a,b \in \mathbb{Z}$ and assume $a \sim b$. Therefore $a + b$ is even. Since $a + b = b + a$, it follows that $b+a$ is also even and therefore $b \sim a$. Hence $\sim$ is symmetric. 
	\end{minipage} \\ \\

	\qquad\begin{minipage}{\dimexpr\textwidth-2cm}
		(Transitivity)\quad Let $a,b,c \in \mathbb{Z}$ and assume that $a \sim b$ and $b \sim c$. Therefore $a + b$ and $b + c$ are both even, meaning $\exists m,n \in \mathbb{Z}$ such that $a + b = 2m$ and $b + c = 2n$. Then
		\begin{align*}
			(a + b) + (b + c) &= 2m + 2n \\
			a + 2b + c &= 2m + 2n \\
			a + c &= 2m + 2n - 2b \\
			a + c &= 2(m + n - b)
		.\end{align*}
		Since $m + n - b \in \mathbb{Z}, a + c$ is even and therefore $a \sim c$. Hence $\sim$ is transitive.
	\end{minipage} \\ \\

	Since $\sim$ is reflexive, symmetric, and transitive it is an equivalence relation.
\end{proof}

The two possible equivalence classes are $2\mathbb{Z}$ and $\mathbb{Z}\setminus 2\mathbb{Z}$. Note that if $x$ is an odd integer, then
\[
	[a] = \qty{b : a + b \text{ is odd}}
.\]
Since $a$ is odd, then $b$ has to also be odd for $a + b$ to be even, meaning the equivalence class of an odd integer is the odd integers. If $a$ is an even integer, then
\[
	[a] = \qty{b : a + b \text{ is even}}
.\]
Since $a$ is now an even integer, $b$ must also be an even integer for $a + b$ to be even, meaning the equivalence class of an even integer is the even integers.

\subsection*{Part B}

If $even$ is replaced with $odd$, then $\sim$ will be symmetric but not reflexive or transitive. Consider reflexivity. Let $a = 1$ and note that $a + a = 2$ which is not odd, hence $\sim$ would not be reflexive. Consider transitivity. Note $1 \sim 2$ and $2 \sim 3$ since $1 + 2 = 3$ and $2 + 3 = 5$, but $1 \not\sim 3$ because $1 + 3 = 4$ which is even. Consider symmetry. Let $a,b \in \mathbb{Z}$ and assume $a \sim b$. Then $\exists n \in \mathbb{N}$ such that $a + b = 2n + 1$. Equivalently $b + a = 2n+1$, therefore $b \sim a$. Hence $\sim$ would be symmetric.

\section*{Problem 7.3.6}
For the purposes of this question, we call a real number $x$ small if $|x| \leq 1$. Let $\mathcal{R}$ be the relation on the set of real numbers defined by
\[
	x\mathcal{R}y \Longleftrightarrow x - y \text{ is small}
.\]
\textit{Prove or disprove}: $\mathcal{R}$ is an equivalence relation on $\mathbb{R}$.

\subsection*{Solution}
\begin{proof}
	Proof that $\mathcal{R}$ is not an equivalence relation on $\mathbb{R}$. Let $a = 1, b = 2, c= 3$. Note that $|a - b| = 1 \leq 1$ and $|b - c| = 1 \leq 1$, therefore $a \mathcal{R} b$ and $b \mathcal{R} c$. However $|a - c| = 2 \nleq 1$, meaning $a$ does not relate $c$. Therefore $\mathcal{R}$ is not transitive and hence not an equivalence relation.
\end{proof}

\section*{Problem 7.3.10}
Let $A = \qty{2m : m \in \mathbb{Z}}$. A relation $\sim$ is defined on the set $\mathbb{Q}^+$ of positive rational numbers by
\[
	a \sim b \Longleftrightarrow \frac{a}{b} \in A
.\]
\begin{enumerate}
	\item[(a)] Show that $\sim$ is an equivalence relation.
	\item[(b)] Describe the elements in the equivalence class $[3]$.
\end{enumerate}

\subsection*{Solution}
\subsection*{Part A}

\begin{proof}
	Proceed to show that $\sim$ is an equivalence relation on $\mathbb{Q}^+$. \\

	\qquad\begin{minipage}{\dimexpr\textwidth-2cm}
		(Reflexivity)\quad Let $a \in \mathbb{Q}^+$. Note that $\frac{a}{a} = 1 = 2^0$. Since $2^0 \in A$, $\frac{a}{a} \in A$ meaning $a \sim a$. Therefore $\sim$ is reflexive.
	\end{minipage} \\ \\

	\qquad\begin{minipage}{\dimexpr\textwidth-2cm}
		(Symmetry)\quad Let $a, b \in \mathbb{Q}^+$ and assume that $a \sim b$. Therefore $\frac{a}{b} \in A$, meaning $\exists m \in \mathbb{Z}$ such that $\frac{a}{b} = 2^m$. Note that $\frac{b}{a} = 2^{-m}$. Since $2^{-m} \in A$, $\frac{b}{a} \in A$ meaning $b \sim a$. Therefore $\sim$ is symmetric.
	\end{minipage} \\ \\

	\qquad\begin{minipage}{\dimexpr\textwidth-2cm}
		(Transitivity)\quad Let $a,b,c \in \mathbb{Q}^+$ and assume that $a \sim b$ and $b \sim c$. Therefore $\frac{a}{b} \in A$ and $\frac{b}{c} \in A$, meaning $\exists m,n \in \mathbb{Z}$ such that $\frac{a}{b} = 2^m$ and $\frac{b}{c} = 2^n$. Then it follows that
		\begin{align*}
			\frac{a}{b} \cdot \frac{b}{c} &= 2^m \cdot 2^n \\
			\frac{a}{c} &= 2^{m+n}
		.\end{align*}
		Since $m + n \in \mathbb{Z}$, then $\frac{a}{c} \in A$. Therefore $a \sim c$ meaning $\sim$ is transitive.
	\end{minipage} \\ \\

	Since $\sim$ is reflexive, symmetric, and transitive it is an equivalence relation.
\end{proof}

\subsection*{Part B}
The equivalence class of $3$ can be described as
\begin{align*}
	[3] &= \qty{y: 3 \sim y} \\
			&= \qty{y: y \sim 3} \\
			&= \qty{y: \frac{y}{3} = 2^m, \forall m \in \mathbb{Z}} \\
			&= \qty{y: y = 3\cdot 2^m, \forall m \in \mathbb{Z}} \\
			&= \qty{3, \frac{3}{2}, 6, \frac{3}{4}, 12, \ldots}
.\end{align*}

\section*{Problem 7.4.3}
Let $X = \qty{1,2,3}$. Define a relation $\mathcal{R} = \qty{(1,1),(1,2),(1,3),(2,1),(2,2),(3,1),(3,3)}$ on $X$.

\begin{enumerate}
	\item[(a)] Which of the properties reflexive, symmetric, transitive are satisfied by $\mathcal{R}$?
	\item[(b)] Compute $A_1, A_2, A_3$ where $A_n = \qty{x \in X : x \mathcal{R} n}$. Show that $\qty{A_1, A_2, A_3}$ do not form a partition of $X$.
	\item[(c)] Repeat parts (a) and (b) for the relations $\mathcal{S}$ and $\mathcal{T}$ on $X$, where
		\begin{align*}
			\mathcal{S} &= \qty{(1, 1), (1, 3), (3, 1), (3, 3)} \\
			\mathcal{T} &= \qty{(1,1),(1,2),(1,3),(2,1),(2,2),(2,3),(3,3)}
		.\end{align*}
\end{enumerate}

\subsection*{Solution}
\subsection*{Part A}
$\mathcal{R}$ is reflexive and symmetric but not transitive.

\subsection*{Part B}
\begin{align*}
	A_1 &= \qty{1, 2, 3} \\
	A_2 &= \qty{1, 2} \\
	A_3 &= \qty{1, 3}
.\end{align*}

Since $A_1 \cap A_2 \neq \varnothing$, $A_1, A_2, A_3$ do not form a partition of $X$.

\section*{Part C}
{
\setlength{\columnsep}{1cm}
\setlength{\columnseprule}{1pt}
\begin{multicols}{2}
	\subsection*{Relation $\mathcal{S}$}

	$\mathcal{S}$ is symmetric but not transitive or reflexive.

	\begin{align*}
		A_1 &= \qty{1, 3} \\
		A_2 &= \varnothing \\
		A_3 &= \qty{1, 3}
	.\end{align*}

	These sets do not form a partition of $X$ since $A_1 \cup A_2 \cup A_3 = \qty{1,3} \neq X$.

	\columnbreak

	\subsection*{Relation $\mathcal{T}$}
	$\mathcal{T}$ is reflexive but not transitive or symmetric

	\begin{align*}
		A_1 &= \qty{1, 2} \\
		A_2 &= \qty{1, 2} \\
		A_3 &= \qty{1, 2, 3}
	.\end{align*}

	Since $A_1 \cap A_2 \neq \varnothing$, $A_1, A_2, A_3$ do not form a partition of $X$.

\end{multicols}
}

\end{document}
