\documentclass[12pt]{extarticle}

% Document Layout and Font
\usepackage{subfiles}
\usepackage[margin=2cm, headheight=15pt]{geometry}
\usepackage{fancyhdr}
\usepackage{enumitem}	
\usepackage{wrapfig}
\usepackage{float}
\usepackage{multicol}

\usepackage[p,osf]{scholax}

\renewcommand*\contentsname{Table of Contents}
\renewcommand{\headrulewidth}{0pt}
\pagestyle{fancy}
\fancyhf{}
\fancyfoot[R]{$\thepage$}
\setlength{\parindent}{0cm}
\setlength{\headheight}{17pt}
\hfuzz=9pt

% Figures
\usepackage{svg}

% Utility Management
\usepackage{color}
\usepackage{colortbl}
\usepackage{xcolor}
\usepackage{xpatch}
\usepackage{xparse}

\definecolor{gBlue}{HTML}{7daea3}
\definecolor{gOrange}{HTML}{e78a4e}
\definecolor{gGreen}{HTML}{a9b665}
\definecolor{gPurple}{HTML}{d3869b}

\definecolor{links}{HTML}{1c73a5}
\definecolor{bar}{HTML}{584AA8}

% Math Packages
\usepackage{mathtools, amsmath, amsthm, thmtools, amssymb, physics}
\usepackage[scaled=1.075,ncf,vvarbb]{newtxmath}

\newcommand\B{\mathbb{C}}
\newcommand\C{\mathbb{C}}
\newcommand\R{\mathbb{R}}
\newcommand\Q{\mathbb{Q}}
\newcommand\N{\mathbb{N}}
\newcommand\Z{\mathbb{Z}}

\DeclareMathOperator{\lcm}{lcm}

% Probability Theory
\newcommand\Prob[1]{\mathbb{P}\qty(#1)}
\newcommand\Var[1]{\text{Var}\qty(#1)}
\newcommand\Exp[1]{\mathbb{E}\qty[#1]}

% Analysis
\newcommand\ball[1]{\B\qty(#1)}
\newcommand\conj[1]{\overline{#1}}
\DeclareMathOperator{\Arg}{Arg}
\DeclareMathOperator{\cis}{cis}

% Linear Algebra
\DeclareMathOperator{\dom}{dom}
\DeclareMathOperator{\range}{range}
\DeclareMathOperator{\spann}{span}
\DeclareMathOperator{\nullity}{nullity}

% TIKZ
\usepackage{tikz}
\usepackage{pgfplots}
\usetikzlibrary{arrows.meta}
\usetikzlibrary{math}
\usetikzlibrary{cd}

% Boxes and Theorems
\usepackage[most]{tcolorbox}
\tcbuselibrary{skins}
\tcbuselibrary{breakable}
\tcbuselibrary{theorems}

\newtheoremstyle{default}{0pt}{0pt}{}{}{\bfseries}{\normalfont.}{0.5em}{}
\theoremstyle{default}

\renewcommand*{\proofname}{\textit{\textbf{Proof.}}}
\renewcommand*{\qedsymbol}{$\blacksquare$}
\tcolorboxenvironment{proof}{
	breakable,
	coltitle = black,
	colback = white,
	frame hidden,
	boxrule = 0pt,
	boxsep = 0pt,
	borderline west={3pt}{0pt}{bar},
	% borderline west={3pt}{0pt}{gPurple},
	sharp corners = all,
	enhanced,
}

\newtheorem{theorem}{Theorem}[section]{\bfseries}{}
\tcolorboxenvironment{theorem}{
	breakable,
	enhanced,
	boxrule = 0pt,
	frame hidden,
	coltitle = black,
	colback = blue!7,
	% colback = gBlue!30,
	left = 0.5em,
	sharp corners = all,
}

\newtheorem{corollary}{Corollary}[section]{\bfseries}{}
\tcolorboxenvironment{corollary}{
	breakable,
	enhanced,
	boxrule = 0pt,
	frame hidden,
	coltitle = black,
	colback = white!0,
	left = 0.5em,
	sharp corners = all,
}

\newtheorem{lemma}{Lemma}[section]{\bfseries}{}
\tcolorboxenvironment{lemma}{
	breakable,
	enhanced,
	boxrule = 0pt,
	frame hidden,
	coltitle = black,
	colback = green!7,
	left = 0.5em,
	sharp corners = all,
}

\newtheorem{definition}{Definition}[section]{\bfseries}{}
\tcolorboxenvironment{definition}{
	breakable,
	coltitle = black,
	colback = white,
	frame hidden,
	boxsep = 0pt,
	boxrule = 0pt,
	borderline west = {3pt}{0pt}{orange},
	% borderline west = {3pt}{0pt}{gOrange},
	sharp corners = all,
	enhanced,
}

\newtheorem{example}{Example}[section]{\bfseries}{}
\tcolorboxenvironment{example}{
	% title = \textbf{Example},
	% detach title,
	% before upper = {\tcbtitle\quad},
	breakable,
	coltitle = black,
	colback = white,
	frame hidden,
	boxrule = 0pt,
	boxsep = 0pt,
	borderline west={3pt}{0pt}{green!70!black},
	% borderline west={3pt}{0pt}{gGreen},
	sharp corners = all,
	enhanced,
}

\newtheoremstyle{remark}{0pt}{4pt}{}{}{\bfseries\itshape}{\normalfont.}{0.5em}{}
\theoremstyle{remark}
\newtheorem*{remark}{Remark}


% TColorBoxes
\newtcolorbox{week}{
	colback = black,
	coltext = white,
	fontupper = {\large\bfseries},
	width = 1.2\paperwidth,
	size = fbox,
	halign upper = center,
	center
}

\newcommand{\banner}[2]{
    \pagebreak
    \begin{week}
   		\section*{#1}
    \end{week}
    \addcontentsline{toc}{section}{#1}
    \addtocounter{section}{1}
    \setcounter{subsection}{0}
}

% Hyperref
\usepackage{hyperref}
\hypersetup{
	colorlinks=true,
	linktoc=all,
	linkcolor=links,
	bookmarksopen=true
}

% Error Handling
\PackageWarningNoLine{ExtSizes}{It is better to use one of the extsizes 
                          classes,^^J if you can}


\fancyhead[R]{Homework \#4}
\fancyhead[L]{Eli Griffiths}
\renewcommand{\headrulewidth}{1pt}
\setlength\parindent{0pt}

\begin{document}

\section*{Problem 1}
\subsection*{Part A}
$X$ is a geometric random variable with parameter $p = 0.2$.

\subsection*{Part B}
\[
	\Exp{X} = \frac{1}{p} = \frac{1}{0.2} = 5
.\]

\subsection*{Part C}
\begin{align*}
	\Var{X} &= \frac{1-p}{p^2} = \frac{0.8}{0.04} = 20 \\
	\sigma &= \sqrt{\Var{X}} = \sqrt{20}
.\end{align*}

\subsection*{Part D}
\[
	\Prob{X \leq 4} = \verb|pgeom(3,0.2)| = 0.5904
.\]

\subsection*{Part E}
Since the events are independent,
\[
	P(\text{Both hits} \leq 4) = P(X \leq 4)^2 = (0.5904)^2 = 0.3486
.\]

\subsection*{Part F}
Let $H_1$ and $H_2$ be the number of attempts it takes to hit the target a first time then the target a second. Then
\[
	\Exp{H_1 + H_2} = \Exp{H_1} + \Exp{H_2} = 2 \cdot \Exp{X} = 10
.\]

\section*{Problem 2}
\subsection*{Part A}
Since the distribution is a poisson distribution with parameter $\lambda = 10$,
\begin{align*}
	&\text{Expected number of cars in $1$ hour } = \Exp{X} = 10 \\
	&\text{Expected number of cars in $3$ hours } = \Exp{3X} = 3 \cdot\Exp{X} = 3 \cdot 10 = 30
.\end{align*}

\subsection*{Part B}
\[
	\Prob{X \leq 15} = e^{-10} \sum_{n=0}^{15} \frac{10^n}{n!} = \verb|ppois(15,10)| = 0.9513
.\]

\subsection*{Part C}
This problem corresponds to a poisson distribution with parameter $\lambda = 30$. Therefore
\[
	\Prob{X \leq 45} = e^{-30} \sum_{n=0}^{45} \frac{30^n}{n!} = \verb|ppois(45,30)| = 0.996
.\]

\subsection*{Part D}
\begin{align*}
	\Prob{\text{10 Arrive and all pass}} &= \Prob{\text{10 Arrive}}\cdot\Prob{\text{All pass} | \text{10 Arrive}} \\
																			 &= \Prob{X = 10}\cdot\Prob{10 \text{Successes}} \\
																			 &= \qty(\frac{10^{10} \cdot e^{-10}}{10!} \cdot \qty(\frac{1}{2})^{10}) \\
																			 &= \texttt{dpois(10,10)*dbinom(10,10,0.5)} \\
																			 &= 0.0001222
.\end{align*}

\section*{Problem 3}
\subsection*{Part A}
\[
	\renewcommand\arraystretch{1.9}
	\large
	\begin{array}{|c|c|c|c|c|c|c|} \hline
		m & 1 & 2 & 3 & 4 & 5 & 6 \\\hline
		\Prob{M = m} & \frac{1}{36} & \frac{3}{36} & \frac{3}{36} & \frac{7}{36} & \frac{9}{36} & \frac{11}{36}   \\\hline
	\end{array}
\]

\subsection*{Part B}
\[
	\Exp{M} = 1\cdot\frac{1}{36} + 2\cdot\frac{3}{36} + 3\cdot\frac{5}{36} + 4\cdot\frac{7}{36} + 5\cdot\frac{9}{36} + 6\cdot\frac{11}{36} = 4.47222
.\]

\subsection*{Part C}
\begin{align*}
	\Var{M} &= \Exp{M^2} - \qty(\Exp{M})^2 \\
	&= 1^2\cdot\frac{1}{36} + 2^2\cdot\frac{3}{36} + 3^2\cdot\frac{5}{36} + 4^2\cdot\frac{7}{36} + 5^2\cdot\frac{9}{36} + 6^2\cdot\frac{11}{36} - 4.47222 \\
	&= 1.9716
.\end{align*}

\subsection*{Part D}
These are the events in which the max is a $4$,
\begin{align*}
	&(1,4),
	(2,4),
	(3,4), \\
	&(4,1), 
	(4,2), 
	(4,3), \\
	&(4,4)
.\end{align*}
Of the events, there are $2$ that have a roll of $2$, therefore the probability is $\frac{2}{7}$

\end{document}
