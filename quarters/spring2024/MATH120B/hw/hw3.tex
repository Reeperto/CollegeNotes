\documentclass[12pt,titlepage]{extarticle}
% Document Layout and Font
\usepackage{subfiles}
\usepackage[margin=2cm, headheight=15pt]{geometry}
\usepackage{fancyhdr}
\usepackage{enumitem}	
\usepackage{wrapfig}
\usepackage{float}
\usepackage{multicol}

\usepackage[p,osf]{scholax}

\renewcommand*\contentsname{Table of Contents}
\renewcommand{\headrulewidth}{0pt}
\pagestyle{fancy}
\fancyhf{}
\fancyfoot[R]{$\thepage$}
\setlength{\parindent}{0cm}
\setlength{\headheight}{17pt}
\hfuzz=9pt

% Figures
\usepackage{svg}

% Utility Management
\usepackage{color}
\usepackage{colortbl}
\usepackage{xcolor}
\usepackage{xpatch}
\usepackage{xparse}

\definecolor{gBlue}{HTML}{7daea3}
\definecolor{gOrange}{HTML}{e78a4e}
\definecolor{gGreen}{HTML}{a9b665}
\definecolor{gPurple}{HTML}{d3869b}

\definecolor{links}{HTML}{1c73a5}
\definecolor{bar}{HTML}{584AA8}

% Math Packages
\usepackage{mathtools, amsmath, amsthm, thmtools, amssymb, physics}
\usepackage[scaled=1.075,ncf,vvarbb]{newtxmath}

\newcommand\B{\mathbb{C}}
\newcommand\C{\mathbb{C}}
\newcommand\R{\mathbb{R}}
\newcommand\Q{\mathbb{Q}}
\newcommand\N{\mathbb{N}}
\newcommand\Z{\mathbb{Z}}

\DeclareMathOperator{\lcm}{lcm}

% Probability Theory
\newcommand\Prob[1]{\mathbb{P}\qty(#1)}
\newcommand\Var[1]{\text{Var}\qty(#1)}
\newcommand\Exp[1]{\mathbb{E}\qty[#1]}

% Analysis
\newcommand\ball[1]{\B\qty(#1)}
\newcommand\conj[1]{\overline{#1}}
\DeclareMathOperator{\Arg}{Arg}
\DeclareMathOperator{\cis}{cis}

% Linear Algebra
\DeclareMathOperator{\dom}{dom}
\DeclareMathOperator{\range}{range}
\DeclareMathOperator{\spann}{span}
\DeclareMathOperator{\nullity}{nullity}

% TIKZ
\usepackage{tikz}
\usepackage{pgfplots}
\usetikzlibrary{arrows.meta}
\usetikzlibrary{math}
\usetikzlibrary{cd}

% Boxes and Theorems
\usepackage[most]{tcolorbox}
\tcbuselibrary{skins}
\tcbuselibrary{breakable}
\tcbuselibrary{theorems}

\newtheoremstyle{default}{0pt}{0pt}{}{}{\bfseries}{\normalfont.}{0.5em}{}
\theoremstyle{default}

\renewcommand*{\proofname}{\textit{\textbf{Proof.}}}
\renewcommand*{\qedsymbol}{$\blacksquare$}
\tcolorboxenvironment{proof}{
	breakable,
	coltitle = black,
	colback = white,
	frame hidden,
	boxrule = 0pt,
	boxsep = 0pt,
	borderline west={3pt}{0pt}{bar},
	% borderline west={3pt}{0pt}{gPurple},
	sharp corners = all,
	enhanced,
}

\newtheorem{theorem}{Theorem}[section]{\bfseries}{}
\tcolorboxenvironment{theorem}{
	breakable,
	enhanced,
	boxrule = 0pt,
	frame hidden,
	coltitle = black,
	colback = blue!7,
	% colback = gBlue!30,
	left = 0.5em,
	sharp corners = all,
}

\newtheorem{corollary}{Corollary}[section]{\bfseries}{}
\tcolorboxenvironment{corollary}{
	breakable,
	enhanced,
	boxrule = 0pt,
	frame hidden,
	coltitle = black,
	colback = white!0,
	left = 0.5em,
	sharp corners = all,
}

\newtheorem{lemma}{Lemma}[section]{\bfseries}{}
\tcolorboxenvironment{lemma}{
	breakable,
	enhanced,
	boxrule = 0pt,
	frame hidden,
	coltitle = black,
	colback = green!7,
	left = 0.5em,
	sharp corners = all,
}

\newtheorem{definition}{Definition}[section]{\bfseries}{}
\tcolorboxenvironment{definition}{
	breakable,
	coltitle = black,
	colback = white,
	frame hidden,
	boxsep = 0pt,
	boxrule = 0pt,
	borderline west = {3pt}{0pt}{orange},
	% borderline west = {3pt}{0pt}{gOrange},
	sharp corners = all,
	enhanced,
}

\newtheorem{example}{Example}[section]{\bfseries}{}
\tcolorboxenvironment{example}{
	% title = \textbf{Example},
	% detach title,
	% before upper = {\tcbtitle\quad},
	breakable,
	coltitle = black,
	colback = white,
	frame hidden,
	boxrule = 0pt,
	boxsep = 0pt,
	borderline west={3pt}{0pt}{green!70!black},
	% borderline west={3pt}{0pt}{gGreen},
	sharp corners = all,
	enhanced,
}

\newtheoremstyle{remark}{0pt}{4pt}{}{}{\bfseries\itshape}{\normalfont.}{0.5em}{}
\theoremstyle{remark}
\newtheorem*{remark}{Remark}


% TColorBoxes
\newtcolorbox{week}{
	colback = black,
	coltext = white,
	fontupper = {\large\bfseries},
	width = 1.2\paperwidth,
	size = fbox,
	halign upper = center,
	center
}

\newcommand{\banner}[2]{
    \pagebreak
    \begin{week}
   		\section*{#1}
    \end{week}
    \addcontentsline{toc}{section}{#1}
    \addtocounter{section}{1}
    \setcounter{subsection}{0}
}

% Hyperref
\usepackage{hyperref}
\hypersetup{
	colorlinks=true,
	linktoc=all,
	linkcolor=links,
	bookmarksopen=true
}

% Error Handling
\PackageWarningNoLine{ExtSizes}{It is better to use one of the extsizes 
                          classes,^^J if you can}


\def\homeworknumber{2}
\fancyhead[R]{\textbf{Math 140A: Homework \#\homeworknumber}}
\fancyhead[L]{Eli Griffiths}
\renewcommand{\headrulewidth}{1pt}
\setlength\parindent{0pt}


\begin{document}

% §20: 4,6,12,14,27,28,29
% §21: 2, 6–11

\subsection*{20.4}
Note that
\[
    3^{47} = 3^{2 \cdot 22 + 3} = \qty(3^{23 - 1})^2 \cdot 3^3
.\]
By Fermat's Little Theorem, $3^{23 - 1} \equiv 1 \pmod{23}$ and therefore $\qty(3^{23 - 1})^2 \equiv 1 \pmod{23}$. Since $3^3 = 27 \equiv 4 \pmod{23}$ it follows
\[
    3^{47} \equiv 1 \cdot 4 \equiv 4 \pmod{23}
.\]

\subsection*{20.6}
First note that
\[
    2^{17} \equiv \qty(2^4)^4 \cdot 2 \equiv (-2)^4 \cdot 2 \equiv 16 \cdot 2 \equiv 14 \pmod{18}
.\]
Therefore $2^17 = 18m + 14$ for some $m \in \Z$. Hence
\[
    2^{2^{17}} = 2^{18m + 14} = \qty(2^{18})^m \cdot 2^{14} = \qty(2^{19 - 1})^m \cdot 2^{14}
.\]
Since $19$ is prime, then
\[
    2^{18} \equiv 2^{19 - 1} \equiv 1 \pmod{19}
\]
meaning
\[
    2^{2^{17}} \equiv \qty(2^{19 - 1})^m \cdot 2^{14} \equiv 1^m \cdot 2^{14} \equiv 2^{14} \equiv \qty(2^7)^2 \equiv (-5)^2 \equiv 6 \pmod{19}
\]
which adding one gives the final result $7 \pmod {19}$.

\subsection*{20.12}
The congruence relation reduces to
\[
    7x \equiv 5 \pmod{15}
.\]
Since $\gcd(7,15) = 1$ which divides $5$, there exists solutions. Since $7 \cdot 5 = 5 \pmod{15}$ the solutions are
\[
    x = 5m + 15, m \in \Z
.\]

\subsection*{20.14}
The congruence relation reduces to
\[
    21x \equiv 15 \pmod{24}
.\]
Since $\gcd(21, 24) = 3$ which divides $15$, there exists solutions. Consider the congruence relation
\[
    7x \equiv 5 \pmod{8}
.\]
This has a solution $x = 3$ meaning the solutions to the original are the elements of $3 + 8 \Z$.


\subsection*{20.27}
\begin{proof}
    Let $a \in \Z_p$. Then $a^2 - 1 = (a-1)(a+1) = 0$. Since $\Z_p$ is a field, it has no zero divisors meaning $a-1$ or $a+1$ are zero and hence $a=1$ or $a = p - 1$.
\end{proof}

\subsection*{20.28}
\begin{proof}
    Note that
    \[
        (p-1)! = (p-1)(p-2)(p-3) \cdots (3)(2)(1)
    .\]
    For $p \geq 3$, the elements exclusively between $p-1$ and $1$ will have their multiplicative inverse in this factorial expansion meaning
    \[
        (p-1)! = (p-1)(1)\cdots(1)(1) = p - 1 \equiv -1 \pmod{p}
    .\]
    In the case that $p = 2$, $(p-1)! = (2 - 1)! = 1 \equiv -1 \pmod{2}$ and for $p = 1$, $(p-1)! = 0! = 1 \equiv -1 \pmod{1}$.
\end{proof}

\subsection*{20.29}
Consider each prime factor individually. Note that only the cases where $n$ isnt divisible by a prime factor need to be considerd since otherwise if $n$ is divisible by all prime factors, $n^{37} - n = n(n^{36} - 1)$ is as well.

\begin{enumerate}
    \item[37)]
        Since $n^{37} \equiv n \pmod{37}$ it follows $n^{37} - n = 0 \equiv \pmod{37}$ so $37$ dividies
    \item[19)]
        Assume that $19$ doesn't divide $n$. Then $n^{36} - 1 \equiv \qty(n^{18})^2 - 1 \equiv 1^2 - 1 \equiv 0 \pmod{19}$ therefore $19$ divides
    \item[13]
        Assume $13$ doesnt divide $n$. Then $n^{36} - 1 \equiv \qty(n^{12})^3 - 1 \equiv 1^3 - 1 \equiv 0 \pmod{13}$ therefore 13 divides
    \item[7)]
        Assume $7$ doesnt divide $n$. Then $n^{36} - 1 \equiv \qty(n^{6})^6 - 1 \equiv 1^6 - 1 \equiv 0 \pmod{7}$ therefore 7 divides
    \item[3)]
        Assume $3$ doesnt divide $n$. Then $n^{36} - 1 \equiv \qty(n^{2})^{18} - 1 \equiv 1^{18} - 1 \equiv 0 \pmod{3}$ therefore 3 divides
    \item[2)]
        Assume $2$ doesnt divide $n$. Then $n^{36} - 1 \equiv \qty(n^1)^{36} - 1 \equiv 1^{36} - 1 \equiv 0 \pmod{2}$ therefore 2 divides
\end{enumerate}

\subsection*{21.2}
The field of quotients for $D$ are $\qty{q + p\sqrt{2} : p,q \in \Q}$ since the multiplicative inverse of an element in $D$ would look like
\[
    \frac{1}{a + b \sqrt{2}} = \frac{a}{a^2 - 2b^2} + \frac{-b}{a^2 - 2b^2} \sqrt{2}
\]
of which $\frac{a}{a^2 - 2b^2}$ and $\frac{-b}{a^2 - 2b^2}$ are rational numbers.

\subsection*{21.6}
\begin{proof}
    Let $[(a_1,b_1)], [(a_2,b_2)]$ and $[(a_3, b_3)]$ be elements of $F$. Then
    \begin{align*}
        \qty\Big([(a_1, b_1)] + [(a_2, b_2)]) + [(a_3,b_3)] &= [(a_1 b_2 + a_2 b_1, b_1 b_2)] + [(a_3,b_3)] \\
        &= [(
            a_1 b_2 b_3 + a_2 b_1 b_3 + a_3 b_1 b_2, b_1 b_2 b_3
        )]
    \end{align*}
    and
    \begin{align*}
        [(a_1, b_1)] + \qty\Big([(a_2, b_2)] + [(a_3,b_3)]) &= [(a_1,b_1)] + [(a_2 b_3 + a_3 b_2, b_2 b_3)] \\
        &= [(
            a_1 b_1 b_2 + a_2 b_1 b_3 + a_3 b_1 b_2, b_3 b_2 b_1
        )].
    \end{align*}
    Since addition and multiplication for $D$ is associative and abelian, these can be rearranged to equal each other and hence addition on $F$ is associative.
\end{proof}

\subsection*{21.7}
\begin{proof}
    Let $[(a,b)] \in F$. Then
    \[
        [(0,1)] + [(a,b)] = [(0b + 1a, 1b)] = [(a,b)]
    .\]
    Since addition on $F$ is commutative, it follows $[(0,1)]$ is an additive identity in $F$.
\end{proof}

\subsection*{21.8}
\begin{proof}
    Let $[(a,b)] \in F$. Note that
    \[
        [(a,b)] + [(-a, b)] = [(ab + b(-a), b^2)] = [(ab - ab, b^2)] = [(0, b^2)] = [(0, 1)]
    .\]
    Since addition is commutative, it follows $[(-a,b)]$ is the additive inverse for any element in $F$.
\end{proof}

\subsection*{21.9}
\begin{proof}
    Let $[(a_1,b_1)], [(a_2,b_2)]$ and $[(a_3, b_3)]$ be elements of $F$. Then
    \[
        \qty\Big([(a_1,b_1)] [(a_2,b_2)]) [(a_3,b_3)] = [(a_1 a_2, b_1 b_2)] [(a_3, b_3)] = [(a_1 a_2 a_3, b_1 b_2 b_3)]
    \]
    and
    \[
        [(a_1,b_1)] \qty\Big([(a_2,b_2)] [(a_3,b_3)]) = [(a_1, b_1)] [(a_2 a_3, b_2 b_3)] = [(a_1 a_2 a_3, b_1 b_2 b_3)]
    \]
    which are equal. Therefore multiplication on $F$ is associative.
\end{proof}

\subsection*{21.10}
\begin{proof}
    Let $[(a_1,b_1)], [(a_2, b_2)] \in F$. Then
    \[
        [(a_1, b_1)][(a_2, b_2)] = [(a_1 a_2, b_1 b_2)] = [(a_2 a_1, b_2 b_1)] = [(a_2, b_2)] [(a_1, b_1)]
    \]
    since multiplication on $D$ is commutative. Therefore multiplication on $F$ is commutative.
\end{proof}

\subsection*{21.11}
\begin{proof}
    Let $[(a_1,b_1)], [(a_2,b_2)]$ and $[(a_3, b_3)]$ be elements of $F$. Then
    \begin{align*}
        [(a_1, b_1)] \qty\Big([(a_2, b_2)] + [(a_3, b_3)]) &= [(a_1, b_1)] [(a_2 b_3 + a_3 b_2, b_2 b_3)] \\ 
               &= [(a_1 a_2 b_3 + a_1 b_3 b_2, b_1 b_2 b_3)]
    \end{align*}
    and
    \begin{align*}
        [(a_1, b_1)] [(a_2, b_2)] + [(a_1, b_1)] [(a_3, b_3)] &= [(a_1 a_2, b_1 b_2)] + [(a_1 a_3, b_1 b_3)] \\
                  &= [(
                  a_1 a_2 b_1 b_3 + a_1 a_3 b_1 b_2, b_1^2 b_2 b_3
                  )]
    \end{align*}
    which are equal since by the definition of the equivalence for $F$
    \[
        [(
            a_1 a_2 b_1 b_3 + a_1 a_3 b_1 b_2, b_1^2 b_2 b_3
        )] = 
        [(
            a_1 a_2 b_3 + a_1 a_3 b_2, b_1 b_2 b_3
        )]
    .\]
    Since multiplication is commutative on $F$, the right distributive law also holds. Hence both laws hold on $F$.
\end{proof}

\end{document}
