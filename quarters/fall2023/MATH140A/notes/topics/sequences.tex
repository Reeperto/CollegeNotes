\documentclass[../notes.tex]{subfiles}
\graphicspath{
    {'../figures'}
}

\begin{document}
\banner{Sequences}

\subsection{Limits of Sequences}
\begin{definition}[Sequence]
	A sequence is a mapping $s : \mathbb{N}_{\geq m} \to \mathbb{R}$ where $m$ is typically $0$ or $1$. Alternatively, a sequence cant be thought of as an infinite tuple
	\[
		s = (s_m, s_{m+1}, s_{m+2}, \ldots)
	\]
	Define the image of a sequence as $S(\mathbb{N}_{\geq m}) := \qty{s_n : n \geq m}$
\end{definition}

\begin{example}
	Consider $(s_n)_{n \in \mathbb{N}}$ given by $s_n = \frac{(-1)^n}{n^3}$. It is a sequence with $m = 1$ and looks like $(-1, \frac{1}{8}, -\frac{1}{27}, \ldots)$.
\end{example}

\begin{example}
	Consider $(s_n)_{n \in \mathbb{N}_0} = (-1)^n$ which is the sequence $(1, -1, 1, -1, 1, \ldots)$. Note that the image $S(\mathbb{N}_0) = \qty{-1, 1}$
\end{example}

\begin{example}
	Consider $(s_n)_{n \in \mathbb{N}_0} = \qty(1+\frac{1}{n})^n$ which gives a sequence of real numbers that 'appears' to get closer $e$ as $n$ grows large, as seen by the fact that $s_{1,000,000} = 2.718280469319377$.
\end{example}

\subsubsection{Convergence of a Sequence}

\begin{definition}[Sequence Convergence]
	A sequence $(s_n)_{n\in \mathbb{N}_0}$ is said to converge to $s_0 \in \mathbb{R}$ if 
	\[
		\forall \epsilon > 0, \exists N \in \mathbb{N} \text{ s.t. } \qty|s_n - s| < \epsilon, \forall n > N
	\]
	Pictorally, we are creating a neighborhood of $2 \epsilon$ around $s_0$. And if the sequence converges, there is an eventual $s_N$ such that every subsequent number is within the neighbor hood around $s_0$.
	\begin{center}
		\begin{tikzpicture}[scale=2]
			\draw (-1,0) -- (1,0);
		\end{tikzpicture}
	\end{center}
\end{definition}

\begin{example}
	Consider $s_n = \frac{1}{n}$. Take $\epsilon > 0$. Note that $|s_n - 0| = \frac{1}{n} < \epsilon$ by the archimedean property. It is clearer if it is rewritten as $1 < n \epsilon$.
\end{example}

\begin{example}
	Consider $s_n = (-1)^n, n \in \mathbb{N}$. Take $e > 0$. % TODO
\end{example}

\begin{example}
	Consider $s_n = \frac{3n +1}{7n-4}, n \in \mathbb{N}$. A good guess for the limit is $\frac{3}{7}$ since the $3n$ and $7n$ terms 'dominate' as $n \to \infty$. Take $\epsilon > 0$. Then
	\begin{align*}
		\qty|\frac{3n+1}{7n-1} - \frac{3}{7}| &= \qty| \frac{21n + 7 - 21n + 12}{7(7n-4)} | \\
											  &= \qty| \frac{19}{7} \cdot \frac{1}{7n-4} | \\
											  &= \frac{19}{7} \cdot \frac{1}{7n-4} \\
											  &\leq \frac{19}{49} \cdot \frac{1}{n - 1}
	\end{align*}
	Since $\frac{1}{n-1} \to 0$ as $n \to \infty$. % TODO
\end{example}

\begin{example}
	Consider $s_n = \sqrt[n]{n}, n \in \mathbb{N}$. Take $s_0 = 1$, and prove this much later.
\end{example}

\begin{theorem}[Uniqueness of Limits]
	If a limit of a sequence exists, then it is unique.
\end{theorem}

\begin{proof}
	Let $s_n$ be a sequence that converges to $s$ and $s'$ as $n \to \infty$. Then
	\begin{align*}
		\forall \epsilon > 0, \exists N \in \mathbb{N} \text{ s.t. } &\qty|s_n - s| < \frac{\epsilon}{2}, \forall n > N \\
		\forall \epsilon > 0, \exists N \in \mathbb{N} \text{ s.t. } &\qty|s_n - s'| < \frac{\epsilon}{2}, \forall n > N \\
	\end{align*}
	Then
	\begin{align*}
		\qty|s - s'| &= \qty|s - s_n + s_n - s'| \\
					 &\leq \qty|s_n - s| + \qty|s_n - s'| < \epsilon
	\end{align*}
	Therefore $0 \geq \qty|s - s'| < \epsilon$ for all $e > 0$, meaning $s = s'$.
\end{proof}

\begin{example}
	Consider $\lim_{n\to \infty} \frac{1}{n^2}$. Let $s_0 = 0$.
\end{example}

\begin{example}
	Consider $\lim_{n \to \infty} \frac{4n^3 + 3n}{n^3 - 6} \overset{?}{=} 4$. Take $\epsilon > 0$. Then
	\begin{align*}
		\qty|\frac{4n^3 + 3n}{n^3 - 6} - 4| &= \qty|\frac{4n^3 + 3n - 4n^3 + 24}{n^3 - 6}| \\
											&= \frac{3n + 24}{\qty|n^3 - 6|} \\
											\intertext{Note that $3n + 24 \leq 27n$ for all $n \in \mathbb{N}$ and $n^3 - 6 \geq \frac{n^3}{4}$ for $n \geq 2$.}
											&\leq 4 \cdot \frac{27n}{n^3} \\
											&= \frac{108}{n^2} < \epsilon
	\end{align*}
	Take $N \in \mathbb{N} \geq \sqrt{\frac{108}{\epsilon}}$. Then
	\[
		\qty|\frac{4n^3 + 3n}{n^3 - 6} - 4| \leq 108
	\]
\end{example}

\begin{theorem}
	Let $(s_n)_{n\in \mathbb{N}}$ be a sequence $s_n \geq 0$ for all $n$ and $s = \lim_{n\to \infty} s_n$. Then $\lim_{n\to \infty} \sqrt{s_n} = \sqrt{s}$
\end{theorem}

\begin{proof}
	% Prove that $s$ is non negative
	Consider $|\sqrt{s_n} - \sqrt{s}|$.
	\begin{align*}
		|\sqrt{s_n} - \sqrt{s}| &= \qty|\frac{(\sqrt{s_n} - \sqrt{s})(\sqrt{s_n} + \sqrt{s})}{\sqrt{s_n} + \sqrt{s}}| \\
								&= \frac{\qty|s_n - s|}{\sqrt{s_n} - \sqrt{s}}
	\end{align*}
	If $s > 0$, then $\frac{1}{\sqrt{s_n} + \sqrt{s}} \leq \frac{1}{\sqrt{s}}$ meaning we would want $\frac{|s_n - s|}{\sqrt{s}} < \epsilon$ or equivalentely $|s_n - s| < \epsilon\sqrt{s}$. If $s = 0$, we want $\sqrt{s_n} < \epsilon$ or $s_n < \epsilon^2$. Now, formally:


\end{proof}

\begin{theorem}
	Let $(s_n)_{n\in \mathbb{N}}$ be convergent to $s \neq 0$ with $\forall n \in \mathbb{N}, s_n \neq 0$. Then $\inf\qty{|s_n| : n \in \mathbb{N}} > 0$.
\end{theorem}

\begin{proof}
	The idea is that given a neighborhood around $s$, there is a finite amount of values of the sequence outside of it. By choosing a neighborhood size of $\frac{|s|}{2}$, 0 is avoided. Therefore proceed with the formal proof by letting $\epsilon = \frac{|s|}{2}$. Since $s_n$ converges and $\epsilon > 0$, there is an $N \in \mathbb{N}$ such that $|s_n - s| < \frac{|s|}{2}$ for all $n > N$. Note that
	\[
		||s_n| - |s|| \leq |s_n - s| < \frac{|s|}{2}, \forall n > N
	\]
	and that
	\[
		|s_n| \in \qty(s - \epsilon, s + \epsilon), \forall n > N
	\]
\end{proof}

\begin{definition}[Bounded Series]
	A series $(s_n)_{n\in \mathbb{N}}$ is bounded if the image is bounded. Equivalently, it is bounded if $\exists M \in \mathbb{R}$ such that $s_n \leq M$ for all $n \in \mathbb{N}$.
\end{definition}

\begin{theorem}[Convergence Implies Boundedness]
	Let $(s_n)_{n\in \mathbb{N}}$ be a series that converges to $s$. Then the series is bounded.
\end{theorem}

\begin{proof}
	Let $(s_n)_{n\in \mathbb{N}}$ be a series and assume it converges to $s$. Take $\epsilon = 1$ and find $N \in \mathbb{N}$ such that $|s_n - s| < 1$ for all $n > N$. Therefore $s_n$ and $s$ are at most $1$ apart, therefore $|s_n| \leq |s| + 1$. This provides an upperbound on $s_n$ for $n > N$. For $n \leq N$, construct the set $M = \qty{s_1, s_2, \ldots, s_N, |s| + 1}$. Then note that
	\[
		s_n \leq \max M < \infty, \forall n \in \mathbb{N} 
	\]
	Therefore the series is bounded.
\end{proof}

\begin{theorem}[Properties of Limits]
	\label{thm:propsoflimits}
	The following properties hold for all limits of sequences.
	\begin{enumerate}[label=\alph*)]
		\item $\lim_{n\to\infty} s_n = s, c \in \mathbb{R} \implies \lim_{n\to\infty} c\cdot s_n = c\lim_{n\to\infty} s_n$
		\item $\lim_{n\to\infty} s_n = s, \lim_{n\to\infty} t_n = t \implies \lim_{n\to\infty} (s_n + t_n) = s + t$
		\item $\lim_{n\to\infty} s_n = s, \lim_{n\to\infty} t_n = t \implies \lim_{n\to\infty} (s_n \cdot t_n) = st$
		\item $\lim_{n\to\infty} s_n = s, s_n \neq 0 \forall n \in \mathbb{N}, s \neq 0 \implies \lim_{n\to\infty} \frac{1}{s_n} = \frac{1}{s}$
	\end{enumerate}
\end{theorem}

\begin{proof}
	Let $s_n$ and $t_n$ be sequences that converge to $s$ and $t$ respectively.
	\begin{enumerate}[label=\alph*)]
		\item TODO
		\item 
            Since both sequences are converget, they both admit $N_1, N_2 \in \mathbb{N}$ such that for an $\epsilon > 0$
            \begin{align*}
            |s_n - s| &< \frac{\epsilon}{2}, \forall n > N_1 \\
            |t_n - t| &< \frac{\epsilon}{2}, \forall n > N_2
            \end{align*}
            Note then that 
            \[
            |(s_n - s) + (t_n - t)| \leq |s_n - s| + |t_n - t| < \frac{\epsilon}{2} + \frac{\epsilon}{2} = \epsilon, \forall n > \max\qty{N_1, N_2}
            \]
        \item
        \item
            Note that
            \[
                |s_n t_n - st| = |s_n(t_n - t) + (s_n - s)t| \leq |s_n| |t_n - t| + |t| |s_n - s|
            \]
            Since $s_n$ and $t_n$ converge, they are bounded. Therefore, take $\epsilon > 0$ and note
            \begin{align*}
                &\exists M > 0, |s_n| \leq M, |t_n| \leq M, \forall n \in \mathbb{N} \\
                &\exists N_1, |s_n - s| < \frac{\epsilon}{2M}, \forall n > N_1 \\
                &\exists N_2, |t_n - t| < \frac{\epsilon}{2M}, \forall n > N_2 \\
            \end{align*}
            Therefore 
            \[
                |s_n t_n - st| \leq M \cdot \frac{\epsilon}{2M} + M \cdot \frac{\epsilon}{2M} = \epsilon, \forall n > \max\qty{N_1, N_2}
            \]
        \item
            Consider the target expression $\qty|\frac{1}{S_n} - \frac{1}{s}|$. This can be reformed into $\frac{1}{s_n \cdot s} \qty|s_n - s|$. Since $s_n \neq 0$ and $s \neq 0$, $|s_n|$ is bounded below by a positive number $m$ for all $n$. This also means that $s \geq m$. Therefore
            \[
                \qty|\frac{1}{s_n} - \frac{1}{s}| \leq \frac{1}{m^2} |s_n - s|
            \]
            Formally, take $\epsilon > 0$. Since $s_n$ converges, $\exists N \in \mathbb{N}$ such that
            \[
                |s_n - s| < \epsilon m^2, \forall n > N
            \]
            By the previous derivation,
            \[
                \qty|\frac{1}{s_n} - \frac{1}{s}| \leq \frac{m^2 \epsilon}{m^2} = \epsilon
            \]
            Hence $\frac{1}{s_n}$ converges to $\frac{1}{s}$.
	\end{enumerate}
\end{proof}

\begin{example}[Example Limits]
    The following are basic example limits.
    \begin{enumerate}
        \item $\lim_{n\to\infty} \frac{1}{n^p} = 0, \forall p > 0$
        \item $\lim_{n\to\infty} r^n = 0, |r| < 1$
        \item $\lim_{n\to\infty} \sqrt[n]{n} = 1$
        \item $\lim_{n\to\infty} \sqrt[n]{r} = 1, r > 0$
    \end{enumerate}
\end{example}

\begin{proof}
    \begin{enumerate}
        \item %----------------------------------------------------------------
            Take $\epsilon > 0$ and $N > \sqrt[p]{\frac{1}{\epsilon}}$. Then $\frac{1}{n^p} < \epsilon$ for all $n > N$.
        \item %----------------------------------------------------------------
            If $r = 0$, then clearly $r^n$ is $0$ for all $n$. Consider then $r \neq 0$. If $|r| < 1$, then $\exists S$ such that $|r| = \frac{1}{1+S}$. Therefore $|r^n| = \frac{1}{(1+S)^n} \leq \frac{1}{1+Sn}$. Take $\epsilon > 0$ and $N > \frac{1}{S \epsilon}$. Then for all $n > N$, $|r^n| < \epsilon$.
        \item %----------------------------------------------------------------
            Let $s_n = \sqrt[n]{n} - 1 \geq 0$. Note that
            \[
                n = (1+ s_n)^n \geq \underbrace{1 + n s_n + \frac{1}{2} n (n-1) s_n^2}_{\text{truncated binomial theorem}} > \frac{1}{2} n(n-1) s_n^2
            \]
            Therefore $0 \leq s_n < \sqrt{\frac{2}{n-1}}$. Since $\lim_{n\to\infty} \sqrt{\frac{2}{n-1}} = 0$, by the squeeze theorem $\lim_{n\to\infty} s_n = 0$, meaning $\lim_{n\to\infty} \sqrt[n]{n} = 1$.
        \item %----------------------------------------------------------------
            Consider $r \geq 1$. Then there is always an $n \geq r$, meaning $1 \leq r \leq n$. Therefore $1 \leq r^{\frac{1}{n}} \leq n^{\frac{1}{n}} = 1$, hence $\lim_{n\to\infty} \sqrt[n]{r} = 1$. If $0 < r < 1$, then $\frac{1}{r} > 1$ meaning $\qty(\frac{1}{r})^n > 1$. % TODO
    \end{enumerate}
\end{proof}

\begin{example}
    Consider $\lim_{n\to\infty} \frac{n^3 + 6n^2 +7}{4n^3 + 3n - 4}$. This can be rewritten as $\frac{1 + \frac{6}{n} + \frac{7}{n^3}}{4 + \frac{3}{n^2} - \frac{4}{n^3}}$. Then
    \[
        \lim_{n\to\infty} \frac{n^3 + 6n^2 +7}{4n^3 + 3n - 4} = \lim_{n\to\infty} \frac{1 + \frac{6}{n} + \frac{7}{n^3}}{4 + \frac{3}{n^2} - \frac{4}{n^3}} = \frac{1 + 0 + 0}{4 + 0 + 0} = \frac{1}{4}
    \]
\end{example}
\begin{example}
    Consider $\lim_{n\to\infty} \frac{n-5}{n^2 + 7}$. This can be rewritten as $\frac{\frac{1}{n} - \frac{5}{n^2}}{1 + \frac{7}{n^2}}$. Then
    \[
        \lim_{n\to\infty} \frac{n-5}{n^2 + 7} = \lim_{n\to\infty} \frac{\frac{1}{n} - \frac{5}{n^2}}{1 + \frac{7}{n^2}} = \frac{0}{1} = 0
    \]
\end{example}

\subsubsection{Unbounded Limits}
\begin{definition}[Infinite Limit]
    $\lim_{n\to\infty} s_n = \infty$ if $\forall M > 0, \exists N$ s.t. $s_n > M, \forall n > N$. Likewise, $\lim_{n\to\infty} s_n = -\infty$ if $\forall M \leq 0, \exists N$ s.t. $s_n < M, \forall n > N$.
\end{definition}

\begin{theorem}[Implication of Infinite Limits]
    Let $s_n$ and $t_n$ be sequences.
    \begin{enumerate}
        \item If $\lim_{n\to\infty} s_n = \infty$ and $\lim_{n\to\infty} t_n > 0$, then $\lim_{n\to\infty} s_n t_n = \infty$.
        \item $\lim_{n\to\infty} s_n = \infty \Leftrightarrow \lim_{n\to\infty} \frac{1}{s_n} = 0$
    \end{enumerate}
\end{theorem}
\begin{proof}
    % TODO
\end{proof}

\subsubsection{}
\begin{definition}[hi]
    Let $(s_n)_{n \in \mathbb{N}}$ be a real sequence. The statement
    \[
        \sup_{n \geq N} s_n
    \]
    Is the supremum of the tail of the sequence (since it only acts on terms greater than $N$). In the limiting case where $N \to \infty$, this can be written as
    \[
        \lim_{N \to \infty} \sup_{n \geq N} s_n = \limsup_{n\to\infty} s_n
    \]
\end{definition}
\begin{remark}
    If $(s_n)$ is not bounded above $\limsup_{n \to \infty} s_n = \infty$ and if it not bounded below then $\liminf_{n \to \infty} s_n = -\infty$
\end{remark}

\end{document}
