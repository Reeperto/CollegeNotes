\documentclass[12pt]{extarticle}

\usepackage{multicol}

% Document Layout and Font
\usepackage{subfiles}
\usepackage[margin=2cm, headheight=15pt]{geometry}
\usepackage{fancyhdr}
\usepackage{enumitem}	
\usepackage{wrapfig}
\usepackage{float}
\usepackage{multicol}

\usepackage[p,osf]{scholax}

\renewcommand*\contentsname{Table of Contents}
\renewcommand{\headrulewidth}{0pt}
\pagestyle{fancy}
\fancyhf{}
\fancyfoot[R]{$\thepage$}
\setlength{\parindent}{0cm}
\setlength{\headheight}{17pt}
\hfuzz=9pt

% Figures
\usepackage{svg}

% Utility Management
\usepackage{color}
\usepackage{colortbl}
\usepackage{xcolor}
\usepackage{xpatch}
\usepackage{xparse}

\definecolor{gBlue}{HTML}{7daea3}
\definecolor{gOrange}{HTML}{e78a4e}
\definecolor{gGreen}{HTML}{a9b665}
\definecolor{gPurple}{HTML}{d3869b}

\definecolor{links}{HTML}{1c73a5}
\definecolor{bar}{HTML}{584AA8}

% Math Packages
\usepackage{mathtools, amsmath, amsthm, thmtools, amssymb, physics}
\usepackage[scaled=1.075,ncf,vvarbb]{newtxmath}

\newcommand\B{\mathbb{C}}
\newcommand\C{\mathbb{C}}
\newcommand\R{\mathbb{R}}
\newcommand\Q{\mathbb{Q}}
\newcommand\N{\mathbb{N}}
\newcommand\Z{\mathbb{Z}}

\DeclareMathOperator{\lcm}{lcm}

% Probability Theory
\newcommand\Prob[1]{\mathbb{P}\qty(#1)}
\newcommand\Var[1]{\text{Var}\qty(#1)}
\newcommand\Exp[1]{\mathbb{E}\qty[#1]}

% Analysis
\newcommand\ball[1]{\B\qty(#1)}
\newcommand\conj[1]{\overline{#1}}
\DeclareMathOperator{\Arg}{Arg}
\DeclareMathOperator{\cis}{cis}

% Linear Algebra
\DeclareMathOperator{\dom}{dom}
\DeclareMathOperator{\range}{range}
\DeclareMathOperator{\spann}{span}
\DeclareMathOperator{\nullity}{nullity}

% TIKZ
\usepackage{tikz}
\usepackage{pgfplots}
\usetikzlibrary{arrows.meta}
\usetikzlibrary{math}
\usetikzlibrary{cd}

% Boxes and Theorems
\usepackage[most]{tcolorbox}
\tcbuselibrary{skins}
\tcbuselibrary{breakable}
\tcbuselibrary{theorems}

\newtheoremstyle{default}{0pt}{0pt}{}{}{\bfseries}{\normalfont.}{0.5em}{}
\theoremstyle{default}

\renewcommand*{\proofname}{\textit{\textbf{Proof.}}}
\renewcommand*{\qedsymbol}{$\blacksquare$}
\tcolorboxenvironment{proof}{
	breakable,
	coltitle = black,
	colback = white,
	frame hidden,
	boxrule = 0pt,
	boxsep = 0pt,
	borderline west={3pt}{0pt}{bar},
	% borderline west={3pt}{0pt}{gPurple},
	sharp corners = all,
	enhanced,
}

\newtheorem{theorem}{Theorem}[section]{\bfseries}{}
\tcolorboxenvironment{theorem}{
	breakable,
	enhanced,
	boxrule = 0pt,
	frame hidden,
	coltitle = black,
	colback = blue!7,
	% colback = gBlue!30,
	left = 0.5em,
	sharp corners = all,
}

\newtheorem{corollary}{Corollary}[section]{\bfseries}{}
\tcolorboxenvironment{corollary}{
	breakable,
	enhanced,
	boxrule = 0pt,
	frame hidden,
	coltitle = black,
	colback = white!0,
	left = 0.5em,
	sharp corners = all,
}

\newtheorem{lemma}{Lemma}[section]{\bfseries}{}
\tcolorboxenvironment{lemma}{
	breakable,
	enhanced,
	boxrule = 0pt,
	frame hidden,
	coltitle = black,
	colback = green!7,
	left = 0.5em,
	sharp corners = all,
}

\newtheorem{definition}{Definition}[section]{\bfseries}{}
\tcolorboxenvironment{definition}{
	breakable,
	coltitle = black,
	colback = white,
	frame hidden,
	boxsep = 0pt,
	boxrule = 0pt,
	borderline west = {3pt}{0pt}{orange},
	% borderline west = {3pt}{0pt}{gOrange},
	sharp corners = all,
	enhanced,
}

\newtheorem{example}{Example}[section]{\bfseries}{}
\tcolorboxenvironment{example}{
	% title = \textbf{Example},
	% detach title,
	% before upper = {\tcbtitle\quad},
	breakable,
	coltitle = black,
	colback = white,
	frame hidden,
	boxrule = 0pt,
	boxsep = 0pt,
	borderline west={3pt}{0pt}{green!70!black},
	% borderline west={3pt}{0pt}{gGreen},
	sharp corners = all,
	enhanced,
}

\newtheoremstyle{remark}{0pt}{4pt}{}{}{\bfseries\itshape}{\normalfont.}{0.5em}{}
\theoremstyle{remark}
\newtheorem*{remark}{Remark}


% TColorBoxes
\newtcolorbox{week}{
	colback = black,
	coltext = white,
	fontupper = {\large\bfseries},
	width = 1.2\paperwidth,
	size = fbox,
	halign upper = center,
	center
}

\newcommand{\banner}[2]{
    \pagebreak
    \begin{week}
   		\section*{#1}
    \end{week}
    \addcontentsline{toc}{section}{#1}
    \addtocounter{section}{1}
    \setcounter{subsection}{0}
}

% Hyperref
\usepackage{hyperref}
\hypersetup{
	colorlinks=true,
	linktoc=all,
	linkcolor=links,
	bookmarksopen=true
}

% Error Handling
\PackageWarningNoLine{ExtSizes}{It is better to use one of the extsizes 
                          classes,^^J if you can}


\fancyhead[R]{Homework \#$3$}
\fancyhead[L]{Eli Griffiths}
\renewcommand{\headrulewidth}{1pt}
\setlength\parindent{0pt}

% Section 4:
% 17, 18, 19, 20, 21

\begin{document}

\section*{4.17}
The set of $n\times n$ upper-triangular matrices with determinant $1$ under matrix multiplication is a group.

\begin{proof}
	Let $M_n$ denote the set of $n \times n$ upper triangular matrices with determinant 1. First note that the multiplication of two upper triangular matrices also results in an upper triangular matrix. Let $A, B \in M_n$ and $C = AB$. An entry $C_{ij}$ from $C$ with $i > j$ is given by
	\[
		C_{ij} = \sum_{k=1}^n a_{ik} b_{jk}
	.\]
	The sum can be split into two parts, resulting in
	\begin{align*}
		C_{ij} &= \sum_{k=1}^n a_{ik} b_{jk} \\
					 &= \sum_{k=1}^{i-1} a_{ik} b_{jk} + \sum_{k=i}^{n} a_{ik} b_{jk} \\
					 &= 0 + 0 = 0
	.\end{align*}
	Therefore the entries below the diagonal of $C$ are $0$, meaning $C$ is also upper-triangular. Additionally, $\det(C) = \det(AB) = \det(A)\det(B) = 1$. Therefore $M_n$ is closed under matrix multiplication. Consider now the three group axioms.
	\begin{enumerate}
		\item[$\mathcal{G}_1.)$]
		Associativity is satisfied since matrix multiplication is associative.
		\item[$\mathcal{G}_2.)$]
		The identity matrix $I_n$ is an upper-triangular matrix with $\det(I_n) = 1$, therefore $M_n$ has an identity element.
		\item[$\mathcal{G}_3.)$]
			Let $A \in M_n$. Note that the inverse of $A$ can be found by row reducing the augmented matrix $\qty[A | I]$ to $\qty[I | A^{-1} ]$. This will look like
			\[
				\left( 
				\begin{array}{cccc | cccc}
					a_{11} & a_{12} & \cdots & a_{1n} & 1 & 0 & \cdots & 0 \\
					0 & a_{22} & \cdots & a_{2n} & 0 & 1 & \cdots & 0 \\
					0 & 0 & \ddots & \vdots & 0 & 0 & \ddots & 0 \\
					0 & 0 & 0 & a_{nn} & 0 & 0 & 0 & 1 \\
				\end{array}
				\right)
			.\]
			Since $A$ is in upper triangular form, its augmented form can be row-reduced using back substitution which will maintain the upper triangular form on the right side. Therefore once the matrix is in the form $\qty[I | A^{-1}]$, the inverse matrix will be upper-triangular as well. Additionally, $\det(A^{-1}) = \frac{1}{\det(A)} = 1$. Therefore $A^{-1} \in M_n$, meaning every element in $M_n$ has an inverse.
	\end{enumerate}
	Since $M_n$ under matrix multiplication satisfies the group axioms, it is a group.
\end{proof}

\section*{4.18}
All $n \times n$ matrices with determinant either $1$ or $-1$ under matrix multiplication forms a group
\begin{proof}
	Let $M_n$ denote all $n \times n$ matrices with determinant $1$ or $-1$. Let $A,B \in M_n$. Their product is an $n \times n$ matrix since both are $n \times n$. Additionally $\det(AB) = \det(A) \det(B)$. Therefore the determinant of their product is also $\pm 1$, hence $M_n$ is closed under matrix multiplication. Consider now the three group axioms.
	\begin{enumerate}
		\item[$\mathcal{G}_1.)$]
		Associativity is satisfied since matrix multiplication is associative.
		\item[$\mathcal{G}_2.)$]
		The identity matrix $I_n$ is an $n \times n$ matrix and has $\det(I_n) = 1$ meaning $I_n \in M_n$, hence $M_n$ has an identity element.
		\item[$\mathcal{G}_3.)$]
			Let $A \in M_n$. Since $\det(A) \neq 0$ and $\det(A^{-1}) = \frac{1}{\det(A)}$ which is either $1$ or $-1$, $A$ has an inverse $A^{-1}$ such that $A A^{-1} = A^{-1} A = I_n$ with $A^{-1} \in M_n$. Therefore $M_n$ has an inverse for each element.
	\end{enumerate}

	Since $M_n$ under matrix multiplication satisfies the group axioms, it is a group.
\end{proof}

\section*{4.19}
\subsection{Part A}
$*$ is a binary operation on $S$.

\begin{proof}
	Let $S$ be the set $\mathbb{R} \setminus \qty{-1}$ and define the mapping $* : S\times S \to S$ where $a * b = a + b + ab$. Examine if $*$ is a well defined map. Since the addition and multiplication of real numbers is well defined, $*$ can only ever be not well-defined if there exists $a,b \in S$ such that $a * b = -1$. Assume towards contradiction that these $a$ and $b$ exist. Then
\begin{align*}
	a + b + ab &= -1 \\
	a + ab + b + 1 &= 0 \\
	(a+1)(b+1) &= 0
.\end{align*}
However, this implies that one of $a$ or $b$ is $-1$, contradicting the assumption that $a,b \in S$ since elements in $S$ cannot be equal to $-1$. Note also that $a + b + ab$ results in a singular value. Therefore since $*$ maps into $S$ exclusively and has only one associated value for every input, it is a well-defined map and hence a binary operation.
\end{proof}

\subsection{Part B}
$\langle S, * \rangle$ is a group.

\begin{proof}
	Define the binary algebraic structure $\langle S, * \rangle$ with the prior $S$ and $*$. Examine the axioms for $S$ to be a group under $*$.
	\begin{enumerate}[leftmargin=1.4cm]
		\item[$\mathcal{G}_1 .)$]
		Let $a,b,c \in S$. It follows that
		\begin{align*}
			a * (b * c) &= a * (b + c + bc) \\
			&= a + b + c + bc + ab + ac + abc
		.\end{align*}
		Additionally,
		\begin{align*}
			(a * b) * c &= (a + b + ab) * c \\
			&= a + b + ab + c + ca + cb + cab \\
			&= a + b + c + bc + ab + ac + abc
		.\end{align*}
		Since $a * (b * c) = (a * b) * c$, associativity is satisfied.
		\item[$\mathcal{G}_2 .)$]
		Consider the element $0 \in S$. Let $a \in S$. Then
		\[
			a * 0 = 0 * a = a + 0 + a(0) = a
		.\]
		Therefore $0$ is the identity element of $S$.
		\item[$\mathcal{G}_3 .)$]
			Let $a \in S$. Choose $a' = -\frac{a}{1+a}$. Note then that
			\begin{align*}
				a * a' = a' * a &= a - \frac{a}{1+a} - a \cdot \frac{a}{1 + a} \\
				&= \frac{a(1+a)}{1+a} - \frac{a}{1+a} - \frac{a^2}{1+a} \\
				&= \frac{a+a^2 - a - a^2}{1+a} \\
				&= \frac{0}{1+a} \\
				&= 0
			.\end{align*}
			Since $a \neq -1$, the inverse is well defined and therefore there is an inverse for every element in $S$.
	\end{enumerate}
	
	Since $S$ under $*$ satisfies the group axioms, it is a group.
\end{proof}

\subsection*{Part C}
Note the operation is commutative (because $a + b + ab = b + a + ba$).
\begin{align*}
	2 * x * 3 &= 7 \\
	x * 3 * 2 &= 7 \\
	x * (3 + 2 + 3\cdot 2) &= 7 \\
	x * 11 &= 7 \\
	x * 11 * 11' &= 7 * 11' \\
	x * 0 &= 7 * 11' \\
	x &= 7 * 11' \\
	x &= 7 * \qty(-\frac{11}{12}) \\
	x &= 7 - \frac{11}{12} - \frac{77}{12} \\
	x &= \frac{84}{12} - \frac{11}{12} - \frac{77}{12} \\
	x &= \frac{84 - 11 - 77}{12} \\
	x &= -\frac{4}{12} \\
	x &= -\frac{1}{3}
.\end{align*}

\section*{4.20}
Displayed are all of the $4$ element groups.
\[
	\begin{array}{c|c|c|c|c}
			& e & a & b & c \\\hline
		e & e & a & b & c \\\hline
		a & a & e & c & b \\\hline
		b & b & c & e & a \\\hline
		c & c & b & a & e 
	\end{array}
	\qquad
	\begin{array}{c|c|c|c|c}
			& e & a & b & c \\\hline
		e & e & a & b & c \\\hline
		a & a & b & c & e \\\hline
		b & b & c & e & a \\\hline
		c & c & e & a & b 
	\end{array}
	\qquad
	\begin{array}{c|c|c|c|c}
			& e & a & b & c \\\hline
		e & e & a & b & c \\\hline
		a & a & e & c & b \\\hline
		b & b & c & a & e \\\hline
		c & c & b & e & a 
	\end{array}
\]

The second table can be made into the third table by swapping all instances of $a$ with $b$, resulting in
\[
	\begin{array}{c|c|c|c|c}
			& e & b & a & c \\\hline
		e & e & b & a & c \\\hline
		b & a & a & c & e \\\hline
		a & b & c & e & b \\\hline
		c & c & e & b & a 
	\end{array}
\]
and then rearranging the order back to $e,a,b,c$ provides
\[
	\begin{array}{c|c|c|c|c}
			& e & a & b & c \\\hline
		e & e & a & b & c \\\hline
		a & b & e & c & b \\\hline
		b & a & c & a & e \\\hline
		c & c & b & e & a 
	\end{array}
\]

\subsection*{Part A}
Every table is symmetric across its diagonal, hence every group of $4$ elements is abelian.

\subsection*{Part B}
The second table is isomorphic to $U_4$ with the mapping
\begin{align*}
	e &\to 1 \\
	a &\to i \\
	b &\to -1 \\
	c &\to -i
.\end{align*}

This is true since the table
\[
	\renewcommand{\arraystretch}{1.5}
	\begin{array}{c|c|c|c|c}
			& 1 & i & -1 & -i  \\\hline
		1 & 1 & i & -1 & -i  \\\hline
		i & i & -1 & -i & 1  \\\hline
		-1 & -1 & -i & 1 & i \\\hline
		-i & -i & 1 & i & -1 
	\end{array}
\]
is equivalent under the mapping outlined above.

\subsection*{Part C}
Consider the first table. Choose $n = 2$ and define the following matrices
\[
	E = \mqty[1 & 0 \\ 0 & 1],\; A = \mqty[-1 & 0 \\ 0 & 1 ],\; B = \mqty[1 & 0 \\ 0 & -1],\; C = \mqty[-1 & 0 \\ 0 & -1]
.\]

These are all in the group outlined in Example $14$ since all their determinants are $1$ or $-1$. If the following mapping is used
\begin{align*}
	e &\to E \\
	a &\to A \\
	b &\to B \\
	c &\to C
\end{align*}
then the same structure is achieved between the two groups. This can be checked by the fact that the table is the Klein-$4$ group, therefore if $A^2 = B^2 = C^2 = E$, the isomorphism is correct.
\begin{align*}
	A^2 &= \mqty[1 & 0 \\ 0 & 1] \mqty[1 & 0 \\ 0 & 1] = \mqty[1 & 0 \\ 0 & 1] = E \\
	B^2 &= \mqty[-1 & 0 \\ 0 & 1] \mqty[-1 & 0 \\ 0 & 1] = \mqty[1 & 0 \\ 0 & 1] = E \\
	C^2 &= \mqty[-1 & 0 \\ 0 & -1] \mqty[-1 & 0 \\ 0 & -1] = \mqty[1 & 0 \\ 0 & 1] = E
.\end{align*}

\section*{4.21}
Let $S$ be a set of $3$ elements. That is $S = \qty{x_1, x_2, x_3}$. For a group structure to emerge from a binary operation on $S$, one of the elements must be chosen as an identity element. Therefore there are $3$ possible choices for an identity element. There is only one group structure for a given identity element as seen in the following table:
\[
	\begin{array}{c|c|c|c}
		   & e & a & b \\\hline 
		 e & e & a & b \\\hline 
		 a & a & b & e \\\hline 
		 b & b & e & a 
	\end{array}
\]

Therefore since there is only one associated group structure for every choice of an identity element and there are $3$ choices for an identity element, there are $3$ binary operations that give a group structure over a set of $3$ elements.


\end{document}
