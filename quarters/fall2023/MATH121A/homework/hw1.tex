\documentclass[12pt,titlepage]{extarticle}
% Document Layout and Font
\usepackage{subfiles}
\usepackage[margin=2cm, headheight=15pt]{geometry}
\usepackage{fancyhdr}
\usepackage{enumitem}	
\usepackage{wrapfig}
\usepackage{float}
\usepackage{multicol}

\usepackage[p,osf]{scholax}

\renewcommand*\contentsname{Table of Contents}
\renewcommand{\headrulewidth}{0pt}
\pagestyle{fancy}
\fancyhf{}
\fancyfoot[R]{$\thepage$}
\setlength{\parindent}{0cm}
\setlength{\headheight}{17pt}
\hfuzz=9pt

% Figures
\usepackage{svg}

% Utility Management
\usepackage{color}
\usepackage{colortbl}
\usepackage{xcolor}
\usepackage{xpatch}
\usepackage{xparse}

\definecolor{gBlue}{HTML}{7daea3}
\definecolor{gOrange}{HTML}{e78a4e}
\definecolor{gGreen}{HTML}{a9b665}
\definecolor{gPurple}{HTML}{d3869b}

\definecolor{links}{HTML}{1c73a5}
\definecolor{bar}{HTML}{584AA8}

% Math Packages
\usepackage{mathtools, amsmath, amsthm, thmtools, amssymb, physics}
\usepackage[scaled=1.075,ncf,vvarbb]{newtxmath}

\newcommand\B{\mathbb{C}}
\newcommand\C{\mathbb{C}}
\newcommand\R{\mathbb{R}}
\newcommand\Q{\mathbb{Q}}
\newcommand\N{\mathbb{N}}
\newcommand\Z{\mathbb{Z}}

\DeclareMathOperator{\lcm}{lcm}

% Probability Theory
\newcommand\Prob[1]{\mathbb{P}\qty(#1)}
\newcommand\Var[1]{\text{Var}\qty(#1)}
\newcommand\Exp[1]{\mathbb{E}\qty[#1]}

% Analysis
\newcommand\ball[1]{\B\qty(#1)}
\newcommand\conj[1]{\overline{#1}}
\DeclareMathOperator{\Arg}{Arg}
\DeclareMathOperator{\cis}{cis}

% Linear Algebra
\DeclareMathOperator{\dom}{dom}
\DeclareMathOperator{\range}{range}
\DeclareMathOperator{\spann}{span}
\DeclareMathOperator{\nullity}{nullity}

% TIKZ
\usepackage{tikz}
\usepackage{pgfplots}
\usetikzlibrary{arrows.meta}
\usetikzlibrary{math}
\usetikzlibrary{cd}

% Boxes and Theorems
\usepackage[most]{tcolorbox}
\tcbuselibrary{skins}
\tcbuselibrary{breakable}
\tcbuselibrary{theorems}

\newtheoremstyle{default}{0pt}{0pt}{}{}{\bfseries}{\normalfont.}{0.5em}{}
\theoremstyle{default}

\renewcommand*{\proofname}{\textit{\textbf{Proof.}}}
\renewcommand*{\qedsymbol}{$\blacksquare$}
\tcolorboxenvironment{proof}{
	breakable,
	coltitle = black,
	colback = white,
	frame hidden,
	boxrule = 0pt,
	boxsep = 0pt,
	borderline west={3pt}{0pt}{bar},
	% borderline west={3pt}{0pt}{gPurple},
	sharp corners = all,
	enhanced,
}

\newtheorem{theorem}{Theorem}[section]{\bfseries}{}
\tcolorboxenvironment{theorem}{
	breakable,
	enhanced,
	boxrule = 0pt,
	frame hidden,
	coltitle = black,
	colback = blue!7,
	% colback = gBlue!30,
	left = 0.5em,
	sharp corners = all,
}

\newtheorem{corollary}{Corollary}[section]{\bfseries}{}
\tcolorboxenvironment{corollary}{
	breakable,
	enhanced,
	boxrule = 0pt,
	frame hidden,
	coltitle = black,
	colback = white!0,
	left = 0.5em,
	sharp corners = all,
}

\newtheorem{lemma}{Lemma}[section]{\bfseries}{}
\tcolorboxenvironment{lemma}{
	breakable,
	enhanced,
	boxrule = 0pt,
	frame hidden,
	coltitle = black,
	colback = green!7,
	left = 0.5em,
	sharp corners = all,
}

\newtheorem{definition}{Definition}[section]{\bfseries}{}
\tcolorboxenvironment{definition}{
	breakable,
	coltitle = black,
	colback = white,
	frame hidden,
	boxsep = 0pt,
	boxrule = 0pt,
	borderline west = {3pt}{0pt}{orange},
	% borderline west = {3pt}{0pt}{gOrange},
	sharp corners = all,
	enhanced,
}

\newtheorem{example}{Example}[section]{\bfseries}{}
\tcolorboxenvironment{example}{
	% title = \textbf{Example},
	% detach title,
	% before upper = {\tcbtitle\quad},
	breakable,
	coltitle = black,
	colback = white,
	frame hidden,
	boxrule = 0pt,
	boxsep = 0pt,
	borderline west={3pt}{0pt}{green!70!black},
	% borderline west={3pt}{0pt}{gGreen},
	sharp corners = all,
	enhanced,
}

\newtheoremstyle{remark}{0pt}{4pt}{}{}{\bfseries\itshape}{\normalfont.}{0.5em}{}
\theoremstyle{remark}
\newtheorem*{remark}{Remark}


% TColorBoxes
\newtcolorbox{week}{
	colback = black,
	coltext = white,
	fontupper = {\large\bfseries},
	width = 1.2\paperwidth,
	size = fbox,
	halign upper = center,
	center
}

\newcommand{\banner}[2]{
    \pagebreak
    \begin{week}
   		\section*{#1}
    \end{week}
    \addcontentsline{toc}{section}{#1}
    \addtocounter{section}{1}
    \setcounter{subsection}{0}
}

% Hyperref
\usepackage{hyperref}
\hypersetup{
	colorlinks=true,
	linktoc=all,
	linkcolor=links,
	bookmarksopen=true
}

% Error Handling
\PackageWarningNoLine{ExtSizes}{It is better to use one of the extsizes 
                          classes,^^J if you can}


\def\homeworknumber{1}
\fancyhead[R]{\textbf{Math 140A: Homework \#\homeworknumber}}
\fancyhead[L]{Eli Griffiths}
\renewcommand{\headrulewidth}{1pt}
\setlength\parindent{0pt}


% Section 1.1: 2, 3. 
% Section 1.2: 1, 13, 17, 19. 
% Section 1.3: 1, 5, 12, 22. 

\begin{document}

\subsection*{1.1.2}
\begin{enumerate}[label=\alph*)]
	\item $(-5, 7, 1) - (3, -2, 4) = (-8, 9, -3)$. Therefore the line is $\vec{x}(t) = (3, -2, 4) + t(-8, 9, -3)$.
	\item $(-3, -6,0) - (2,4,0) = (-5, -10, 0)$. Therefore the line is $\vec{x}(t) = (2,4,0) + t(-5,-10,0)$.
	\item $(3,7,-8)-(3,7,2) = (0, 0, -10)$. Therefore the line is $\vec{x}(t) = (3, 7, -8) + t(0, 0, -10)$.
	\item $(3,9,7)-(-2,-1,5) = (5,10,2)$. Therefore the line is $\vec{x}(t) = (3,9,7) + t(5,10,2)$.
\end{enumerate}

\subsection*{1.1.3}
\begin{enumerate}[label=\alph*)]
	\item Let $A = (2, -5, -1), B = (0,4,6), C = (-3,7,1)$. Then $B - A = (-2, 9, 7)$ and $C - A = (-5, 12, 2)$. Therefore the plane equation is $\vec{x}(s,t) = (2, -5, -1) + s(-2,9,7) + t(-5, 12, 2)$.
	\item Let $A = (3, -6, 7), B = (-2,0,-4), C = (5,-9,2)$. Then $B - A = (-5, 6, -11)$ and $C - A = (2, -3, -5)$. Therefore the plane equation is $\vec{x}(s,t) = (3, -6, -7) + s(-5, 6, -11) + t(2, -3, -5)$.
	\item Let $A = (-8, 2, 0), B = (1,3,0), C = (6,-5,0)$. Then $B - A = (9, 1, 0)$ and $C - A = (14, -7, 0)$. Therefore the plane equation is $\vec{x}(s,t) = (-8, 2, 0) + s(9, 1, 0) + t(14, -7, 0)$.
	\item Let $A = (1,1,1), B = (5,5,5), C = (-6,4,2)$. Then $B - A = (4,4,4)$ and $C - A = (-7, 3, 1)$. Therefore the plane equation is $\vec{x}(s,t) = (1,1,1) + s(4,4,4) + t(-7,3,1)$.
\end{enumerate}

\subsection*{1.2.1}
\begin{multicols}{3}
	\begin{enumerate}[label=\alph*)]
		\item True
		\item False
		\item False
		\item False
		\item True
		\item False
		\item False
		\item False
		\item True
		\item True
		\item True
	\end{enumerate}
\end{multicols}

\subsection*{1.2.13}
It is not a vector space. The zero vector must be $0 (a_1, a_2) = (0, a_2)$ for all $a_1, a_2 \in \mathbb{F}$. However since $a_2$ is variable, the zero vector is not unique and therefore $V$ is not a vector space.

\subsection*{1.2.17}
It is not a vector space. The zero vector must be $0 (a_1, a_2) = (a_1, 0)$ for all $a_1, a_2 \in \mathbb{F}$. This fails the same way the last question does. Therefore $V$ is not a vector space.

\subsection*{1.2.19}
It is not a vector space. If $V$ were a vector space, then $(2+3) \cdot (3,4) = 2\cdot (3, 4) + 3\cdot (3,4)$. Calculating both sides:
\begin{align*}
	(2 + 3) \cdot (3,4) &= 5 \cdot (3,4) \\
	&= (15, \frac{4}{5}) \\
	\intertext{and}
	(2+3) \cdot (3,4) &= 2\cdot (3,4) + 3\cdot (3,4) \\
					  &= (6, 2) + (9, \frac{4}{3}) \\
					  &= (15, \frac{10}{3})
\end{align*}
Since $(15, \frac{10}{3}) \neq (15, \frac{4}{5})$, distributivity doesn't hold and so $V$ is not a vector space.

\subsection*{1.3.1}
\begin{multicols}{2}
	\begin{enumerate}[label=\alph*)]
		\item False
		\item False
		\item True
		\item False
		\item True
		\item False
		\item False
	\end{enumerate}
\end{multicols}

\subsection*{1.3.5}
\begin{proof}
	Let $A$ be a square matrix. Then
	\[
		(A + A^\transp)^\transp = A^\transp + A = A + A^\transp
	\]
	Therefore since $(A + A^\transp)^\transp = A + A^\transp$, $A + A^\transp$ is a symmetric matrix.
\end{proof}

\subsection*{1.3.12}
\begin{proof}
	Let $W = \qty{m\times n\text{ upper triangular matrices}}$. Note that $W \subset M_{m\times n}(\mathbb{F})$. Let $w, k \in W$. $k + w$ will result in an upper triangular matrix since every element below the main diagonal is zero in both and therefore their sum below the main diagonal will be zero. Therefore $k + w \in W$. Let $c \in \mathbb{F}$. $c \cdot w$ will give an upper triangular matrix since every zero element below the main diagonal multiplied by a constant will remain zero, resulting in an upper triangular matrix. Therefore $cw \in W$. Note that the zero matrix is upper triangular and hence is in $W$. Since $W$ is closed under addition and scalar multiplication and the zero matrix is in $W$, $W$ is a subspace of $M_{m\times n} (\mathbb{F})$.
\end{proof}

\subsection*{1.3.22}
\def\funfield{\mathcal{F}(\mathbb{F}_1, \mathbb{F}_2)}

Let $\mathbb{F}_1$ and $\mathbb{F}_2$ be fields. First consider even functions:
\begin{proof}
	Let $E = \qty{g\text{ is even} : g\in\funfield}$. Note that $E \subset \funfield$. Let $f,g \in E$. Let $h(t) = f(t) + g(t)$ where $t \in \mathbb{F}_1$. Note that
	\[
		h(-t) = f(-t) + g(-t) = f(t) + g(t) = h(t)
	\]
	Therefore $h(t)$ is also an even function and hence $h \in E$. Let $c \in \mathbb{F}_2$ and $j(t) = c \cdot f(t)$. Since $j(-t) = c \cdot f(-t) = c \cdot f(t) = j(t)$, $j(t)$ is an even function and hence $j \in E$. Additionally, the zero function is even and odd so it is in $E$. Since $E$ is closed under addition and scalar multiplication and contains the zero function, $E$ is a subspace of $\funfield$.
\end{proof}

Next consider odd functions:
\begin{proof}
	Let $O = \qty{g\text{ is odd} : g\in\funfield}$. Note that $O \subset \funfield$. Let $f,g \in O$. Let $h(t) = f(t) + g(t)$ where $t \in \mathbb{F}_1$. Note that
	\[
		h(-t) = f(-t) + g(-t) = -f(t) - g(t) = -(f(t) + g(t)) = -h(t)
	\]
	Therefore $h(t)$ is also an odd function and hence $h \in O$. Let $c \in \mathbb{F}_2$ and $j(t) = c \cdot f(t)$. Note that $j(-t) = c \cdot f(-t) = -c \cdot f(t) = -j(t)$. Therefore $j$ is an odd function and hence $j \in O$. Additionally, the zero function is even and odd so it is in $O$. Since $O$ is closed under addition and scalar multiplication and contains the zero function, $O$ is a subspace of $\funfield$.
\end{proof}

\end{document}
