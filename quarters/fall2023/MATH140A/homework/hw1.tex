\documentclass[12pt,titlepage]{extarticle}
% Document Layout and Font
\usepackage{subfiles}
\usepackage[margin=2cm, headheight=15pt]{geometry}
\usepackage{fancyhdr}
\usepackage{enumitem}	
\usepackage{wrapfig}
\usepackage{float}
\usepackage{multicol}

\usepackage[p,osf]{scholax}

\renewcommand*\contentsname{Table of Contents}
\renewcommand{\headrulewidth}{0pt}
\pagestyle{fancy}
\fancyhf{}
\fancyfoot[R]{$\thepage$}
\setlength{\parindent}{0cm}
\setlength{\headheight}{17pt}
\hfuzz=9pt

% Figures
\usepackage{svg}

% Utility Management
\usepackage{color}
\usepackage{colortbl}
\usepackage{xcolor}
\usepackage{xpatch}
\usepackage{xparse}

\definecolor{gBlue}{HTML}{7daea3}
\definecolor{gOrange}{HTML}{e78a4e}
\definecolor{gGreen}{HTML}{a9b665}
\definecolor{gPurple}{HTML}{d3869b}

\definecolor{links}{HTML}{1c73a5}
\definecolor{bar}{HTML}{584AA8}

% Math Packages
\usepackage{mathtools, amsmath, amsthm, thmtools, amssymb, physics}
\usepackage[scaled=1.075,ncf,vvarbb]{newtxmath}

\newcommand\B{\mathbb{C}}
\newcommand\C{\mathbb{C}}
\newcommand\R{\mathbb{R}}
\newcommand\Q{\mathbb{Q}}
\newcommand\N{\mathbb{N}}
\newcommand\Z{\mathbb{Z}}

\DeclareMathOperator{\lcm}{lcm}

% Probability Theory
\newcommand\Prob[1]{\mathbb{P}\qty(#1)}
\newcommand\Var[1]{\text{Var}\qty(#1)}
\newcommand\Exp[1]{\mathbb{E}\qty[#1]}

% Analysis
\newcommand\ball[1]{\B\qty(#1)}
\newcommand\conj[1]{\overline{#1}}
\DeclareMathOperator{\Arg}{Arg}
\DeclareMathOperator{\cis}{cis}

% Linear Algebra
\DeclareMathOperator{\dom}{dom}
\DeclareMathOperator{\range}{range}
\DeclareMathOperator{\spann}{span}
\DeclareMathOperator{\nullity}{nullity}

% TIKZ
\usepackage{tikz}
\usepackage{pgfplots}
\usetikzlibrary{arrows.meta}
\usetikzlibrary{math}
\usetikzlibrary{cd}

% Boxes and Theorems
\usepackage[most]{tcolorbox}
\tcbuselibrary{skins}
\tcbuselibrary{breakable}
\tcbuselibrary{theorems}

\newtheoremstyle{default}{0pt}{0pt}{}{}{\bfseries}{\normalfont.}{0.5em}{}
\theoremstyle{default}

\renewcommand*{\proofname}{\textit{\textbf{Proof.}}}
\renewcommand*{\qedsymbol}{$\blacksquare$}
\tcolorboxenvironment{proof}{
	breakable,
	coltitle = black,
	colback = white,
	frame hidden,
	boxrule = 0pt,
	boxsep = 0pt,
	borderline west={3pt}{0pt}{bar},
	% borderline west={3pt}{0pt}{gPurple},
	sharp corners = all,
	enhanced,
}

\newtheorem{theorem}{Theorem}[section]{\bfseries}{}
\tcolorboxenvironment{theorem}{
	breakable,
	enhanced,
	boxrule = 0pt,
	frame hidden,
	coltitle = black,
	colback = blue!7,
	% colback = gBlue!30,
	left = 0.5em,
	sharp corners = all,
}

\newtheorem{corollary}{Corollary}[section]{\bfseries}{}
\tcolorboxenvironment{corollary}{
	breakable,
	enhanced,
	boxrule = 0pt,
	frame hidden,
	coltitle = black,
	colback = white!0,
	left = 0.5em,
	sharp corners = all,
}

\newtheorem{lemma}{Lemma}[section]{\bfseries}{}
\tcolorboxenvironment{lemma}{
	breakable,
	enhanced,
	boxrule = 0pt,
	frame hidden,
	coltitle = black,
	colback = green!7,
	left = 0.5em,
	sharp corners = all,
}

\newtheorem{definition}{Definition}[section]{\bfseries}{}
\tcolorboxenvironment{definition}{
	breakable,
	coltitle = black,
	colback = white,
	frame hidden,
	boxsep = 0pt,
	boxrule = 0pt,
	borderline west = {3pt}{0pt}{orange},
	% borderline west = {3pt}{0pt}{gOrange},
	sharp corners = all,
	enhanced,
}

\newtheorem{example}{Example}[section]{\bfseries}{}
\tcolorboxenvironment{example}{
	% title = \textbf{Example},
	% detach title,
	% before upper = {\tcbtitle\quad},
	breakable,
	coltitle = black,
	colback = white,
	frame hidden,
	boxrule = 0pt,
	boxsep = 0pt,
	borderline west={3pt}{0pt}{green!70!black},
	% borderline west={3pt}{0pt}{gGreen},
	sharp corners = all,
	enhanced,
}

\newtheoremstyle{remark}{0pt}{4pt}{}{}{\bfseries\itshape}{\normalfont.}{0.5em}{}
\theoremstyle{remark}
\newtheorem*{remark}{Remark}


% TColorBoxes
\newtcolorbox{week}{
	colback = black,
	coltext = white,
	fontupper = {\large\bfseries},
	width = 1.2\paperwidth,
	size = fbox,
	halign upper = center,
	center
}

\newcommand{\banner}[2]{
    \pagebreak
    \begin{week}
   		\section*{#1}
    \end{week}
    \addcontentsline{toc}{section}{#1}
    \addtocounter{section}{1}
    \setcounter{subsection}{0}
}

% Hyperref
\usepackage{hyperref}
\hypersetup{
	colorlinks=true,
	linktoc=all,
	linkcolor=links,
	bookmarksopen=true
}

% Error Handling
\PackageWarningNoLine{ExtSizes}{It is better to use one of the extsizes 
                          classes,^^J if you can}


\def\homeworknumber{1}
\fancyhead[R]{\textbf{Math 140A: Homework \#\homeworknumber}}
\fancyhead[L]{Eli Griffiths}
\renewcommand{\headrulewidth}{1pt}
\setlength\parindent{0pt}


% Section 1: 1, 4, 8/9
% Section 2: 1, 5, 7, 8
% Section 3: 1, 3, 4

\begin{document}

\subsection*{1.1}
\begin{proof}
	Proceed with induction. Consider the base when $n = 1$. Then $1^2 = 1 = \frac{1}{6} \cdot 6 = \frac{1}{6} 1(1+1)(2 + 1)$. Therefore the base case holds. Assume for a fixed $n \in \mathbb{N}$ that $1^2 + 2^2 + \ldots + n^2 = \frac{1}{6} n (n+1) (2n+1)$. Note that
	\[
		\frac{1}{6} n (n+1) (2n +1) = \frac{2n^3 + 3n^2 + n}{6}
	\]
	Consider the equation
	\begin{align*}
		\frac{1}{6} (n+1) (n + 2) (2n +3) &= \frac{1}{6}\qty[(n^3 + 3n + 2)(2n + 3)] \\
										  &= \frac{1}{6}\qty[2n^3 + 6n^2 + 4n + 3n^2 + 9n +6] \\
										  &= \frac{1}{6}\qty[2n^3 + 9n^2 + 13n + 6] \\
										  &= \frac{2n^3 + 3n^2 + n}{6} + n^2 + 2n + 1 \\
										  &= \frac{2n^3 + 3n^2 + n}{6} + (n+1)^2 \\
		\intertext{Applying the induction hypothesis,}
										  &= 1^2 + 2^2 + \ldots + n^2+ (n+1)^2
	\end{align*}
	therefore the $n + 1$ case holds. Therefore for all $n \in \mathbb{N}$, $1^2 + 2^2 + \ldots + n^2 = \frac{1}{6} n (n+1) (2n+1)$.
\end{proof}

\subsection*{1.4}
\subsubsection*{Part A}
\[
	1 + 3 + \ldots + (2n - 1) = n^2
\]

\subsubsection*{Part B}
\begin{proof}
	Proceed with induction. Consider the base case where $n = 1$. Then $2n - 1 = 2 - 1 = 1 = 1^2$, therefore the base case holds. Assume for a fixed $n \in \mathbb{N}$ that $1 + 3 + \ldots + (2n - 1) = n^2$. Then
	\begin{align*}
		(n+1)^2 &= n^2 + 2n + 1 \\
		\intertext{Applying the induction hypothesis to $n^2$,}
		(n+1)^2 &= 1 + 3 + \ldots + (2n - 1) + (2n + 1).
	\end{align*}
	Since $2n + 1 = (2(n+1) - 1)$, the $n+1$ case holds. Therefore for all $n \in \mathbb{N}$, $1 + 3 + \ldots + (2n-1) = n^2$.
\end{proof}

\subsection*{1.9}
\subsubsection*{Part A}
$2^n > n^2$ for all $n \geq 5$.

\subsubsection*{Part B}
\begin{proof}
	Proceed with induction. Consider the base case where $n = 5$. Then $2^5 = 32 > 25 = 5^2$, therefore the base case holds. Assume for a fixed $n \in \mathbb{N} \geq 5$ that $2^n > n^2$. Then
	\begin{align*}
		2^{n+1} = 2 (2^n) > 2n^2 > n^2 + 2n + 1 = (n+1)^2
	\end{align*}
	$2n^2 > n^2 + 2n + 1$ is true because in 1.8 it is established that $n^2 > n + 1$, which leads to $2n^2 > 2n + 2 > 2n + 1$. $2n^2 > n^2 + 2n + 1$ can be rewritten as $n^2 > 2n + 1$ and hence is true by the previous derivation. Transversing the inequalities gives $2^{n+1} > (n+1)^2$, meaning the $n+1$ case holds. Therefore for all $n \in \mathbb{N} \geq 5$, $2^{n+1} > (n+1)^2$.
\end{proof}

\subsection*{2.1}
\begin{enumerate}
	\item Let $x = \sqrt{3}$. Then $x^2 - 3 = 0$. By Corollary 2.3, the only rational solutions are integers that divide $-3$, meaning $\pm 1, \pm 3$. $(\pm 1)^2 - 3 = -2$ and $(\pm 3)^2 - 3 = 6$. None of the possible rational solutions work, so $\sqrt{3}$ is not rational.
	\item Let $x = \sqrt{5}$. Then $x^2 - 5 = 0$. By Corollary 2.3, the only rational solutions are integers that divide $-5$, meaning $\pm 1, \pm 5$. $(\pm 1)^2 - 5 = -4$ and $(\pm 5)^2 - 5 = 20$. None of the possible rational solutions work, so $\sqrt{5}$ is not rational.
	\item Let $x = \sqrt{7}$. Then $x^2 - 7 = 0$. By Corollary 2.3, the only rational solutions are integers that divide $-7$, meaning $\pm 1, \pm 7$. $(\pm 1)^2 - 7 = -6$ and $(\pm 7)^2 - 7 = 42$. None of the possible rational solutions work, so $\sqrt{7}$ is not rational.
	\item Let $x = \sqrt{24}$. Then $x^2 - 24 = 0$. By Corollary 2.3, the only rational solutions are integers that divide $-24$, meaning $\pm 1, \pm 2, \pm 3, \pm 6, \pm 8, \pm 12, \pm 24$. Checking each:
		\begin{align*}
			(\pm 1)^2 - 24 &= 1 - 24 \neq 0\\
			(\pm 2)^2 - 24 &= 4 - 24 \neq 0\\
			(\pm 3)^2 - 24 &= 9 -24 \neq 0 \\
			(\pm 4)^2 - 24 &= 16 - 24 \neq 0 \\
			(\pm 6)^2 - 24 &= 36 - 24 \neq 0 \\
			(\pm 8)^2 - 24 &= 64 - 24 \neq 0 \\
			(\pm 12)^2 - 24 &= 144 - 24 \neq 0 \\
		\end{align*}
		None of the possible rational solutions work, so $\sqrt{24}$ is not rational.
	\item Let $x = \sqrt{31}$. Then $x^2 - 31 = 0$. By Corollary 2.3, the only rational solutions are integers that divide $-31$, meaning $\pm 1, \pm 31$. $(\pm 1)^2 - 31 = -30$ and $(\pm 31)^2 - 31 = 930$. None of the possible rational solutions work, so $\sqrt{31}$ is not rational.
\end{enumerate}

\subsection*{2.5}
Let $x = \qty(3 + \sqrt{2})^{\frac{2}{3}}$. Then $x^6 - 22x^3 + 49 = 0$. By Corollary 2.3, the only rational solutions are integers that divide $49$, meaning $\pm 1, \pm 7$. Checking each:
\begin{align*}
	1^6 - 22 \cdot 1^3 + 49 &= 28 \neq 0 \\
	(-1)^6 - 22 \cdot (-1)^3 + 49 &= 72 \neq 0 \\
	7^6 - 22 \cdot 7^3 + 49 &>> 0 \\
	(-7)^6 - 22 \cdot (-7)^3 + 49 &>> 0 \\
\end{align*}
Since none of the possible rational solutions work, $\qty(3 + \sqrt{2})^{\frac{2}{3}}$ is not rational.

\subsection*{2.7}
\subsubsection*{Part A}
Let $x = \sqrt{4 + 2 \sqrt{3}} - \sqrt{3}$. Then $x^4 - 14x^2 + 24x - 11 = 0$. By Corollary 2.3, the rational solutions are integers that divide $-11$ meaning $\pm 1, \pm 11$. Consider $1$.
\[
	1^4 - 14 \cdot 1^2 + 24 - 11 = 1 - 14 + 24 - 11 = 0
\]
Therefore $1$ is a rational solution. Note then that
\begin{align*}
	1 &= \sqrt{4 + 2 \sqrt{3}} - \sqrt{3} \\
	1 + \sqrt{3} &= \sqrt{4 + 2 \sqrt{3}} \\
	1 + 2\sqrt{3} + 3 &= 4 + 2 \sqrt{3} \\
	1 + 3 &= 4 \\
	4 &= 4 \\
\end{align*}
Therefore $x = 1$ and is hence rational.

\subsubsection*{Part B}
Let $x = \sqrt{6 + 4 \sqrt{2}} - \sqrt{2}$. Then $x^4 - 16x^2  + 32x - 16 = 0$. By Corollary 2.3, the rational solutions are integers that divide $-16$ meaning $\pm 1, \pm 2, \pm 4, \pm 8$. Consider $2$.
\[
	2^4 - 16 \cdot 2^2 + 32 \cdot 2 - 16 = 16 - 64 + 64 - 16 = 0
\]
Therefore $2$ is a rational solution. Note then that
\begin{align*}
	2 &= \sqrt{6 + 4 \sqrt{2}} - \sqrt{2} \\
	2 + \sqrt{2} &= \sqrt{6 + 4 \sqrt{2}} \\
	6 + 4\sqrt{2} &= 6 + 4 \sqrt{2} \\
	6  &= 6  \\
\end{align*}
Therefore $x = 2$ and is hence rational.

\subsection*{2.8}
By Corollary 2.3, the rational solutions of $x^8  - 4x^5 + 13x^3 - 7x + 1 = 0$ are the integers that divide $1$, meaning $\pm 1$. Checking $1$ and $-1$:
\begin{align*}
	1^8  - 4\cdot 1^5 + 13\cdot 1^3 - 7\cdot 1 + 1 &= 1 - 4 + 13 - 7 + 1 = 4 \neq 0 \\
	(-1)^8  - 4\cdot (-1)^5 + 13\cdot (-1)^3 - 7\cdot (-1) + 1 &= 1 + 4 - 13 + 7 + 1 = 0
\end{align*}
Therefore the only rational solution is $x = -1$.

\subsection*{3.1}
For $\mathbb{N}$ A4 and M4 fails and for $\mathbb{Z}$ only M4 fails.

\subsection*{3.3}
\subsubsection*{iv.)}
\begin{proof}
	Let $a,b \in \mathbb{F}$. Consider the equation
	\begin{align*}
		-ab + (-a)(-b) &= -(a)b + (-a)(-b) \tag{By iii} \\
					   &= -(a)(b + (-b)) \tag{By DL} \\
					   &= -(a) \cdot 0 \tag{By A4} \\
					   &= 0 \tag{By ii}
	\end{align*}
	Therefore $-ab + (-a)(-b) = 0$, meaning $(-a)(-b) = ab$.
\end{proof}

\subsubsection*{v.)}
\begin{proof}
	Let $a,b,c \in \mathbb{F}$ with $c \neq 0$. Consider the equation $ac = bc$. Since $c$ is non-zero it has an inverse $c^{-1}$. Then
	\begin{align*}
		ac &= bc \\
		(ac)c^{-1} &= (bc)c^{-1} \\
		a(cc^{-1}) &= b(cc^{-1}) \tag{By M1} \\
		a\cdot 1 &= b\cdot 1 \tag{By M4} \\
		a &= b \tag{By M3}
	\end{align*}
	Therefore $a = b$.
\end{proof}

\subsection*{3.4}
\subsubsection*{v.)}
\begin{proof}
	By (iv), for all $a \in \mathbb{F}$, $0 <= a^2$. Therefore $0 <= 1^2$. Since $1^2 = 1\cdot 1 = 1$ by M3, $0 \leq 1$. Since $0 \neq 1$, $0 < 1$.
\end{proof}

\subsubsection*{vii.)}
\begin{proof}
	Let $a,b \in \mathbb{F}$ and assume that $0 < a < b$. Since $a > 0$ and $b > 0$, they have inverses and, with $c = a^{-1} b^{-1}$, $c > 0$. Since $a < b$ and $c > 0$, $ac < bc$ meaning
	\[
		aa^{-1}b^{-1} < ba^{-1}b^{-1}
	\]
	which by commutivity, associativity, and inverses results in
	\[
		b^{-1} < a^{-1}
	.\]
	Since $a > 0$ and $b > 0$, their inverses are also greater than $0$ meaning overall
 that $0 < b^{-1} < a^{-1}$.
\end{proof}

\end{document}
