\documentclass{article}

% Document Layout and Font
\usepackage{subfiles}
\usepackage[margin=2cm, headheight=15pt]{geometry}
\usepackage{fancyhdr}
\usepackage{enumitem}	
\usepackage{wrapfig}
\usepackage{float}
\usepackage{multicol}

\usepackage[p,osf]{scholax}

\renewcommand*\contentsname{Table of Contents}
% \renewcommand{\headrulewidth}{0pt}
\pagestyle{fancy}
\fancyhf{}
\fancyfoot[R]{$\thepage$}
\setlength{\parindent}{0cm}
\setlength{\headheight}{17pt}
\hfuzz=9pt

% Figures
\usepackage{svg}

% Utility Management
\usepackage{color}
\usepackage{colortbl}
\usepackage{xcolor}
\usepackage{xpatch}
\usepackage{xparse}

\definecolor{gBlue}{HTML}{7daea3}
\definecolor{gOrange}{HTML}{e78a4e}
\definecolor{gGreen}{HTML}{a9b665}
\definecolor{gPurple}{HTML}{d3869b}

\definecolor{links}{HTML}{1c73a5}
\definecolor{bar}{HTML}{584AA8}

% Math Packages
\usepackage{mathtools, amsmath, amsthm, thmtools, amssymb, physics}
\usepackage[scaled=1.075,ncf,vvarbb]{newtxmath}

\newcommand\B{\mathbb{C}}
\newcommand\C{\mathbb{C}}
\newcommand\R{\mathbb{R}}
\newcommand\Q{\mathbb{Q}}
\newcommand\N{\mathbb{N}}
\newcommand\Z{\mathbb{Z}}

\DeclareMathOperator{\lcm}{lcm}

% Probability Theory
\newcommand\Prob[1]{\mathbb{P}\qty(#1)}
\newcommand\Var[1]{\text{Var}\qty(#1)}
\newcommand\Exp[1]{\mathbb{E}\qty[#1]}

% Analysis
\newcommand\ball[1]{\B\qty(#1)}
\newcommand\conj[1]{\overline{#1}}
\DeclareMathOperator{\Arg}{Arg}
\DeclareMathOperator{\cis}{cis}

% Linear Algebra
\DeclareMathOperator{\dom}{dom}
\DeclareMathOperator{\spann}{span}
\DeclareMathOperator{\nullity}{nullity}

% TIKZ
\usepackage{tikz}
\usepackage{pgfplots}
\usetikzlibrary{arrows.meta}
\usetikzlibrary{math}
\usetikzlibrary{cd}

% Boxes and Theorems
\usepackage[most]{tcolorbox}
\tcbuselibrary{skins}
\tcbuselibrary{breakable}
\tcbuselibrary{theorems}

\newtheoremstyle{default}{0pt}{0pt}{}{}{\bfseries}{\normalfont.}{0.5em}{}
\theoremstyle{default}

\renewcommand*{\proofname}{\textit{\textbf{Proof.}}}
\renewcommand*{\qedsymbol}{$\blacksquare$}
\tcolorboxenvironment{proof}{
	breakable,
	coltitle = black,
	colback = white,
	frame hidden,
	boxrule = 0pt,
	boxsep = 0pt,
	borderline west={3pt}{0pt}{bar},
	% borderline west={3pt}{0pt}{gPurple},
	sharp corners = all,
	enhanced,
}

\newtheorem{theorem}{Theorem}[section]{\bfseries}{}
\tcolorboxenvironment{theorem}{
	breakable,
	enhanced,
	boxrule = 0pt,
	frame hidden,
	coltitle = black,
	colback = blue!7,
	% colback = gBlue!30,
	left = 0.5em,
	sharp corners = all,
}

\newtheorem{corollary}{Corollary}[section]{\bfseries}{}
\tcolorboxenvironment{corollary}{
	breakable,
	enhanced,
	boxrule = 0pt,
	frame hidden,
	coltitle = black,
	colback = white!0,
	left = 0.5em,
	sharp corners = all,
}

\newtheorem{lemma}{Lemma}[section]{\bfseries}{}
\tcolorboxenvironment{lemma}{
	breakable,
	enhanced,
	boxrule = 0pt,
	frame hidden,
	coltitle = black,
	colback = green!7,
	left = 0.5em,
	sharp corners = all,
}

\newtheorem{definition}{Definition}[section]{\bfseries}{}
\tcolorboxenvironment{definition}{
	breakable,
	coltitle = black,
	colback = white,
	frame hidden,
	boxsep = 0pt,
	boxrule = 0pt,
	borderline west = {3pt}{0pt}{orange},
	% borderline west = {3pt}{0pt}{gOrange},
	sharp corners = all,
	enhanced,
}

\newtheorem{example}{Example}[section]{\bfseries}{}
\tcolorboxenvironment{example}{
	% title = \textbf{Example},
	% detach title,
	% before upper = {\tcbtitle\quad},
	breakable,
	coltitle = black,
	colback = white,
	frame hidden,
	boxrule = 0pt,
	boxsep = 0pt,
	borderline west={3pt}{0pt}{green!70!black},
	% borderline west={3pt}{0pt}{gGreen},
	sharp corners = all,
	enhanced,
}

\newtheoremstyle{remark}{0pt}{4pt}{}{}{\bfseries\itshape}{\normalfont.}{0.5em}{}
\theoremstyle{remark}
\newtheorem*{remark}{Remark}


% TColorBoxes
\newtcolorbox{week}{
	colback = black,
	coltext = white,
	fontupper = {\large\bfseries},
	width = 1.2\paperwidth,
	size = fbox,
	halign upper = center,
	center
}

\newcommand{\banner}[2]{
    \pagebreak
    \begin{week}
   		\section*{#1}
    \end{week}
    \addcontentsline{toc}{section}{#1}
    \addtocounter{section}{1}
    \setcounter{subsection}{0}
}

% Hyperref
\usepackage{hyperref}
\hypersetup{
	colorlinks=true,
	linktoc=all,
	linkcolor=links,
	bookmarksopen=true
}

% Error Handling
\PackageWarningNoLine{ExtSizes}{It is better to use one of the extsizes 
                          classes,^^J if you can}


\usepackage{fancyhdr}
\pagestyle{fancy}
\fancyhead[lcr]{}
\fancyhead[l]{Eli Griffiths}
\fancyhead[c]{MATH $121$B}
\fancyhead[r]{HW \#$1$}

\DeclareMathOperator*{\argmin}{argmin}

\NewDocumentCommand\iprd{ s m }{% s = star, m = mandatory arg
   \IfBooleanTF{#1}{%
    \langle #2 \rangle
   }{%
    \left\langle #2 \right\rangle
   }%
}

\begin{document}

\section*{Problem 1}
\begin{proof}
    We can first rewrite the norm in terms of the inner product,
    \begin{align*}
        \norm{\sum_i a_i v_i}^2 &= \qty(\sqrt{
            \iprd{\sum_i a_i v_i, \sum_i a_i v_i}
        })^2 \\
        &= \iprd{\sum_i a_i v_i, \sum_i a_i v_i} \\
    \intertext{Since the inner product is linear, we can pass the sums through}
        &= \sum_i \iprd{a_i v_i, \sum_j a_j v_j} \\
        &= \sum_i a_i \sum_j \iprd{v_i, a_j v_j} \\
    \intertext{$S$ is orthogonal as well meaning $\iprd{v_i, v_j} = 0$ if $i = j$, hence}
        &= \sum_i a_i \iprd{v_i, a_i v_i} \\
        &= \sum_i a_i \conj{a_i} \iprd{v_i, v_i} \\
        &= \sum_i |a_i|^2 \norm{v_i}^2 \qedhere
    \end{align*}
\end{proof}

\section*{Problem 2}
\begin{proof}
   Since we are dealing with a norm induced by an inner product, we can rewrite the left hand side as
   \begin{align*}
       \norm{x+y}^2 + \norm{x-y}^2 &= \iprd{x+y, x+y} + \iprd{x-y, x-y}
       \intertext{We can use the linearity of the inner product to split apart the terms,}
                                   &= \iprd{x, x+y} + \iprd{y, x+y} + \iprd{x, x-y} + \iprd{-y,x-y} \\
       \intertext{and the $-1$ attached to $-y$ can be moved into the second position since $\conj{-1} = -1$,}
                                   &= \iprd{x, x+y} + \iprd{y, x+y} + \iprd{x, x-y} + \iprd{y,y-x} \\
                                   &= \iprd{x, (x+y) + (x-y)} + \iprd{y, (x+y) + (y-x)} \\
                                   &= \iprd{x, 2x} + \iprd{y, 2y} \\
                                   &= 2 \iprd{x, x} + 2\iprd{y, y} \\
                                   &= 2 \norm{x}^2 + 2\norm{y}^2.
   \end{align*}
   Hence the equality holds.
\end{proof}

\section*{Problem 3}
No, it does not hold in general. Consider $\R^2$ with the infinity norm $\norm{x}_{\infty} = \displaystyle\max_i |x_i|$. Then if $x = (1,0)$ and $y = (0, 1)$ we have
\[
    \norm{x + y}_{\infty}^2 + \norm{x - y}_{\infty}^2 = 1^2 + 1^2 = 2 \neq 4 = 2 + 2 = 2 \norm{x}_{\infty}^2 + 2 \norm{y}_{\infty}^2
.\]

\section*{Problem 4}
\begin{proof}
    Let $y \in V$ and
    \[
        y_{\parallel} \coloneq \sum_{i=1}^k \iprd{y, w_i} w_i
    .\]
    We will show that
    \[
        \argmin_{x \in W} \norm{y - x} = y_\parallel
    \]
    Let $z = y - y_{\parallel}$. Then for any $w_j \in S$
    \begin{align*}
        \iprd{z, w_j} &= \iprd{\qty(y - \sum_{i=1}^k \iprd{y, w_i} w_i), w_j} \\
                      &= \iprd{y, w_j} - \iprd{\sum_{i=1}^k \iprd{y, w_i} w_i, w_j} \\
                      &= \iprd{y, w_j} - \sum_{i=1}^k \iprd{\iprd{y, w_i} w_i, w_j} \\
                      &= \iprd{y, w_j} - \sum_{i=1}^k \iprd{y, w_i} \iprd{w_i, w_j}.
    \end{align*}
    Since $S$ is orthonormal, we have $\iprd{w_i, w_j} = \delta_{ij}$ and
    \[
        \iprd{z, w_j} = \iprd{y, w_j} - \sum_{i=1}^k \iprd{y, w_i} \iprd{w_i, w_j} = \iprd{y, w_j} - \iprd{y, w_j} = 0
    .\]
    Therefore $z \in W^\perp$. Now consider some $x \in W$. Since $y_{\parallel}$ is defined as a linear combination of vectors from $W$, it must also be in $W$ meaning $y_\parallel - x \in W$. Then $y_{\parallel} - x$ is orthogonal to $z$. With this orthogonality,
    \begin{align*}
        \norm{y - x}^2 &= \norm{y_{\parallel} + z - x}^2 \\
                       &= \norm{(y_{\parallel} - x) + z} \\
                       &= \norm{y_{\parallel} - x}^2 + \norm{z}^2 \geq \norm{z}^2 = \norm{y - y_{\parallel}}.
    \end{align*}
    Reading the extreme ends of both sides and taking the square root gives $\norm{y-x} \geq \norm{y - y_\parallel}$. Thus any choice of $x$ will give a value greater than or equal to the proposed $y_{\parallel}$. Therefore we know that at least $y_\parallel$ is in the $\argmin$. If $\norm{y-x} = \norm{y-y_{\parallel}}$ then we have from the previous derivation equality and
    \[
        \norm{y_{\parallel} - x}^2 + \norm{x}^2 = \norm{x}^2 \implies \norm{y_\parallel - x} = 0
    .\]
    But this means $y_{\parallel} = x$. Therefore we have $y_\parallel$ as the unique vector in the $\argmin$.
\end{proof}

\section*{Problem 5}
\begin{proof}
    Let $x \in R(T^*)^\perp$. Then for any $y = T^* y' \in R(T^*)'$ with $y' \in V_1$ it follows
    \[
        0 = \iprd{x,y} = \iprd{x, T^* y'} = \iprd{Tx, y'}
    .\]
    Therefore $Tx$ must be zero since it holds for any $y'$ meaning $x \in N(T)$. Now let $x \in N(T)$. Then
    \[
        0 = \iprd{0, y} = \iprd{Tx, y} = \iprd{x, T^* y}
    \]
    for any $y \in V_1$. Therefore $x \in R(T^*)^\perp$, hence $R(T^*)^\perp = N(T)$. Since $V_1$ and $V_2$ are finite dimensional, then we have $\qty(W^\perp)^\perp = W$ for any subspace $W$ in both spaces which gives $N(T)^\perp = (R(T^*)^\perp)^\perp = R(T^*)$.
\end{proof}

\end{document}
