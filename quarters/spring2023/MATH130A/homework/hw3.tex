\documentclass[12pt]{extarticle}

% Document Layout and Font
\usepackage{subfiles}
\usepackage[margin=2cm, headheight=15pt]{geometry}
\usepackage{fancyhdr}
\usepackage{enumitem}	
\usepackage{wrapfig}
\usepackage{float}
\usepackage{multicol}

\usepackage[p,osf]{scholax}

\renewcommand*\contentsname{Table of Contents}
\renewcommand{\headrulewidth}{0pt}
\pagestyle{fancy}
\fancyhf{}
\fancyfoot[R]{$\thepage$}
\setlength{\parindent}{0cm}
\setlength{\headheight}{17pt}
\hfuzz=9pt

% Figures
\usepackage{svg}

% Utility Management
\usepackage{color}
\usepackage{colortbl}
\usepackage{xcolor}
\usepackage{xpatch}
\usepackage{xparse}

\definecolor{gBlue}{HTML}{7daea3}
\definecolor{gOrange}{HTML}{e78a4e}
\definecolor{gGreen}{HTML}{a9b665}
\definecolor{gPurple}{HTML}{d3869b}

\definecolor{links}{HTML}{1c73a5}
\definecolor{bar}{HTML}{584AA8}

% Math Packages
\usepackage{mathtools, amsmath, amsthm, thmtools, amssymb, physics}
\usepackage[scaled=1.075,ncf,vvarbb]{newtxmath}

\newcommand\B{\mathbb{C}}
\newcommand\C{\mathbb{C}}
\newcommand\R{\mathbb{R}}
\newcommand\Q{\mathbb{Q}}
\newcommand\N{\mathbb{N}}
\newcommand\Z{\mathbb{Z}}

\DeclareMathOperator{\lcm}{lcm}

% Probability Theory
\newcommand\Prob[1]{\mathbb{P}\qty(#1)}
\newcommand\Var[1]{\text{Var}\qty(#1)}
\newcommand\Exp[1]{\mathbb{E}\qty[#1]}

% Analysis
\newcommand\ball[1]{\B\qty(#1)}
\newcommand\conj[1]{\overline{#1}}
\DeclareMathOperator{\Arg}{Arg}
\DeclareMathOperator{\cis}{cis}

% Linear Algebra
\DeclareMathOperator{\dom}{dom}
\DeclareMathOperator{\range}{range}
\DeclareMathOperator{\spann}{span}
\DeclareMathOperator{\nullity}{nullity}

% TIKZ
\usepackage{tikz}
\usepackage{pgfplots}
\usetikzlibrary{arrows.meta}
\usetikzlibrary{math}
\usetikzlibrary{cd}

% Boxes and Theorems
\usepackage[most]{tcolorbox}
\tcbuselibrary{skins}
\tcbuselibrary{breakable}
\tcbuselibrary{theorems}

\newtheoremstyle{default}{0pt}{0pt}{}{}{\bfseries}{\normalfont.}{0.5em}{}
\theoremstyle{default}

\renewcommand*{\proofname}{\textit{\textbf{Proof.}}}
\renewcommand*{\qedsymbol}{$\blacksquare$}
\tcolorboxenvironment{proof}{
	breakable,
	coltitle = black,
	colback = white,
	frame hidden,
	boxrule = 0pt,
	boxsep = 0pt,
	borderline west={3pt}{0pt}{bar},
	% borderline west={3pt}{0pt}{gPurple},
	sharp corners = all,
	enhanced,
}

\newtheorem{theorem}{Theorem}[section]{\bfseries}{}
\tcolorboxenvironment{theorem}{
	breakable,
	enhanced,
	boxrule = 0pt,
	frame hidden,
	coltitle = black,
	colback = blue!7,
	% colback = gBlue!30,
	left = 0.5em,
	sharp corners = all,
}

\newtheorem{corollary}{Corollary}[section]{\bfseries}{}
\tcolorboxenvironment{corollary}{
	breakable,
	enhanced,
	boxrule = 0pt,
	frame hidden,
	coltitle = black,
	colback = white!0,
	left = 0.5em,
	sharp corners = all,
}

\newtheorem{lemma}{Lemma}[section]{\bfseries}{}
\tcolorboxenvironment{lemma}{
	breakable,
	enhanced,
	boxrule = 0pt,
	frame hidden,
	coltitle = black,
	colback = green!7,
	left = 0.5em,
	sharp corners = all,
}

\newtheorem{definition}{Definition}[section]{\bfseries}{}
\tcolorboxenvironment{definition}{
	breakable,
	coltitle = black,
	colback = white,
	frame hidden,
	boxsep = 0pt,
	boxrule = 0pt,
	borderline west = {3pt}{0pt}{orange},
	% borderline west = {3pt}{0pt}{gOrange},
	sharp corners = all,
	enhanced,
}

\newtheorem{example}{Example}[section]{\bfseries}{}
\tcolorboxenvironment{example}{
	% title = \textbf{Example},
	% detach title,
	% before upper = {\tcbtitle\quad},
	breakable,
	coltitle = black,
	colback = white,
	frame hidden,
	boxrule = 0pt,
	boxsep = 0pt,
	borderline west={3pt}{0pt}{green!70!black},
	% borderline west={3pt}{0pt}{gGreen},
	sharp corners = all,
	enhanced,
}

\newtheoremstyle{remark}{0pt}{4pt}{}{}{\bfseries\itshape}{\normalfont.}{0.5em}{}
\theoremstyle{remark}
\newtheorem*{remark}{Remark}


% TColorBoxes
\newtcolorbox{week}{
	colback = black,
	coltext = white,
	fontupper = {\large\bfseries},
	width = 1.2\paperwidth,
	size = fbox,
	halign upper = center,
	center
}

\newcommand{\banner}[2]{
    \pagebreak
    \begin{week}
   		\section*{#1}
    \end{week}
    \addcontentsline{toc}{section}{#1}
    \addtocounter{section}{1}
    \setcounter{subsection}{0}
}

% Hyperref
\usepackage{hyperref}
\hypersetup{
	colorlinks=true,
	linktoc=all,
	linkcolor=links,
	bookmarksopen=true
}

% Error Handling
\PackageWarningNoLine{ExtSizes}{It is better to use one of the extsizes 
                          classes,^^J if you can}

\usepackage{svg}
\svgsetup{inkscapeexe=inkscape, inkscapearea=drawing, inkscapeversion=1}
\svgpath{{figures/}}

\fancyhead[R]{\textbf{Homework \#$3$}}
\fancyhead[L]{Eli Griffiths}
\renewcommand{\headrulewidth}{1pt}
\setlength\parindent{0pt}

\begin{document}
\section*{Problem 1}

The possible values of $X$ are
\[
	X = \qty{-2, -1, 0, 1, 2, 4}
\]
with associated probabilities
\begin{align*}
		\Prob{X = -2} = \frac{\binom{8}{2}}{\binom{14}{2}} &= \frac{28}{91} \\
		\Prob{X = -1} = \frac{\binom{8}{1} \binom{2}{1}}{\binom{14}{2}} &= \frac{16}{91} \\
		\Prob{X =  0} = \frac{\binom{2}{2}}{\binom{14}{2}} &= \frac{1}{91} \\
		\Prob{X =  1} = \frac{\binom{4}{1} \binom{8}{1}}{\binom{14}{2}} &= \frac{32}{91} \\
		\Prob{X =  2} = \frac{\binom{4}{1} \binom{2}{1}}{\binom{14}{12}} &= \frac{8}{91} \\
		\Prob{X =  4} = \frac{\binom{4}{2}}{\binom{14}{2}} &= \frac{6}{91}
.\end{align*}

\section*{Problem 2}
The possible values for $X$ are
\begin{align*}
	X &= \qty{n - 2i : i = 0, 1, 2, \ldots, n }
.\end{align*}


\section*{Problem 3}
{
	\renewcommand{\arraystretch}{1.5}
\[
	\begin{array}{c|c}
		i  & \Prob{X = i} \\\hline
		1  & \frac{1}{2} \\\hline
		2  & \frac{5}{10} \cdot \frac{5}{9} = \frac{5}{18} \\\hline
		3  & \frac{5}{10} \cdot \frac{4}{9} \cdot \frac{5}{8} = \frac{5}{36} \\\hline
		4  & \frac{5}{10} \cdot \frac{4}{9} \cdot \frac{3}{8} \cdot \frac{5}{7} = \frac{5}{84} \\\hline
		5  & \frac{5}{10} \cdot \frac{4}{9} \cdot \frac{3}{8} \cdot\frac{2}{7} \cdot \frac{5}{6} = \frac{5}{252} \\\hline
		6  & \frac{5}{10} \cdot \frac{4}{9} \cdot \frac{3}{8} \cdot \frac{2}{7} \cdot \frac{1}{6} = \frac{1}{252} \\\hline
		7  & 0 \\\hline
		8  & 0 \\\hline
		9  & 0 \\\hline
		10 & 0
	\end{array}
.\]
}

\section*{Problem 4}

\begin{figure}[h!]
	\centering
	\includesvg[width=0.6\columnwidth,inkscapelatex=false]{figures/pmassfunction.svg}
\end{figure}

\section*{Problem 5}
Let $(n_1, n_2, n_3, n_4, n_5)$ denote an order of the players where $n_1$ is the player with the largest number and $n_5$ the player with the smallest.
\[
	\everymath{\displaystyle}
	\renewcommand\arraystretch{3}
	\begin{array}{rll}
		\Prob{X = 0} =& \Prob{2\text{ beats }1} &= \frac{(1+2+3+4)\cdot 3!}{5!} = \frac{1}{2} \\
		\Prob{X = 1} =& \Prob{(3,1,2,\ldots)} &= \frac{1}{3}\cdot\frac{1}{2} = \frac{1}{6} \\
		\Prob{X = 2} =& \Prob{(4,1,\ldots)} &= \frac{1}{4}\cdot\frac{1}{3} = \frac{1}{12} \\
		\Prob{X = 3} =& \Prob{(5,1,\ldots)} &= \frac{1}{5}\cdot\frac{1}{4} = \frac{1}{20} \\
		\Prob{X = 5} =& \Prob{(1,\ldots)} &= \frac{1}{5}
	\end{array}
.\]

\section*{Problem 6}
\[
	\Prob{X = i | X > 0} = \frac{\Prob{X = i}}{\Prob{X > 0}}
.\]
The probability of getting a positive amount is
\[
	\Prob{X > 0} =  \sum_{n=1}^{3} \Prob{X = i} = p(1) + p(2) + p(3) = \frac{13}{55} + \frac{1}{11} + \frac{1}{165} = \frac{1}{3}
.\]
Thefore
\[
	\everymath{\displaystyle}
	\renewcommand\arraystretch{3}
	\begin{array}{rll}
		\Prob{X = 1 | X > 0} =& \frac{\frac{13}{55}}{\frac{1}{3}} &= \frac{39}{55} \\
		\Prob{X = 2 | X > 0} =& \frac{\frac{1}{11}}{\frac{1}{3}} &= \frac{3}{11} \\
		\Prob{X = 3 | X > 0} =& \frac{\frac{1}{165}}{\frac{1}{3}} &= \frac{1}{55} \\
	\end{array}
\]

\section*{Problem 7}
\[
	\everymath{\displaystyle}
	\begin{array}{lcccl}
		\Prob{X = 0} &=& \binom{6}{0} \qty(\frac{1}{2})^6 &=& \frac{1}{64} \\
		\Prob{X = 1} &=& \binom{6}{1} \qty(\frac{1}{2})^6 &=& \frac{6}{64} \\
		\Prob{X = 2} &=& \binom{6}{2} \qty(\frac{1}{2})^6 &=& \frac{15}{64} \\
		\Prob{X = 3} &=& \binom{6}{3} \qty(\frac{1}{2})^6 &=& \frac{20}{64} \\
		\Prob{X = 4} &=& \binom{6}{4} \qty(\frac{1}{2})^6 &=& \frac{15}{64} \\
		\Prob{X = 5} &=& \binom{6}{5} \qty(\frac{1}{2})^6 &=& \frac{6}{64} \\
		\Prob{X = 6} &=& \binom{6}{6} \qty(\frac{1}{2})^6 &=& \frac{1}{64} \\
	\end{array}
.\]
By inspection $X = 3$ is the most likely outcome.

\section*{Problem 8}
\[
	p(x) = \Prob{X = x} = \binom{3}{x} (0.7)^x (0.3)^{3-x}
.\]

\section*{Problem 9}
Let $X$ denote the number of $6$'s rolled by $3$ fair dice. Note that
\[
	X \sim \text{Binom}\qty(3, \frac{1}{6})
.\]
The probability that at most one $6$ is rolled is
\[
	\Prob{X \leq 1} = \Prob{X = 0} + \Prob{X = 1}
.\]
Since $X$ follows a binomial distribution,
\begin{align*}
	\Prob{X \leq 1} &= \Prob{X = 0} + \Prob{X = 1} \\
									&= \binom{3}{0} \qty(\frac{1}{6})^0 \qty(\frac{5}{6})^3 + \binom{3}{1} \qty(\frac{1}{6})^1 \qty(\frac{5}{6})^2 \\
									&= \qty(\frac{5}{6})^3 + \frac{1}{2} \qty(\frac{5}{6})^2 \\
									&= \frac{25}{27}
.\end{align*}

\section*{Problem 10}
Let $X$ denote the number of multiple choice questions the student gets right. Note that
\[
	X \sim \text{Binom}\qty(5,\frac{1}{3})
.\]
The probability that the student gets four or more questions by guessing is
\[
	\Prob{X \geq 4} = \Prob{X = 4} + \Prob{X = 5}
.\]
Since $X$ follows a binomial distribution,
\begin{align*}
	\Prob{X \leq 4} &= \Prob{X = 4} + \Prob{X = 5} \\
									&= \binom{5}{4} \qty(\frac{1}{3})^4 \qty(\frac{2}{3})^1 + \binom{5}{5} \qty(\frac{1}{3})^5 \qty(\frac{2}{3})^0 \\
									&= \frac{10}{243} + \frac{1}{243} \\
									&= \frac{11}{243}
.\end{align*}

\end{document}
