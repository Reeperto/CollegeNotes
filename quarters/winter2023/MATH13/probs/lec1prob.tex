\documentclass[12pt]{extarticle}

% Document Layout and Font
\usepackage{subfiles}
\usepackage[margin=2cm, headheight=15pt]{geometry}
\usepackage{fancyhdr}
\usepackage{enumitem}	
\usepackage{wrapfig}
\usepackage{float}
\usepackage{multicol}

\usepackage[p,osf]{scholax}

\renewcommand*\contentsname{Table of Contents}
\renewcommand{\headrulewidth}{0pt}
\pagestyle{fancy}
\fancyhf{}
\fancyfoot[R]{$\thepage$}
\setlength{\parindent}{0cm}
\setlength{\headheight}{17pt}
\hfuzz=9pt

% Figures
\usepackage{svg}

% Utility Management
\usepackage{color}
\usepackage{colortbl}
\usepackage{xcolor}
\usepackage{xpatch}
\usepackage{xparse}

\definecolor{gBlue}{HTML}{7daea3}
\definecolor{gOrange}{HTML}{e78a4e}
\definecolor{gGreen}{HTML}{a9b665}
\definecolor{gPurple}{HTML}{d3869b}

\definecolor{links}{HTML}{1c73a5}
\definecolor{bar}{HTML}{584AA8}

% Math Packages
\usepackage{mathtools, amsmath, amsthm, thmtools, amssymb, physics}
\usepackage[scaled=1.075,ncf,vvarbb]{newtxmath}

\newcommand\B{\mathbb{C}}
\newcommand\C{\mathbb{C}}
\newcommand\R{\mathbb{R}}
\newcommand\Q{\mathbb{Q}}
\newcommand\N{\mathbb{N}}
\newcommand\Z{\mathbb{Z}}

\DeclareMathOperator{\lcm}{lcm}

% Probability Theory
\newcommand\Prob[1]{\mathbb{P}\qty(#1)}
\newcommand\Var[1]{\text{Var}\qty(#1)}
\newcommand\Exp[1]{\mathbb{E}\qty[#1]}

% Analysis
\newcommand\ball[1]{\B\qty(#1)}
\newcommand\conj[1]{\overline{#1}}
\DeclareMathOperator{\Arg}{Arg}
\DeclareMathOperator{\cis}{cis}

% Linear Algebra
\DeclareMathOperator{\dom}{dom}
\DeclareMathOperator{\range}{range}
\DeclareMathOperator{\spann}{span}
\DeclareMathOperator{\nullity}{nullity}

% TIKZ
\usepackage{tikz}
\usepackage{pgfplots}
\usetikzlibrary{arrows.meta}
\usetikzlibrary{math}
\usetikzlibrary{cd}

% Boxes and Theorems
\usepackage[most]{tcolorbox}
\tcbuselibrary{skins}
\tcbuselibrary{breakable}
\tcbuselibrary{theorems}

\newtheoremstyle{default}{0pt}{0pt}{}{}{\bfseries}{\normalfont.}{0.5em}{}
\theoremstyle{default}

\renewcommand*{\proofname}{\textit{\textbf{Proof.}}}
\renewcommand*{\qedsymbol}{$\blacksquare$}
\tcolorboxenvironment{proof}{
	breakable,
	coltitle = black,
	colback = white,
	frame hidden,
	boxrule = 0pt,
	boxsep = 0pt,
	borderline west={3pt}{0pt}{bar},
	% borderline west={3pt}{0pt}{gPurple},
	sharp corners = all,
	enhanced,
}

\newtheorem{theorem}{Theorem}[section]{\bfseries}{}
\tcolorboxenvironment{theorem}{
	breakable,
	enhanced,
	boxrule = 0pt,
	frame hidden,
	coltitle = black,
	colback = blue!7,
	% colback = gBlue!30,
	left = 0.5em,
	sharp corners = all,
}

\newtheorem{corollary}{Corollary}[section]{\bfseries}{}
\tcolorboxenvironment{corollary}{
	breakable,
	enhanced,
	boxrule = 0pt,
	frame hidden,
	coltitle = black,
	colback = white!0,
	left = 0.5em,
	sharp corners = all,
}

\newtheorem{lemma}{Lemma}[section]{\bfseries}{}
\tcolorboxenvironment{lemma}{
	breakable,
	enhanced,
	boxrule = 0pt,
	frame hidden,
	coltitle = black,
	colback = green!7,
	left = 0.5em,
	sharp corners = all,
}

\newtheorem{definition}{Definition}[section]{\bfseries}{}
\tcolorboxenvironment{definition}{
	breakable,
	coltitle = black,
	colback = white,
	frame hidden,
	boxsep = 0pt,
	boxrule = 0pt,
	borderline west = {3pt}{0pt}{orange},
	% borderline west = {3pt}{0pt}{gOrange},
	sharp corners = all,
	enhanced,
}

\newtheorem{example}{Example}[section]{\bfseries}{}
\tcolorboxenvironment{example}{
	% title = \textbf{Example},
	% detach title,
	% before upper = {\tcbtitle\quad},
	breakable,
	coltitle = black,
	colback = white,
	frame hidden,
	boxrule = 0pt,
	boxsep = 0pt,
	borderline west={3pt}{0pt}{green!70!black},
	% borderline west={3pt}{0pt}{gGreen},
	sharp corners = all,
	enhanced,
}

\newtheoremstyle{remark}{0pt}{4pt}{}{}{\bfseries\itshape}{\normalfont.}{0.5em}{}
\theoremstyle{remark}
\newtheorem*{remark}{Remark}


% TColorBoxes
\newtcolorbox{week}{
	colback = black,
	coltext = white,
	fontupper = {\large\bfseries},
	width = 1.2\paperwidth,
	size = fbox,
	halign upper = center,
	center
}

\newcommand{\banner}[2]{
    \pagebreak
    \begin{week}
   		\section*{#1}
    \end{week}
    \addcontentsline{toc}{section}{#1}
    \addtocounter{section}{1}
    \setcounter{subsection}{0}
}

% Hyperref
\usepackage{hyperref}
\hypersetup{
	colorlinks=true,
	linktoc=all,
	linkcolor=links,
	bookmarksopen=true
}

% Error Handling
\PackageWarningNoLine{ExtSizes}{It is better to use one of the extsizes 
                          classes,^^J if you can}


\begin{document}

\section*{Lecture Problem}

With the given definition

\begin{definition}
	We say a mapping $f: A \rightarrow B$ is well-defined if
	\begin{enumerate}
		\item[(a)] $\forall a \in A, f(a) \in B$
		\item[(b)] $\forall a_1, a_2 \in A, a_1 = a_2 \implies f(a_1) = f(a_2)$
	\end{enumerate}
\end{definition}

\subsection*{Part A}

\begin{enumerate}
	\item Give and example of $f$ that doesn't satisfy (a)
	\item Give and example of $f$ that doesn't satisfy (b)
	\item Give and example of $f$ that satisfies (a) and (b) and prove it
\end{enumerate}

\subsubsection*{Part 1}

Let $f$ be defined as 

\[
	f: [0,2] \rightarrow [0,1] : x \mapsto x^2
.\]

In this case $f$ doesn't satisfy (a) since $2 \in [0,2]$, but $f(2) = 4 \notin [0,1]$.

\subsubsection*{Part 2}
Let $f$
\[
	f: [0,1] \rightarrow [-1,1] : x \mapsto \pm \sqrt{x}
.\]

In this case $f$ doesn't satisfy (b) since $f(1) = 1$ and $f(1) = -1$.

\subsubsection*{Part 3}

Let $f$ be defined as

\[
	f: A \rightarrow B
.\]

Where $A = \qty{0,1}$ and $B = \qty{2,3}$ where
\begin{align*}
	0 \mapsto 2 \\
	1 \mapsto 3
.\end{align*}

\begin{proof}
	Proof that $f$ satisfies both conditions (a) and (b). Note that $\forall a \in A, f(a) \in B$ since $f(0) = 2 \in B$ and $f(1) = 3 \in B$. Let $a_1, a_2 \in A$ and assume $a_1 = a_2$. If $a_1 = a_2 = 0$, then $f(a_1) = f(a_2) = 2$ and if $a_1 = a_2 = 1$, then $f(a_1) = f(a_2) = 3$. Therefore both (a) and (b) are true for $f$.
\end{proof}

\subsection*{Part B}

Let
\[
	f: \mathbb{Q} \rightarrow \mathbb{Q} : \frac{a}{b} \mapsto a
.\]

Show that $f$ is not well-defined

\subsubsection*{Solution}

\begin{proof}
	Let $q_1 = \frac{1}{2}$ and $q_2 = \frac{2}{4}$. Then $q_1 = q_2$ but $f(q_1) = 1 \neq 2 =f(q_2)$, hence $f$ is not well-defined.
\end{proof}

\subsection*{Part C}

Let
\[
	f: \mathbb{Q} \rightarrow \mathbb{Q} : \frac{a}{b} \mapsto \qty(\frac{a}{b})^2
\]

Show that $f$ is well-defined.

\begin{proof}
	Let $q = \frac{a}{b} \in \mathbb{Q}$. Then $f(q) = f(\frac{a}{b}) = \frac{a^2}{b^2} \in \mathbb{Q}$. Let $q_1 = \frac{a_1}{b_1} \in \mathbb{Q}$ and $q_2 = \frac{a_2}{b_2} \in \mathbb{Q}$. Assume $q_1 = q_2$. Then
	\begin{align*}
		q_1 &= q_2 \\
		\frac{a_1}{b_1} &= \frac{a_2}{b_2} \\
		\qty(\frac{a_1}{b_1})^2 &= \qty(\frac{a_2}{b_2})^2 \\
		f(q_1) &= f(q_2)
	.\end{align*}
\end{proof}

\end{document}
