\documentclass[12pt]{extarticle}

% Document Layout and Font
\usepackage{subfiles}
\usepackage[margin=2cm, headheight=15pt]{geometry}
\usepackage{fancyhdr}
\usepackage{enumitem}	
\usepackage{wrapfig}
\usepackage{float}
\usepackage{multicol}

\usepackage[p,osf]{scholax}

\renewcommand*\contentsname{Table of Contents}
\renewcommand{\headrulewidth}{0pt}
\pagestyle{fancy}
\fancyhf{}
\fancyfoot[R]{$\thepage$}
\setlength{\parindent}{0cm}
\setlength{\headheight}{17pt}
\hfuzz=9pt

% Figures
\usepackage{svg}

% Utility Management
\usepackage{color}
\usepackage{colortbl}
\usepackage{xcolor}
\usepackage{xpatch}
\usepackage{xparse}

\definecolor{gBlue}{HTML}{7daea3}
\definecolor{gOrange}{HTML}{e78a4e}
\definecolor{gGreen}{HTML}{a9b665}
\definecolor{gPurple}{HTML}{d3869b}

\definecolor{links}{HTML}{1c73a5}
\definecolor{bar}{HTML}{584AA8}

% Math Packages
\usepackage{mathtools, amsmath, amsthm, thmtools, amssymb, physics}
\usepackage[scaled=1.075,ncf,vvarbb]{newtxmath}

\newcommand\B{\mathbb{C}}
\newcommand\C{\mathbb{C}}
\newcommand\R{\mathbb{R}}
\newcommand\Q{\mathbb{Q}}
\newcommand\N{\mathbb{N}}
\newcommand\Z{\mathbb{Z}}

\DeclareMathOperator{\lcm}{lcm}

% Probability Theory
\newcommand\Prob[1]{\mathbb{P}\qty(#1)}
\newcommand\Var[1]{\text{Var}\qty(#1)}
\newcommand\Exp[1]{\mathbb{E}\qty[#1]}

% Analysis
\newcommand\ball[1]{\B\qty(#1)}
\newcommand\conj[1]{\overline{#1}}
\DeclareMathOperator{\Arg}{Arg}
\DeclareMathOperator{\cis}{cis}

% Linear Algebra
\DeclareMathOperator{\dom}{dom}
\DeclareMathOperator{\range}{range}
\DeclareMathOperator{\spann}{span}
\DeclareMathOperator{\nullity}{nullity}

% TIKZ
\usepackage{tikz}
\usepackage{pgfplots}
\usetikzlibrary{arrows.meta}
\usetikzlibrary{math}
\usetikzlibrary{cd}

% Boxes and Theorems
\usepackage[most]{tcolorbox}
\tcbuselibrary{skins}
\tcbuselibrary{breakable}
\tcbuselibrary{theorems}

\newtheoremstyle{default}{0pt}{0pt}{}{}{\bfseries}{\normalfont.}{0.5em}{}
\theoremstyle{default}

\renewcommand*{\proofname}{\textit{\textbf{Proof.}}}
\renewcommand*{\qedsymbol}{$\blacksquare$}
\tcolorboxenvironment{proof}{
	breakable,
	coltitle = black,
	colback = white,
	frame hidden,
	boxrule = 0pt,
	boxsep = 0pt,
	borderline west={3pt}{0pt}{bar},
	% borderline west={3pt}{0pt}{gPurple},
	sharp corners = all,
	enhanced,
}

\newtheorem{theorem}{Theorem}[section]{\bfseries}{}
\tcolorboxenvironment{theorem}{
	breakable,
	enhanced,
	boxrule = 0pt,
	frame hidden,
	coltitle = black,
	colback = blue!7,
	% colback = gBlue!30,
	left = 0.5em,
	sharp corners = all,
}

\newtheorem{corollary}{Corollary}[section]{\bfseries}{}
\tcolorboxenvironment{corollary}{
	breakable,
	enhanced,
	boxrule = 0pt,
	frame hidden,
	coltitle = black,
	colback = white!0,
	left = 0.5em,
	sharp corners = all,
}

\newtheorem{lemma}{Lemma}[section]{\bfseries}{}
\tcolorboxenvironment{lemma}{
	breakable,
	enhanced,
	boxrule = 0pt,
	frame hidden,
	coltitle = black,
	colback = green!7,
	left = 0.5em,
	sharp corners = all,
}

\newtheorem{definition}{Definition}[section]{\bfseries}{}
\tcolorboxenvironment{definition}{
	breakable,
	coltitle = black,
	colback = white,
	frame hidden,
	boxsep = 0pt,
	boxrule = 0pt,
	borderline west = {3pt}{0pt}{orange},
	% borderline west = {3pt}{0pt}{gOrange},
	sharp corners = all,
	enhanced,
}

\newtheorem{example}{Example}[section]{\bfseries}{}
\tcolorboxenvironment{example}{
	% title = \textbf{Example},
	% detach title,
	% before upper = {\tcbtitle\quad},
	breakable,
	coltitle = black,
	colback = white,
	frame hidden,
	boxrule = 0pt,
	boxsep = 0pt,
	borderline west={3pt}{0pt}{green!70!black},
	% borderline west={3pt}{0pt}{gGreen},
	sharp corners = all,
	enhanced,
}

\newtheoremstyle{remark}{0pt}{4pt}{}{}{\bfseries\itshape}{\normalfont.}{0.5em}{}
\theoremstyle{remark}
\newtheorem*{remark}{Remark}


% TColorBoxes
\newtcolorbox{week}{
	colback = black,
	coltext = white,
	fontupper = {\large\bfseries},
	width = 1.2\paperwidth,
	size = fbox,
	halign upper = center,
	center
}

\newcommand{\banner}[2]{
    \pagebreak
    \begin{week}
   		\section*{#1}
    \end{week}
    \addcontentsline{toc}{section}{#1}
    \addtocounter{section}{1}
    \setcounter{subsection}{0}
}

% Hyperref
\usepackage{hyperref}
\hypersetup{
	colorlinks=true,
	linktoc=all,
	linkcolor=links,
	bookmarksopen=true
}

% Error Handling
\PackageWarningNoLine{ExtSizes}{It is better to use one of the extsizes 
                          classes,^^J if you can}


\fancyhead[R]{Homework \#$6$}
\fancyhead[L]{Eli Griffiths}
\renewcommand{\headrulewidth}{1pt}
\setlength\parindent{0pt}

\begin{document}
\DeclarePairedDelimiter\bangle\langle\rangle

% Section 10: 2, 4, 12, 15, 27, 28, 30, 32, 35, 36

\section*{Problem 2}
\begin{align*}
	4 \mathbb{Z} + 0 &= \qty{\ldots, -8, -4, 0, 4, 8, \ldots} \\
	4 \mathbb{Z} + 2 &= \qty{\ldots, -6, -2, 2, 4, 6, \ldots}
.\end{align*}

\section*{Problem 4}
\begin{align*}
	\bangle{2} + 0 &= \qty{0, 2, 4, 6, 8, 10} \\
	\bangle{2} + 1 &= \qty{1, 3, 5, 7, 9, 11}
.\end{align*}

\section*{Problem 12}
The cosets of $\bangle{3}$ in $\mathbb{Z}_{24}$ are
\begin{align*}
	\bangle{3} + 0 &= \qty{0,3,6,9,12,15,18,21} \\
	\bangle{3} + 1 &= \qty{1,4,7,10,13,16,19,22} \\
	\bangle{3} + 2 &= \qty{2,5,8,11,14,17,20,23}
.\end{align*}

Therefore
\[
	\qty[\mathbb{Z}_{24} : \bangle{3}] = 3
.\]

\section*{Problem 15}
Rewriting $\sigma$ as disjoint cycles gives
\[
	\sigma = (1, 2, 3, 5, 4)
.\]
meaning that $|\sigma| = 5$, hence
\[
	\qty[S_5 : \bangle{\sigma}] = \frac{|S_5|}{|\sigma|} = \frac{5!}{5} = 24
.\]

\section*{Problem 27}
\begin{proof}
	Let $G$ be a group, $H$ be a subgroup of $G$, and $g \in G$. Define the mapping $\phi : H \to Hg$ where $\phi(h) = hg$. Let $a,b \in H$ and assume $\phi(a) = \phi(b)$. Then $ag = bg$ which be the cancellation law implies $a = b$, hence $\phi$ is injective. Let $y \in Hg$. Then there exists some $x \in H$ such that $y = xg$. $\phi(x) = g$, meaning $\phi$ is onto. Therefore $\phi$ is onto and one-to-one.
\end{proof}


\section*{Problem 28}
\begin{proof}
	Let $H$ be a subgroup of a group $G$ such that $g^{-1} h g \in H$ for all $g \in G$ and $h \in H$. Let $g \in G$ and consider its cosets. Let $x \in Hg$. Then $\exists h_1 \in H$ such that $x = hg = g g^{-1} h g$. Let $r = g^{-1}$. Then $x = g r^{-1} h_1 r$, meaning $x \in H$ and therefore $x \in gH$. Therefore $Hg \subseteq gH$. Let $x \in gH$. Then $\exists h_2 \in H$ such that $x = g h_2 = g h_2 g^{-1} g = r^{-1} h_2 r g$, meaning $x \in Hg$. Therefore $gH \subseteq Hg$. Hence $gH = Hg$ for any $g$, meaning all left cosets and right cosets of $H$ are the same.
\end{proof}

\section*{Problem 30}
Consider the subgroup $\qty{\rho_0, \mu_1}$ of $S_3$. It follows that
\[
	\rho_1 \qty{\rho_0, \mu_1} = \mu_3 \qty{\rho_0, \mu_1} = \qty{\rho_1, \mu_2}
.\]
However,
\[
	\qty{\rho_0, \mu_1}\rho_1 = \qty{\rho_1, \mu_2} \neq \qty{\rho_2, \mu_3} = \qty{\rho_0, \mu_1} \mu_3
.\]

\section*{Problem 32}
It is true. Since $H \leq G$, $\qty{h^{-1} : h \in H} = H$. Therefore 
\[
	Ha^{-1} = \qty{ha^{-1} : h \in H} = \qty{h^{-1} a^{-1} : h \in H} = \qty{(ah)^{-1} : h \in H}
.\]
That is $Ha^{-1}$ contains all the inverses of $aH$. If $aH = bH$, then the inverse of all the elements in both cosets are the same, meaning $Ha^{-1} = Hb^{-1}$.

\section*{Problem 35}
\begin{proof}
	Let $G$ be a group and $H \leq G$. Define the mapping $\phi$ where $\phi(aH) = Ha^{-1}$ for all $a \in G$. First check if $\phi$ is well defined. Assume that $aH = bH$. Then $a^{-1} b H = H$ meaning $a^{-1} b \in H$. Note that $a^{-1} b = a^{-1} \qty(b^{-1})^{-1} \in H$. Since $a^{-1} \qty(b^{-1})^{-1} \in H$, $H a^{-1} \qty{b^{-1}}^{-1} = H$ implying $Ha^{-1} = Hb^{-1}$. Therefore $\phi$ is well defined. Let $x,y \in G$ and assume $\phi(xH) = \phi(yH)$. Then $Hx^{-1} = Hy^{-1}$, meaning there is some $h \in H$ such that $x^{-1} = hy^{-1}$. Therefore $h = x^{-1} y$ and $h^{-1} = y^{-1} x$. Since $H \leq G$, $h^{-1} \in H$ and therefore $y^{-1} x \in H$. Therefore $y^{-1} x H = H$ meaning $xH = yH$, hence $\phi$ is injective. Let $Ha$ be a right coset of $G$. Note that $\phi(a^{-1} H) = Ha$, hence $\phi$ is surjective. Therefore $\phi$ is a bijection between the left and right cosets of $H$, meaning the left and right cosets of $H$ are equinumerous.
\end{proof}

\section*{Problem 36}
\begin{proof}
	Let $G$ be an abelian group of order $2n$ where $n$ is an odd number. Suppose that there are two elements $a$ and $b$ of $G$ both with order 2. That is $a^2 = e$ and $b^2 = e$. Note that $(ab)^2 = a^2 b^2 = e$. $ab \neq e$ since $a$ is its own inverse and $b \neq a$. Therefore $ab$ also has an order of 2. It also follows that $\qty{e, a, b, ab}$ is a subgroup with order 4. Since $n$ is odd, $\exists k \in \mathbb{Z}$ such that $n = 2k +1$, therefore $|G| = 4k + 2$. Therefore $4 \nmid |G|$, meaning the subgroup $\qty{e,a,b,ab}$ can not exist, cotnradicting the assumption that there are 2 elements of order 2. Since the group has an even order, there must exist some element that is it's own inverse, meaning there does exist a unique element of order 2.
\end{proof}

\end{document}
