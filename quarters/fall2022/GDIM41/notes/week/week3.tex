\documentclass{standalone}

\begin{document}

\subsection{Game Representation}

Some basic elements of representation are concept such as perspective, dimensionality, type of play space, off-space, scroll direction, and exploration limitations.

\subsection{The M.D.A. Framework}

\centerline{Mechanics $\implies$ Dynamics $\implies$ Aesthetics}
\textbf{Mechanics} is the actual components of the game that define what players can do; \textit{the rules
}. \textbf{Dynamics} is the "run time" behaviour of the game. Aesthetics is the final experiential goal the designers have in mind for their players.

\subsubsection{Types of Aesthetics/Fun}

\begin{itemize}
				\tightlist
				\item \textbf{Sensation} - Games as a sensory or pleasurable environment
				\begin{itemize}
								\item Can be real world tactile sensations or satisfying experience with physics simulations, etc.
				\end{itemize}
				\item \textbf{Fantasy} - Games as make believe
				\item \textbf{Narrative} - Games as a drama or story
				\item \textbf{Challenge} - Games as an obstacle course
				\item \textbf{Fellowship} - Games as a social framework
				\item \textbf{Discovery} - Games as an uncharted territory to be explored and understood
				\item \textbf{Expression} - Games as self-discovery
				\item \textbf{Submission} - Games as a pastime
\end{itemize}

\end{document}
