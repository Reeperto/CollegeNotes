\documentclass[12pt,titlepage]{extarticle}
% Document Layout and Font
\usepackage{subfiles}
\usepackage[margin=2cm, headheight=15pt]{geometry}
\usepackage{fancyhdr}
\usepackage{enumitem}	
\usepackage{wrapfig}
\usepackage{float}
\usepackage{multicol}

\usepackage[p,osf]{scholax}

\renewcommand*\contentsname{Table of Contents}
\renewcommand{\headrulewidth}{0pt}
\pagestyle{fancy}
\fancyhf{}
\fancyfoot[R]{$\thepage$}
\setlength{\parindent}{0cm}
\setlength{\headheight}{17pt}
\hfuzz=9pt

% Figures
\usepackage{svg}

% Utility Management
\usepackage{color}
\usepackage{colortbl}
\usepackage{xcolor}
\usepackage{xpatch}
\usepackage{xparse}

\definecolor{gBlue}{HTML}{7daea3}
\definecolor{gOrange}{HTML}{e78a4e}
\definecolor{gGreen}{HTML}{a9b665}
\definecolor{gPurple}{HTML}{d3869b}

\definecolor{links}{HTML}{1c73a5}
\definecolor{bar}{HTML}{584AA8}

% Math Packages
\usepackage{mathtools, amsmath, amsthm, thmtools, amssymb, physics}
\usepackage[scaled=1.075,ncf,vvarbb]{newtxmath}

\newcommand\B{\mathbb{C}}
\newcommand\C{\mathbb{C}}
\newcommand\R{\mathbb{R}}
\newcommand\Q{\mathbb{Q}}
\newcommand\N{\mathbb{N}}
\newcommand\Z{\mathbb{Z}}

\DeclareMathOperator{\lcm}{lcm}

% Probability Theory
\newcommand\Prob[1]{\mathbb{P}\qty(#1)}
\newcommand\Var[1]{\text{Var}\qty(#1)}
\newcommand\Exp[1]{\mathbb{E}\qty[#1]}

% Analysis
\newcommand\ball[1]{\B\qty(#1)}
\newcommand\conj[1]{\overline{#1}}
\DeclareMathOperator{\Arg}{Arg}
\DeclareMathOperator{\cis}{cis}

% Linear Algebra
\DeclareMathOperator{\dom}{dom}
\DeclareMathOperator{\range}{range}
\DeclareMathOperator{\spann}{span}
\DeclareMathOperator{\nullity}{nullity}

% TIKZ
\usepackage{tikz}
\usepackage{pgfplots}
\usetikzlibrary{arrows.meta}
\usetikzlibrary{math}
\usetikzlibrary{cd}

% Boxes and Theorems
\usepackage[most]{tcolorbox}
\tcbuselibrary{skins}
\tcbuselibrary{breakable}
\tcbuselibrary{theorems}

\newtheoremstyle{default}{0pt}{0pt}{}{}{\bfseries}{\normalfont.}{0.5em}{}
\theoremstyle{default}

\renewcommand*{\proofname}{\textit{\textbf{Proof.}}}
\renewcommand*{\qedsymbol}{$\blacksquare$}
\tcolorboxenvironment{proof}{
	breakable,
	coltitle = black,
	colback = white,
	frame hidden,
	boxrule = 0pt,
	boxsep = 0pt,
	borderline west={3pt}{0pt}{bar},
	% borderline west={3pt}{0pt}{gPurple},
	sharp corners = all,
	enhanced,
}

\newtheorem{theorem}{Theorem}[section]{\bfseries}{}
\tcolorboxenvironment{theorem}{
	breakable,
	enhanced,
	boxrule = 0pt,
	frame hidden,
	coltitle = black,
	colback = blue!7,
	% colback = gBlue!30,
	left = 0.5em,
	sharp corners = all,
}

\newtheorem{corollary}{Corollary}[section]{\bfseries}{}
\tcolorboxenvironment{corollary}{
	breakable,
	enhanced,
	boxrule = 0pt,
	frame hidden,
	coltitle = black,
	colback = white!0,
	left = 0.5em,
	sharp corners = all,
}

\newtheorem{lemma}{Lemma}[section]{\bfseries}{}
\tcolorboxenvironment{lemma}{
	breakable,
	enhanced,
	boxrule = 0pt,
	frame hidden,
	coltitle = black,
	colback = green!7,
	left = 0.5em,
	sharp corners = all,
}

\newtheorem{definition}{Definition}[section]{\bfseries}{}
\tcolorboxenvironment{definition}{
	breakable,
	coltitle = black,
	colback = white,
	frame hidden,
	boxsep = 0pt,
	boxrule = 0pt,
	borderline west = {3pt}{0pt}{orange},
	% borderline west = {3pt}{0pt}{gOrange},
	sharp corners = all,
	enhanced,
}

\newtheorem{example}{Example}[section]{\bfseries}{}
\tcolorboxenvironment{example}{
	% title = \textbf{Example},
	% detach title,
	% before upper = {\tcbtitle\quad},
	breakable,
	coltitle = black,
	colback = white,
	frame hidden,
	boxrule = 0pt,
	boxsep = 0pt,
	borderline west={3pt}{0pt}{green!70!black},
	% borderline west={3pt}{0pt}{gGreen},
	sharp corners = all,
	enhanced,
}

\newtheoremstyle{remark}{0pt}{4pt}{}{}{\bfseries\itshape}{\normalfont.}{0.5em}{}
\theoremstyle{remark}
\newtheorem*{remark}{Remark}


% TColorBoxes
\newtcolorbox{week}{
	colback = black,
	coltext = white,
	fontupper = {\large\bfseries},
	width = 1.2\paperwidth,
	size = fbox,
	halign upper = center,
	center
}

\newcommand{\banner}[2]{
    \pagebreak
    \begin{week}
   		\section*{#1}
    \end{week}
    \addcontentsline{toc}{section}{#1}
    \addtocounter{section}{1}
    \setcounter{subsection}{0}
}

% Hyperref
\usepackage{hyperref}
\hypersetup{
	colorlinks=true,
	linktoc=all,
	linkcolor=links,
	bookmarksopen=true
}

% Error Handling
\PackageWarningNoLine{ExtSizes}{It is better to use one of the extsizes 
                          classes,^^J if you can}


\def\homeworknumber{1}
\fancyhead[R]{\textbf{Math 140A: Homework \#\homeworknumber}}
\fancyhead[L]{Eli Griffiths}
\renewcommand{\headrulewidth}{1pt}
\setlength\parindent{0pt}


\begin{document}

% Section 2: #2
% Section 3: #1 (show your work)
% Section 6: #7, 10, 13
% Section 9: #5, 6, 8
% Section 11: #3, 5
% Section 12: #1, 4

\subsection*{2.2}
Let $z = x + iy$ such that $\Re(z) = x$ and $\Im(z) = y$.
\subsubsection*{Part A}

\[
    \Re(iz) = \Re(i(x + iy)) = \Re(ix - y) = \Re(- y + ix) = -y = -\Im(z)
.\]

\subsubsection*{Part B}
\[
    \Im(iz) = \Im(i(x + iy)) = \Im(- y + ix) = x = \Re(z)
.\]

\subsection*{3.1}
\subsubsection*{Part A}
\begin{align*}
    \frac{1+2i}{3-4i} + \frac{2-i}{5i} &=
    \frac{(1+2i)(\overline{3-4i})}{3^2 + (-4)^2} + \frac{(2-i)(\overline{5i})}{0^2 + 5^2} \\
    &= \frac{(1+2i)(3+4i)}{25} + \frac{(2-i)(-5i)}{25} \\
    &= \frac{3 + 4i + 6i + 8i^2}{25} + \frac{-10i + 5i^2}{25} \\
    &= \frac{-5 + 10i}{25} + \frac{-5 -10i}{25} \\
    &= -\frac{10}{25} = - \frac{2}{5}
\end{align*}

\subsubsection*{Part B}
\begin{align*}
    \frac{5i}{(1-i)(2-i)(3-i)} &= \frac{5i}{(2 - i - 2i + i^2)(3-i)} \\
    &= \frac{5i}{(1 - 3i)(3 - i)} \\
    &= \frac{5i}{3 - i - 9i + 3i^2} \\
    &= \frac{5i}{-10i} \\
    &= -\frac{5}{10} = -\frac{1}{2}
\end{align*}

\subsubsection*{Part C}
\[
    (1-i)^2 = 1 - 2i + i^2 = -2i \implies (1-i)^4 = (-2i)^2 = 4i^2 = -4
.\]

\subsection*{6.7}
\begin{align*}
    |\Re(2+\overline{z} + z^3)| &= \qty|\frac{2 + \overline{z} + z^3 + (\overline{2 + \overline{z} + z^3})}{2}| \\
    &= \qty|\frac{2 + \overline{z} + z^3 + 2 + z + \overline{z}^3}{2}| \\
    &= \qty|\frac{4 + z + \overline{z} + z^3 + 2}{2}| \\
    &\leq \frac{2 + |z| + |\overline{z}| + |z|^3 + |\overline{z}|^3}{2} \\
    &= \frac{2 + 2|z| + 2|z|^3}{2} \\
    &\leq \frac{2 + 2 + 2}{2} = 3 \leq 4
\end{align*}

\subsection*{6.10}
\begin{proof}
    Let $z = x + iy$.
    \begin{enumerate}
        \item[$\Rightarrow)$]
            Assume that $z$ is real. That is, $y = 0$. Then $z = x + 0y = x = x - 0y = \overline{z}$. Therefore $z = \overline{z}$
        \item[$\Leftarrow)$]
            Assume that $z = \overline{z}$. Then
            \[
                x + iy = x - iy
            .\]
            Equating the imaginary components gives $iy = - iy$ or equivalently $y = -y$. This is only true if $y = 0$. Therefore $z = x + 0y = x$ and hence $z$ is real.
    \end{enumerate}
    Both directions hence prove the if and only if.
\end{proof}

\subsection*{6.13}
\begin{align*}
    |z - z_0| = R \implies |z - z_0|^2 &= R^2 \\
    (z - z_0)\conj{(z - z_0)} &= R^2 \\
    (z - z_0) (\conj{z} - \conj{z_0}) &= R^2 \\
    z \conj{z} - z \conj{z_0} - \conj{z} z_0 + z_0 \conj{z_0} &= R^2 \\
    |z|^2 - z \conj{z_0} - \conj{z \conj{z_0}} + |z_0|^2 &= R^2 \\
    |z|^2 - (z \conj{z_0} + \conj{z \conj{z_0}}) + |z_0|^2 &= R^2 \\
    |z|^2 - 2 \Re(z \conj{z_0}) + |z_0|^2 &= R^2
\end{align*}

\subsection*{9.5}
\subsubsection*{Part A}
Since
\begin{align*}
    i &\Leftrightarrow e^{i\frac{\pi}{2}} \\
    1 - i\sqrt{3} &\Leftrightarrow 2e^{-i \frac{\pi}{3}} \\
    \sqrt{3} + i &\Leftrightarrow 2e^{i \frac{\pi}{6}}
\end{align*}
it follows that
\begin{align*}
    i (1 - i\sqrt{3})(\sqrt{3} + i) &= e^{i \frac{\pi}{2}} \cdot 2 e^{-i \frac{\pi}{3}} \cdot 2e^{i \frac{\pi}{6}} \\
    &= 4 e^{i\qty(\frac{\pi}{2} - \frac{\pi}{3} + \frac{\pi}{6})} \\
    &= 4e^{i \frac{\pi}{3}} \\
    &= 4 \cdot (\frac{1}{2} + i\frac{\sqrt{3}}{2}) = 2 \cdot (1 + i\sqrt{3})
\end{align*}

\subsubsection*{Part B}
Since
\begin{align*}
    5i &\Leftrightarrow 5 e^{i \frac{\pi}{2}} \\
    2 + i &\Leftrightarrow \sqrt{5} e^{i \arctan(\frac{1}{2})}
\end{align*}
Let $\theta = \arctan(\frac{1}{2})$. It follows
\begin{align*}
    \frac{5i}{2+i} &= 5e^{\pi \frac{\pi}{2}} \cdot \frac{1}{\sqrt{5}} e^{-i \theta} \\
                   &= \frac{5}{\sqrt{5}} e^{i \qty(\frac{\pi}{2} - \theta)} \\
                   &= \frac{5}{\sqrt{5}} (\cos(\frac{\pi}{2} - \theta) + i \sin(\frac{\pi}{2} - \theta)) \\
                   &= \frac{5}{\sqrt{5}} (\sin \theta - i \cos \theta) \\
                   &= \frac{5}{\sqrt{5}} (\sin \theta - i \cos \theta) \\
                   &= \frac{5}{\sqrt{5}} (\sin \theta - i \cos \theta) \\
\end{align*}

\subsubsection*{Part C}


\subsubsection*{Part D}

\subsection*{9.6}
\subsection*{9.8}

\subsection*{11.3}
\subsection*{11.5}

\subsection*{12.1}
\subsection*{12.4}


\end{document}
