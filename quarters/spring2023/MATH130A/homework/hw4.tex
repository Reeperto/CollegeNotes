\documentclass[12pt]{extarticle}

% Document Layout and Font
\usepackage{subfiles}
\usepackage[margin=2cm, headheight=15pt]{geometry}
\usepackage{fancyhdr}
\usepackage{enumitem}	
\usepackage{wrapfig}
\usepackage{float}
\usepackage{multicol}

\usepackage[p,osf]{scholax}

\renewcommand*\contentsname{Table of Contents}
\renewcommand{\headrulewidth}{0pt}
\pagestyle{fancy}
\fancyhf{}
\fancyfoot[R]{$\thepage$}
\setlength{\parindent}{0cm}
\setlength{\headheight}{17pt}
\hfuzz=9pt

% Figures
\usepackage{svg}

% Utility Management
\usepackage{color}
\usepackage{colortbl}
\usepackage{xcolor}
\usepackage{xpatch}
\usepackage{xparse}

\definecolor{gBlue}{HTML}{7daea3}
\definecolor{gOrange}{HTML}{e78a4e}
\definecolor{gGreen}{HTML}{a9b665}
\definecolor{gPurple}{HTML}{d3869b}

\definecolor{links}{HTML}{1c73a5}
\definecolor{bar}{HTML}{584AA8}

% Math Packages
\usepackage{mathtools, amsmath, amsthm, thmtools, amssymb, physics}
\usepackage[scaled=1.075,ncf,vvarbb]{newtxmath}

\newcommand\B{\mathbb{C}}
\newcommand\C{\mathbb{C}}
\newcommand\R{\mathbb{R}}
\newcommand\Q{\mathbb{Q}}
\newcommand\N{\mathbb{N}}
\newcommand\Z{\mathbb{Z}}

\DeclareMathOperator{\lcm}{lcm}

% Probability Theory
\newcommand\Prob[1]{\mathbb{P}\qty(#1)}
\newcommand\Var[1]{\text{Var}\qty(#1)}
\newcommand\Exp[1]{\mathbb{E}\qty[#1]}

% Analysis
\newcommand\ball[1]{\B\qty(#1)}
\newcommand\conj[1]{\overline{#1}}
\DeclareMathOperator{\Arg}{Arg}
\DeclareMathOperator{\cis}{cis}

% Linear Algebra
\DeclareMathOperator{\dom}{dom}
\DeclareMathOperator{\range}{range}
\DeclareMathOperator{\spann}{span}
\DeclareMathOperator{\nullity}{nullity}

% TIKZ
\usepackage{tikz}
\usepackage{pgfplots}
\usetikzlibrary{arrows.meta}
\usetikzlibrary{math}
\usetikzlibrary{cd}

% Boxes and Theorems
\usepackage[most]{tcolorbox}
\tcbuselibrary{skins}
\tcbuselibrary{breakable}
\tcbuselibrary{theorems}

\newtheoremstyle{default}{0pt}{0pt}{}{}{\bfseries}{\normalfont.}{0.5em}{}
\theoremstyle{default}

\renewcommand*{\proofname}{\textit{\textbf{Proof.}}}
\renewcommand*{\qedsymbol}{$\blacksquare$}
\tcolorboxenvironment{proof}{
	breakable,
	coltitle = black,
	colback = white,
	frame hidden,
	boxrule = 0pt,
	boxsep = 0pt,
	borderline west={3pt}{0pt}{bar},
	% borderline west={3pt}{0pt}{gPurple},
	sharp corners = all,
	enhanced,
}

\newtheorem{theorem}{Theorem}[section]{\bfseries}{}
\tcolorboxenvironment{theorem}{
	breakable,
	enhanced,
	boxrule = 0pt,
	frame hidden,
	coltitle = black,
	colback = blue!7,
	% colback = gBlue!30,
	left = 0.5em,
	sharp corners = all,
}

\newtheorem{corollary}{Corollary}[section]{\bfseries}{}
\tcolorboxenvironment{corollary}{
	breakable,
	enhanced,
	boxrule = 0pt,
	frame hidden,
	coltitle = black,
	colback = white!0,
	left = 0.5em,
	sharp corners = all,
}

\newtheorem{lemma}{Lemma}[section]{\bfseries}{}
\tcolorboxenvironment{lemma}{
	breakable,
	enhanced,
	boxrule = 0pt,
	frame hidden,
	coltitle = black,
	colback = green!7,
	left = 0.5em,
	sharp corners = all,
}

\newtheorem{definition}{Definition}[section]{\bfseries}{}
\tcolorboxenvironment{definition}{
	breakable,
	coltitle = black,
	colback = white,
	frame hidden,
	boxsep = 0pt,
	boxrule = 0pt,
	borderline west = {3pt}{0pt}{orange},
	% borderline west = {3pt}{0pt}{gOrange},
	sharp corners = all,
	enhanced,
}

\newtheorem{example}{Example}[section]{\bfseries}{}
\tcolorboxenvironment{example}{
	% title = \textbf{Example},
	% detach title,
	% before upper = {\tcbtitle\quad},
	breakable,
	coltitle = black,
	colback = white,
	frame hidden,
	boxrule = 0pt,
	boxsep = 0pt,
	borderline west={3pt}{0pt}{green!70!black},
	% borderline west={3pt}{0pt}{gGreen},
	sharp corners = all,
	enhanced,
}

\newtheoremstyle{remark}{0pt}{4pt}{}{}{\bfseries\itshape}{\normalfont.}{0.5em}{}
\theoremstyle{remark}
\newtheorem*{remark}{Remark}


% TColorBoxes
\newtcolorbox{week}{
	colback = black,
	coltext = white,
	fontupper = {\large\bfseries},
	width = 1.2\paperwidth,
	size = fbox,
	halign upper = center,
	center
}

\newcommand{\banner}[2]{
    \pagebreak
    \begin{week}
   		\section*{#1}
    \end{week}
    \addcontentsline{toc}{section}{#1}
    \addtocounter{section}{1}
    \setcounter{subsection}{0}
}

% Hyperref
\usepackage{hyperref}
\hypersetup{
	colorlinks=true,
	linktoc=all,
	linkcolor=links,
	bookmarksopen=true
}

% Error Handling
\PackageWarningNoLine{ExtSizes}{It is better to use one of the extsizes 
                          classes,^^J if you can}

\usepackage{svg}
\svgsetup{inkscapeexe=inkscape, inkscapearea=drawing, inkscapeversion=1}
\svgpath{{figures/}}

\fancyhead[R]{\textbf{Homework \#$4$}}
\fancyhead[L]{Eli Griffiths}
\renewcommand{\headrulewidth}{1pt}
\setlength\parindent{0pt}

\begin{document}
\section*{Problem 1}
Let $X$ denote the number of $6$'s that appear in $3$ rolls. Then $X \sim \text{Binom}(3, \frac{1}{6})$. Therefore
\begin{align*}
	\Prob{X \leq 1} &= \Prob{X = 0} + \Prob{X = 1} \\
									&= \binom{3}{0}\qty(\frac{5}{6})^3 + \binom{3}{1}\qty(\frac{1}{6})\qty(\frac{5}{6})^2 \\
									&= \frac{25}{27}
.\end{align*}

\section*{Problem 2}
Since there is replacement, each drawing is independent. Let $X$ denote the number of white balls drawn after $4$ drawings. Then $X \sim \text{Binom}(4, \frac{1}{2})$. Therefore
\begin{align*}
	\Prob{X = 2} &= \binom{4}{2} \cdot \frac{1}{2}^4 \\
							 &= 6 \cdot \frac{1}{16} = \frac{3}{8}
.\end{align*}

\section*{Problem 3}
\begin{align*}
	\Exp{X} &= 25\qty(\frac{25}{148}) + 33\qty(\frac{33}{148}) + 40\qty(\frac{40}{148}) + 50\qty(\frac{50}{148}) \\
					&= \frac{2907}{74} \approx 39.28 \\
					\\
	\Exp{Y} &= \frac{1}{4} \cdot (25 + 33 + 40 + 50) \\
	&= \frac{148}{4} = 37
\end{align*}

\section*{Problem 4}
Let $I$ denote the revenue the company makes. Let $X$ denote the profit the company makes. Note that $X = \qty{I, I - A}$ with $\Prob{X = I-A} = p$ and $\Prob{X = I} = p$. Therefore the expected profit is
\[
	\Exp{X} = I (1-p) + (I - A)p = I - pA
.\]
Therefore since the company wants their expected profit to be $10\%$ of $A$,
\begin{align*}
	\Exp{X} &= \frac{A}{10} \\
	I - pA &= \frac{A}{10} \\
	I &= \frac{A}{10} + pA \implies \boxed{I = A\qty(p + \frac{1}{10})}
.\end{align*}

\section*{Problem 5}
Let $X_1 \sim \text{Bern}(0.6)$ represent the first flip and $X_2 \sim \text{Bern}(0.7)$ represent the second flip. Then $X = X_1 + X_2$. Therefore 
\begin{align*}
	\Prob{X = 1} &= \Prob{X_1 = 1, X_2 = 0} + \Prob{X_1 = 0, X_2 = 1} \\
	&= (0.6)(1-0.7) + (0.7)(1-0.6) \\
	&= 0.18 + 0.28 = 0.46
.\end{align*}
and
\begin{align*}
	\Exp{X} &= \Exp{X_1 + X_2} \\
					&= \Exp{X_1} + \Exp{X_2} \\
					&= 0.6 + 0.7 = 1.3
.\end{align*}

\section*{Problem 6}
The probability a tails appears on the $n^{\text{th}}$ flip is $\qty(\frac{1}{2})^n$, therefore the expected value is
\begin{align*}
	\Exp{X} &= \sum_{n=1}^{\infty} 2^n \cdot\qty(\frac{1}{2})^n \\
	        &= \sum_{n=1}^{\infty} 2^n \cdot\frac{1}{2^n} \\
	        &= \sum_{n=1}^{\infty} 1 \to +\infty
.\end{align*}
For (1), no because to get a net positive amount back one would have to flip $19$ heads in a row which has a probability of 0.0000019073, meaning it's very unlikely one would recover their million dollars. However, for (2), it would advanteagous to pay a million because the expectation is infinite, meaning eventually after enough tries one could easily make more than a million dollars by playing the game contiuonsly.

\section*{Problem 7}
Let $X$ denote the winnings. Then
\begin{align*}
	\Exp{X} = -1 \cdot \qty(2\cdot\frac{\binom{5}{2}}{\binom{10}{2}}) + 1.1 \cdot \qty(2\cdot\frac{\binom{5}{1}}{\binom{10}{2}}) = -0.2
.\end{align*}
and
\[
	\Var{X} = (-1)^2 \cdot \qty(2\cdot\frac{\binom{5}{2}}{\binom{10}{2}}) + (1.1)^2 \cdot \qty(2\cdot\frac{\binom{5}{1}}{\binom{10}{2}}) + 0.2 = 0.913
.\]

\section*{Problem 8}
Let $D \sim \text{Binom}\qty(10, \frac{1}{3})$ denote the daily demand. Let $n$ denote the number of papers he buys and $X_n$ denote the associated profit. If $D > n$, then $X = 0.05 n$. If $D \leq n$, then $X = 0.04 D - 0.10(n - X)$. Therefore
\[
	\Exp{X_n} = \sum_{i=0}^{n} \qty(0.05 i - 0.10(n - i)) \Prob{D = i} + \sum_{k=n+1}^{10} (0.05 n) \Prob{D = k}
.\]

Calculating $\Exp{X_n}$ for $0 \leq n \leq 10$ is shown in the table.
\begin{figure}[h!]
	\centering
	\includesvg[width=1\textwidth]{figures/profit.svg}
\end{figure}
Therefore the profit maximizing amount of papers the boy should buy is $3$.

\section*{Problem 9}
Note that $\Var{X} = \Exp{X^2} - \Exp{X}^2 \implies \Exp{X^2} = \Var{X} + \Exp{X}^2$. \\

\begin{multicols}{2}
\subsubsection*{(1)}
\begin{align*}
	\Exp{(2+X^2)} &= \Exp{X^2 + 4X + 4} \\
								&= \Exp{X^2} + 4\cdot\Exp{X} + 4 \\
								&= \Var{X} + \Exp{X}^2 + 4\cdot\Exp{X} + 4 \\
								&= 5 + 1^2 + 4\cdot 1 + 4 \\
								&= 14
.\end{align*}
\columnbreak
\subsubsection*{(2)}
\begin{align*}
	\Var{4 + 3X} &= \Var{3X} \\
							 &= 3^2 \cdot \Var{X} \\
							 &= 9 \cdot 5 \\ 
							 &= 45
.\end{align*}
\end{multicols}

\section*{Problem 10}
Consider the case where $i = 2$. Let $X_2$ denote the number of games played before a team wins $2$ times. Note that $\mathcal{R}_{X_2} = \qty{2,3}$ with the following possible game scenarios
\[
	\begin{array}{rl}
		AA & ABA \\
		BB & BAA \\
		   & BAB \\
		   & ABB \\
	\end{array}
\]
Therefore
\begin{align*}
	\Exp{X_2} &= 2\cdot\qty[ p^2 + (1-p)^2 ] + 3\cdot\qty[2p^2(1-p) + 2(1-p)^2 p] \\
	&= 2 + 2 p - 2 p^2
.\end{align*}

Consider the case where $i = 3$. Let $X_3$ denote the number of games played before a teame wins $3$ times. Note that $\mathcal{R}_{X_3} = \qty{3,4,5}$ with the following possible game scenarios
\[
	\begin{array}{rcl}
		AAA & AABA & BBAAA \\
		BBB & ABAA & BABAA \\
		    & BAAA & BAABA \\
		    & BBAB & ABABA \\
		    & BABB & AABBA \\
				& ABBB & AABBB \\
		    &      & ABABB \\
		    &      & ABBAB \\
		    &      & BABAB \\
		    &      & BBAAB \\
	\end{array}
\]
Therefore
\begin{align*}
	\Exp{E_3} &= 3\cdot\qty[p^3 + (1-p)^3] + 4\cdot\qty[3p^3(1-p) + 3p(1-p)^3] + 5\cdot\qty[4p^3(1-p)^2+4p^2(1-p)^3] \\
						&= 3 + 3p - 7p^2 + 8p^3 - 4p^4
.\end{align*}

Consider the derivatives of each expectation.
\begin{align*}
	\dv{\Exp{X_2}}{p} &= 2 - 4p = 0 \implies p = \frac{1}{2} \\
	\dv{\Exp{X_3}}{p} &= 3-14p + 24p^2 - 15p^3 = 0 \implies p = \frac{1}{2}
.\end{align*}

Since both are concave down and $p=\frac{1}{2}$ is the only critical point for both, the expectations are maximized by $p=\frac{1}{2}$.

\end{document}
