\documentclass[12pt]{extarticle}

% Document Layout and Font
\usepackage{subfiles}
\usepackage[margin=2cm, headheight=15pt]{geometry}
\usepackage{fancyhdr}
\usepackage{enumitem}	
\usepackage{wrapfig}
\usepackage{float}
\usepackage{multicol}

\usepackage[p,osf]{scholax}

\renewcommand*\contentsname{Table of Contents}
\renewcommand{\headrulewidth}{0pt}
\pagestyle{fancy}
\fancyhf{}
\fancyfoot[R]{$\thepage$}
\setlength{\parindent}{0cm}
\setlength{\headheight}{17pt}
\hfuzz=9pt

% Figures
\usepackage{svg}

% Utility Management
\usepackage{color}
\usepackage{colortbl}
\usepackage{xcolor}
\usepackage{xpatch}
\usepackage{xparse}

\definecolor{gBlue}{HTML}{7daea3}
\definecolor{gOrange}{HTML}{e78a4e}
\definecolor{gGreen}{HTML}{a9b665}
\definecolor{gPurple}{HTML}{d3869b}

\definecolor{links}{HTML}{1c73a5}
\definecolor{bar}{HTML}{584AA8}

% Math Packages
\usepackage{mathtools, amsmath, amsthm, thmtools, amssymb, physics}
\usepackage[scaled=1.075,ncf,vvarbb]{newtxmath}

\newcommand\B{\mathbb{C}}
\newcommand\C{\mathbb{C}}
\newcommand\R{\mathbb{R}}
\newcommand\Q{\mathbb{Q}}
\newcommand\N{\mathbb{N}}
\newcommand\Z{\mathbb{Z}}

\DeclareMathOperator{\lcm}{lcm}

% Probability Theory
\newcommand\Prob[1]{\mathbb{P}\qty(#1)}
\newcommand\Var[1]{\text{Var}\qty(#1)}
\newcommand\Exp[1]{\mathbb{E}\qty[#1]}

% Analysis
\newcommand\ball[1]{\B\qty(#1)}
\newcommand\conj[1]{\overline{#1}}
\DeclareMathOperator{\Arg}{Arg}
\DeclareMathOperator{\cis}{cis}

% Linear Algebra
\DeclareMathOperator{\dom}{dom}
\DeclareMathOperator{\range}{range}
\DeclareMathOperator{\spann}{span}
\DeclareMathOperator{\nullity}{nullity}

% TIKZ
\usepackage{tikz}
\usepackage{pgfplots}
\usetikzlibrary{arrows.meta}
\usetikzlibrary{math}
\usetikzlibrary{cd}

% Boxes and Theorems
\usepackage[most]{tcolorbox}
\tcbuselibrary{skins}
\tcbuselibrary{breakable}
\tcbuselibrary{theorems}

\newtheoremstyle{default}{0pt}{0pt}{}{}{\bfseries}{\normalfont.}{0.5em}{}
\theoremstyle{default}

\renewcommand*{\proofname}{\textit{\textbf{Proof.}}}
\renewcommand*{\qedsymbol}{$\blacksquare$}
\tcolorboxenvironment{proof}{
	breakable,
	coltitle = black,
	colback = white,
	frame hidden,
	boxrule = 0pt,
	boxsep = 0pt,
	borderline west={3pt}{0pt}{bar},
	% borderline west={3pt}{0pt}{gPurple},
	sharp corners = all,
	enhanced,
}

\newtheorem{theorem}{Theorem}[section]{\bfseries}{}
\tcolorboxenvironment{theorem}{
	breakable,
	enhanced,
	boxrule = 0pt,
	frame hidden,
	coltitle = black,
	colback = blue!7,
	% colback = gBlue!30,
	left = 0.5em,
	sharp corners = all,
}

\newtheorem{corollary}{Corollary}[section]{\bfseries}{}
\tcolorboxenvironment{corollary}{
	breakable,
	enhanced,
	boxrule = 0pt,
	frame hidden,
	coltitle = black,
	colback = white!0,
	left = 0.5em,
	sharp corners = all,
}

\newtheorem{lemma}{Lemma}[section]{\bfseries}{}
\tcolorboxenvironment{lemma}{
	breakable,
	enhanced,
	boxrule = 0pt,
	frame hidden,
	coltitle = black,
	colback = green!7,
	left = 0.5em,
	sharp corners = all,
}

\newtheorem{definition}{Definition}[section]{\bfseries}{}
\tcolorboxenvironment{definition}{
	breakable,
	coltitle = black,
	colback = white,
	frame hidden,
	boxsep = 0pt,
	boxrule = 0pt,
	borderline west = {3pt}{0pt}{orange},
	% borderline west = {3pt}{0pt}{gOrange},
	sharp corners = all,
	enhanced,
}

\newtheorem{example}{Example}[section]{\bfseries}{}
\tcolorboxenvironment{example}{
	% title = \textbf{Example},
	% detach title,
	% before upper = {\tcbtitle\quad},
	breakable,
	coltitle = black,
	colback = white,
	frame hidden,
	boxrule = 0pt,
	boxsep = 0pt,
	borderline west={3pt}{0pt}{green!70!black},
	% borderline west={3pt}{0pt}{gGreen},
	sharp corners = all,
	enhanced,
}

\newtheoremstyle{remark}{0pt}{4pt}{}{}{\bfseries\itshape}{\normalfont.}{0.5em}{}
\theoremstyle{remark}
\newtheorem*{remark}{Remark}


% TColorBoxes
\newtcolorbox{week}{
	colback = black,
	coltext = white,
	fontupper = {\large\bfseries},
	width = 1.2\paperwidth,
	size = fbox,
	halign upper = center,
	center
}

\newcommand{\banner}[2]{
    \pagebreak
    \begin{week}
   		\section*{#1}
    \end{week}
    \addcontentsline{toc}{section}{#1}
    \addtocounter{section}{1}
    \setcounter{subsection}{0}
}

% Hyperref
\usepackage{hyperref}
\hypersetup{
	colorlinks=true,
	linktoc=all,
	linkcolor=links,
	bookmarksopen=true
}

% Error Handling
\PackageWarningNoLine{ExtSizes}{It is better to use one of the extsizes 
                          classes,^^J if you can}


\fancyhead[R]{Homework \#$6$}
\fancyhead[L]{Eli Griffiths}
\renewcommand{\headrulewidth}{1pt}
\setlength\parindent{0pt}

\begin{document}
\DeclarePairedDelimiter\bangle\langle\rangle

% Section 9: 1, 6, 9, 13(a), 16, 29. 34.

\section*{9.1}
\begin{align*}
	O_\sigma (1) = O_\sigma (2) = O_\sigma (5) &= \qty{1, 5, 2} \\
	O_\sigma (4) = O_\sigma (6) &=  \qty{4,6} \\
	O_\sigma (3) &= \qty{3}
.\end{align*}

\section*{9.6}
\begin{align*}
	O_\sigma (3n) &= 3 \mathbb{Z} \\
	O_\sigma (3n + 1) &= 3 \mathbb{Z} + 1 \\
	O_\sigma (3n + 2) &= 3 \mathbb{Z} + 2
.\end{align*}

\section*{9.9}
\[
	(1, 2)(4, 7, 8)(2, 1)(7, 2, 8, 1, 5) = (1, 5, 8) (2, 4, 7)
.\]

\section*{9.13(a)}
Let $\sigma = (1, 4, 5, 7)$.
\begin{align*}
	\sigma^2 &= (1,5) (4, 7) \\
	\sigma^3 &= (1, 7, 5, 4) \\
	\sigma^4 &= e 
.\end{align*}
Therefore $|\sigma| = 4$.

\section*{9.16}
The maximum order of an element is going to be the maximum value that can be obtained from $\text{lcm}(a,b)$ where $a + b = 7$ and $a,b \geq 0$. This is because the order of an element is the least common multiple of the order's of it's disjoint cycles, which is maximized with 2 cycles. Consider all the ways to add 2 numbers to get seven and their lcm:
\begin{align*}
	1 + 6 &\implies 6 \\
	2 + 5 &\implies 10 \\
	3 + 4 &\implies 12
.\end{align*}
Therefore the maximum order of an element in $S_7$ is 12.

\section*{9.29}
\begin{proof}
	Let $H \leq S_n$ with $n \geq 2$. If $H$ does not contain any odd permutations, then all the permutations of $H$ are even. Examine the case if there exists an odd permutation in $H$, that is some $\sigma \in H$ that is odd. Define the mapping $\phi : H \to H : \mu \mapsto \sigma\mu$. Let $\mu_1, \mu_2 \in H$ and assume that $\phi(\mu_1) = \phi(\mu_2)$. Then $\sigma \mu_1 = \sigma \mu_2$ which by cancellation implies $\mu_1 = \mu_2$, hence $\phi$ is one-to-one. Let $\mu \in H$. Note that $\phi(\sigma^{-1} \mu) = \mu$, hence $\phi$ is onto. Therefore $\phi$ is a bijection on $H$. Note that $\phi$ takes an odd permutation to an even permutation and takes an even permutation to an odd permutation. Therefore since $\phi$ is bijective, there must be an equal amount of even elements as odd elements otherwise the swapping would not be one-to-one and onto. Therefore if $H$ has an odd permutation, exactly half of its permutations are even.
\end{proof}

\section*{9.34}
\begin{proof}
	It can be assumed without loss of generality that an odd cycle can be represented as $\sigma = (1,2,3,\ldots,m)$ where $m$ is an odd number. Computing its square results in
	\[
		\sigma^2 = (1,3,5,\ldots,m,2,4,6,\ldots,m-1)
	\]
	which is a cycle.
\end{proof}

\end{document}
