\documentclass[../notes.tex]{subfiles}
\graphicspath{
    {"../figures"}
}

\begin{document}

\banner{Basic Probability}

\begin{definition}[Sample Space]
	The collection of all possible outcomes in an experiment.
\end{definition}

Probability can be understood from the perspective of gambling. The simplest way to gamble to flip a coin. There are only two possibilities, \textbf{H} or \textbf{T}. Therefore
\[
	S_{\text{coin}} = \qty{ \textbf{H}, \textbf{T} }
.\]
Considering a singular die, there are only six outcomes corresponding to the 6 sides of the die. Therefore
\[
	S_{\text{die}} = \qty{1,2,3,4,5,6}
.\]

\begin{definition}
	An event is a subset of a sample space. Equivalently a set $E$ is an event of a sample space if $E \subseteq S$.
\end{definition}
An event in the first example would be tossing a coin and getting tails ($E = \qty{\textbf{T}}$). Given two events $E$ and $F$, their intersection is denoted as
\[
	EF \Leftrightarrow E \cap F
.\]

\begin{definition}[Probability]
	Given some event $E \subset S$ where $S$ is a sample space, $\mathbb{P}(E)$ is a number assigned to the event such that
	\begin{enumerate}
		\item $0 \leq \mathbb{P}(E) \leq 1$
		\item $\mathbb{P}(S) = 1$
		\item For mutually exclusive events $E_1, E_2, E_3, \ldots, E_n, \ldots$ then
			\[
				\mathbb{P}\qty(\;\bigcup_{n=1}^\infty E_n) = \sum_{n=1}^\infty \mathbb{P}(E_n)
			.\]
	\end{enumerate}
\end{definition}

\end{document}
