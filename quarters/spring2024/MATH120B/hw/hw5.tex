\documentclass[12pt,titlepage]{extarticle}
% Document Layout and Font
\usepackage{subfiles}
\usepackage[margin=2cm, headheight=15pt]{geometry}
\usepackage{fancyhdr}
\usepackage{enumitem}	
\usepackage{wrapfig}
\usepackage{float}
\usepackage{multicol}

\usepackage[p,osf]{scholax}

\renewcommand*\contentsname{Table of Contents}
\renewcommand{\headrulewidth}{0pt}
\pagestyle{fancy}
\fancyhf{}
\fancyfoot[R]{$\thepage$}
\setlength{\parindent}{0cm}
\setlength{\headheight}{17pt}
\hfuzz=9pt

% Figures
\usepackage{svg}

% Utility Management
\usepackage{color}
\usepackage{colortbl}
\usepackage{xcolor}
\usepackage{xpatch}
\usepackage{xparse}

\definecolor{gBlue}{HTML}{7daea3}
\definecolor{gOrange}{HTML}{e78a4e}
\definecolor{gGreen}{HTML}{a9b665}
\definecolor{gPurple}{HTML}{d3869b}

\definecolor{links}{HTML}{1c73a5}
\definecolor{bar}{HTML}{584AA8}

% Math Packages
\usepackage{mathtools, amsmath, amsthm, thmtools, amssymb, physics}
\usepackage[scaled=1.075,ncf,vvarbb]{newtxmath}

\newcommand\B{\mathbb{C}}
\newcommand\C{\mathbb{C}}
\newcommand\R{\mathbb{R}}
\newcommand\Q{\mathbb{Q}}
\newcommand\N{\mathbb{N}}
\newcommand\Z{\mathbb{Z}}

\DeclareMathOperator{\lcm}{lcm}

% Probability Theory
\newcommand\Prob[1]{\mathbb{P}\qty(#1)}
\newcommand\Var[1]{\text{Var}\qty(#1)}
\newcommand\Exp[1]{\mathbb{E}\qty[#1]}

% Analysis
\newcommand\ball[1]{\B\qty(#1)}
\newcommand\conj[1]{\overline{#1}}
\DeclareMathOperator{\Arg}{Arg}
\DeclareMathOperator{\cis}{cis}

% Linear Algebra
\DeclareMathOperator{\dom}{dom}
\DeclareMathOperator{\range}{range}
\DeclareMathOperator{\spann}{span}
\DeclareMathOperator{\nullity}{nullity}

% TIKZ
\usepackage{tikz}
\usepackage{pgfplots}
\usetikzlibrary{arrows.meta}
\usetikzlibrary{math}
\usetikzlibrary{cd}

% Boxes and Theorems
\usepackage[most]{tcolorbox}
\tcbuselibrary{skins}
\tcbuselibrary{breakable}
\tcbuselibrary{theorems}

\newtheoremstyle{default}{0pt}{0pt}{}{}{\bfseries}{\normalfont.}{0.5em}{}
\theoremstyle{default}

\renewcommand*{\proofname}{\textit{\textbf{Proof.}}}
\renewcommand*{\qedsymbol}{$\blacksquare$}
\tcolorboxenvironment{proof}{
	breakable,
	coltitle = black,
	colback = white,
	frame hidden,
	boxrule = 0pt,
	boxsep = 0pt,
	borderline west={3pt}{0pt}{bar},
	% borderline west={3pt}{0pt}{gPurple},
	sharp corners = all,
	enhanced,
}

\newtheorem{theorem}{Theorem}[section]{\bfseries}{}
\tcolorboxenvironment{theorem}{
	breakable,
	enhanced,
	boxrule = 0pt,
	frame hidden,
	coltitle = black,
	colback = blue!7,
	% colback = gBlue!30,
	left = 0.5em,
	sharp corners = all,
}

\newtheorem{corollary}{Corollary}[section]{\bfseries}{}
\tcolorboxenvironment{corollary}{
	breakable,
	enhanced,
	boxrule = 0pt,
	frame hidden,
	coltitle = black,
	colback = white!0,
	left = 0.5em,
	sharp corners = all,
}

\newtheorem{lemma}{Lemma}[section]{\bfseries}{}
\tcolorboxenvironment{lemma}{
	breakable,
	enhanced,
	boxrule = 0pt,
	frame hidden,
	coltitle = black,
	colback = green!7,
	left = 0.5em,
	sharp corners = all,
}

\newtheorem{definition}{Definition}[section]{\bfseries}{}
\tcolorboxenvironment{definition}{
	breakable,
	coltitle = black,
	colback = white,
	frame hidden,
	boxsep = 0pt,
	boxrule = 0pt,
	borderline west = {3pt}{0pt}{orange},
	% borderline west = {3pt}{0pt}{gOrange},
	sharp corners = all,
	enhanced,
}

\newtheorem{example}{Example}[section]{\bfseries}{}
\tcolorboxenvironment{example}{
	% title = \textbf{Example},
	% detach title,
	% before upper = {\tcbtitle\quad},
	breakable,
	coltitle = black,
	colback = white,
	frame hidden,
	boxrule = 0pt,
	boxsep = 0pt,
	borderline west={3pt}{0pt}{green!70!black},
	% borderline west={3pt}{0pt}{gGreen},
	sharp corners = all,
	enhanced,
}

\newtheoremstyle{remark}{0pt}{4pt}{}{}{\bfseries\itshape}{\normalfont.}{0.5em}{}
\theoremstyle{remark}
\newtheorem*{remark}{Remark}


% TColorBoxes
\newtcolorbox{week}{
	colback = black,
	coltext = white,
	fontupper = {\large\bfseries},
	width = 1.2\paperwidth,
	size = fbox,
	halign upper = center,
	center
}

\newcommand{\banner}[2]{
    \pagebreak
    \begin{week}
   		\section*{#1}
    \end{week}
    \addcontentsline{toc}{section}{#1}
    \addtocounter{section}{1}
    \setcounter{subsection}{0}
}

% Hyperref
\usepackage{hyperref}
\hypersetup{
	colorlinks=true,
	linktoc=all,
	linkcolor=links,
	bookmarksopen=true
}

% Error Handling
\PackageWarningNoLine{ExtSizes}{It is better to use one of the extsizes 
                          classes,^^J if you can}


\def\homeworknumber{5}
\fancyhead[R]{\textbf{Math 140A: Homework \#\homeworknumber}}
\fancyhead[L]{Eli Griffiths}
\renewcommand{\headrulewidth}{1pt}
\setlength\parindent{0pt}


\begin{document}

\subsection*{26.4}
\[
    \def\arraystretch{1.5}
    \begin{array}{c|c|c|c|c}
        +        & 8 \Z     & 2 + 8 \Z & 4 + 8 \Z & 6 + 8 \Z \\\hline
        8 \Z     & 8 \Z     & 2 + 8 \Z & 4 + 8 \Z & 6 + 8 \Z \\\hline
        2 + 8 \Z & 2 + 8 \Z & 4 + 8 \Z & 6 + 8 \Z & 8 \Z     \\\hline
        4 + 8 \Z & 4 + 8 \Z & 6 + 8 \Z & 8 \Z     & 2 + 8 \Z \\\hline
        6 + 8 \Z & 6 + 8 \Z & 8 \Z     & 2 + 8 \Z & 4 + 8 \Z \\
    \end{array}
    \hspace{1cm}
    \begin{array}{c|c|c|c|c}
        \times   & 8 \Z     & 2 + 8 \Z & 4 + 8 \Z & 6 + 8 \Z \\\hline
        8 \Z     &     8 \Z &     8 \Z &     8 \Z &     8 \Z \\\hline
        2 + 8 \Z &     8 \Z & 4 + 8 \Z &     8 \Z & 4 + 8 \Z \\\hline
        4 + 8 \Z &     8 \Z &     8 \Z &     8 \Z &     8 \Z \\\hline
        6 + 8 \Z &     8 \Z & 4 + 8 \Z &     8 \Z & 4 + 8 \Z \\
    \end{array}
\]

The rings cannot be isomorphic since $2 \Z / 8 \Z$ doesn't have unity but $\Z_4$ does.

\subsection*{26.9}
Let $\phi : \Z \to \Z \times \Z : n \mapsto (n, 0)$. $\phi$ is a homormorphism, however the unity of $\Z$ is $1$ and $\phi(1) = (1, 0)$ which is not the unity of $\Z \times \Z$.

\subsection*{26.12}
The factor ring $\Z / 2 \Z$ is a field since $\Z / 2 \Z$ is isomorphic to $\Z_2$ which is a field.

\subsection*{26.13}
The factor ring $\Z / 6 \Z$ is isomorphic to $\Z_6$ which has the zero divisor $3$.

\subsection*{26.14}
In $\Z \times \Z$, $(1,0)(0,1) = 0$ meaning it has zero divisors, however the factor ring $(\Z \times \Z) / (\Z \times \qty{0})$ is isomorphic to $\Z$ which has no zero divisors.

\subsection*{26.15}
The subring $\qty{(n, n) : n \in \Z}$ is a subring, however $(a,b)(n,n) = (an, bn)$ will not be an element in the subring if $a \neq b$.

\subsection*{26.18}
\begin{proof}
    Let $\phi : F \to R$ be a ring homorphism and $N = \ker \phi$. Assume that $N \neq \qty{0}$. Then there is some element $a \in F$ and $a \in N$. Since $F$ is a field, $a$ is a unit, meaning $a^{-1} \in F$. Since $N$ is also an ideal, then $a^{-1} a \in N$ meaning $N$ contains unity. Therefore $x 1 = x \in N$ for all $x \in F$, hence $N = F$. Therefore every element of $F$ is mapped to $0$. If $N = \qty{0}$, then $\phi$ must be one-to-one.
\end{proof}

\subsection*{26.20}
\begin{proof}
    By the binomial theorem,
    \[
        (a+b)^p = \sum_{n=0}^p \binom{p}{n} a^{n} b^{p-n}
    .\]
    Since $p$ is prime, the coefficients $\binom{p}{n}$ for $1 \leq n \leq p - 1$ will all be multiples of $p$. Since $R$ is of characteristic $p$, these terms all will go to zero. Therefore for any $a,b \in R$, it follows $(a+b)^p = a^p + b^p$. Therefore 
    \[
        \phi_p(a+b) = (a+b)^p = a^p + b^p = \phi_p(a) + \phi_p(b)
    .\]
    Trivially $\phi_p(ab) = (ab)^p = a^p b^p = \phi_p(a) \phi_p(b)$. Therefore $\phi_p$ is a ring homomorphism.
\end{proof}

\subsection*{26.22}
\subsubsection*{Part A}
\begin{proof}
    Since an ideal is also a subring, then $\phi[N]$ is a subring of $R'$. Let $r \in R$ and $n \in N$. Since $N$ is an ideal, then $rn \in N$ and $nr \in N$. Since $\phi$ is a ring homomorphism
    \begin{align*}
        \phi(r) \phi(n) &= \phi(rn) \in \phi[N] \\
        \phi(n) \phi(r) &= \phi(nr) \in \phi[N]
    \end{align*}
    Therefore $\phi[N]$ is an ideal of $\phi[R]$.
\end{proof}

\subsubsection*{Part B}
Take $\phi : \Z \to \Q : x \mapsto x$ and the ideal $3 \Z$ of $\Z$. Since 
\[
    \frac{1}{3} \cdot 3 = 1
\]
then if $3 \Z$ was an ideal of $\Q$, it would have to contain unity but it does not. So $3 \Z$ is an ideal of $\Z$ but $\phi[3 \Z]$ is not an ideal of $\Q$.

\subsubsection*{Part C}
\begin{proof}
    Note that $N'$ is a subring of $R$. Let $r \in R$ and $n \in \phi^{-1}[N']$. That is $\phi(n) \in N'$. Then $\phi(rn) = \phi(r) \phi(n)$. Since $N'$ is an ideal of $\phi[R]$ or $R'$, then $\phi(r) \phi(n) \in N'$ and the same argument shows $\phi(nr) \in N'$. Therefore $\phi^{-1}[N']$ is an ideal of $R$.
\end{proof}

\subsection*{26.26}
\begin{proof}
    Let $x,y \in I_a$. Then $ax = 0$ and $ay = 0$. Note that then
    \[
        ax + ay = 0 \implies a(x + y) = 0
    .\]
    Therefore $x + y \in I_a$. Since $a0 = 0$, $I_a$ contains the additive identity. Furthermore, $a(-x) = -(ax) = 0$, therefore $I_a$ contains additive inverses. Let $r \in R$. Note that $rax = r 0 = 0$ and $axr = 0 r = 0$ meaning $rI_a \subseteq I_a$ and $I_a r \subseteq I_a$. Therefore $I_a$ is an ideal of $R$.
\end{proof}

\end{document}
