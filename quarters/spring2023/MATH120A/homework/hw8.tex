\documentclass[12pt]{extarticle}

% Document Layout and Font
\usepackage{subfiles}
\usepackage[margin=2cm, headheight=15pt]{geometry}
\usepackage{fancyhdr}
\usepackage{enumitem}	
\usepackage{wrapfig}
\usepackage{float}
\usepackage{multicol}

\usepackage[p,osf]{scholax}

\renewcommand*\contentsname{Table of Contents}
\renewcommand{\headrulewidth}{0pt}
\pagestyle{fancy}
\fancyhf{}
\fancyfoot[R]{$\thepage$}
\setlength{\parindent}{0cm}
\setlength{\headheight}{17pt}
\hfuzz=9pt

% Figures
\usepackage{svg}

% Utility Management
\usepackage{color}
\usepackage{colortbl}
\usepackage{xcolor}
\usepackage{xpatch}
\usepackage{xparse}

\definecolor{gBlue}{HTML}{7daea3}
\definecolor{gOrange}{HTML}{e78a4e}
\definecolor{gGreen}{HTML}{a9b665}
\definecolor{gPurple}{HTML}{d3869b}

\definecolor{links}{HTML}{1c73a5}
\definecolor{bar}{HTML}{584AA8}

% Math Packages
\usepackage{mathtools, amsmath, amsthm, thmtools, amssymb, physics}
\usepackage[scaled=1.075,ncf,vvarbb]{newtxmath}

\newcommand\B{\mathbb{C}}
\newcommand\C{\mathbb{C}}
\newcommand\R{\mathbb{R}}
\newcommand\Q{\mathbb{Q}}
\newcommand\N{\mathbb{N}}
\newcommand\Z{\mathbb{Z}}

\DeclareMathOperator{\lcm}{lcm}

% Probability Theory
\newcommand\Prob[1]{\mathbb{P}\qty(#1)}
\newcommand\Var[1]{\text{Var}\qty(#1)}
\newcommand\Exp[1]{\mathbb{E}\qty[#1]}

% Analysis
\newcommand\ball[1]{\B\qty(#1)}
\newcommand\conj[1]{\overline{#1}}
\DeclareMathOperator{\Arg}{Arg}
\DeclareMathOperator{\cis}{cis}

% Linear Algebra
\DeclareMathOperator{\dom}{dom}
\DeclareMathOperator{\range}{range}
\DeclareMathOperator{\spann}{span}
\DeclareMathOperator{\nullity}{nullity}

% TIKZ
\usepackage{tikz}
\usepackage{pgfplots}
\usetikzlibrary{arrows.meta}
\usetikzlibrary{math}
\usetikzlibrary{cd}

% Boxes and Theorems
\usepackage[most]{tcolorbox}
\tcbuselibrary{skins}
\tcbuselibrary{breakable}
\tcbuselibrary{theorems}

\newtheoremstyle{default}{0pt}{0pt}{}{}{\bfseries}{\normalfont.}{0.5em}{}
\theoremstyle{default}

\renewcommand*{\proofname}{\textit{\textbf{Proof.}}}
\renewcommand*{\qedsymbol}{$\blacksquare$}
\tcolorboxenvironment{proof}{
	breakable,
	coltitle = black,
	colback = white,
	frame hidden,
	boxrule = 0pt,
	boxsep = 0pt,
	borderline west={3pt}{0pt}{bar},
	% borderline west={3pt}{0pt}{gPurple},
	sharp corners = all,
	enhanced,
}

\newtheorem{theorem}{Theorem}[section]{\bfseries}{}
\tcolorboxenvironment{theorem}{
	breakable,
	enhanced,
	boxrule = 0pt,
	frame hidden,
	coltitle = black,
	colback = blue!7,
	% colback = gBlue!30,
	left = 0.5em,
	sharp corners = all,
}

\newtheorem{corollary}{Corollary}[section]{\bfseries}{}
\tcolorboxenvironment{corollary}{
	breakable,
	enhanced,
	boxrule = 0pt,
	frame hidden,
	coltitle = black,
	colback = white!0,
	left = 0.5em,
	sharp corners = all,
}

\newtheorem{lemma}{Lemma}[section]{\bfseries}{}
\tcolorboxenvironment{lemma}{
	breakable,
	enhanced,
	boxrule = 0pt,
	frame hidden,
	coltitle = black,
	colback = green!7,
	left = 0.5em,
	sharp corners = all,
}

\newtheorem{definition}{Definition}[section]{\bfseries}{}
\tcolorboxenvironment{definition}{
	breakable,
	coltitle = black,
	colback = white,
	frame hidden,
	boxsep = 0pt,
	boxrule = 0pt,
	borderline west = {3pt}{0pt}{orange},
	% borderline west = {3pt}{0pt}{gOrange},
	sharp corners = all,
	enhanced,
}

\newtheorem{example}{Example}[section]{\bfseries}{}
\tcolorboxenvironment{example}{
	% title = \textbf{Example},
	% detach title,
	% before upper = {\tcbtitle\quad},
	breakable,
	coltitle = black,
	colback = white,
	frame hidden,
	boxrule = 0pt,
	boxsep = 0pt,
	borderline west={3pt}{0pt}{green!70!black},
	% borderline west={3pt}{0pt}{gGreen},
	sharp corners = all,
	enhanced,
}

\newtheoremstyle{remark}{0pt}{4pt}{}{}{\bfseries\itshape}{\normalfont.}{0.5em}{}
\theoremstyle{remark}
\newtheorem*{remark}{Remark}


% TColorBoxes
\newtcolorbox{week}{
	colback = black,
	coltext = white,
	fontupper = {\large\bfseries},
	width = 1.2\paperwidth,
	size = fbox,
	halign upper = center,
	center
}

\newcommand{\banner}[2]{
    \pagebreak
    \begin{week}
   		\section*{#1}
    \end{week}
    \addcontentsline{toc}{section}{#1}
    \addtocounter{section}{1}
    \setcounter{subsection}{0}
}

% Hyperref
\usepackage{hyperref}
\hypersetup{
	colorlinks=true,
	linktoc=all,
	linkcolor=links,
	bookmarksopen=true
}

% Error Handling
\PackageWarningNoLine{ExtSizes}{It is better to use one of the extsizes 
                          classes,^^J if you can}


\fancyhead[R]{Homework \#$8$}
\fancyhead[L]{Eli Griffiths}
\renewcommand{\headrulewidth}{1pt}
\setlength\parindent{0pt}

\begin{document}
\DeclarePairedDelimiter\bangle\langle\rangle

% Section 11: 1, 2, 4, 9, 16, 20, 24, 29, 46, 54

\section*{11.1}
\[
	\mathbb{Z}_2 \times \mathbb{Z}_4 = 
	\qty{\begin{array}{l}
		(0,0), (0,1), (0,2), (0,3) \\
		(1,0), (1,1), (1,2), (1,3)
	\end{array}}
.\]

\[
	\begin{array}{ll}
		|(0,0)| = 1 & |(1,0)| = 2 \\
		|(0,1)| = 4 & |(1,1)| = 4 \\
		|(0,2)| = 2 & |(1,2)| = 2 \\
		|(0,3)| = 4 & |(1,3)| = 4
	\end{array}
.\]

\section*{11.2}
\[
	\mathbb{Z}_3 \times \mathbb{Z}_4 = 
	\qty{\begin{array}{l}
		(0,0), (0,1), (0,2), (0,3) \\
		(1,0), (1,1), (1,2), (1,3) \\
		(2,0), (2,1), (2,2), (2,3)
	\end{array}}
.\]

\[
	\begin{array}{lll}
		|(0,0)| = 1 & |(1,0)| = 3  & |(2,0)| = 3 \\
		|(0,1)| = 4 & |(1,1)| = 12 & |(2,1)| = 12 \\
		|(0,2)| = 2 & |(1,2)| = 6  & |(2,2)| = 6 \\
		|(0,3)| = 4 & |(1,3)| = 12 & |(2,3)| = 12
	\end{array}
.\]

\section*{11.4}
\[
	|(2,3)| = \text{lcm}(3, 5) = 15
.\]

\section*{11.9}
\[
	\mathbb{Z}_2 \times \mathbb{Z}_2 = 
	\qty{\begin{array}{l}
		(0,0), (0,1) \\
		(1,0), (1,1)
	\end{array}}
.\]
The proper non-trivial subgroups will be those of order $2$, hence
\begin{align*}
	&\qty{(0,0), (1,1)} \\
	&\qty{(0,0), (1,0)} \\
	&\qty{(0,0), (0,1)}
.\end{align*}

\section*{11.16}
Yes they are isomorphic since
\[
	\mathbb{Z}_2 \times \mathbb{Z}_{12} \simeq 
	\mathbb{Z}_2 \times \mathbb{Z}_3 \times \mathbb{Z}_4 \simeq
	\mathbb{Z}_4 \times \mathbb{Z}_6
.\]

\section*{11.20}
Yes they are isomorphic since
\[
	\mathbb{Z}_4 \times \mathbb{Z}_{18} \times \mathbb{Z}_{15} \simeq
	\mathbb{Z}_2 \times \mathbb{Z}_3 \times \mathbb{Z}_4 \times \mathbb{Z}_5 \times \mathbb{Z}_9 \simeq
	\mathbb{Z}_3 \times \mathbb{Z}_{36} \times \mathbb{Z}_{10}
.\]

\section*{11.24}
Note that $720 = 2^4 \cdot 3^2 \cdot 5$. From the table in 11.29, there are 5 finite abelian groups of order $2^4$, 2 finite abelian groups of order $3^2$, and there is one finite abelian group of order $5^1$. Therefore there are $1 \cdot 2 \cdot 5 = 10$ finite abelian groups of order $720$.

\section*{11.29}
\subsection*{Part A}

\begin{table}[h!]
	\centering
	\renewcommand\arraystretch{1.5}
	\begin{tabular}{c|c}
		$n$ & \# of Groups \\\hline
		2 & 2   \\\hline
		3 & 3   \\\hline
		4 & 5   \\\hline
		5 & 7   \\\hline
		6 & 11  \\\hline
		7 & 15  \\\hline
		8 & 22  \\\hline
	\end{tabular}
\end{table}

% Let an integer sequence in middle column of the table represent the direct product of $\mathbb{Z}_n$'s.
% \begin{table}[h!]
% 	\centering
% 	\begin{tabular}{c|l|c}
% 		Factors & Sequence & Total \\\hline
% 		1 & $[8]$ & 1 \\\hline
% 		2 & $[7,1], [6,2], [5,3], [4,4]$ & 4 \\\hline
% 		3 & $[6,1,1], [5,2,1], [4,3,1], [4,2,2], [3,3,2]$ & 5 \\\hline
% 		4 & $[5,1,1,1], [4,2,1,1], [3,3,1,1], [3,2,2,1], [2,2,2,2]$ & 5 \\\hline
% 		5 & $[4,1,1,1,1], [3,2,1,1,1], [2,2,2,1,1]$ & 3 \\\hline
% 		6 & $[3,1,1,1,1,1], [2,2,1,1,1,1]$ & 2 \\\hline
% 		7 & $[2,1,1,1,1,1,1]$ & 1 \\\hline
% 		8 & $[1,1,1,1,1,1,1,1]$ & 1 \\\hline
% 	\end{tabular}
% \end{table}

\subsection*{Part B}
\begin{align*}
	p^3 q^4 r^7 &\implies 3 \cdot 5 \cdot 15 = 225 \\
	q^7 r^7 &\implies 15^2 = 225 \\
	q^8 r^4 &\implies 22 \cdot 5 = 110
.\end{align*}

\section*{11.46}
\begin{proof}
	Let $G_1, G_2, \ldots G_n$ be a collection of abelian groups. Consider $G_1 \times G_2 \times \ldots \times G_n$. Let $a,b$ be elements of the direct product and $*_n$ denote the binary operation of the $n$th group in the collection.
	\begin{align*}
		ab &= (a_1, a_2, \ldots, a_n)(b_1, b_2, \ldots, b_n) \\
		&= (a_1 *_1 b_1, a_2 *_2 b_2, \ldots, a_n *_n b_n) \\
		&= (b_1 *_1 a_1, b_2 *_2 a_2, \ldots, b_n *_n a_n) \\
		&= (b_1, b_2, \ldots, b_n)(a_1, a_2, \ldots, a_n) = ba
	\end{align*}
	Therefore the direct product of abelian groups is abelian.
\end{proof}

\section*{11.54}
\begin{proof}
	Let $G,H, K$ be finitely generated abelian groups. By the Fundamental Theorem of Finitely Generated Abelian Groups, each group will have a unqiue decomposition. Since $G\times K \simeq H\times K$, the decomposition of $G\times K$ and $H\times K$ must be the same. The decomposition of both can be written by placing the decomposition of $K$ at the end. Considering then the decomposition excluding $K$'s decomposition leaves behind that the decomposition of $G$ and $H$ are isomorphic. Therefore $G$ and $H$ are isomorphic since their decompositions are isomorphic. 
\end{proof}

\end{document}
