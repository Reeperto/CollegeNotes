\documentclass[../notes.tex]{subfiles}
\graphicspath{
    {'../figures'}
}

\begin{document}

\banner{Series}

\begin{definition}[Summation]
    Given a sequence $(a_n)$ starting at $m$, then
    \[
        S_n \coloneq \sum_{k = m}^{n} a_k, n \geq m
    \]
    and $(S_n)_{n \geq m}$ is the sequence of partial sums. Then
    \[
        \sum_{k = m}^\infty a_k \text{ converges } \Leftrightarrow (S_n)_{n \geq m} \text{ converges.}
    \]
    Furthermore, if $\lim S_n = s$, then
    \[
        \lim_{n\to \infty} \sum_{k = m}^n a_k = s
    .\]
\end{definition}
\begin{remark}
    Note the following properties for the sequence of partial sums
    \begin{enumerate}[label=\alph*)]
        \item $a_k \geq 0$ for all $k \geq m$, then $(S_n)_{n \geq m}$ is an increasing sequence and either converges or diverges to $\infty$.
        \item As a consequence, $\displaystyle \sum_{k = m}^n a_k$ is always meaningful.
    \end{enumerate}
\end{remark}

The last property motivates defining another form of convergence.

\begin{definition}[Absolute Convergence]
    $\displaystyle \sum_{k = m}^\infty a_k$ converges absolutely if $\displaystyle \sum_{k=m}^\infty |a_k|$ converges.
\end{definition}

\begin{example}
    Note that $(1-r)(1 + r + r^2 + \ldots + r^n) = 1 + r^{n+1}$. Therefore
    \[
        (1 + r + r^2 + \ldots + r^n) = \sum_{k = 0}^n r^k = \frac{1 + r^{n+1}}{1-r}, \forall n \geq 0
    .\]
\end{example}

\end{document}
