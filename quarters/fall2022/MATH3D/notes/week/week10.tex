\documentclass[../notes.tex]{subfiles}
\graphicspath{
    {"../figures"}
}

\begin{document}

\subsection{More on Power Series}

Consider the geometric power series
\[
\sum_{k=0}^{\infty} x^{k} = \frac{1}{1-x}
.\]

The equivalent power series then for its square would be
\[
    \qty(\frac{1}{1-x})^2 = \sum_{k=0}^{\infty} x^{k} \cdot \sum_{k=0}^{\infty} x^{k}
.\]
Given two power series multiplied, they can be expressed as a new power series of the form
\begin{align*}
    f(x) \cdot g(x) &= \sum_{n=1}^{\infty} c_n x^{k} \\
    c_n &= \sum_{j=1}^{n} a_j b_{n-j}
.\end{align*}
Therefore for this case, $c_n = k+1$, meaning
\[
    \qty(\frac{1}{1-x})^2 = \sum_{k=0}^{\infty} (k+1)x^{k}
.\]
Consider a more difficult example
\begin{example}{Find the power series to $\frac{x}{1-4x+4x^2}$}
    Notice first that the x can be ignored in then numerator as it will just be a multiplier in the final answer. Note that
    \[
        \frac{1}{1-4x+4x^2} = \frac{1}{(1-2x)^2}
    .\]
While the power series for $\frac{1}{1-2x}$ is unknown, it is simply a scaled input to the power series previously mentioned. Therefore
\[
    \frac{1}{1-2x} = \sum_{k=0}^{\infty} 2^{k}x^{k}
.\]
Squaring the result necessitates finding the $c_n$ terms.
\begin{align*}
    a_n &= 2^{n} \\
    b_n &= 2^{n} \\
    a_j b_{n-j} &= 2^{n} \\
    c_n & =\sum_{k=1}^{n} 2^{k} \\
    c_n &= (n+1)2^{n}
.\end{align*}
Therefore the final power series representation is
\[
    \frac{x}{1-4x+4x^2} = \sum_{n=1}^{\infty} (n+1)2^{n}x^{n+1}
.\]
\end{example}

\end{document}
