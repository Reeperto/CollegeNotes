\documentclass{article}

\usepackage[margin=1in]{geometry}
\usepackage{xcolor}
\usepackage{tikz}

\usepackage{amsmath, amssymb, amsthm}

\begin{document}

{\Large This sentence is abnormally big! But it can also}
{\small become really tiny as well!}
{\scriptsize Even tinier :3.}

\textbf{\color{red}\Large This sentence is abnormally big! But it can also}
\textbf{\color{blue}\small become really tiny as well!}
\textbf{\color{green}\scriptsize Even tinier :3.}

\vspace{2cm}

Let $G = (V,E)$ be a finite graph and $H$ be a subgraph of $G$. We know from \cite{DAS2004715} that the eigenvalues of $L_G$ and $L_H$ are interwoven. That is
\[
    \lambda_1(G) \geq \lambda_1(H) \geq \lambda_2(G) \geq \lambda_2(H) \geq \ldots \geq \lambda_n(G) = \lambda_n(H) = 0
.\]

\vspace{2cm}
\noindent
An example of a quotient space in topolgy is $[0,1] / \{0,1\} \sim S^1$. Pictorally
\begin{figure}[h!]
    \centering
    \begin{tikzpicture}[
        c/.style={insert path={circle[radius=2pt,fill=black]}}
        ]
        \filldraw (-4,1) [c] -- ++(3,0) [c] node[pos=0.5,anchor=south] {$[0,1]$};
        \filldraw[dashed] (0,0) [c] -- ++(3,0) [c] node[pos=0.5,anchor=south] {$[0,1]$};
        \filldraw (-2, -1) -- ++(3,3);
        \node at (4, 0.5) {\Large $\simeq$};
        \draw (7,0.5) arc (0:360:1);
        \fill[fill=black] (5, 0.5) circle[radius=2pt] node[anchor=west] {$S^1$};
    \end{tikzpicture}
    \caption{Quotient Space of $[0,1] / \{0,1\}$}
\end{figure}

\vspace{2cm}

\noindent
The following truth table establishes logical equivalency between an implication and its contrapositive.
\[
    \begin{array}{|c|c||c|c|c|c|}
        \hline
        P & Q & P \implies Q & \lnot Q & \lnot P & \lnot Q \implies P \\
        \hline \hline
        T & T & T & F & F & T \\
        T & F & F & T & F & F \\
        F & T & T & F & T & T \\
        F & F & T & T & T & T \\\hline
    \end{array}
\]

\newpage
\bibliographystyle{alpha}
\bibliography{citations.bib}
\end{document}
