\documentclass[12pt,titlepage]{extarticle}
% Document Layout and Font
\usepackage{subfiles}
\usepackage[margin=2cm, headheight=15pt]{geometry}
\usepackage{fancyhdr}
\usepackage{enumitem}	
\usepackage{wrapfig}
\usepackage{float}
\usepackage{multicol}

\usepackage[p,osf]{scholax}

\renewcommand*\contentsname{Table of Contents}
\renewcommand{\headrulewidth}{0pt}
\pagestyle{fancy}
\fancyhf{}
\fancyfoot[R]{$\thepage$}
\setlength{\parindent}{0cm}
\setlength{\headheight}{17pt}
\hfuzz=9pt

% Figures
\usepackage{svg}

% Utility Management
\usepackage{color}
\usepackage{colortbl}
\usepackage{xcolor}
\usepackage{xpatch}
\usepackage{xparse}

\definecolor{gBlue}{HTML}{7daea3}
\definecolor{gOrange}{HTML}{e78a4e}
\definecolor{gGreen}{HTML}{a9b665}
\definecolor{gPurple}{HTML}{d3869b}

\definecolor{links}{HTML}{1c73a5}
\definecolor{bar}{HTML}{584AA8}

% Math Packages
\usepackage{mathtools, amsmath, amsthm, thmtools, amssymb, physics}
\usepackage[scaled=1.075,ncf,vvarbb]{newtxmath}

\newcommand\B{\mathbb{C}}
\newcommand\C{\mathbb{C}}
\newcommand\R{\mathbb{R}}
\newcommand\Q{\mathbb{Q}}
\newcommand\N{\mathbb{N}}
\newcommand\Z{\mathbb{Z}}

\DeclareMathOperator{\lcm}{lcm}

% Probability Theory
\newcommand\Prob[1]{\mathbb{P}\qty(#1)}
\newcommand\Var[1]{\text{Var}\qty(#1)}
\newcommand\Exp[1]{\mathbb{E}\qty[#1]}

% Analysis
\newcommand\ball[1]{\B\qty(#1)}
\newcommand\conj[1]{\overline{#1}}
\DeclareMathOperator{\Arg}{Arg}
\DeclareMathOperator{\cis}{cis}

% Linear Algebra
\DeclareMathOperator{\dom}{dom}
\DeclareMathOperator{\range}{range}
\DeclareMathOperator{\spann}{span}
\DeclareMathOperator{\nullity}{nullity}

% TIKZ
\usepackage{tikz}
\usepackage{pgfplots}
\usetikzlibrary{arrows.meta}
\usetikzlibrary{math}
\usetikzlibrary{cd}

% Boxes and Theorems
\usepackage[most]{tcolorbox}
\tcbuselibrary{skins}
\tcbuselibrary{breakable}
\tcbuselibrary{theorems}

\newtheoremstyle{default}{0pt}{0pt}{}{}{\bfseries}{\normalfont.}{0.5em}{}
\theoremstyle{default}

\renewcommand*{\proofname}{\textit{\textbf{Proof.}}}
\renewcommand*{\qedsymbol}{$\blacksquare$}
\tcolorboxenvironment{proof}{
	breakable,
	coltitle = black,
	colback = white,
	frame hidden,
	boxrule = 0pt,
	boxsep = 0pt,
	borderline west={3pt}{0pt}{bar},
	% borderline west={3pt}{0pt}{gPurple},
	sharp corners = all,
	enhanced,
}

\newtheorem{theorem}{Theorem}[section]{\bfseries}{}
\tcolorboxenvironment{theorem}{
	breakable,
	enhanced,
	boxrule = 0pt,
	frame hidden,
	coltitle = black,
	colback = blue!7,
	% colback = gBlue!30,
	left = 0.5em,
	sharp corners = all,
}

\newtheorem{corollary}{Corollary}[section]{\bfseries}{}
\tcolorboxenvironment{corollary}{
	breakable,
	enhanced,
	boxrule = 0pt,
	frame hidden,
	coltitle = black,
	colback = white!0,
	left = 0.5em,
	sharp corners = all,
}

\newtheorem{lemma}{Lemma}[section]{\bfseries}{}
\tcolorboxenvironment{lemma}{
	breakable,
	enhanced,
	boxrule = 0pt,
	frame hidden,
	coltitle = black,
	colback = green!7,
	left = 0.5em,
	sharp corners = all,
}

\newtheorem{definition}{Definition}[section]{\bfseries}{}
\tcolorboxenvironment{definition}{
	breakable,
	coltitle = black,
	colback = white,
	frame hidden,
	boxsep = 0pt,
	boxrule = 0pt,
	borderline west = {3pt}{0pt}{orange},
	% borderline west = {3pt}{0pt}{gOrange},
	sharp corners = all,
	enhanced,
}

\newtheorem{example}{Example}[section]{\bfseries}{}
\tcolorboxenvironment{example}{
	% title = \textbf{Example},
	% detach title,
	% before upper = {\tcbtitle\quad},
	breakable,
	coltitle = black,
	colback = white,
	frame hidden,
	boxrule = 0pt,
	boxsep = 0pt,
	borderline west={3pt}{0pt}{green!70!black},
	% borderline west={3pt}{0pt}{gGreen},
	sharp corners = all,
	enhanced,
}

\newtheoremstyle{remark}{0pt}{4pt}{}{}{\bfseries\itshape}{\normalfont.}{0.5em}{}
\theoremstyle{remark}
\newtheorem*{remark}{Remark}


% TColorBoxes
\newtcolorbox{week}{
	colback = black,
	coltext = white,
	fontupper = {\large\bfseries},
	width = 1.2\paperwidth,
	size = fbox,
	halign upper = center,
	center
}

\newcommand{\banner}[2]{
    \pagebreak
    \begin{week}
   		\section*{#1}
    \end{week}
    \addcontentsline{toc}{section}{#1}
    \addtocounter{section}{1}
    \setcounter{subsection}{0}
}

% Hyperref
\usepackage{hyperref}
\hypersetup{
	colorlinks=true,
	linktoc=all,
	linkcolor=links,
	bookmarksopen=true
}

% Error Handling
\PackageWarningNoLine{ExtSizes}{It is better to use one of the extsizes 
                          classes,^^J if you can}


\def\homeworknumber{3}
\fancyhead[R]{\textbf{Math 140A: Homework \#\homeworknumber}}
\fancyhead[L]{Eli Griffiths}
\renewcommand{\headrulewidth}{1pt}
\setlength\parindent{0pt}


% Section 2.1: 1, 5, 9, 15, 17, 22. 
% Section 2.2: 1, 4, 10, 14, 16. 
% Section 2.3: 3, 9, 11, 16.

\begin{document}

\subsection*{2.1.1}
\begin{enumerate}[label=\alph*)]
    \item True
    \item False
    \item False
    \item True
    \item False
    \item False
    \item True
    \item False
\end{enumerate}

\subsection*{2.1.5}
\begin{proof}
    Let $T : P_2 (\mathbb{R}) \to P_3 (\mathbb{R})$ defined by $T(f(x)) = x \cdot f(x) + f'(x)$. Let $f,g \in P_2 (\mathbb{R})$ and $c \in \mathbb{R}$. Then
    \[
        T(f + g) = x (f(x) + g(x)) + f'(x) + g'(x) = x f(x) + f'(x) + x g(x) + g'(x) = T(f) + T(g)
    \]
    and
    \[
        T(cf) = x (c\cdot f(x)) + c f'(x) = c (x f(x) + f'(x)) = c T(f).
    \]
    Therefore $T$ is a linear transformation.
\end{proof}
\begin{align*}
    \beta_{N(T)} &= \qty{ 0 } &\implies \dim(N(T)) = 0 \\
    \beta_{R(T)} &= \qty{ x, x^2 + 1, x^3 } &\implies \dim(R(T)) = 3
\end{align*}

Since $N(T) = \qty{0}$, $T$ is one-to-one but not onto since $\rank(T) \neq \dim(P_3(\mathbb{R}))$.

\subsection*{2.1.9}
\begin{enumerate}
    \item $T(0,0) = (1,0) \neq (0,0)$
    \item $cT(a_1, a_2) = (ca_1 ca_1^2) \neq (ca_1, c^2 a_1^2) = T(ca_1, ca_2)$
    \item $T(2\cdot \frac{\pi}{2}, 0) = (0,0) \neq (2,0) = 2\cdot T(\frac{\pi}{2},0)$
    \item $T((1,0) + (-1,0)) = (0,0) \neq (2,0) = T(1,0) + T(-1,0)$
    \item $T(0,0) = (1, 0) \neq (0,0)$
\end{enumerate}

\subsection*{2.1.15}
Since the only function when integrated equals zero is the zero function itself. Therefore $N(T) = \qty{0}$, therefore $T$ is one-to-one. Note as well that
\[
    T(a_n x^n + a_{n-1} x^{n-1} + \ldots + a_{2} x + a_1) = \frac{a_n}{n+1} x^{n+1} + \frac{a_{n-1}}{n-1} x^n + \ldots + \frac{a_{2}}{2} x^2 + a_1 x
\]
Since there is no constant term in the output, all constant polynomials dont have a corresponding polynomial that under $T$ would equal it. Therefore $T$ cannot be onto.

\subsection*{2.1.17}
\subsubsection*{Part A}
Since $\rank T \leq \dim V < \dim W$, $\rank T < \dim W$ and therefore $T$ is not onto.

\subsubsection*{Part B}
Since $\nullity T = \dim V - \rank T \geq \dim V - \dim W > 0$, $N(T) \neq \qty{0}$ and therefore $T$ cannot be one-to-one. 

\subsection*{2.1.22}
For $T : \mathbb{R}^3 \to \mathbb{R}$, let $a = T(1,0,0), b = T(0,1,0), c = T(0,0,1)$. Note then that
\[
    T(x,y,z) = x T(1,0,0) + y T(0,1,0) + z T(0,0,1) = ax + by + cz
\]
Now generally:
\begin{theorem}
    Let $T : \mathbb{F}^n \to \mathbb{F}$ be linear. Then there exists scalars $a_i \in \mathbb{F}$ such that $T(x_1, x_2, \ldots, x_n) = a_1 x_1 + a_2 x_2 + \ldots a_n x_n$.
\end{theorem}
\begin{proof}
    Let $T : \mathbb{F}^n \to \mathbb{F}$ be linear. Let $e_i$ denote the vector where the $i$th position is one and all other's are zero. Let $a_i = T(e_i)$ where $1 \leq i \leq n$. Note then that
    \[
        T(x_1, x_2, x_3, \ldots, x_n) = \sum_{i=1}^{n} x_i T(e_i) = \sum_{i=1}^n a_i x_i.
    \]
\end{proof}

\subsection*{2.2.1}
\begin{enumerate}[label=\alph*)]
    \item True
    \item True
    \item False
    \item True
    \item True
    \item False
\end{enumerate}

\subsection*{2.2.4}
\[
    [T]_\beta^\gamma = \mqty(
    1 & 1 & 0 & 0 \\
    0 & 0 & 0 & 2 \\
    0 & 1 & 0 & 0
    )
\]

\subsection*{2.2.10}
\[
    [T]_\beta = \mqty(
    1 & 0 & \cdots & 0 & 0\\
    1 & 1 & 0 &   & 0\\
    0 & 1 & 1 & \ddots & \vdots\\
    \vdots &   & \ddots & \ddots & 0\\
    0 & \cdots & 0 & 1 & 1\\
    )
\]

\subsection*{2.2.14}
\begin{proof}
    Let $V = P(\mathbb{R})$ and $T_j(f) = f^{(j)}(x)$. Let $n\in \mathbb{N}$ and assume that $\sum_{j=0}^{n} a_i T_i = 0$. Note that $T_j(x^n) = \frac{n!}{(n-j)!} x^{n-j}$. It is clear that for different $j$, the results are linearly independent since the degrees are different. Therefore $\sum_{j=0}^{n} a_i T_i(x^n) = 0$ implies that $a_i = 0$ for all $i$. Hence $\qty{T_1, T_2, \ldots, T_n}$ is linearly independent. 
\end{proof}

\subsection*{2.2.16}
\begin{proof}
    Let $V$ and $W$ be vector spaces such that $\dim V = \dim W = n$ and let $T : V \to W$ be linear. Let $\qty{u_1, u_2, u_3, \ldots u_m}$ be a basis for $N(T)$ and extend it to be a basis $\beta = \qty{u_1, \ldots, u_m, v_{m+1}, v_{m+2}, \ldots v_n}$ of $V$. Examine the linear indepence of the set $\qty{T(v_{m+1}), T(v_{m+2}), \ldots, T(v_n)}$. Assume that
    \[
        a_{m+1} T(v_{m+1}) + a_{m+2} T(v_{m+2}) + \ldots + a_n T(v_n) = 0.
    \]
    Then
    \[
        T(a_{m+2} v_{m+1} + a_{m+2} v_{m+2} + \ldots + a_n v_n) = 0.
    \]
    Therefore $a_{m+2} v_{m+1} + a_{m+2} v_{m+2} + \ldots + a_n v_n \in N(T)$ and therefore can be expressed as linear combination of basis vectors of $N(T)$. That is
    \[
        \sum_{i=m+1}^{n} a_i v_i = \sum_{i=1}^{m} b_i u_i \implies \sum_{i=1}^{m} b_i u_i - \sum_{i=m+1}^{n} a_i v_i = w = 0.
    \]
    However, note that $w$ is a linear combination of the basis $\beta$. Therefore the summation is linearly independent meaning all coefficients, and therefore all $a_i$, must equal $0$. Therefore $\qty{T(v_{m+1}), T(v_{m+2}), \ldots, T(v_n)}$ is linearly independent and hence can be extened to be a basis $\gamma$ of $W$ with $\gamma = \qty{w_1, w_2, \ldots, w_m, T(v_{m+1}), T(v_{m+2}), \ldots, T(v_n)}$. Therefore
    \[
        [T]_{\beta}^\gamma =
        \mqty(
         0 & & & & &\\
         & \ddots & & & &\\
         & &0 & & & \\
         & & & 1 & & \\
         & & & & \ddots & \\
         & & & & & 1
         )
    \]
    which is a diagonal matrix.
\end{proof}

\subsection*{2.3.3}
\subsubsection*{Part A}
\begin{align*}
    [U]_{\beta}^{\gamma} &= \mqty(
        1 & 1  & 0 \\
        0 & 0  & 1 \\
        1 & -1 & 0 \\
    ) \\
    [T]_{\beta} &= \mqty(
        2 & 3 & 0 \\
        0 & 3 & 8 \\
        0 & 0 & 2 \\
    ) \\
    [UT]_{\beta}^\gamma &= \mqty(
        2 & 6 & 8 \\
        0 & 0 & 2 \\
        2 & 0 & -8 \\
    ) = \mqty(
        1 & 1  & 0 \\
        0 & 0  & 1 \\
        1 & -1 & 0 \\
    ) \times \mqty(
        2 & 3 & 0 \\
        0 & 3 & 8 \\
        0 & 0 & 2 \\
    ) \;\; \checkmark
\end{align*}

\subsection*{2.3.9}
Take $T(a, b) = (0, a)$ and $U(a,b) = (a, 0)$. Note then that
\[
    UT(a,b) = U(T(a,b)) = U(0,a) = (0,0)
\]
but that
\[
    TU(a,b) = T(U(a,b)) = T(a,0) = (0, a) \neq (0,0)
\]
Therefore by using the standard basis for $\mathbb{F}^2$,
\begin{align*}
    \mqty(
    1 & 0 \\
    0 & 0 \\
    )
    \mqty(
    0 & 0 \\
    1 & 0 \\
    ) &= \mqty(
    0 & 0 \\  
    0 & 0 \\  
    ) \\
    \mqty(
    0 & 0 \\
    1 & 0 \\
    )
    \mqty(
    1 & 0 \\
    0 & 0 \\
    ) &= \mqty(
    0 & 0 \\  
    1 & 0 \\  
    ) \\
\end{align*}

\subsection*{2.3.11}
\begin{proof}
    Let $V$ be a vector space and $T : V \to V$ be linear.
    \begin{enumerate}
        \item[$\Rightarrow)$] %------------------------------------------------
        Assume that $T^2 = T_0$. Let $w \in R(T)$. Then $\exists v \in V$ such that $w = T(v)$.Note then that $T(w) = T(T(v)) = 0$. Therefore $w \in N(T)$, meaning $R(T) \subseteq N(T)$.
        \item[$\Leftarrow)$] %------------------------------------------------
        Assume that $R(T) \subseteq N(T)$. That is, an element $w \in R(T)$ is in $N(T)$ meaning $T(w) = 0$. Since $w \in R(T)$, $\exists v \in V$ such that $T(v) = w$. Hence $T(T(v)) = 0$ meaning $T^2 = T_0$.
    \end{enumerate}
\end{proof}

\subsection*{2.3.16}
Let $V$ be a finite dimensional vector space and $T : V \to V$ be linear.

\subsubsection*{Part A}
\begin{proof}
    Assume that $\rank T = \rank T^2$. First, note that $N(T) \subseteq N(T^2)$ since for $x \in T(V)$, $T^2(x) = T(0) = 0$ hence $x \in N(T^2)$. Furthermore, $N(T) = N(T^2)$ since $\nullity T^2 = \dim V - \rank T^2 = \dim V - \rank T = \nullity T$. Let $v \in N(T) \cap R(T)$. Then $v \in R(T)$ meaning there is $u \in V$ such that $T(u) = v$. Since $v \in N(T)$ as well, $T(T(u)) = T(v) = 0$, meaning $u \in N(T^2)$ and therefore $u \in N(T)$. This means that $T(u) = 0 = v$ and therefore $v = 0$, hence $R(T) \cap N(T) = \qty{0}$.
\end{proof}

Note that since $R(T) \cap N(T) = \qty{0}$, it follows that $R(T) \oplus N(T)$ is well defined and that $R(T) \oplus N(T) = R(T) + N(T)$. Since $R(T) \subseteq V$ and $N(T)\subseteq V$, $R(T) + N(T) \subseteq V$. Note that $\dim(R(T) + N(T)) = \rank T + \nullity T - \dim(R(T) \cap N(t)) = \rank T + \nullity T + 0 = \dim V$. Therefore it follows that $V = R(T) \oplus N(T)$.

\subsubsection*{Part B}
\begin{proof}
    Note that $R(T^{k+1}) \subseteq R(T^k)$ for any $k$ since $v \in R(T^{k+1})$ means there is some $a \in V$ such that $T^{k+1}(a) = T^k(T(a)) = v$ meaning $v \in R(T^{k})$. Assume towards contradiction that there is no $k \in \mathbb{N}_0$ such that $R(T^k) \subseteq R(T^{k+1})$. Then $R(T^{k+1}) \subset R(T^k)$ meaning $\rank T^{k+1} < \rank T^k$ for all $k$. This gives rise to an infinite chain 
    \[
        0 \leq \ldots <
        \rank T^3 <
        \rank T^2 <
        \rank T
    \]
    However, since $\rank T \leq \dim V$, it follows that there is an infinite chain of distinct powers of $T$ between $0$ and $\dim V$ which is only finite. Therefore there must be some $k \in \mathbb{N}_0$ such that $R(T^k) \subset R(T^{k+1})$ and therefore $R(T^k) = R(T^{k+1})$. Hence $\rank T^k = \rank T^{k+1}$. By using the arguement from part A and using the fact that $\rank T^k = \rank T^{k+1}$ instead of $\rank T = \rank T^2$, it will follow that $V = R(T^k) \oplus N(T^k)$.
\end{proof}

\end{document}
