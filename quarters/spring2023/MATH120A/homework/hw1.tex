\documentclass[12pt]{extarticle}

\usepackage{multicol}

% Document Layout and Font
\usepackage{subfiles}
\usepackage[margin=2cm, headheight=15pt]{geometry}
\usepackage{fancyhdr}
\usepackage{enumitem}	
\usepackage{wrapfig}
\usepackage{float}
\usepackage{multicol}

\usepackage[p,osf]{scholax}

\renewcommand*\contentsname{Table of Contents}
\renewcommand{\headrulewidth}{0pt}
\pagestyle{fancy}
\fancyhf{}
\fancyfoot[R]{$\thepage$}
\setlength{\parindent}{0cm}
\setlength{\headheight}{17pt}
\hfuzz=9pt

% Figures
\usepackage{svg}

% Utility Management
\usepackage{color}
\usepackage{colortbl}
\usepackage{xcolor}
\usepackage{xpatch}
\usepackage{xparse}

\definecolor{gBlue}{HTML}{7daea3}
\definecolor{gOrange}{HTML}{e78a4e}
\definecolor{gGreen}{HTML}{a9b665}
\definecolor{gPurple}{HTML}{d3869b}

\definecolor{links}{HTML}{1c73a5}
\definecolor{bar}{HTML}{584AA8}

% Math Packages
\usepackage{mathtools, amsmath, amsthm, thmtools, amssymb, physics}
\usepackage[scaled=1.075,ncf,vvarbb]{newtxmath}

\newcommand\B{\mathbb{C}}
\newcommand\C{\mathbb{C}}
\newcommand\R{\mathbb{R}}
\newcommand\Q{\mathbb{Q}}
\newcommand\N{\mathbb{N}}
\newcommand\Z{\mathbb{Z}}

\DeclareMathOperator{\lcm}{lcm}

% Probability Theory
\newcommand\Prob[1]{\mathbb{P}\qty(#1)}
\newcommand\Var[1]{\text{Var}\qty(#1)}
\newcommand\Exp[1]{\mathbb{E}\qty[#1]}

% Analysis
\newcommand\ball[1]{\B\qty(#1)}
\newcommand\conj[1]{\overline{#1}}
\DeclareMathOperator{\Arg}{Arg}
\DeclareMathOperator{\cis}{cis}

% Linear Algebra
\DeclareMathOperator{\dom}{dom}
\DeclareMathOperator{\range}{range}
\DeclareMathOperator{\spann}{span}
\DeclareMathOperator{\nullity}{nullity}

% TIKZ
\usepackage{tikz}
\usepackage{pgfplots}
\usetikzlibrary{arrows.meta}
\usetikzlibrary{math}
\usetikzlibrary{cd}

% Boxes and Theorems
\usepackage[most]{tcolorbox}
\tcbuselibrary{skins}
\tcbuselibrary{breakable}
\tcbuselibrary{theorems}

\newtheoremstyle{default}{0pt}{0pt}{}{}{\bfseries}{\normalfont.}{0.5em}{}
\theoremstyle{default}

\renewcommand*{\proofname}{\textit{\textbf{Proof.}}}
\renewcommand*{\qedsymbol}{$\blacksquare$}
\tcolorboxenvironment{proof}{
	breakable,
	coltitle = black,
	colback = white,
	frame hidden,
	boxrule = 0pt,
	boxsep = 0pt,
	borderline west={3pt}{0pt}{bar},
	% borderline west={3pt}{0pt}{gPurple},
	sharp corners = all,
	enhanced,
}

\newtheorem{theorem}{Theorem}[section]{\bfseries}{}
\tcolorboxenvironment{theorem}{
	breakable,
	enhanced,
	boxrule = 0pt,
	frame hidden,
	coltitle = black,
	colback = blue!7,
	% colback = gBlue!30,
	left = 0.5em,
	sharp corners = all,
}

\newtheorem{corollary}{Corollary}[section]{\bfseries}{}
\tcolorboxenvironment{corollary}{
	breakable,
	enhanced,
	boxrule = 0pt,
	frame hidden,
	coltitle = black,
	colback = white!0,
	left = 0.5em,
	sharp corners = all,
}

\newtheorem{lemma}{Lemma}[section]{\bfseries}{}
\tcolorboxenvironment{lemma}{
	breakable,
	enhanced,
	boxrule = 0pt,
	frame hidden,
	coltitle = black,
	colback = green!7,
	left = 0.5em,
	sharp corners = all,
}

\newtheorem{definition}{Definition}[section]{\bfseries}{}
\tcolorboxenvironment{definition}{
	breakable,
	coltitle = black,
	colback = white,
	frame hidden,
	boxsep = 0pt,
	boxrule = 0pt,
	borderline west = {3pt}{0pt}{orange},
	% borderline west = {3pt}{0pt}{gOrange},
	sharp corners = all,
	enhanced,
}

\newtheorem{example}{Example}[section]{\bfseries}{}
\tcolorboxenvironment{example}{
	% title = \textbf{Example},
	% detach title,
	% before upper = {\tcbtitle\quad},
	breakable,
	coltitle = black,
	colback = white,
	frame hidden,
	boxrule = 0pt,
	boxsep = 0pt,
	borderline west={3pt}{0pt}{green!70!black},
	% borderline west={3pt}{0pt}{gGreen},
	sharp corners = all,
	enhanced,
}

\newtheoremstyle{remark}{0pt}{4pt}{}{}{\bfseries\itshape}{\normalfont.}{0.5em}{}
\theoremstyle{remark}
\newtheorem*{remark}{Remark}


% TColorBoxes
\newtcolorbox{week}{
	colback = black,
	coltext = white,
	fontupper = {\large\bfseries},
	width = 1.2\paperwidth,
	size = fbox,
	halign upper = center,
	center
}

\newcommand{\banner}[2]{
    \pagebreak
    \begin{week}
   		\section*{#1}
    \end{week}
    \addcontentsline{toc}{section}{#1}
    \addtocounter{section}{1}
    \setcounter{subsection}{0}
}

% Hyperref
\usepackage{hyperref}
\hypersetup{
	colorlinks=true,
	linktoc=all,
	linkcolor=links,
	bookmarksopen=true
}

% Error Handling
\PackageWarningNoLine{ExtSizes}{It is better to use one of the extsizes 
                          classes,^^J if you can}


\newcommand{\powerset}[1]{\mathcal{P}(#1)}
\fancyhead[R]{\textbf{Math 120A: Homework \#1}}
\fancyhead[L]{Eli Griffiths}
\renewcommand{\headrulewidth}{1pt}
\setlength\parindent{0pt}

\usetikzlibrary{arrows.meta}

\begin{document}

\section*{Problem 0.11}

\[
	\qty{a,b,c} \times \qty{1,2,c} = \qty{(a,1),(a,2),(a,c),(b,1),(b,2),(b,c),(c,1),(c,2),(c,c)}
.\]

\section*{Problem 0.12}
\subsection*{Part A}

The relation is a function mapping $A$ into $B$. The function is not one-to-one since $(1,4)$ and $(2,4)$ are in the relation. The function is not onto $B$ either since there is no $a \in A$ that maps to a $b=2$.

\subsection*{Part B}

The relation is a function mapping $A$ into $B$. The function is not one-to-one since $(1,4)$ and $(3,4)$ are in the relation. The function is not onto $B$ either since there is no $a \in A$ that maps to a $b=2$.

\section*{Problem 0.18}

\begin{proof}
	Let $A$ be any set and let $B^A$ be the set of all functions mapping $A$ into the set $B = \qty{0,1}$. Define a map $\phi: B^A \to \powerset{A}$ where given a function $f \in B^A$, it is defined that $\phi(f) = \qty{a \in A : f(a) = 1}$. Proceed to show that $\phi$ is bijective. \\

	\qquad \begin{minipage}{\dimexpr\textwidth - 3cm}
		(One-to-One)\quad Let $f,g \in B^A$ and assume that $\phi(f) = \phi(g)$. Therefore for some element $x \in A$, $f(x) = 1$ if and only if $g(x) = 0$. Since $f$ and $g$ only take on two possible values, it also follows that $f(x) = 0$ if and only if $g(x) = 0$. Therefore $f = g$, meaning $\phi$ is one-to-one.
	\end{minipage} \\
	\\

	\qquad \begin{minipage}{\dimexpr\textwidth - 3cm}
		(Onto)\quad Let $S \in \powerset{A}$. Therefore $S \subseteq A$. Define $\theta : A \rightarrow \qty{0,1}$ by 
		\[
			\theta(x) = \begin{cases}
				1, &x \in S \\ 
				0, &x \notin S
			\end{cases}
		.\]
		Note that $\phi(\theta) = S$, hence $\phi$ is onto.
	\end{minipage} \\
	\\

	Since $\phi$ is both one-to-one and onto, it is a bijection between $B^A$ and $\powerset{A}$ meaning they have the same cardinality.
\end{proof}

\section*{Problem 0.19}

\begin{proof}
	Assume towards contradiction there exists a one-to-one map $\phi : A \rightarrow \powerset{A}$. Define $S = \qty{x \in A : x \notin \phi(x)}$. Let $a \in A$ and consider two cases. If $a \in \phi(a)$, then $a \notin S$. If $a \notin \phi(a)$, then $a \in S$. Therefore $S$ and $\phi(a)$ are different subsets of $A$ since $a$ is only ever in one but not the either. Hence $\phi$ cannot be one-to-one, which is a contradiction. Therefore $\phi$ cannot exist.
\end{proof}

The set of everything is not a logically acceptable. If $T$ denotes this supposed set of everything, then $\powerset{T}$ would be larger than $T$, contradicting the fact that it already the set of everything.

\section*{Problem 0.32}

\begin{proof}
	Proof that $\mathcal{R}$ is not an equivalence relation. Examine transitivity. Let $a,b,c \in \mathbb{R}$ with $a = 1, b = 3$ and $c = 6$. Note that $a \mathcal{R} b$ since $|a - b| = |1 - 3| = 2 \leq 3$ and $b \mathcal{R} c$ since $|b - c| = |3 - 6| = 3 \leq 3$. However $a$ is not related to $c$ since $|a - c| = |1 - 6| = 5 \nleq 3$. Therefore $\mathcal{R}$ is not transitive, meaning it is not an equivalence relation.
\end{proof}

\section*{Problem 1.4}
\[
	(-i)^{35} = (-1)^{35} \cdot i^{35} = -(i^3) = i
.\]

\section*{Problem 1.20}

\begin{align*}
	z^6 = 1 \implies &|z^6| (\cos(6\theta) + i \sin(6\theta)) = 1 \\
	&|z^6| (\cos(6\theta) + i \sin(6\theta)) = 1(1 + 0\cdot i)
.\end{align*}

Therefore $|z| = 1, \cos(6\theta) = 1$ and $\sin(6\theta) = 0$. This implies that $\theta = \frac{\pi n}{3}$ for $n \in \mathbb{Z}$. Thus the solutions when $\theta \in [0, 2\pi)$ are
\[
	z = \qty{e^{\frac{\pi n}{3}} : n \in \qty{0,1,2,3,4,5}}
.\]

\section*{Problem 1.22}
\[
	10 +_{17} 16 = 9
.\]

\section*{Problem 1.25}

\[
	\frac{1}{2} +_1 \frac{7}{8} = \frac{3}{8}
.\]

\section*{Problem 1.26}

\[
	\frac{3\pi}{4} +_{2\pi} \frac{3\pi}{2} = \frac{\pi}{4}
.\]

\section*{Problem 1.29}
Solutions will be when $x + 7 = 15 + 3$, meaning $x = 11$. All other possible values of $x \in \mathbb{Z}_15$ do not work.

\section*{Problem 1.36}
\begin{align*}
	\zeta &\leftrightarrow 4 \\
	\zeta^2 = \zeta \cdot \zeta &\leftrightarrow 4+_7 4 = 1 \\
	\zeta^3 = \zeta^2 \cdot \zeta &\leftrightarrow 4+_7 1 = 5 \\
	\zeta^4 = \zeta^2 \cdot \zeta^2 &\leftrightarrow 1+_7 1 = 2 \\
	\zeta^5 = \zeta^4 \cdot \zeta &\leftrightarrow 5+_7 1 = 6 \\
	\zeta^6 = \zeta^5 \cdot \zeta &\leftrightarrow 6+_7 4 = 3 \\
	\zeta^0 = \zeta^5 \cdot \zeta^2 &\leftrightarrow 6+_7 1 = 0
\end{align*}

\section*{Problem 1.37}
If there was an isomorphism between $U_6$ and $\mathbb{Z}_6$, then $\zeta^2 = \zeta \cdot \zeta \leftrightarrow 4 +_6 4 = 2$ and $\zeta^4 = \zeta^2 \cdot \zeta^2 \leftrightarrow 2 +_6 2 = 4$. However that means $\zeta$ and $\zeta^4$ correspond to the same value which contradicts the requirement that an ismorphism is one-to-one.

\section*{Problem 2.1}
\begin{align*}
	b * d &= e \\
	\\
	c * c &= b \\
	\\
	[(a * c) * e] * a &= [c * e] * a  \\
										&= a * a \\
										&= a
.\end{align*}

\section*{Problem 2.3}

\begin{align*}
	(b * d) * c &= e * c \\
							&= a
.\end{align*}

\begin{align*}
	b * (d * c) &= b * b \\
							&= c
.\end{align*}

This computation does imply that $*$ in this instance is not associative since the results of each computation are not equal to each other.

\section*{Problem 2.5}

\begin{center}
	\begin{tabular}{c|c|c|c|c}
		$*$ & $a$ & $b$ & $c$ & $d$ \\\hline
		$a$ & $a$ & $b$ & $c$ & $d$ \\\hline
		$b$ & $b$ & $d$ & $a$ & $c$ \\\hline
		$c$ & $c$ & $a$ & $d$ & $b$ \\\hline
		$d$ & $d$ & $c$ & $b$ & $a$ \\
	\end{tabular}
\end{center}

\section*{Problem 2.8}

$*$ is commutatitive but not associative.

\begin{proof}
	First examine commutativity. Let $a,b \in \mathbb{Q}$. Note then that
	\[
		a * b = ab + 1 = ba + 1 = b * a
	.\]
	Therefore $*$ is commutative. Now examine associativity. Let $a = 1, b = 2, c = 3$. Then note that
	\begin{align*}
		(a * b) * c &= 3 * c \\
								&= 3 * 3 \\
								&= 10
	.\end{align*}
	and that
	\begin{align*}
		a * (b * c) &= a * 7 \\
								&= 1 * 7 \\
								&= 8
	.\end{align*}
	In this case $(a * b) * c \neq a * (b * c)$, meaning $*$ is not associative.
\end{proof}

\section*{Problem 2.10}

$*$ is commutative but not associative.

\begin{proof}
	First examine commutativity. Let $a,b \in \mathbb{Z}^+$. Note then that
	\[
		a * b = 2^{ab} = 2^{ba} = b * a
	.\]
	Therefore $*$ is commutative. Now examine associativity. Let $a = 1, b = 2, c = 3$. Then note that
	\begin{align*}
		(a * b) * c &= 2^{2} * c \\
								&= 4 * 3 \\
								&= 2^{12}
	.\end{align*}
	and that
	\begin{align*}
		a * (b * c) &= a * 2^{6} \\
								&= 1 * 64 \\
								&= 2^{64}
	.\end{align*}
	In this case $(a * b) * c \neq a * (b * c)$, meaning $*$ is not associative.
\end{proof}

\section*{Problem 2.12}

Consider the tabular representation of $*$. Given the set $S$ that $*$ is over, define $n = |S|$. The table will therefore have $n^2$ entries in it. Each entry has $n$ choices as it can be any element of $S$. Therefore since you have $n$ choices $n^2$ times
\[
	\textit{Number of possible binary operations} = n^{(n^2)}
.\]

\section*{Problem 2.17}

$*$ does not give a binary operation since it breaks Condition $2$. Consider $a = 1, b = 2$. Both $1$ and $2$ are in $\mathbb{Z}^+$, however $a * b = 1 - 2 = -1 \notin \mathbb{Z}^+$.

\section*{Problem 2.20}

$*$ is a binary operation since it obeys both conditions.

\section*{Problem 2.23}

$H$ is closed under under both matrix addition and matrix multiplication.

\begin{proof}
	Consider first matrix addition on $H$. Let $a,b,c,d \in \mathbb{R}$
\[
	\mqty[a & -b \\ b & a] + \mqty[c & -d \\ d & c] = \mqty[(a+c) & - (b + d) \\ (b + d) & (a + c)]
.\]

\[
	\mqty[a & -b \\ b & a] \mqty[c & -d \\ d & c] =
	\mqty[
	ac - bd & -(ad + bc) \\
	ad + bc & ac - bd
	]
.\]
\end{proof}


\section*{Problem 2.24}
\section*{Problem 2.36}


\end{document}

% 20 * 18 * 16 * 14 * 12 * 10 = 9,676,800 ways to select 6 non couple people
% order doesn't matter therefore divide out the ways to possibly arrange these 6 people. Therefore
%
% Choices = 13,440

% Additionally, there are 10 choose 6 ways to pick 6 couples and there are 2^6 ways to choose a person from each couple,
% Therefore there are 2^6 * (10 choose 6) = 13,440 ways. Same result

% 10 * 8 * 6 men choices
% 10 * 8 * 6 women choices

% 230,400 / 14,400 = 16
