\documentclass[12pt]{extarticle}

% Document Layout and Font
\usepackage{subfiles}
\usepackage[margin=2cm, headheight=15pt]{geometry}
\usepackage{fancyhdr}
\usepackage{enumitem}	
\usepackage{wrapfig}
\usepackage{float}
\usepackage{multicol}

\usepackage[p,osf]{scholax}

\renewcommand*\contentsname{Table of Contents}
\renewcommand{\headrulewidth}{0pt}
\pagestyle{fancy}
\fancyhf{}
\fancyfoot[R]{$\thepage$}
\setlength{\parindent}{0cm}
\setlength{\headheight}{17pt}
\hfuzz=9pt

% Figures
\usepackage{svg}

% Utility Management
\usepackage{color}
\usepackage{colortbl}
\usepackage{xcolor}
\usepackage{xpatch}
\usepackage{xparse}

\definecolor{gBlue}{HTML}{7daea3}
\definecolor{gOrange}{HTML}{e78a4e}
\definecolor{gGreen}{HTML}{a9b665}
\definecolor{gPurple}{HTML}{d3869b}

\definecolor{links}{HTML}{1c73a5}
\definecolor{bar}{HTML}{584AA8}

% Math Packages
\usepackage{mathtools, amsmath, amsthm, thmtools, amssymb, physics}
\usepackage[scaled=1.075,ncf,vvarbb]{newtxmath}

\newcommand\B{\mathbb{C}}
\newcommand\C{\mathbb{C}}
\newcommand\R{\mathbb{R}}
\newcommand\Q{\mathbb{Q}}
\newcommand\N{\mathbb{N}}
\newcommand\Z{\mathbb{Z}}

\DeclareMathOperator{\lcm}{lcm}

% Probability Theory
\newcommand\Prob[1]{\mathbb{P}\qty(#1)}
\newcommand\Var[1]{\text{Var}\qty(#1)}
\newcommand\Exp[1]{\mathbb{E}\qty[#1]}

% Analysis
\newcommand\ball[1]{\B\qty(#1)}
\newcommand\conj[1]{\overline{#1}}
\DeclareMathOperator{\Arg}{Arg}
\DeclareMathOperator{\cis}{cis}

% Linear Algebra
\DeclareMathOperator{\dom}{dom}
\DeclareMathOperator{\range}{range}
\DeclareMathOperator{\spann}{span}
\DeclareMathOperator{\nullity}{nullity}

% TIKZ
\usepackage{tikz}
\usepackage{pgfplots}
\usetikzlibrary{arrows.meta}
\usetikzlibrary{math}
\usetikzlibrary{cd}

% Boxes and Theorems
\usepackage[most]{tcolorbox}
\tcbuselibrary{skins}
\tcbuselibrary{breakable}
\tcbuselibrary{theorems}

\newtheoremstyle{default}{0pt}{0pt}{}{}{\bfseries}{\normalfont.}{0.5em}{}
\theoremstyle{default}

\renewcommand*{\proofname}{\textit{\textbf{Proof.}}}
\renewcommand*{\qedsymbol}{$\blacksquare$}
\tcolorboxenvironment{proof}{
	breakable,
	coltitle = black,
	colback = white,
	frame hidden,
	boxrule = 0pt,
	boxsep = 0pt,
	borderline west={3pt}{0pt}{bar},
	% borderline west={3pt}{0pt}{gPurple},
	sharp corners = all,
	enhanced,
}

\newtheorem{theorem}{Theorem}[section]{\bfseries}{}
\tcolorboxenvironment{theorem}{
	breakable,
	enhanced,
	boxrule = 0pt,
	frame hidden,
	coltitle = black,
	colback = blue!7,
	% colback = gBlue!30,
	left = 0.5em,
	sharp corners = all,
}

\newtheorem{corollary}{Corollary}[section]{\bfseries}{}
\tcolorboxenvironment{corollary}{
	breakable,
	enhanced,
	boxrule = 0pt,
	frame hidden,
	coltitle = black,
	colback = white!0,
	left = 0.5em,
	sharp corners = all,
}

\newtheorem{lemma}{Lemma}[section]{\bfseries}{}
\tcolorboxenvironment{lemma}{
	breakable,
	enhanced,
	boxrule = 0pt,
	frame hidden,
	coltitle = black,
	colback = green!7,
	left = 0.5em,
	sharp corners = all,
}

\newtheorem{definition}{Definition}[section]{\bfseries}{}
\tcolorboxenvironment{definition}{
	breakable,
	coltitle = black,
	colback = white,
	frame hidden,
	boxsep = 0pt,
	boxrule = 0pt,
	borderline west = {3pt}{0pt}{orange},
	% borderline west = {3pt}{0pt}{gOrange},
	sharp corners = all,
	enhanced,
}

\newtheorem{example}{Example}[section]{\bfseries}{}
\tcolorboxenvironment{example}{
	% title = \textbf{Example},
	% detach title,
	% before upper = {\tcbtitle\quad},
	breakable,
	coltitle = black,
	colback = white,
	frame hidden,
	boxrule = 0pt,
	boxsep = 0pt,
	borderline west={3pt}{0pt}{green!70!black},
	% borderline west={3pt}{0pt}{gGreen},
	sharp corners = all,
	enhanced,
}

\newtheoremstyle{remark}{0pt}{4pt}{}{}{\bfseries\itshape}{\normalfont.}{0.5em}{}
\theoremstyle{remark}
\newtheorem*{remark}{Remark}


% TColorBoxes
\newtcolorbox{week}{
	colback = black,
	coltext = white,
	fontupper = {\large\bfseries},
	width = 1.2\paperwidth,
	size = fbox,
	halign upper = center,
	center
}

\newcommand{\banner}[2]{
    \pagebreak
    \begin{week}
   		\section*{#1}
    \end{week}
    \addcontentsline{toc}{section}{#1}
    \addtocounter{section}{1}
    \setcounter{subsection}{0}
}

% Hyperref
\usepackage{hyperref}
\hypersetup{
	colorlinks=true,
	linktoc=all,
	linkcolor=links,
	bookmarksopen=true
}

% Error Handling
\PackageWarningNoLine{ExtSizes}{It is better to use one of the extsizes 
                          classes,^^J if you can}


\fancyhead[R]{Homework \#8}
\fancyhead[L]{Eli Griffiths}
\renewcommand{\headrulewidth}{1pt}
\setlength\parindent{0pt}

\begin{document}

\section*{Problem 1}
\subsection*{Part A}
\begin{align*}
	H_0 &: \mu_1 - \mu_2 = 0 \\
	H_\alpha &: \mu_1 - \mu_2 \neq 0
.\end{align*}

\[
	t = \frac{\overline{X}_1 - \overline{X}_2}{\sqrt{
			\frac{s_1^2}{n_1} + \frac{s_2^2}{n_2}
	}} = 0.8769
.\]
The p-value in this instance will be $2\cdot\Prob{T > 0.8769}$

\subsection*{Part B}
If the the p-value is closer to 1 and larger than 0.05, then we would be unable to make a conclusion about the null or alternative hypothesis as we will have only failed to reject the null hypothesis.

\subsection*{Part C}
The form of the confidence interval will be $(\overline{X}_1 - \overline{X}_2) \pm t^* \cdot \sqrt{\frac{s_1^2}{n_1} + \frac{s_2^2}{n_2}}$ where $\overline{X}_1 = 12.5, \overline{X}_2 = 12.2, s_1^2 = 9, s_2^2 = 16, n_1 = 196, n_2 = 225, t^* = 1.972$. The resulting interval is
\[
	(-0.3746, 0.9746)
.\]

\subsection*{Part D}
Since the confidence interval contains 0, there is not evidence at a 95\% confidence level to say that there is a difference between the two servers, which is the same conclusion reached using the p-value analysis. Both fail to show sufficient evidence that the two server response times are different.

\section*{Problem 2}
\subsection*{Part A}
\begin{align*}
	H_0 &: \mu_1 - \mu_2 \leq 0 \\
	H_\alpha &: \mu_1 - \mu_2 > 0
.\end{align*}
In this instance, $\mu_1$ is length of the morning commute for week 1 and $\mu_2$ for week 2. The meaning of $\mu_1 - \mu_2 \leq 0$ is that the commute of week 1 compared to week 2 is eithe same or longer.

\subsection*{Part B}
\[
	t = \frac{-1.714}{\frac{2.984}{\sqrt{7}}} = -1.5197
.\]

\subsection*{Part C}
\[
	p = \Prob{T > -1.5197} = \verb|1 - dt(-1.5197, 7)| = 0.8769
.\]

\subsection*{Part D}
Since $0.8769 > 0.15$, we fail to reject the null hypothesis. That is we do not have sufficient evidence to claim that the the commute time of week 2 is less than that of week 1.

\section*{Problem 3}
\subsection*{Part A}
\begin{align*}
	H_0 &: \mu \geq 35 \\
	H_\alpha &: \mu < 35
.\end{align*}
In this instance, $\mu$ is the sample dollar amount per transaction. The null hypothesis is that the amount per transaction hasnt changed or has increased, and the alternative hypothesis is that the amount per transaction has decreased.

\subsection*{Part B}
\[
	t = \frac{32.5 - 35}{\frac{2.5}{\sqrt{40}}} = -6.32456
.\]

\subsection*{Part C}
\[
	p = \Prob{T < -6.32456} = \verb|pt(-6.32456, 39)| = 9.175\times 10^{-8}
.\]

\subsection*{Part D}
Since $p << 0.15$, we have sufficient evidence to reject the null hypothesis, meaning there is evidence that the dollar amount per transaction the business makes is less than 35 dollars.

\section*{Problem 4}
\subsection*{Part A}
\begin{align*}
	H_0 &: \mu_1 - \mu_2 = 0 \\
	H_\alpha &: \mu_1 - \mu_2 \neq 0
.\end{align*}
In this instance, $\mu_1$ is the average clicks per user of the first page and $\mu_2$ the second. The null hypothesis is that the average clicks per user between the two pages is the same and the alternative hypothesis is that the average clicks per user is different between the two pages.

\subsection*{Part B}
\[
	t = \frac{\mu_1 - \mu_2}{\sqrt{
			\frac{s_1^2}{n_1} + \frac{s_2^2}{n_2}
	}} = -7.2682
.\]

\subsection*{Part C}
\[
	p = 2\cdot\Prob{T > |t|} = \verb|2 * (1 - pt(7.2682, 29))| \approx 5.278 \times 10^{-8}
.\]

\subsection*{Part D}
Since $p << 0.05$, we have sufficient evidence to reject the null hypothesis, meaning there is evidence that the average clicks per user between the two pages is different.

\end{document}
