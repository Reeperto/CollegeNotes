\documentclass{standalone}

\begin{document}

\subsection{Offered Definitions:}\label{offered-definitions}

\subsubsection{Bernard Suits}\label{bernard-suits}

\begin{itemize}
\item
  Games inherently are inefficient as a result of their rules
\end{itemize}

\subsubsection{Salen and Zimmerman}\label{salen-and-zimmerman}

\begin{itemize}
\tightlist
\item
  Games are systems that induce artificial conflict within the
  confinement of rules

  \begin{itemize}
  \item
    There is a quantifiable outcome
  \end{itemize}
\end{itemize}

\subsection{Rules:}\label{rules}

\begin{itemize}
\tightlist
\item
  The formal structure of games arise from rules
\item
  Rules do not confine player's experiences
\item
  Rules are not strategies
\item
  Game rules are inherently artificial and disconnected from other
  social contexts (etiquette, law, war, etc.)
\end{itemize}

\subsubsection{Characteristics}\label{characteristics}

\begin{itemize}
\tightlist
\item
  Limit player action
\item
  Explicit / Understood
\item
  Fixed
\item
  Binding
\item
  Repeatable
\end{itemize}

\subsubsection{Types:}\label{types}

From \emph{Salen and Zimmerman}:

\begin{itemize}
\tightlist
\item
  Constitutive

  \begin{itemize}
  \item
    Encapsulate core game logic
  \item
    Do not necessarily indicate enforcement of rules
  \item
    Ex. How chess pieces move
  \end{itemize}
\item
  Operational

  \begin{itemize}
  \item
    ``Rules of play''
  \item
    The stuff you'd find in a game manual
  \item
    Ex. Chess is played by 2 people, turn based
  \end{itemize}
\item
  Implicit

  \begin{itemize}
  \item
    The ``unwritten rules''
  \item
    Concern the etiquette of the game
  \end{itemize}
\end{itemize}

\subsubsection{Representation}\label{representation}

\subsubsection{Geography}\label{geography}

\subsubsection{Time}\label{time}

\subsubsection{Uncertainty}\label{uncertainty}

``Games allow us to encounter uncertainty in a non-threatening way''
(Costikyan)

\subsubsection{Number of Players}\label{number-of-players}

\subsection{Dynamics}\label{dynamics}

\subsubsection{Game Balance}\label{game-balance}

\begin{itemize}
\tightlist
\item
  Balance controls:

  \begin{itemize}
  \tightlist
  \item
    Difficulty (ex. Potentially dynamic or geography based)
  \item
    Pacing
  \item
    Fairness
  \item
    Symmetry
  \item
    \emph{Illusion of Winnability}\footnote{https://book.huihoo.com/the-art-of-computer-game-design/Chapter6.html}
  \end{itemize}
\end{itemize}

\subsubsection{Flow}\label{flow}

\begin{itemize}
\item
  Flow conditions:

  \begin{itemize}
  \item
    \emph{TODO: grab from slides}
  \end{itemize}
\end{itemize}

Flow is not intrinsic to just games. Flow is a general principle that
applies to many activities. Examples include surfing, coding, and of
course games.

\end{document}
