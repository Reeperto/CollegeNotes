\documentclass[../notes.tex]{subfiles}
\graphicspath{
    {'../figures'}
}
\begin{document}

\banner{Extending the Naturals}

\subsection{Rational Numbers}

\begin{definition}
	The rational numbers is the set of numbers of the form $\frac{p}{q}$ where $p \in \mathbb{Z}$ and $q \in \mathbb{N}$.
\end{definition}

Rational numbers are the first number system that provides a nice comprehensive structure. Multiplication, division, addition, and subtraction are all closed operations making it a strong number system.

\begin{theorem}[Rational Root Theorem]
	Let $c_0, c_1, \ldots, c_{n} \in \mathbb{Z}$. If $r$ solves $c_{n} x^n + c_{n-1} x^{n-1} + \ldots + c_1 x + c_0 = 0$, $c_{n} \neq 0 \neq c_1$ and $r = \frac{p}{q}$ where $p$ and $q$ are coprime
	\[
		p \vert c_0, \;\; q \vert c_{n}
	\]
\end{theorem}
\begin{proof}
	Let $r$ be a rational solution to the polynomial equation $c_{n} x^n + c_{n-1} x^{n-1} + \ldots + c_1 x + c_0 = 0$. Since $r \in \mathbb{Q}$, $r = \frac{p}{q}$ where $p \in \mathbb{Z}$ and $q \in \mathbb{N}$. Then
	\begin{align*}
		c_{n} \qty(\frac{p}{q})^n + c_{n-1} \qty(\frac{p}{q})^{n-1} + \ldots + c_1 \qty(\frac{p}{q}) + c_0 &= 0 \\
		c_{n} p^n + c_{n-1} qp^{n-1} + \ldots + c_1 q^{n-1}p + c_0 q^n &= 0 \\
		-c_{n} p^n - c_{n-1} qp^{n-1} - \ldots - c_1 q^{n-1}p &= c_0 q^n \\
		-p \qty[c_{n} p^{n-1} - c_{n-1} qp^{n-2} - \ldots - c_1 q^{n-1}] &= c_0 q^n
	\end{align*}
	Therefore $p \vert c_0 q^n$. Since $p$ and $q$ are coprime, $p$ must divide $c_0$. By solving for $c_{n} p^n$ instead, it follows that $q$ divides $c_{n}$.
\end{proof}

While rationals are quite nice, there are many equations that have solutions that cannot be represented by a rational number.

\begin{example}[$\sqrt{2}$]
	Consider the equation $x^2 - 2$. Its solutions by the Rational Root Theorem must be an integer. However no integer satisfies the equation and therefore there is no rational root for $x^2-2$.
\end{example}

\subsection{Algebraic Numbers}

\begin{definition}[Algebraic Number]
	A number is called algebraic if it is the root of an integer coeffecient polynomial. That is, it is a solution to
	\[
		c_{n} x^n + c_{n-1} x^{n-1} + \ldots + c_1 x + c_0 = 0
	\]
	where $c_i \in \mathbb{Z}, c_i \neq 0$ and $n \geq 1$.
\end{definition}

Many numbers that are used day to day are algebraic. It follows clearly that all integers are algebraic and all rationals are algebraic. Other numbers such as the $\sqrt{2}$ are algebraic. Even the number $\sqrt{2 + \sqrt[3]{5}}$ is algebraic. However, there are infinitely many other numbers that are not algebraic such as $\pi$ and $e$.

\subsection*{Real Numbers}

As seen above, both the rationals and algebraic numbers can be very useful but fail to encapsulate important types of numbers. That is, both $\mathbb{Q}$ and the algebraic numbers have gaps in them, that is the irrationals for $\mathbb{Q}$ and transcendtals for algebraic numbers.
\subsubsection{Ordering Structure}
\begin{definition}[Ordered Field]
	We say a field with a relation $(\mathbb{F}, +, \cdot, \leq)$ is an ordered field if it satisfies the following properties:
	\begin{enumerate}
		\item $p \leq q$ or $q \leq p$ for all $p,q \in \mathbb{F}$
		\item $p \leq q$ and $q \leq p \implies p = q$ 
		\item $p \leq q$ and $q \leq r \implies p \leq r$
		\item $p \leq q \implies p + r \leq q + r$ 
		\item $p \leq q \implies pr \leq qr$ for all $r \in \mathbb{F} \geq 0$ 
	\end{enumerate}
\end{definition}

Certain properties are derivable from the properties and ordering of $\mathbb{R}$.
\begin{theorem}[Properties of $\mathbb{R}$]
	\label{thm:propsreals}
	For all $p,q,r \in \mathbb{R}$
	\begin{enumerate}
		\item $p + r = q + r \implies p = q$
		\item $p \cdot 0 = 0 = 0 \cdot p$
		\item $(-p)q = -(pq)$
		\item $(-p)(-q) = pq$
		\item $pr = qr \implies p = q$ if $r \neq 0$
		\item $pq = 0 \implies p=0$ or $q = 0$
	\end{enumerate}
\end{theorem}

\begin{proof}
	Let $p, q, r \in \mathbb{R}$ for the following.
	\begin{enumerate}
		\item[(1)] 
			Assume that $p + r = q + r$. Since additive inverses exist, $p + r + (-r) = q + r + (-r)$. By associativity, $p + (r + (-r)) = q + (r + (-r))$. By definition of inverses, $p + 0 = q + 0$. By the additive identity, $p = q$.
		\item[(2)]
			Examine $p \cdot 0$. Note that $p \cdot 0 = p \cdot (0 + 0)$. By distribution, $p \cdot 0 + p \cdot 0 = p \cdot 0$. This means that $p \cdot 0$ does not change when added to itself, which is by definition the additive identity. Therefore $p \cdot 0 = 0$.
		\item[(3)]
			Consider the expression $pq + (-p)q$. By distributivity, $pq + (-p)q = (p + (-p))q$. By inverses, $pq + (-p)q = 0 \cdot q = 0$. Therefore $-pq = (-p)q$.
		\item[(4)] To be completed
		\item[(5)] To be completed
		\item[(6)] Assume that $pq = 0$. WLOG, let $q \neq 0$. Since multiplicative inverses exist, $0 = q^{-1} \cdot 0 = 0 \cdot q^{-1} = pqq^{-1} = p(qq^{-1}) = p$. Therefore $p = 0$. 
	\end{enumerate}
\end{proof}

When considering the ordered field of the reals, more properties are derivable.

\begin{theorem}[Properties of Ordered Reals]
	\label{thm:propsorderedreals}
	Let $p,q,r \in \mathbb{R}$
	\begin{enumerate}
		\item $p \leq q \implies -q \leq -p$
		\item $p \leq q, r \leq 0 \implies qr \leq pr$
		\item $p \geq 0, q\geq 0 \implies pq \geq 0$
		\item $p^2 \geq 0$
		\item $0 < 1$
		\item $p > 0 \implies p^{-1} > 0$
		\item $0 < p < q \implies 0 < q^{-1} < p^{-1}$
	\end{enumerate}

	\begin{remark}
		$p < q$ is defined as $p \leq q$ and $p \neq q$.
	\end{remark}
\end{theorem}

\begin{proof}
	Let $p,q,r \in \mathbb{R}$
	\begin{enumerate}
		\item[(1)]
			Assume that $p \leq q$. Let $r = (-p) + (-q)$. Since adding a number to both sides of a inequality preserves it, $p + r \leq q + r$. Then $p + (-p) + (-q) \leq q + (-p) + (-q)$. By commutativity and associativity, $(p + (-p)) + (-q) \leq (q + (-q)) + (-p)$. By inverses and additive identity, $-q \leq -p$.
		\item[(2)]
			Assume that $p \leq q$ and that $r \leq 0$. By (1), $-r \leq 0$. Therefore, $p(-r) \leq q(-r)$ hence $-pr \leq -qr$. By (1), $qr \leq pr$.
		\item[(3)] To complete
		\item[(4)]
			By the properties of an ordered field, $p \leq 0$ or $p \geq 0$. If $p \geq 0$, then by (3), $p^2 = p\cdot p \geq 0$. If $p \leq 0$, then $-p \geq 0$. By \ref{thm:propsorderedreals}.4, $p^2 = (-p)(-p) \geq 0$ by the first case.
		\item[(5)] To complete
		\item[(6)]
			Assume towards contradiction that $p > 0$ and $p^{-1} \leq 0$. By (1), $-p^{-1} \geq 0$. Since $p$ and $-p^{-1}$ are non-negative, $p (-p^{-1}) \geq 0$. This means that $-1 \geq 0$ or equivalently $1 \leq 0$. By (5), this is a contradiction.
	\end{enumerate}
\end{proof}

\subsubsection{Absolute Value}

\begin{definition}[Absolute Value]
	Let $p, q \in \mathbb{R}$.
	\[
		|p| := \begin{cases}
			p & p \geq 0 \\
			-p & p \leq 0
		\end{cases}
	\]
	Additionally, define the distance between two reals as
	\[
		\operatorname{dist}(p,q) = | p - q |
	\]
\end{definition}

\begin{theorem}[Properties of Absolute Value]
	\label{thm:propsabsvalue}
	Let $p,q \in \mathbb{R}$.
	\begin{enumerate}
		\item $|p| \geq 0$
		\item $|pq| = |p||q|$
		\item \label{itm:triangle_inequality} $|p + q| \leq |p| + |q|$
	\end{enumerate}
\end{theorem}

\begin{proof}
	Let $p,q \in \mathbb{R}$.
	\begin{enumerate}
		\item[(1)]
			If $p \geq 0$, then $|p| \geq 0$. If $p \leq 0$, then $|p| \geq 0$. Therefore $|p| \geq 0$ for all $p$.
		\item[(2)]
			If $p \geq 0, q \geq 0$. Then $|pq| = pq = |p||q|$. If $p \leq 0, q \leq 0$, then $-p \geq 0, -q \geq 0$ and $|p||q| = (-p)(-q) = pq = |pq|$.
		\item[(3)]
			Note that $-|p| \leq p \leq |p|$. This is because $p$ either is $|p|$ or $|p| = -p$ meaning $p = -|p|$. Same is true for $q$. Therefore
			\begin{align*}
				-|p| + (-|q|) \leq -|p| + q &\leq p + q \leq |p| + q \leq |p| + |q| \\
				-(|p| + |q|) &\leq p + q \leq |p| + |q|
			\end{align*}
			The derived inequality shows that $p+q \leq |p| + |q|$ and $-(p+q) \leq |p| + |q|$. Since $|p + q|$ is either $p + q$ or $-(p+q)$, $|p+q| \leq |p| + |q|$.
	\end{enumerate}
\end{proof}

\begin{corollary}[Distance Triangle Inequality]
	\[
		\operatorname{dist}(p, r) \leq \operatorname{dist}(p, q) + \operatorname{dist}(q,r)
	\]
\end{corollary}

% \begin{proof}
% 	Consider $\abs| p - q + q - r|$. By the triangle inequality,
% 	% \begin{align*}
% 	% 	\abs| p - q + q - r| &\leq \abs|p-q| + \abs|q-r| \\
% 	% 	\abs| p - r| &\leq \abs|p-q| + \abs|q-r| \\
% 	% 	\operatorname{dist}(p, r) &\leq \operatorname{dist}(p, q) + \operatorname{dist}(q,r)
% 	% \end{align*}
% \end{proof}

\ref{thm:propsabsvalue}.\ref{itm:triangle_inequality} is an important property of the absolute value, usually referred to as the \textit{triangle inequality}. 

\end{document}
