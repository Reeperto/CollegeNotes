\documentclass[12pt]{extarticle}

% Document Layout and Font
\usepackage{subfiles}
\usepackage[margin=2cm, headheight=15pt]{geometry}
\usepackage{fancyhdr}
\usepackage{enumitem}	
\usepackage{wrapfig}
\usepackage{float}
\usepackage{multicol}

\usepackage[p,osf]{scholax}

\renewcommand*\contentsname{Table of Contents}
\renewcommand{\headrulewidth}{0pt}
\pagestyle{fancy}
\fancyhf{}
\fancyfoot[R]{$\thepage$}
\setlength{\parindent}{0cm}
\setlength{\headheight}{17pt}
\hfuzz=9pt

% Figures
\usepackage{svg}

% Utility Management
\usepackage{color}
\usepackage{colortbl}
\usepackage{xcolor}
\usepackage{xpatch}
\usepackage{xparse}

\definecolor{gBlue}{HTML}{7daea3}
\definecolor{gOrange}{HTML}{e78a4e}
\definecolor{gGreen}{HTML}{a9b665}
\definecolor{gPurple}{HTML}{d3869b}

\definecolor{links}{HTML}{1c73a5}
\definecolor{bar}{HTML}{584AA8}

% Math Packages
\usepackage{mathtools, amsmath, amsthm, thmtools, amssymb, physics}
\usepackage[scaled=1.075,ncf,vvarbb]{newtxmath}

\newcommand\B{\mathbb{C}}
\newcommand\C{\mathbb{C}}
\newcommand\R{\mathbb{R}}
\newcommand\Q{\mathbb{Q}}
\newcommand\N{\mathbb{N}}
\newcommand\Z{\mathbb{Z}}

\DeclareMathOperator{\lcm}{lcm}

% Probability Theory
\newcommand\Prob[1]{\mathbb{P}\qty(#1)}
\newcommand\Var[1]{\text{Var}\qty(#1)}
\newcommand\Exp[1]{\mathbb{E}\qty[#1]}

% Analysis
\newcommand\ball[1]{\B\qty(#1)}
\newcommand\conj[1]{\overline{#1}}
\DeclareMathOperator{\Arg}{Arg}
\DeclareMathOperator{\cis}{cis}

% Linear Algebra
\DeclareMathOperator{\dom}{dom}
\DeclareMathOperator{\range}{range}
\DeclareMathOperator{\spann}{span}
\DeclareMathOperator{\nullity}{nullity}

% TIKZ
\usepackage{tikz}
\usepackage{pgfplots}
\usetikzlibrary{arrows.meta}
\usetikzlibrary{math}
\usetikzlibrary{cd}

% Boxes and Theorems
\usepackage[most]{tcolorbox}
\tcbuselibrary{skins}
\tcbuselibrary{breakable}
\tcbuselibrary{theorems}

\newtheoremstyle{default}{0pt}{0pt}{}{}{\bfseries}{\normalfont.}{0.5em}{}
\theoremstyle{default}

\renewcommand*{\proofname}{\textit{\textbf{Proof.}}}
\renewcommand*{\qedsymbol}{$\blacksquare$}
\tcolorboxenvironment{proof}{
	breakable,
	coltitle = black,
	colback = white,
	frame hidden,
	boxrule = 0pt,
	boxsep = 0pt,
	borderline west={3pt}{0pt}{bar},
	% borderline west={3pt}{0pt}{gPurple},
	sharp corners = all,
	enhanced,
}

\newtheorem{theorem}{Theorem}[section]{\bfseries}{}
\tcolorboxenvironment{theorem}{
	breakable,
	enhanced,
	boxrule = 0pt,
	frame hidden,
	coltitle = black,
	colback = blue!7,
	% colback = gBlue!30,
	left = 0.5em,
	sharp corners = all,
}

\newtheorem{corollary}{Corollary}[section]{\bfseries}{}
\tcolorboxenvironment{corollary}{
	breakable,
	enhanced,
	boxrule = 0pt,
	frame hidden,
	coltitle = black,
	colback = white!0,
	left = 0.5em,
	sharp corners = all,
}

\newtheorem{lemma}{Lemma}[section]{\bfseries}{}
\tcolorboxenvironment{lemma}{
	breakable,
	enhanced,
	boxrule = 0pt,
	frame hidden,
	coltitle = black,
	colback = green!7,
	left = 0.5em,
	sharp corners = all,
}

\newtheorem{definition}{Definition}[section]{\bfseries}{}
\tcolorboxenvironment{definition}{
	breakable,
	coltitle = black,
	colback = white,
	frame hidden,
	boxsep = 0pt,
	boxrule = 0pt,
	borderline west = {3pt}{0pt}{orange},
	% borderline west = {3pt}{0pt}{gOrange},
	sharp corners = all,
	enhanced,
}

\newtheorem{example}{Example}[section]{\bfseries}{}
\tcolorboxenvironment{example}{
	% title = \textbf{Example},
	% detach title,
	% before upper = {\tcbtitle\quad},
	breakable,
	coltitle = black,
	colback = white,
	frame hidden,
	boxrule = 0pt,
	boxsep = 0pt,
	borderline west={3pt}{0pt}{green!70!black},
	% borderline west={3pt}{0pt}{gGreen},
	sharp corners = all,
	enhanced,
}

\newtheoremstyle{remark}{0pt}{4pt}{}{}{\bfseries\itshape}{\normalfont.}{0.5em}{}
\theoremstyle{remark}
\newtheorem*{remark}{Remark}


% TColorBoxes
\newtcolorbox{week}{
	colback = black,
	coltext = white,
	fontupper = {\large\bfseries},
	width = 1.2\paperwidth,
	size = fbox,
	halign upper = center,
	center
}

\newcommand{\banner}[2]{
    \pagebreak
    \begin{week}
   		\section*{#1}
    \end{week}
    \addcontentsline{toc}{section}{#1}
    \addtocounter{section}{1}
    \setcounter{subsection}{0}
}

% Hyperref
\usepackage{hyperref}
\hypersetup{
	colorlinks=true,
	linktoc=all,
	linkcolor=links,
	bookmarksopen=true
}

% Error Handling
\PackageWarningNoLine{ExtSizes}{It is better to use one of the extsizes 
                          classes,^^J if you can}

\usepackage{svg}
\svgsetup{inkscapeexe=inkscape, inkscapearea=drawing, inkscapeversion=1}
\svgpath{{figures/}}

\fancyhead[R]{\textbf{Homework \#$5$}}
\fancyhead[L]{Eli Griffiths}
\renewcommand{\headrulewidth}{1pt}
\setlength\parindent{0pt}

\begin{document}

\section*{Problem 1}
\subsection*{Part 1}
For $f(x)$ to be a valid probability density function, it must be true that $\int_{\mathbb{S}_X} f(x) \dd x = 1$. Therefore
\begin{align*}
	\int_{-\infty}^{\infty} f(x) \dd x &= 1 \\
	\int_{-1}^{1} c(1-x^2) \dd x &= 1 \\
	c \int_{-1}^{1} 1-x^2 \dd x &= 1 \\
	c \qty[x - \frac{x^3}{3}]_{-1}^{1} &= 1 \\
	c \qty[\frac{2}{3} - \qty(-\frac{2}{3})] &= 1 \\
	c \cdot \frac{4}{3} &= 1 \implies \boxed{c = \frac{3}{4}}
.\end{align*}

\subsection*{Part 2}
\begin{align*}
	F(a) = \Prob{X \leq a} = \frac{3}{4} \int_{-1}^{a} 1 - x^2 \dd x = \frac{3}{4} \qty[x - \frac{x^3}{3}]_{-1}^a &= \frac{3}{4} \qty(a - \frac{a^3}{3} + \frac{2}{3}) \\
																												  &= \boxed{\frac{1}{2} + \frac{3a}{4} - \frac{a^3}{4}}
.\end{align*}

\section*{Problem 2}
\begin{align*}
	\Exp{X} &= \frac{3}{5} \\
	\int_0^1 x \qty(ax + bx^2) \dd x &= \frac{3}{5} \\
	\int_0^1 ax^2 + bx^3 \dd x &= \frac{3}{5} \\
	\frac{ax^3}{3} + \frac{bx^4}{4} \eval_0^1 &= \frac{3}{5} \\
	\frac{a}{3} + \frac{b}{4} &= \frac{3}{5} \\
	4a + 3b &= \frac{36}{5}
.\end{align*}
Additionally,
\begin{align*}
	\int_0^1 ax + bx^2 \dd x &= 1 \\
	\frac{ax^2}{2} + \frac{bx^3}{3} \eval_0^1 &= 1 \\
	\frac{a}{2} + \frac{b}{3} &= 1 \\
	3a + 2b &= 6
.\end{align*}
Therefore we have a system of equations
\begin{align*}
	4a + 3b &= \frac{36}{5} \\
	3a + 2b &= 6
.\end{align*}
Solving this system gives $a = \frac{18}{5}, b = -\frac{12}{5}$.

\section*{Problem 3}
% 
% x^2  e^{-x}
% 2x  -e^{-x}
% 2    e^{-x}
% 0   -e^{-x}
% 
By the definition of expectation of a continuous random variable,
\[
	\Exp{X} = \int_0^\infty x^2 e^{-x} \dd x
.\]
Using the tabular form of integration by parts results in
\begin{center}
	\[
		\renewcommand\arraystretch{2}
		\begin{array}{c | c}
			dv & u \\\hline
			x^2 & e^{-x} \\
			2x & -e^{-x} \\
			2 & e^{-x} \\
			0 & -e^{-x}
		\end{array}
	.\]
\end{center}
meaning that
\[
	\Exp{X} = -x^2 e^{-x} - 2x e^{-x} - 2e^{-x} \eval_0^{\infty} = 2
.\]

\section*{Problem 4}
\subsection*{Part 1}
\begin{align*}
	\int_0^2 c\qty(4x - 2x^2) \dd x &= 1 \\
	c \int_0^2 4x - 2x^2 \dd x &= 1 \\
	c \qty[2x^2 - \frac{2}{3}x^3 ]_0^2 &= 1 \\
	c \cdot \frac{8}{3} &= 1 \implies \boxed{c = \frac{3}{8}}
.\end{align*}

\subsection*{Part 2}
\begin{align*}
	\Prob{\frac{1}{2} < X < \frac{3}{2}} &= \frac{3}{8} \int_{\frac{1}{2}}^{\frac{3}{2}} 4x - 2x^2 \dd x \\
										 &= \frac{3}{8} \qty[2x^2 - \frac{2}{3} x^3]_{\frac{1}{2}}^{\frac{3}{2}} \\
										 &= \frac{3}{8} \qty[2\cdot \frac{9}{4} - \frac{2}{3} \cdot \frac{27}{8} - \qty(2 \cdot \frac{1}{4} - \frac{2}{3} \cdot \frac{1}{8})] \\
										 &= \frac{3}{8} \qty[\frac{9}{2} - \frac{9}{4} - \qty(\frac{1}{2} - \frac{1}{12})] \\
										 &= \frac{3}{8} \qty[\frac{9}{2} - \frac{9}{4} - \frac{5}{12}] \\
										 &= \frac{3}{8} \cdot \frac{11}{6} = \boxed{\frac{11}{16}}
.\end{align*}

\section*{Problem 5}
\begin{align*}
	& F_Y (x) = \Prob{Y \leq x} = \Prob{e^X \leq x} = \Prob{X \leq \ln(x)} \\
	& \Downarrow \dv{x} \\
	& f_Y (x) = F_X'(\ln(x)) \cdot \frac{1}{x} = f_X (\ln(x)) \cdot \frac{1}{x} = \frac{1}{x}
.\end{align*}
Therefore
\[
	f_Y(x) = 
	\begin{cases}
		\frac{1}{x} & 1 < x < e \\
		0 & \text{otherwise}
	\end{cases}
.\]

\section*{Problem 6}
\subsection*{Part 1}
\begin{align*}
	\Prob{|X| > \frac{1}{2}} &= \Prob{\qty(X > \frac{1}{2}) \cup \qty(X < -\frac{1}{2})} \\
							 &= \Prob{X > \frac{1}{2}} + \Prob{X < -\frac{1}{2}} \\
							 &= \frac{1}{2}\cdot\qty(\frac{1}{2} + \frac{1}{2}) = \frac{1}{2}
.\end{align*}


\subsection*{Part 2}
Let $Y = |X|$.
\begin{align*}
	& F_Y (y) = \Prob{Y \leq y} = \Prob{|X| \leq y} = \Prob{-y \geq X \leq y} = \int_{-y}^0 \frac{1}{2} \dd y + \int_0^y \frac{1}{2} \dd y \\
	& \Downarrow \dv{y} \\
	& f_Y (y) = \dv{y} \qty[\int_{-y}^0 \frac{1}{2} \dd y + \int_0^y \frac{1}{2} \dd y] = \dv{y} (y) = 1
\end{align*}
Therefore
\[
	f_Y (x) =
	\begin{cases}
		1 & x \in [0,1) \\
		0 & x \notin [0,1)
	\end{cases}
.\]

\section*{Problem 7}
\subsection*{Part 1}
\begin{align*}
	\Prob{X > 10} = \int_{10}^{30} \frac{1}{30} \dd x = \frac{2}{3}
.\end{align*}

\subsection*{Part 2}
\begin{align*}
	\Prob{X > 25 | X > 15} = \frac{\Prob{X > 25}}{\Prob{X > 15}} = \frac{\int_{25}^{30} \dd x}{\int_{15}^{30} \dd x} = \frac{30-25}{30-15} = \frac{1}{3}
.\end{align*}

\section*{Problem 8}
\begin{align*}
	\Exp{X^n} &= \int_0^1 x^n \dd x \\
			  &= \int_0^1 x^n \dd x \\
			  &= \frac{x^{n+1}}{n+1} \eval_0^1 = \frac{1}{n+1}
.\end{align*}

\section*{Problem 9}
Let $c$ be the capacity of the tank.
\begin{align*}
	\Prob{X \geq c} &= 0.01 \\
	\int_c^1 5(1-x)^4 \dd x &= 0.01 \\
	5 \int_c^1 (1-x)^4 \dd x &= 0.01 \\
	-5 \int_{1-c}^0 u^4 \dd u &= 0.01 \\
	5 \int_0^{1-c} u^4 \dd u &= 0.01 \\
	5 \cdot \frac{(1-c)^5}{5} &= 0.01 \\
	(1-c)^5 &= 0.01 \\
	c &= 1 - \sqrt[5]{0.01} \implies \boxed{c \approx 0.6019}
\end{align*}

\section*{Problem 10}
The position of the point that is randomnly chosen can be interpreted as a uniform random variable $X$ from $0$ to $L$. Consider the position of a point denoted by $a$. There are two seperate ratios to consider. If $0$ to $a$ is the short segment, then the ratio is
\[
	\frac{a}{L - a}
.\]
If the $a$ to $L-a$ segment is the short segment, then the ratio is
\[
	\frac{L-a}{a}
.\]
These events are mutually exclusive. Therefore
\begin{align*}
	\Prob{\frac{\text{short}}{\text{long}} < \frac{1}{4}} &= \Prob{\qty(\frac{X}{L - X} < \frac{1}{4}) \cup \qty(\frac{L - X}{X} < \frac{1}{4})} \\
														  &= \Prob{X < \frac{L}{5}} + \Prob{X > \frac{4L}{5}} \\
														  &= \frac{1}{5} + \frac{1}{5} = \boxed{\frac{2}{5}}
\end{align*}

\end{document}
