\documentclass[12pt,titlepage]{extarticle}
% Document Layout and Font
\usepackage{subfiles}
\usepackage[margin=2cm, headheight=15pt]{geometry}
\usepackage{fancyhdr}
\usepackage{enumitem}	
\usepackage{wrapfig}
\usepackage{float}
\usepackage{multicol}

\usepackage[p,osf]{scholax}

\renewcommand*\contentsname{Table of Contents}
\renewcommand{\headrulewidth}{0pt}
\pagestyle{fancy}
\fancyhf{}
\fancyfoot[R]{$\thepage$}
\setlength{\parindent}{0cm}
\setlength{\headheight}{17pt}
\hfuzz=9pt

% Figures
\usepackage{svg}

% Utility Management
\usepackage{color}
\usepackage{colortbl}
\usepackage{xcolor}
\usepackage{xpatch}
\usepackage{xparse}

\definecolor{gBlue}{HTML}{7daea3}
\definecolor{gOrange}{HTML}{e78a4e}
\definecolor{gGreen}{HTML}{a9b665}
\definecolor{gPurple}{HTML}{d3869b}

\definecolor{links}{HTML}{1c73a5}
\definecolor{bar}{HTML}{584AA8}

% Math Packages
\usepackage{mathtools, amsmath, amsthm, thmtools, amssymb, physics}
\usepackage[scaled=1.075,ncf,vvarbb]{newtxmath}

\newcommand\B{\mathbb{C}}
\newcommand\C{\mathbb{C}}
\newcommand\R{\mathbb{R}}
\newcommand\Q{\mathbb{Q}}
\newcommand\N{\mathbb{N}}
\newcommand\Z{\mathbb{Z}}

\DeclareMathOperator{\lcm}{lcm}

% Probability Theory
\newcommand\Prob[1]{\mathbb{P}\qty(#1)}
\newcommand\Var[1]{\text{Var}\qty(#1)}
\newcommand\Exp[1]{\mathbb{E}\qty[#1]}

% Analysis
\newcommand\ball[1]{\B\qty(#1)}
\newcommand\conj[1]{\overline{#1}}
\DeclareMathOperator{\Arg}{Arg}
\DeclareMathOperator{\cis}{cis}

% Linear Algebra
\DeclareMathOperator{\dom}{dom}
\DeclareMathOperator{\range}{range}
\DeclareMathOperator{\spann}{span}
\DeclareMathOperator{\nullity}{nullity}

% TIKZ
\usepackage{tikz}
\usepackage{pgfplots}
\usetikzlibrary{arrows.meta}
\usetikzlibrary{math}
\usetikzlibrary{cd}

% Boxes and Theorems
\usepackage[most]{tcolorbox}
\tcbuselibrary{skins}
\tcbuselibrary{breakable}
\tcbuselibrary{theorems}

\newtheoremstyle{default}{0pt}{0pt}{}{}{\bfseries}{\normalfont.}{0.5em}{}
\theoremstyle{default}

\renewcommand*{\proofname}{\textit{\textbf{Proof.}}}
\renewcommand*{\qedsymbol}{$\blacksquare$}
\tcolorboxenvironment{proof}{
	breakable,
	coltitle = black,
	colback = white,
	frame hidden,
	boxrule = 0pt,
	boxsep = 0pt,
	borderline west={3pt}{0pt}{bar},
	% borderline west={3pt}{0pt}{gPurple},
	sharp corners = all,
	enhanced,
}

\newtheorem{theorem}{Theorem}[section]{\bfseries}{}
\tcolorboxenvironment{theorem}{
	breakable,
	enhanced,
	boxrule = 0pt,
	frame hidden,
	coltitle = black,
	colback = blue!7,
	% colback = gBlue!30,
	left = 0.5em,
	sharp corners = all,
}

\newtheorem{corollary}{Corollary}[section]{\bfseries}{}
\tcolorboxenvironment{corollary}{
	breakable,
	enhanced,
	boxrule = 0pt,
	frame hidden,
	coltitle = black,
	colback = white!0,
	left = 0.5em,
	sharp corners = all,
}

\newtheorem{lemma}{Lemma}[section]{\bfseries}{}
\tcolorboxenvironment{lemma}{
	breakable,
	enhanced,
	boxrule = 0pt,
	frame hidden,
	coltitle = black,
	colback = green!7,
	left = 0.5em,
	sharp corners = all,
}

\newtheorem{definition}{Definition}[section]{\bfseries}{}
\tcolorboxenvironment{definition}{
	breakable,
	coltitle = black,
	colback = white,
	frame hidden,
	boxsep = 0pt,
	boxrule = 0pt,
	borderline west = {3pt}{0pt}{orange},
	% borderline west = {3pt}{0pt}{gOrange},
	sharp corners = all,
	enhanced,
}

\newtheorem{example}{Example}[section]{\bfseries}{}
\tcolorboxenvironment{example}{
	% title = \textbf{Example},
	% detach title,
	% before upper = {\tcbtitle\quad},
	breakable,
	coltitle = black,
	colback = white,
	frame hidden,
	boxrule = 0pt,
	boxsep = 0pt,
	borderline west={3pt}{0pt}{green!70!black},
	% borderline west={3pt}{0pt}{gGreen},
	sharp corners = all,
	enhanced,
}

\newtheoremstyle{remark}{0pt}{4pt}{}{}{\bfseries\itshape}{\normalfont.}{0.5em}{}
\theoremstyle{remark}
\newtheorem*{remark}{Remark}


% TColorBoxes
\newtcolorbox{week}{
	colback = black,
	coltext = white,
	fontupper = {\large\bfseries},
	width = 1.2\paperwidth,
	size = fbox,
	halign upper = center,
	center
}

\newcommand{\banner}[2]{
    \pagebreak
    \begin{week}
   		\section*{#1}
    \end{week}
    \addcontentsline{toc}{section}{#1}
    \addtocounter{section}{1}
    \setcounter{subsection}{0}
}

% Hyperref
\usepackage{hyperref}
\hypersetup{
	colorlinks=true,
	linktoc=all,
	linkcolor=links,
	bookmarksopen=true
}

% Error Handling
\PackageWarningNoLine{ExtSizes}{It is better to use one of the extsizes 
                          classes,^^J if you can}


\def\homeworknumber{2}
\fancyhead[R]{\textbf{Math 140A: Homework \#\homeworknumber}}
\fancyhead[L]{Eli Griffiths}
\renewcommand{\headrulewidth}{1pt}
\setlength\parindent{0pt}


\usepackage{algorithm}
\usepackage{algpseudocode}

\begin{document}

% 4.1: 3,7,11,13,15,17,27,31,33,43

\subsection*{2.2.1}
\begin{tasks}
    \task $A \cap B$ is the set of students within a mile of the school that walk to class 
    \task $A \cup B$ is the set of students who are within a mile of the school or walk to class
    \task $A - B$ is the set of students who are within a mile of the school but do not walk to class
    \task $B - A$ is the set of students who walk to class that are not within a mile of the school
\end{tasks}

\subsection*{2.2.3}
\begin{tasks}
    \task $\qty{0,1,2,3,4,5,6}$
    \task $\qty{3}$
    \task $\qty{1,2,4,5}$
    \task $\qty{0,6}$
\end{tasks}

\subsection*{2.2.9}
\begin{tasks}
    \task We will show that both sets are subsets of each other.
    \begin{proof}
        Let $x \in U$. If $x \in A$, then $x \in A \cup \conj{A}$. If $x \notin A$, then $x \in \conj{A}$ and hence $x \in A \cup \conj{A}$. Therefore $U \subseteq A \cup \conj{A}$. Since $A$ and $\conj{A}$ are subsets of $U$ by definition, then $A \cup \conj{A} \subset U$. Therefore $A \cup \conj{A} = U$.
    \end{proof}
    \task We will show that $A \cap \conj{A}$ contains no elements.
    \begin{proof}
        Let $x \in A$. Then by the definition of the complement, $x \notin \conj{A}$. Therefore none of the elements in $A$ are in  $\conj{A}$ and hence $A \cap \conj{A} = \varnothing$.
    \end{proof}
\end{tasks}

\subsection*{2.2.13}
\begin{proof}
    Let $x \in A$. Then $x \in A \cup B$ and hence $x \in A \cap (A \cup B)$. Therefore $A \subseteq A \cap (A \cup B)$. Let $x \in A \cap (A \cup B)$. Then $x \in A$ and hence $A \cap (A \cup B) \subseteq A$. Therefore $A \cap (A \cup B) = A$.
\end{proof}

\subsection*{2.2.21}
\subsubsection*{Part A}
\begin{proof}
    Let $x \in A - B$. Then $x \in A$ and $x \notin B$. Therefore $x \in \conj{B}$ meaning $x \in A \cap \conj{B}$. Hence $A - B \subseteq A \cap \conj{B}$. Let $x \in A \cap \conj{B}$. Then $x \in A$ and $x \in \conj{B}$. Therefore $x \notin B$ meaning $x \in A - B$. Hence $A \cap \conj{B} \subseteq A - B$ giving $A - B = A \cap \conj{B}$.
\end{proof}

\subsubsection*{Part B} % TODO
\begin{proof}
    
\end{proof}

\subsection*{2.2.27}
\begin{tasks}
    \task $\qty{4,6}$
    \task $\qty{0,1,2,3,4,5,6,7,8,9,10}$
    \task $\qty{4,5,6,8,10}$
\end{tasks}

\subsection*{2.2.51}
\begin{proof}
    Assume towards contradiction that $A$ is an infinite set and $A \cup B$ is a finite set. Since $A \cup B$ is finite, it has $n$ elements where $n$ is a natural number. However, $A$ has more than $n$ elements since it is infinite. Since all the elements of $A$ are in $A \cup B$, then $A \cup B$ has more than $n$ elements, a contradiction.
\end{proof}

\subsection*{2.3.7}
\begin{tasks}
    \task $\dom(f) = \Z^+ \times \Z^+$ and $\range(f) = \Z^+$
    \task $\dom(f) = \Z^+$ and $\range(f) = \qty{n \in \Z : 0 \leq n \leq 9}$
    \task $\dom(f) = $ set of all bit strings and $\range(f) = \N_0$
    \task $\dom(f) = $ set of all bit strings and $\range(f) = \N_0$
\end{tasks}

\subsection*{2.3.11}
Only $(a)$ is onto since every element of $\qty{a,b,c,d}$ is in the range of $f$.

\subsection*{2.3.13}
Only $(a)$ and $(d)$ are onto. $(b)$ is not onto since it maps integers to strictly positive integers. $(c)$ is not onto since there is no integer whose cube is $2$.

\subsection*{2.3.15}
\begin{tasks}
    \task The function is onto since for any $n \in \Z$ we have $f(n, 0) = n$.
    \task The function is not onto since there is no $m,n \in \Z$ such that $m^2 + n^2 = 10$.
    \task The function is onto for any $n \in \Z$ we have $f(n, 0) = n$.
    \task The function is not onto since $f[\Z \times \Z] = \Z^+ \neq \Z$
\end{tasks}

\subsection*{2.3.23}
Only $(a)$ and $(c)$ are bijections since the image of the $\R$ under both is $\R$ itself. $(b)$ and $(d)$ are not bijections because they are not onto.

\subsection*{2.3.31}
\begin{tasks}(2)
    \task $f(S) = \qty{1, 0, 0, 0, 1, 3}$
    \task $f(S) = \qty{0, 0, 1, 3, 5, 8}$
    \task $f(S) = \qty{0, 8, 16, 40}$
    \task $f(S) = \qty{1, 12, 33, 65}$
\end{tasks}

\subsection*{2.3.37}
\begin{proof}
    It is not true in general. Consider $f : 2 \Z \to \Z : n \mapsto \frac{n}{2}$ and $g : \Z \to \Z : n \mapsto 2n$. Note that $f$ is onto and $(f \circ g)(n) = n$ and is also onto. However $g$ is not onto.
\end{proof}

\subsection*{2.3.47}
\begin{proof}
    Note that
    \[
        f^{-1}(\conj{S}) = \qty{a \in A : f(a) \notin S} = \conj{\qty{a \in A : f(a) \in S}} = \conj{f^{-1}(S)}
    .\]
    Hence both are equal.
\end{proof}

\subsection*{2.3.73}
\begin{proof}
    Let $x \in U$ and $f_S$ be defined as given.
    \begin{enumerate}[label=\alph*)]
        \item % ----------------------------------------------------------------
        If $x \in A \cap B$, then $x \in A$ and $x \in B$. Therefore $f_{A \cap B}(x) = 1$ and $f_{A}(x) \cdot f_{B}(x) = 1 \cdot 1 = 1$. If $x \notin A \cap B$, then $x \notin A$ or $x \notin B$. Therefore $f_{A \cap B}(x) = 0$ and since one of $f_{A}(x)$ or $f_{B}(x)$ are zero meaning $f_{A}(x) \cdot f_{B}(x) = 0$. Therefore in either case $f_{A \cap B}(x) = f_{A}(x) \cdot f_{B}(x)$.
        \item % ----------------------------------------------------------------
        There are three cases to consider.
        \begin{itemize}
            \item Assume that $x \in A \cup B$ and $x \in A \cap B$. Then $f_{A \cup B}(x) = 1$ and 
            \[
                f_A(x) + f_B(x) - f_A(x) \cdot f_B(x) = 1 + 1 - f_{A\cap B}(x) = 1 + 1 - 1 = 1
            .\]
        \item Assume that $x \in A \cup B$ and $x \notin A \cap B$. Then $f_{A \cup B}(x) = 1$. Since $x \notin A \cap B$, then either $x \notin A$ or $x \notin B$. Therefore
            \[
                f_A(x) + f_B(x) - f_A(x) \cdot f_B(x) = 0 + 1 + f_{A \cap B}(x) = 0 + 1 + 0 = 1
            .\]
        \item Assume that $x \notin A \cup B$. Then $f_{A \cup B}(x) = 0$. Since $x \notin A \cup B$, $x \notin A$ and $x \notin B$ meaning
            \[
                f_A(x) + f_B(x) - f_A(x) \cdot f_B(x) = 0 + 0 - 0 \cdot 0 = 0
            .\]
        \end{itemize}
        In all possible cases, the two sides are equal.
        \item % ----------------------------------------------------------------
        Let $x \in \conj{A}$. Then $f_{\conj{A}}(x) = 1$. Since $x \in \conj{A}$, $x \notin A$ meaning $1 - f_A(x) = 1 - 0 = 1$. Same holds for $x \notin \conj{A}$.
        \item % ----------------------------------------------------------------
        Let $x \in A \bigoplus B$. Then $x \in A, x \notin B$ or $x \notin A, x \in B$. Therefore
        \[
            f_A(x) + f_B(x) - 2 f_A(x) \cdot f_B(x) = 1
        .\]
        Same holds for $x \notin A \bigoplus B$.
    \end{enumerate}
\end{proof}

\subsection*{2.4.3}
\begin{tasks}(2)
    \task $a_0 = 2, a_1 = 3, a_2 = 5, a_3 = 9$
    \task $a_0 = 1, a_1 = 4, a_2 = 27, a_3 = 256$
    \task $a_0 = 0, a_1 = 0, a_2 = 1, a_3 = 1$
    \task $a_0 = 0, a_1 = 1, a_2 = 2, a_3 = 3$
\end{tasks}

\subsection*{2.4.7}
The sequences $a_n = 2^n$, $a_n = 1 + \left\lfloor \frac{(n+1)^2}{3} \right\rfloor$, and $a_{n+1} = a_{n} + (n-1)$ all satisfy this condition.

\subsection*{2.4.11}
\begin{tasks}
    \task $a_0 = 6, a_1 = 17, a_2 = 49, a_3 = 143, a_4 = 421$
    \task Note that
    \begin{itemize}
        \item $a_2 = 49 = 85 - 36 = 5(17) - 6(6) = 5 a_1 - 5 a_0$ 
        \item $a_3 = 143 = 245 - 102 = 5(49) - 6(17) = 5 a_2 - 6 a_1$
        \item $a_4 = 421 = 715 - 294 = 5(143) - 6(49) = 5 a_3 - 6 a_2$
    \end{itemize}
    \task 
    For $n \geq 2$,
    \begin{align*}
        5 a_{n-1} - 6 a_{n-2} &= 5 (2^{n-1} + 5 \cdot 3^{n-1}) - 6(2^{n-2} + 5\cdot 3^{n-2}) \\
                              &= 5 \cdot 2^{n-1} + 25 \cdot 3^{n-1} - 6 \cdot 2^{n-2} - 30 \cdot 3^{n-2} \\
                              &= 5 \cdot 2^{n-1} + 25 \cdot 3^{n-1} - 3 \cdot 2^{n-1} - 10 \cdot 3^{n-1} \\
                              &= 2 \cdot 2^{n-1} + 15 \cdot 3^{n-1} \\
                              &= 2^n + 5 \cdot 3^n \\
                              &= a_n
    \end{align*}
\end{tasks}

\subsection*{2.4.13}
\begin{tasks}(3)
    \task Yes
    \task No
    \task No
    \task Yes 
    \task Yes
    \task Yes
    \task No
    \task No
\end{tasks}

\subsection*{2.4.15} % TODO

\subsection*{2.4.25}
\begin{tasks}
    \task 1 one followed by 1 zero, 2 ones followed by 2 zeroes, etc. Next three terms would be $1,1,1$
    \task Each integer greater than 1 with each even integer repeated twice. Next three terms would be $9, 10, 10$
    \task $a_n = 2^{\frac{n}{2}}$ for even $n$ and $0$ otherwise. Next three terms would be $32, 0, 64$
    \task $a_n = 3 \cdot 2^n$. Next three terms would be $384, 768, 1536$
    \task $a_n = 15 - 7n$. Next three terms would be $-34, -41, -48$
    \task $a_n = \frac{(n+1)^2 + n + 5}{2}$. Next three terms would be $57, 68, 80$
    \task $a_n = 2n^3$. Next three terms would be $1024, 1458, 2000$
    \task $a_n = 1 + n!$. Next three terms would be $362881, 3628801, 39916801$
\end{tasks}

\subsection*{2.4.29}
\begin{tasks}(2)
    \task $\displaystyle \sum_{k=1}^5 k + 1 = 5 + \sum_{k = 1}^5 k = 5 + \frac{5\cdot 6}{2} = 20$
    \task $\displaystyle \sum_{j=0}^4 (-2)^j = 1 - 2 + 4 - 8 + 16 = 11$
    \task $\displaystyle \sum_{i=1}^{10} 3 = 30$
    \task $\displaystyle \sum_{j=0}^8 2^{j+1} - 2^j = \sum_{j=0}^8 2^j = 2^9 - 1$
\end{tasks}

\subsection*{2.4.31}
\begin{tasks}(2)
    \task $S_8 = 1533$
    \task $S_8 - 1 = 510$
    \task $S_8 - (1 - 3) = 4923$
    \task $S_8 = 9842$
\end{tasks}

\subsection*{2.4.33}
\begin{tasks}(2)
    \task $21$
    \task $78$
    \task $18$
    \task $18$
\end{tasks}

\subsection*{2.4.35}
Since for every term $a_i$ with $0 < i < n$ appears both as a positive and negative term in the sum, they all cancel out leaving behind just $a_n - a_0$.

\subsection*{2.4.39}
\[
    \sum_{k=100}^{200} k = \sum_{k=0}^{100} (k+100) = 100 \cdot 101 + \frac{100(101)}{2} = 15150
.\]

\subsection*{2.5.1}
\begin{tasks}(2)
    \task Countably infinite; $−1, −2, −3, −4, \ldots$
    \task Countably infinite; $0, 2, -2, 4, -4,\ldots$
    \task Countably infinite; $99, 98, 97, \ldots$
    \task Uncountable 
    \task Finite 
    \task Countably infinite; $0, 7, −7, 14, −14, \ldots$
\end{tasks}

\subsection*{2.5.3}
\begin{tasks}(2)
    \task Countable; map $n$ to the bit string contain $n$ ones
    \task Countable; follow the diagonalization arguement excluding top three rows
    \task Uncountable 
    \task Uncountable
\end{tasks}

\subsection*{2.5.7}
For each $n$, take the guest in room $2n$ and put them in room $n$ of the original building and the guest in room $2n + 1$ into room $n$ of the new building.

\subsection*{2.5.11}
\begin{tasks}
    \task $[-1,0] \cap [0,1] = \qty{0}$
    \task $([-2,-1] \cup \Z) \cap ([1,2] \cup \Z) = \Z$
    \task $[-2, 2] \cap [-1, 1] = [-1, 1]$
\end{tasks}

\subsection*{2.5.17}
No. Notice that $[-4, 4] - [-1, 1] = [-4, -1) \cup (1, 4]$ which is still uncountable.

\subsection*{2.5.27}
\begin{proof}
    Let $A_1, A_2, A_3, \ldots$ be a list of countably countable sets. For a given set $A_i$, its elements can be listed out since it is countable in the manner
    \[
        a_{i1}, a_{i 2}, a_{i 3}, \ldots
    .\]
    Listing out the elements of $\bigcup A_i$ is simply listing out all elements $a_{ij}$ where $i + j = 2$, then $i + j = 3$, and so forth.
\end{proof}

\subsection*{2.5.31} % TODO

\subsection*{2.6.1}
\begin{tasks}(3)
    \task $A$ is a $3\times 4$ matrix
    \task $\mqty[1 \\ 4 \\ 3]$
    \task $\mqty[2 & 0 & 4 & 6]$
    \task $1$
    \task $A^t = \mqty[
    1 & 2 & 1 \\
    1 & 0 & 1 \\
    1 & 4 & 3 \\
    3 & 6 & 7 \\
    ]$
\end{tasks}

\subsection*{2.6.5}
\[
    A = \mqty[
    \frac{9}{5} & -\frac{6}{5} \\
    -\frac{1}{5} & \frac{4}{5}
    ]
.\]

\subsection*{2.6.7}
Note that for any entry in $A_{ij}$ that
\[
    A_{ij} + \mathbf{0}_{ij} = \mathbf{0}_{ij} + A_{ij} = A_{ij} + 0 = A_{ij}
.\]
Therefore the resultant matrix remains unchanged.

\subsection*{2.6.11}
If $AB$ has size $m \times n$, then $BA$ has size $n \times m$.

\subsection*{2.6.15}
\[
    A^n = \mqty[
    1 & n \\
    0 & 1 \\
    ]
.\]

This can be shown via induction using the base case
\[
    A^2  = AA = \mqty[
    1 & 1 \\
    0 & 1 \\
    ] \mqty[
    1 & 1 \\
    0 & 1 \\
    ] = \mqty[
    1 & 2 \\
    0 & 1 \\
    ]
.\]

\subsection*{2.6.25}
The system can be expressed as a matrix linear equation
\[
    \underbrace{\mqty[
    7 & -8 & 5 \\
    -4 & 5 & -3 \\
    1 & -1 & 1
    ]}_A 
    \underbrace{\mqty[
    x_1 \\
    x_2 \\
    x_3
    ]}_x =
    \underbrace{\mqty[
    5 \\
    -3 \\
    0
    ]}_b
.\]

The solution is then $A^{-1} b$ where $A^{-1}$ is given in exercise 18. Hence
\[
    x_1 = 1, x_2 = -1, x_3 = -2
.\]

\subsection*{2.6.27}
\begin{tasks}(3)
    \task
    \[
        A \lor B = \mqty[
        1 & 1 & 1 \\
        1 & 1 & 1 \\
        1 & 0 & 1 \\
        ]
    .\]
    \task
    \[
        A \land B = \mqty[
        0 & 0 & 1 \\
        1 & 0 & 0 \\
        0 & 0 & 1 \\
        ]
    .\]
    \task
    \[
        A \odot B = \mqty[
        1 & 1 & 1 \\
        1 & 1 & 1 \\
        1 & 0 & 1 \\
        ]
    .\]
\end{tasks}

\subsection*{2.6.29}
\begin{tasks}(3)
    \task
    \[
        A^{[2]} = \mqty[
        1 & 0 & 0 \\
        1 & 1 & 0 \\
        1 & 0 & 1
        ]
    .\]
    \task
    \[
        A^{[3]} = \mqty[
        1 & 0 & 0 \\
        1 & 0 & 1 \\
        1 & 1 & 0
        ]
    .\]
    \task
    \[
        A \lor A^{[2]} \lor A^{[3]} = \mqty[
        1 & 0 & 0 \\
        1 & 1 & 1 \\
        1 & 1 & 1
        ]
    .\]
\end{tasks}

\subsection*{3.1.1}
max := 1 $\to$
i := 2 $\to$
max := 8 $\to$
i := 3 $\to$
max := 12 $\to$
i := 4 $\to$
i := 5 $\to$
i := 6 $\to$
i := 7 $\to$
max := 14 $\to$
i := 8 $\to$
i := 9 $\to$
i := 10 $\to$
i := 11

\subsection*{3.1.3}
\begin{algorithm}[H]
    \caption{Sum of sequece of integers $a_1, \ldots, a_n$}
    \begin{algorithmic}[1]
        \Function{sum\_sequence}{$a_1, \ldots, a_n$}
            \State sum $\coloneq a_1$
            \For{$i = 2$ to ${n}$}
                \State $\text{sum} = \text{sum} + a_i$
            \EndFor
            \State\Return sum
        \EndFunction
    \end{algorithmic}
\end{algorithm}

\subsection*{3.1.9}
\begin{algorithm}[H]
    \caption{Determine if a given string is a palindrome}
    \begin{algorithmic}[1]
        \Function{is\_palindrome}{$s_0 s_1 \ldots s_{n-1}:$ string}
            \For{$i = 0$ to $\left\lfloor \frac{n}{2} \right\rfloor - 1$}
            \If{$s_i \neq s_{n - i - 1}$}
                \State \Return False
            \EndIf
            \EndFor
            \State\Return True
        \EndFunction
    \end{algorithmic}
\end{algorithm}

\subsection*{3.1.13}
\begin{enumerate}[leftmargin=3cm]
    \item[Linear Search)] i := 1 $\to$ i := 2 $\to$ i := 3 $\to$ i := 4 $\to$ i := 5 $\to$ i := 6 $\to$ i := 7 $\to$ location := 7
    \item[Binary Search)] i := 1 $\to$ j := 8 $\to$ m := 4 $\to$ i := 5 $\to$ m := 6 $\to$ i := 7 $\to$ m := 7 $\to$ j := 7 $\to$ location := 7
\end{enumerate}

\subsection*{3.2.1}
\begin{tasks}(3)
    \task $C = 1, k = 10$
    \task $C = 4, k = 7 $
    \task Is not $O(x)$ 
    \task $C = 5, k = 1 $
    \task $C = 1, k = 0 $
    \task $C = 1, k = 2 $
\end{tasks}

\subsection*{3.2.3}
Since $x^4 + 9x^3 + 4x + 7 \leq 10000x^3$ for $x > 10$, we have witness $C = 10000, k = 10$.

\subsection*{3.2.7}
\begin{tasks}(2)
    \task $n = 3$
    \task $n = 3$
    \task $n = 1$
    \task $n = 0$
\end{tasks}

\subsection*{3.2.11}
Since we are dealing with polynomials,
\[
    O\qty(3x^4 + 1) = O\qty(x^4) = O\qty(\frac{x^4}{2})
.\]

\subsection*{3.3.3}
It is $O\qty(n^2)$ since 
\[
    \sum_{i=0}^n \sum_{k=i}^n 1 \sim n^2
.\]

\subsection*{3.3.11}
\subsubsection*{Part A}
\begin{algorithm}[H]
    \caption{Determine if a disjoint subset pair exists}
    \begin{algorithmic}[1]
        \Function{exists\_disjoint}{$S_1, \ldots, S_n$: subsets of $\qty{1, \ldots, n}$}
        \State disjoint $\coloneq$ False
        \For{$i = 1$ to $n$}
            \For{$j = i+1$ to $n$}
                \State disjoint = True 
                \For{$k = 1$ to $n$}
                \If{$k \in S_i \land k \in S_j$}
                    \State disjoint = False
                \EndIf
                \EndFor
            \EndFor
        \EndFor
        \State \Return disjoint
        \EndFunction
    \end{algorithmic}
\end{algorithm}

\subsubsection*{Part B}
Since there are $O\qty(n^2)$ sets considered and $O\qty(n)$ integers tested for each, then in total the answer is on the order of $O\qty(n^3)$.

\end{document}
