\documentclass[12pt]{extarticle}

\usepackage{multicol}

% Document Layout and Font
\usepackage{subfiles}
\usepackage[margin=2cm, headheight=15pt]{geometry}
\usepackage{fancyhdr}
\usepackage{enumitem}	
\usepackage{wrapfig}
\usepackage{float}
\usepackage{multicol}

\usepackage[p,osf]{scholax}

\renewcommand*\contentsname{Table of Contents}
\renewcommand{\headrulewidth}{0pt}
\pagestyle{fancy}
\fancyhf{}
\fancyfoot[R]{$\thepage$}
\setlength{\parindent}{0cm}
\setlength{\headheight}{17pt}
\hfuzz=9pt

% Figures
\usepackage{svg}

% Utility Management
\usepackage{color}
\usepackage{colortbl}
\usepackage{xcolor}
\usepackage{xpatch}
\usepackage{xparse}

\definecolor{gBlue}{HTML}{7daea3}
\definecolor{gOrange}{HTML}{e78a4e}
\definecolor{gGreen}{HTML}{a9b665}
\definecolor{gPurple}{HTML}{d3869b}

\definecolor{links}{HTML}{1c73a5}
\definecolor{bar}{HTML}{584AA8}

% Math Packages
\usepackage{mathtools, amsmath, amsthm, thmtools, amssymb, physics}
\usepackage[scaled=1.075,ncf,vvarbb]{newtxmath}

\newcommand\B{\mathbb{C}}
\newcommand\C{\mathbb{C}}
\newcommand\R{\mathbb{R}}
\newcommand\Q{\mathbb{Q}}
\newcommand\N{\mathbb{N}}
\newcommand\Z{\mathbb{Z}}

\DeclareMathOperator{\lcm}{lcm}

% Probability Theory
\newcommand\Prob[1]{\mathbb{P}\qty(#1)}
\newcommand\Var[1]{\text{Var}\qty(#1)}
\newcommand\Exp[1]{\mathbb{E}\qty[#1]}

% Analysis
\newcommand\ball[1]{\B\qty(#1)}
\newcommand\conj[1]{\overline{#1}}
\DeclareMathOperator{\Arg}{Arg}
\DeclareMathOperator{\cis}{cis}

% Linear Algebra
\DeclareMathOperator{\dom}{dom}
\DeclareMathOperator{\range}{range}
\DeclareMathOperator{\spann}{span}
\DeclareMathOperator{\nullity}{nullity}

% TIKZ
\usepackage{tikz}
\usepackage{pgfplots}
\usetikzlibrary{arrows.meta}
\usetikzlibrary{math}
\usetikzlibrary{cd}

% Boxes and Theorems
\usepackage[most]{tcolorbox}
\tcbuselibrary{skins}
\tcbuselibrary{breakable}
\tcbuselibrary{theorems}

\newtheoremstyle{default}{0pt}{0pt}{}{}{\bfseries}{\normalfont.}{0.5em}{}
\theoremstyle{default}

\renewcommand*{\proofname}{\textit{\textbf{Proof.}}}
\renewcommand*{\qedsymbol}{$\blacksquare$}
\tcolorboxenvironment{proof}{
	breakable,
	coltitle = black,
	colback = white,
	frame hidden,
	boxrule = 0pt,
	boxsep = 0pt,
	borderline west={3pt}{0pt}{bar},
	% borderline west={3pt}{0pt}{gPurple},
	sharp corners = all,
	enhanced,
}

\newtheorem{theorem}{Theorem}[section]{\bfseries}{}
\tcolorboxenvironment{theorem}{
	breakable,
	enhanced,
	boxrule = 0pt,
	frame hidden,
	coltitle = black,
	colback = blue!7,
	% colback = gBlue!30,
	left = 0.5em,
	sharp corners = all,
}

\newtheorem{corollary}{Corollary}[section]{\bfseries}{}
\tcolorboxenvironment{corollary}{
	breakable,
	enhanced,
	boxrule = 0pt,
	frame hidden,
	coltitle = black,
	colback = white!0,
	left = 0.5em,
	sharp corners = all,
}

\newtheorem{lemma}{Lemma}[section]{\bfseries}{}
\tcolorboxenvironment{lemma}{
	breakable,
	enhanced,
	boxrule = 0pt,
	frame hidden,
	coltitle = black,
	colback = green!7,
	left = 0.5em,
	sharp corners = all,
}

\newtheorem{definition}{Definition}[section]{\bfseries}{}
\tcolorboxenvironment{definition}{
	breakable,
	coltitle = black,
	colback = white,
	frame hidden,
	boxsep = 0pt,
	boxrule = 0pt,
	borderline west = {3pt}{0pt}{orange},
	% borderline west = {3pt}{0pt}{gOrange},
	sharp corners = all,
	enhanced,
}

\newtheorem{example}{Example}[section]{\bfseries}{}
\tcolorboxenvironment{example}{
	% title = \textbf{Example},
	% detach title,
	% before upper = {\tcbtitle\quad},
	breakable,
	coltitle = black,
	colback = white,
	frame hidden,
	boxrule = 0pt,
	boxsep = 0pt,
	borderline west={3pt}{0pt}{green!70!black},
	% borderline west={3pt}{0pt}{gGreen},
	sharp corners = all,
	enhanced,
}

\newtheoremstyle{remark}{0pt}{4pt}{}{}{\bfseries\itshape}{\normalfont.}{0.5em}{}
\theoremstyle{remark}
\newtheorem*{remark}{Remark}


% TColorBoxes
\newtcolorbox{week}{
	colback = black,
	coltext = white,
	fontupper = {\large\bfseries},
	width = 1.2\paperwidth,
	size = fbox,
	halign upper = center,
	center
}

\newcommand{\banner}[2]{
    \pagebreak
    \begin{week}
   		\section*{#1}
    \end{week}
    \addcontentsline{toc}{section}{#1}
    \addtocounter{section}{1}
    \setcounter{subsection}{0}
}

% Hyperref
\usepackage{hyperref}
\hypersetup{
	colorlinks=true,
	linktoc=all,
	linkcolor=links,
	bookmarksopen=true
}

% Error Handling
\PackageWarningNoLine{ExtSizes}{It is better to use one of the extsizes 
                          classes,^^J if you can}


\fancyhead[R]{Homework \#2}
\fancyhead[L]{Eli Griffiths}
\renewcommand{\headrulewidth}{1pt}
\setlength\parindent{0pt}

% Section 3: 
% 2, 8, 11, 19, 26, 28, 33

% Section 4: 
% 6, 9, 11, 29, 31, 32, 33, 34, 37, 41

\begin{document}
\section*{Problem 3.2}
$\phi$ is an isomorphism since it is one-to-one and onto and $\phi(n+m) = - (n + m) = (-n) + (-m) = \phi(n) + \phi(m)$ for all $m,n \in \mathbb{Z}$.

\section*{Problem 3.8}
$\phi$ is not an isomorphism because it is not one-to-one. Consider the following two matrices
\[
	A = \mqty[3 & 0 \\ 0 & 2], B = \mqty[2 & 0 \\ 0 & 3]
.\]
It is clear that $A \neq B$. However $\det(A) = \det(B) = 6$, hence $\phi$ is not one-to-one and therefore not an isomorphism.

\section*{Problem 3.11}
$\phi$ is not an isomorphism because it is not one-to-one. Consider $f(x) = x^2 + 3$ and $g(x) = x^2 + 4$. Note that $f'(x) = g'(x) = 2x$. However since $f(x) \neq g(x)$, $\phi$ is not one-ton-one and hence not an isomorphism.

\section*{Problem 3.19}
\subsection*{Part A}
Define the binary operation $*$ by
\[
	a * b = \frac{(a+1)\cdot (b+1)}{3} - 1
.\]
Note that this satisifes the homorphism property since
\[
	\phi(x \cdot y) = 3xy - 1
.\]
and
\begin{align*}
	\phi(x) * \phi(y) &= (3x - 1) * (3y - 1) \\
										&= \frac{(3x - 1 + 1)\cdot (3y - 1 + 1)}{3} - 1 \\
										&= \frac{3x\cdot 3y}{3} - 1 \\
										&= \frac{9xy}{3} - 1 \\
										&= 3xy - 1
.\end{align*}
Therefore since $\phi(x \cdot y) = \phi(x) * \phi(y)$, $\phi$ is homomorphic and since it is a bijection it is an isomorphism between $\langle \mathbb{Q}, \cdot \rangle$ and $\langle \mathbb{Q}, * \rangle$. The identity element for $*$ is $2$ since for all $a \in \mathbb{Q}$,
\begin{align*}
	2 * a &= \frac{(2+1)(a+1)}{3} - 1 \\
				&= a + 1 - 1 = a
.\end{align*}
and
\begin{align*}
	a * 2 &= \frac{(a + 1)(2 + 1)}{3} - 1 \\
	&= a + 1 - 1 = a
.\end{align*}.

\subsection*{Part B}
Since $\phi$ is one-to-one and onto, it is invertible. Therefore
\[
	\phi^{-1}(x) = \frac{x+1}{3}
.\]
Since $\phi^{-1}$ must also be an isomorphism
\begin{align*}
	a * b &= \phi^{-1}(3a - 1) \cdot \phi^{-1}(3b - 1) \\
				&= \phi^{-1} ((3a - 1) \cdot (3b -1)) \\
				&= \phi^{-1} (9ab - 3a - 3b + 1) \\
				&= \frac{9ab - 3a - 3b + 1 + 1}{3} \\
				&= 3ab - a - b + \frac{2}{3}
.\end{align*}
The identity element of $\langle \mathbb{Q}, \cdot \rangle$ is preserved under $\phi$, therefore the identity element of $\langle \mathbb{Q}, * \rangle$ is
\[
	\phi^{-1}(1) = \frac{2}{3}
.\]

\section*{3.26}
\begin{proof}
	Let $\langle S, * \rangle$ and $\langle S', *' \rangle$ be binary algebraic structures and assume there exists an isomorphism $\phi : S \to S'$. Consider the inverse map $\phi^{-1} : S' \to S$. Since $\phi$ is an isomorphism, it is one-to-one and onto and therefore its inverse is also one-to-one and onto. Let $a',b' \in S'$. By the properties of inverses
	\[
		\phi(\phi^{-1} (a' *' b')) = a' *' b'
	.\]
	Since $\phi$ is an isomorphism
	\begin{align*}
		\phi(\phi(a') * \phi(b')) &= \phi(\phi^{-1} (a')) *' \phi(\phi^{-1}(b')) \\
		&= a' *' b'
	.\end{align*}
	Therefore since both equations are equal to $a' *' b'$, it follows that
	\begin{align*}
		\phi(\phi^{-1}(a' *' b')) &= \phi(\phi^{-1}(a') * \phi^{-1}(b')) \\
		\phi^{-1}(a' *' b') &= \phi^{-1}(a') * \phi^{-1}(b')
	,\end{align*}
	meaning $\phi^{-1}$ is a homorphism. Therefore since $\phi^{-1}$ is one-to-one, onto, and homomorphic, it is an isomorphism from $\langle S', *' \rangle$ to $\langle S, * \rangle$.
\end{proof}

\section*{3.28}
\begin{proof}
	Let $A$ be a set of binary algebraic structures and define a relation $\simeq$ over $A$ such that
	\[
		\langle S, * \rangle \simeq \langle S', *' \rangle \Longleftrightarrow \langle S, * \rangle\text{ is isomorphic to }\langle S', *' \rangle
	.\]
	Proceed to show that $\simeq$ is an equivalence relation.
	\\

	\qquad\begin{minipage}{\dimexpr\textwidth-2cm}
		(Reflexivity)\quad Let $\langle S, * \rangle \in A$. Define a mapping $\phi : S \to S : a \mapsto a$. Let $a,b \in S$ and assume $\phi(a) = \phi(b)$. Then $a = b$, hence $\phi$ is one-to-one. Let $b \in S$. Then $\phi(b) = b$, meaning $\phi$ is onto. Additionally, $\phi(a * b) = a*b = \phi(a) * \phi(b)$ meaning $\phi$ is homomorphic. Therefore $\phi$ is an isomorphism, meaning $\langle S, * \rangle \simeq \langle S, * \rangle$.
	\end{minipage} \\
  \\

	\qquad\begin{minipage}{\dimexpr\textwidth-2cm}
		(Symmetry)\quad Let $\langle S, * \rangle, \langle S', *' \rangle \in A$. Assume that $\langle S, * \rangle \simeq \langle S', *' \rangle$. By the result in ($3.26$), it follows there is an isomorphic map from $\langle S', *' \rangle$ to $\langle S, * \rangle$, meaning $\langle S', *' \rangle \simeq \langle S, * \rangle$.
	\end{minipage} \\
	\\

	\qquad\begin{minipage}{\dimexpr\textwidth-2cm}
		(Transitivty)\quad Let $\langle S, * \rangle, \langle S', *' \rangle, \langle S'', *'' \rangle \in A$. For simplicity, denote each structure by its set. Assume $S \simeq S'$ and $S' \simeq S''$. Therefore $S$ is isomorphic to $S'$ and $S'$ is isomorphic to $S''$. By the result in ($3.27$), $S$ is isomorphic to $S''$. Hence $S \simeq S''$
	\end{minipage} \\
	\\

	Since $\simeq$ is reflexive, symmetric, and transitive, it is an equivalent relation.
\end{proof}

\section*{3.33}
\subsection*{Part A}
\begin{proof}
	Let $H \subseteq M_2(\mathbb{R})$ such that an element of $H$ is of the form $\mqty[a & -b \\ b & a]$ with $a,b \in \mathbb{R}$. Define a map $\phi : \mathbb{C} \to H$ such that for a complex number $z$ in its cartesian form $a + bi$
	\[
		\phi(z) = \mqty[a & -b \\ b & a]
	.\]
	Examine the conditions for $\phi$ to be an isomorphism. \\

	\qquad\begin{minipage}{\dimexpr\textwidth-2cm}
		(One-to-One)\quad 
		Let $z_1, z_2 \in \mathbb{C}$. Then there exists $a,b,c,d \in \mathbb{R}$ such that $z_1 = a+bi$ and $z_2 = c+di$. Assume $\phi(z_1) = \phi(z_2)$. Then
		\begin{align*}
			\phi(z_1) &= \phi(z_2) \\
			\phi(a+bi) &= \phi(c+di) \\
			\mqty[a & -b \\ b & a] &= \mqty[c & -d \\ d & c]
		.\end{align*}
		For the two matrices to be equal, $a = c$ and $b = d$. Therefore $z_1 = z_2$, hence $\phi$ is one-to-one.
	\end{minipage} \\
	\\

	\qquad\begin{minipage}{\dimexpr\textwidth-2cm}
		(Onto)\quad 
		Let $M \in H$. Then there exists $a,b \in \mathbb{R}$ such that
		$
			M = \mqty[a & -b \\ b & a]
		$
		Let $z = a+bi \in \mathbb{C}$. Then
		\[
			\phi(z) = \mqty[a & -b \\ b & a] = M
		.\]
		Therefore $\phi$ is onto.
	\end{minipage} \\
	\\

	\qquad\begin{minipage}{\dimexpr\textwidth-2cm}
		(Homomorphic)\quad 
		Let $z_1, z_2 \in \mathbb{C}$. Then there exists $a,b,c,d \in \mathbb{R}$ such that $z_1 = a+bi$ and $z_2 = c+di$. It follows that
		\begin{align*}
			\phi((a+bi) + (c+di)) &= \phi((a+c) + (b+d)i) \\
														&= \mqty[a + c & -b-d \\ b + d & a + c]
		.\end{align*}
		Additionally,
		\begin{align*}
			\phi(a+bi) + \phi(c+di) &= \mqty[a & -b \\ b & a] + \mqty[c & -d \\ d & c] \\
														&= \mqty[a + c & -b-d \\ b + d & a + c]
		.\end{align*}
		Therefore $\phi(z_1 + z_2) = \phi(z_1) + \phi(z_2)$ meaning $\phi$ is homomorphic.
	\end{minipage} \\
	\\

	Since $\phi$ is one-to-one, onto, and homomorphic, it is an isomorphism between $\langle \mathbb{C}, + \rangle$ and $\langle H, + \rangle$.
\end{proof}

\subsection*{Part B}
\begin{proof}
	Let $H \subseteq M_2(\mathbb{R})$ such that an element of $H$ is of the form $\mqty[a & -b \\ b & a]$ with $a,b \in \mathbb{R}$. Define a map $\phi : \mathbb{C} \to H$ such that for a complex number $z$ in its cartesian form $a + bi$
	\[
		\phi(z) = \mqty[a & -b \\ b & a]
	.\]
	From Part A, $\phi$ is one-to-one and onto. Examine $\phi$ for the homomorphism property. \\

	\qquad\begin{minipage}{\dimexpr\textwidth-2cm}
		(Homomorphic)\quad 
		Let $z_1, z_2 \in \mathbb{C}$. Then there exists $a,b,c,d \in \mathbb{R}$ such that $z_1 = a+bi$ and $z_2 = c+di$. It follows that
		\begin{align*}
			\phi((a+bi)(c+di)) &= \phi((ac - bd) + (ad + bc)i) \\
														&= \mqty[ac - bd & -ad-bc \\ ad+bc & ac-bd]
		.\end{align*}
		Additionally,
		\begin{align*}
			\phi(a+bi) + \phi(c+di) &= \mqty[a & -b \\ b & a]\mqty[c & -d \\ d & c] \\
															&= \mqty[ac - bd & -ad - bc \\ ad + bc & ac - bd]
		.\end{align*}
		Therefore $\phi(z_1 + z_2) = \phi(z_1)\phi(z_2)$ meaning $\phi$ is homomorphic.
	\end{minipage} \\
	\\

	Since $\phi$ is one-to-one, onto, and homomorphic, it is an isomorphism between $\langle \mathbb{C}, \cdot \rangle$ and $\langle H, \cdot \rangle$.
\end{proof}

\section*{4.6}
$\langle \mathbb{C}, * \rangle$ is not a group since there is no inverse element for $0$.

\section*{4.9}
Consider the following equation for each respective group with $x$ being an element of a given group, $e$ being the associated identity, and $*$ being the associated operation. Then the equation
\[
	x * x * x = e
\]
will have $1$ solution in $\mathbb{R}$, $1$ solution in $\mathbb{R}^*$, but $3$ solutions in $U$. Therefore $\langle U, \cdot \rangle$ cannot be isomorphic to either $\langle \mathbb{R}, + \rangle$ or $\langle \mathbb{R}^*, \cdot \rangle$.

\section*{4.11}
The set of all $n \times n$ diagonal matrices under matrix addition is a group.

\begin{proof}
	Let $D_n$ denote the set of all $n \times n$ diagonal matrices define the binary structure $\langle D_n, + \rangle$ where $+$ is normal matrix addition. Examine the three axioms of a group. \\

	\quad\begin{minipage}{\dimexpr\textwidth-2cm}
		(Associativity)\quad
		Let $A, B, C \in D_n$. Then it quickly follows that
		\[
			A + (B + C) = (A + B) + C
		\]
		since matrix addition is associative.
	\end{minipage} \\
	\\

	\quad\begin{minipage}{\dimexpr\textwidth-2cm}
		(Identity Element)\quad
		Let $e$ be the $n \times n$ matrix with all zero entries. Clearly $e \in D_n$ and given a matrix $A \in D_n$,
		\[
			A + e = e + A = A
		.\]
		hence $e$ is the indentity element.
	\end{minipage} \\
	\\

	\quad\begin{minipage}{\dimexpr\textwidth-2cm}
		(Inverse)\quad
		Let $A \in D_n$. Let $A'$ be the diagonal matrix where the diagonal is the negation of $A$. Therefore
		\[
			A + A' = A' + A = A - A = e
		.\]
	\end{minipage} \\
	\\

	Since $\langle D_n, + \rangle$ follows the three axioms of a group, it is a group.
\end{proof}

\section*{4.29}
\begin{proof}
	Let $G$ be a finite group with an even number of elements. Consider the following set
	\[
		S = \qty{a \in G : a \neq a'}
	.\]
	Note that $|S|$ must be even since entries are paired by $a, a'$. Since $|G|$ is even and $|S|$ is even, $|G - S|$ must also be even. $|G - S| \neq 0$ since the identity element $e \in G$ is in $G$ but not in $S$, so it is in $G - S$. However, since $|G - S|$ is even, there must be at least one other element in $G - S$, meaning there is another element $a \in G$ that isnt the identity such that $aa = a$
\end{proof}

\section*{4.31}
\begin{proof}
	Let $\langle G, * \rangle$ be a group. Let $e \in G$ denote the identity element of $G$. It is trivial that $e$ is idempotent for $*$ since $e * e = e$. Therefore there is at least one idempotent for $*$. Assume towards contradiction there exists an element $x \in G \neq e$ that is also an idempotent for $*$. Since $x$ is an idempotent,
	\[
		x * x = x
	.\]
	Since $x$ is an element of a group, $x$ has an inverse $x'$. Therefore
	\begin{align*}
		x * x &= x \\
		x * x * x' &= x * x' \\
		x * e &= e \\
		x &= e
	.\end{align*}
	However, this contradicts the assumption that $a \neq e$. Therefore there cannot be any other idempotents for $*$ besides an identity element $e$. By the uniqueness of the identity element, there is only one identity for $G$, hence $e$ is the only idempotent for $*$.
\end{proof}

\section*{4.32}
\begin{proof}
	Let $G$ be a group with identity $*$ and assume that for all $x \in G$ that $x * x = e$. Therefore for all $x \in G$
	\begin{align*}
		x * x &= e \\
		x * x * x' &= e * x' \\
		x * e &= x' \\
		x &= x'
	.\end{align*}
	Let $a,b \in G$. Consider $(a*b) * (a*b)$. Then
	\begin{align*}
		(a*b) * (a*b) &= e \\
		a*b &= (a*b)' \\
		a*b &= b'*a' \\
		a*b &= b*a
	.\end{align*}
	Therefore $G$ is abelian.
\end{proof}

\section*{4.33}
\begin{proof}
	Proceed with induction. Let $G$ be an abelian group with $a, b \in G$. Consider the base case where $n=1$. Then
	\[
		(a * b)^1 = a * b = a^1 * b^1
	.\]
	Therefore the base case holds. Assume for some fixed $n \in \mathbb{Z}^+$ that $(a*b)^n = a^n * b^n$. Then
	\begin{align*}
		(a*b)^{n+1} &= (a*b) * (a*b)^n \\
		&= a*b * a^n * b^n \\
		&= a * a^n * b * b^n \\
		&= a^{n+1} * b^{n+1}
	.\end{align*}
	Therefore the $n+1$ case holds, meaning for $n \in \mathbb{Z}^+$ that for all $a,b \in G$ that $(a*b)^n = a^n * b^n$.
\end{proof}

\section*{4.34}
\begin{proof}
	Let $G$ be a finite group and let $a \in G$. Consider the set $S = \qty{a, a^2, a^3, \ldots, a^m, a^{m+1}}$ where $m = |G|$. Since there are $m+1$ elements in $S$, there has to be a repeat otherwise $S$ would contain $m+1$ unique elements which is larger than $|G|$. Therefore there exists $\alpha, \beta \in \mathbb{Z}^+$ such that $\alpha \neq \beta$ and $a^\alpha = a^\beta$. Without loss of generality let $\alpha < \beta$. Then
	\begin{align*}
		a^\beta &= a^\alpha \\
		a^{\beta - \alpha} &= e
	.\end{align*}
	Since $\alpha < \beta$, $\beta - \alpha > 0$ meaning $\beta - \alpha \in \mathbb{Z}^+$. Therefore for any $a \in G$ there exists a $n \in \mathbb{Z}^+$ such that $a^n = e$.
\end{proof}

\section*{4.37}
\begin{proof}
	Let $G$ be a group and $a,b,c \in G$. Assume that $a * b * c = e$. Then
	\begin{align*}
		a * b * c &= e \\
		a' * a * b * c &= e * a' \\
		b * c &= a' \\
		b * c * c' &= a' * c' \\
		b &= a' * c' \\
		b * c &= a' * c' * c \\
		b * c &= a' \\
		b * c * a &= a' * a \\
		b * c * a &= e
	.\end{align*}
	Therefore for all $a,b,c \in G$, if $a * b * c = e$ then $b * c * a = e$.
\end{proof}

\section*{4.41}
\begin{proof}
	Let $G$ be a group and $g \in G$. Define the map $i_g : G \to G$ such that $i_g(x) = gxg'$ for $x \in G$. Check the conditions that $i_g$ is an isomorphism of $G$ with itself. \\

	\quad\begin{minipage}{\dimexpr\textwidth-2cm}
		(One-to-One)\quad
		Let $a,b \in G$ and assume that $i_g (a) = i_g (b)$. Then
		\begin{align*}
			i_g(a) &= i_g(b) \\
			gag' &= gbg' \\
			gag'g &= gbg'g \\
			ga &= gb \\
			g'ga &= g'gb \\
			a &= b
		.\end{align*}
		Therefore $i_g$ is one-to-one.
	\end{minipage} \\
	\\

	\quad\begin{minipage}{\dimexpr\textwidth-2cm}
		(Onto)\quad
		Let $b \in G$ and let $a = g'bg$. Then
		\begin{align*}
			i_g(a) &= gag' \\
			&= gg'bgg' \\
			&= b
		.\end{align*}
		Therefore $i_g$ is onto.
	\end{minipage} \\
	\\

	\quad\begin{minipage}{\dimexpr\textwidth-2cm}
		(Homomorphic)\quad
		Let $a,b \in G$. Then
		\[
			i_g(ab) = gabg'
		.\]
		and
		\begin{align*}
			i_g(a)i_g(b) &= gag'gbg' \\
			&= gabg'
		.\end{align*}
		Therefore $i_g(ab) = i_g(a) i_g(b)$, meaning $i_g$ is homomorphic.
	\end{minipage} \\
	\\

	Therefore since $i_g$ is one-to-one, onto, and homomorphic, it is an isomorphism of $G$ with itself.

\end{proof}
	
\end{document}
