\documentclass[12pt,titlepage]{extarticle}
% Document Layout and Font
\usepackage{subfiles}
\usepackage[margin=2cm, headheight=15pt]{geometry}
\usepackage{fancyhdr}
\usepackage{enumitem}	
\usepackage{wrapfig}
\usepackage{float}
\usepackage{multicol}

\usepackage[p,osf]{scholax}

\renewcommand*\contentsname{Table of Contents}
\renewcommand{\headrulewidth}{0pt}
\pagestyle{fancy}
\fancyhf{}
\fancyfoot[R]{$\thepage$}
\setlength{\parindent}{0cm}
\setlength{\headheight}{17pt}
\hfuzz=9pt

% Figures
\usepackage{svg}

% Utility Management
\usepackage{color}
\usepackage{colortbl}
\usepackage{xcolor}
\usepackage{xpatch}
\usepackage{xparse}

\definecolor{gBlue}{HTML}{7daea3}
\definecolor{gOrange}{HTML}{e78a4e}
\definecolor{gGreen}{HTML}{a9b665}
\definecolor{gPurple}{HTML}{d3869b}

\definecolor{links}{HTML}{1c73a5}
\definecolor{bar}{HTML}{584AA8}

% Math Packages
\usepackage{mathtools, amsmath, amsthm, thmtools, amssymb, physics}
\usepackage[scaled=1.075,ncf,vvarbb]{newtxmath}

\newcommand\B{\mathbb{C}}
\newcommand\C{\mathbb{C}}
\newcommand\R{\mathbb{R}}
\newcommand\Q{\mathbb{Q}}
\newcommand\N{\mathbb{N}}
\newcommand\Z{\mathbb{Z}}

\DeclareMathOperator{\lcm}{lcm}

% Probability Theory
\newcommand\Prob[1]{\mathbb{P}\qty(#1)}
\newcommand\Var[1]{\text{Var}\qty(#1)}
\newcommand\Exp[1]{\mathbb{E}\qty[#1]}

% Analysis
\newcommand\ball[1]{\B\qty(#1)}
\newcommand\conj[1]{\overline{#1}}
\DeclareMathOperator{\Arg}{Arg}
\DeclareMathOperator{\cis}{cis}

% Linear Algebra
\DeclareMathOperator{\dom}{dom}
\DeclareMathOperator{\range}{range}
\DeclareMathOperator{\spann}{span}
\DeclareMathOperator{\nullity}{nullity}

% TIKZ
\usepackage{tikz}
\usepackage{pgfplots}
\usetikzlibrary{arrows.meta}
\usetikzlibrary{math}
\usetikzlibrary{cd}

% Boxes and Theorems
\usepackage[most]{tcolorbox}
\tcbuselibrary{skins}
\tcbuselibrary{breakable}
\tcbuselibrary{theorems}

\newtheoremstyle{default}{0pt}{0pt}{}{}{\bfseries}{\normalfont.}{0.5em}{}
\theoremstyle{default}

\renewcommand*{\proofname}{\textit{\textbf{Proof.}}}
\renewcommand*{\qedsymbol}{$\blacksquare$}
\tcolorboxenvironment{proof}{
	breakable,
	coltitle = black,
	colback = white,
	frame hidden,
	boxrule = 0pt,
	boxsep = 0pt,
	borderline west={3pt}{0pt}{bar},
	% borderline west={3pt}{0pt}{gPurple},
	sharp corners = all,
	enhanced,
}

\newtheorem{theorem}{Theorem}[section]{\bfseries}{}
\tcolorboxenvironment{theorem}{
	breakable,
	enhanced,
	boxrule = 0pt,
	frame hidden,
	coltitle = black,
	colback = blue!7,
	% colback = gBlue!30,
	left = 0.5em,
	sharp corners = all,
}

\newtheorem{corollary}{Corollary}[section]{\bfseries}{}
\tcolorboxenvironment{corollary}{
	breakable,
	enhanced,
	boxrule = 0pt,
	frame hidden,
	coltitle = black,
	colback = white!0,
	left = 0.5em,
	sharp corners = all,
}

\newtheorem{lemma}{Lemma}[section]{\bfseries}{}
\tcolorboxenvironment{lemma}{
	breakable,
	enhanced,
	boxrule = 0pt,
	frame hidden,
	coltitle = black,
	colback = green!7,
	left = 0.5em,
	sharp corners = all,
}

\newtheorem{definition}{Definition}[section]{\bfseries}{}
\tcolorboxenvironment{definition}{
	breakable,
	coltitle = black,
	colback = white,
	frame hidden,
	boxsep = 0pt,
	boxrule = 0pt,
	borderline west = {3pt}{0pt}{orange},
	% borderline west = {3pt}{0pt}{gOrange},
	sharp corners = all,
	enhanced,
}

\newtheorem{example}{Example}[section]{\bfseries}{}
\tcolorboxenvironment{example}{
	% title = \textbf{Example},
	% detach title,
	% before upper = {\tcbtitle\quad},
	breakable,
	coltitle = black,
	colback = white,
	frame hidden,
	boxrule = 0pt,
	boxsep = 0pt,
	borderline west={3pt}{0pt}{green!70!black},
	% borderline west={3pt}{0pt}{gGreen},
	sharp corners = all,
	enhanced,
}

\newtheoremstyle{remark}{0pt}{4pt}{}{}{\bfseries\itshape}{\normalfont.}{0.5em}{}
\theoremstyle{remark}
\newtheorem*{remark}{Remark}


% TColorBoxes
\newtcolorbox{week}{
	colback = black,
	coltext = white,
	fontupper = {\large\bfseries},
	width = 1.2\paperwidth,
	size = fbox,
	halign upper = center,
	center
}

\newcommand{\banner}[2]{
    \pagebreak
    \begin{week}
   		\section*{#1}
    \end{week}
    \addcontentsline{toc}{section}{#1}
    \addtocounter{section}{1}
    \setcounter{subsection}{0}
}

% Hyperref
\usepackage{hyperref}
\hypersetup{
	colorlinks=true,
	linktoc=all,
	linkcolor=links,
	bookmarksopen=true
}

% Error Handling
\PackageWarningNoLine{ExtSizes}{It is better to use one of the extsizes 
                          classes,^^J if you can}


\def\homeworknumber{4}
\fancyhead[R]{\textbf{Math 140A: Homework \#\homeworknumber}}
\fancyhead[L]{Eli Griffiths}
\renewcommand{\headrulewidth}{1pt}
\setlength\parindent{0pt}


% \usepackage{algorithm}
% \usepackage{algpseudocode}

\begin{document}

% 6.4: 1,7,9,11
% 9.1: 1,3,7,9,53,55,57
% 9.2: 1,5,,7,9,11,17,19
% 9.4: 1,5,7,9,19,25,27
% 9.5: 1,9,27
% 9.6: 1,7,23
% 10.1: 1,5,9 (light overview of the concept)

\subsection*{6.1.5}
Once at Denver, I have 6 choices of flights to get to San Francisco from which I have an independent choice of 7 more flights to get to New York. Therefore there are $6 \cdot 7 = 42$ ways for me to get to New York from Denver.

\subsection*{6.1.7}
There are 26 choices for each letter meaning there are $26^3 = 17576$ three letter initials.

\subsection*{6.1.9}
There is only 1 choice for the the first letter (the letter A) and then 26 choices for each remaining initial, hence there are $1 \cdot 26^2 = 676$ such initials.

\subsection*{6.1.23}
\begin{tasks}
    \task $\left\lfloor \frac{999}{7} \right\rfloor - \left\lceil \frac{100}{7} \right\rceil + 1 = 128$
    \task $(999 - 100 + 1) - \qty(\left\lfloor \frac{999}{7} \right\rfloor - \left\lceil \frac{100}{7} \right\rceil + 1) = 450$
    \task $\frac{999}{111} = 9$
    \task $(999 - 100 + 1) - \qty(\left\lfloor \frac{999}{4} \right\rfloor - \left\lceil \frac{100}{4} \right\rceil + 1) = 900 - (249 - 25 + 1) = 675$
    \task By the inclusion exclusion principle,
    \[
        \qty[\left\lfloor \frac{999}{3} \right\rfloor - \left\lceil \frac{100}{3} \right\rceil + 1] + \qty[\left\lfloor \frac{999}{4} \right\rfloor - \left\lceil \frac{100}{4} \right\rceil + 1] - \qty[\left\lfloor \frac{999}{12} \right\rfloor - \left\lceil \frac{100}{12} \right\rceil + 1] = 450
    .\]
    \task $(999 - 100 + 1) - 450 = 450$
    \task $\qty[\left\lfloor \frac{999}{3} \right\rfloor - \left\lceil \frac{100}{3} \right\rceil + 1] - \qty[\left\lfloor \frac{999}{12} \right\rfloor - \left\lceil \frac{100}{12} \right\rceil + 1] = 225$
    \task $\qty[\left\lfloor \frac{999}{12} \right\rfloor - \left\lceil \frac{100}{12} \right\rceil + 1] = 75$
\end{tasks}

\subsection*{6.1.27}
There are 3 choices of representative for each of the 50 states, hence $3^{50}$ choices.

\subsection*{6.1.29}
There are $26^2 \cdot 10^4 + 10^2 \cdot 26^4 = 52,457,600$ such license plates.

\subsection*{6.1.33}
\begin{tasks}(2)
    \task $(26-5)^8 = 37,822,859,361$
    \task $\frac{21!}{(21-8)!} = 8,204,716,800$
    \task $5 \cdot 26^7 = 40,159,050,880$
    \task $5 \cdot \frac{25!}{(25-7)!} = 12,113,640,000$
    \task $26^8 - 21^8 = 171,004,205,215$
    \task $8 \cdot 5 \cdot 21^7 = 72,043,541,640$
    \task $26^7 - 21^7 = 6,230,721,635$
    \task $26^6 - 21^6 = 223,149,655$
\end{tasks}

\subsection*{6.1.35}
\begin{tasks}(2)
    \task There are none
    \task $5! = 120$
    \task $\frac{6!}{(6-5)!} = 6! = 720$
    \task $\frac{7!}{(7-5)!} = 2520$
\end{tasks}

\subsection*{6.3.1}
\[
    \begin{array}{ccc}
        \qty{a,b,c} & \qty{b,a,c} & \qty{c,a,b} \\
        \qty{c,b,a} & \qty{b,c,a} & \qty{a,c,b}
    \end{array}
\]

\subsection*{6.3.3}
There are $6! = 720$ permutations.

\subsection*{6.3.5}
\begin{tasks}(2)
    \task $P(6,3) = \frac{6!}{3!} = 120$
    \task $P(6,5) = 6! = 720$
    \task $P(8,1) = 8$
    \task $P(8,5) = 8(7)(6)(5)(4) = 6720$
    \task $P(8,8) = 8! = 40,320$
    \task $P(10,9) = 10! = 3,628,800$
\end{tasks}

\subsection*{6.3.9}
\begin{tasks}
    \task $C(5,1) = 5$
    \task $C(5,3) = \frac{5(4)(3)}{3!} = 10$
\end{tasks}

\subsection*{6.3.13}
\subsection*{6.3.17}
\subsection*{6.3.21}
\subsection*{6.3.23}
\subsection*{6.3.29}

\end{document}
