\documentclass[12pt,titlepage]{extarticle}
% Document Layout and Font
\usepackage{subfiles}
\usepackage[margin=2cm, headheight=15pt]{geometry}
\usepackage{fancyhdr}
\usepackage{enumitem}	
\usepackage{wrapfig}
\usepackage{float}
\usepackage{multicol}

\usepackage[p,osf]{scholax}

\renewcommand*\contentsname{Table of Contents}
\renewcommand{\headrulewidth}{0pt}
\pagestyle{fancy}
\fancyhf{}
\fancyfoot[R]{$\thepage$}
\setlength{\parindent}{0cm}
\setlength{\headheight}{17pt}
\hfuzz=9pt

% Figures
\usepackage{svg}

% Utility Management
\usepackage{color}
\usepackage{colortbl}
\usepackage{xcolor}
\usepackage{xpatch}
\usepackage{xparse}

\definecolor{gBlue}{HTML}{7daea3}
\definecolor{gOrange}{HTML}{e78a4e}
\definecolor{gGreen}{HTML}{a9b665}
\definecolor{gPurple}{HTML}{d3869b}

\definecolor{links}{HTML}{1c73a5}
\definecolor{bar}{HTML}{584AA8}

% Math Packages
\usepackage{mathtools, amsmath, amsthm, thmtools, amssymb, physics}
\usepackage[scaled=1.075,ncf,vvarbb]{newtxmath}

\newcommand\B{\mathbb{C}}
\newcommand\C{\mathbb{C}}
\newcommand\R{\mathbb{R}}
\newcommand\Q{\mathbb{Q}}
\newcommand\N{\mathbb{N}}
\newcommand\Z{\mathbb{Z}}

\DeclareMathOperator{\lcm}{lcm}

% Probability Theory
\newcommand\Prob[1]{\mathbb{P}\qty(#1)}
\newcommand\Var[1]{\text{Var}\qty(#1)}
\newcommand\Exp[1]{\mathbb{E}\qty[#1]}

% Analysis
\newcommand\ball[1]{\B\qty(#1)}
\newcommand\conj[1]{\overline{#1}}
\DeclareMathOperator{\Arg}{Arg}
\DeclareMathOperator{\cis}{cis}

% Linear Algebra
\DeclareMathOperator{\dom}{dom}
\DeclareMathOperator{\range}{range}
\DeclareMathOperator{\spann}{span}
\DeclareMathOperator{\nullity}{nullity}

% TIKZ
\usepackage{tikz}
\usepackage{pgfplots}
\usetikzlibrary{arrows.meta}
\usetikzlibrary{math}
\usetikzlibrary{cd}

% Boxes and Theorems
\usepackage[most]{tcolorbox}
\tcbuselibrary{skins}
\tcbuselibrary{breakable}
\tcbuselibrary{theorems}

\newtheoremstyle{default}{0pt}{0pt}{}{}{\bfseries}{\normalfont.}{0.5em}{}
\theoremstyle{default}

\renewcommand*{\proofname}{\textit{\textbf{Proof.}}}
\renewcommand*{\qedsymbol}{$\blacksquare$}
\tcolorboxenvironment{proof}{
	breakable,
	coltitle = black,
	colback = white,
	frame hidden,
	boxrule = 0pt,
	boxsep = 0pt,
	borderline west={3pt}{0pt}{bar},
	% borderline west={3pt}{0pt}{gPurple},
	sharp corners = all,
	enhanced,
}

\newtheorem{theorem}{Theorem}[section]{\bfseries}{}
\tcolorboxenvironment{theorem}{
	breakable,
	enhanced,
	boxrule = 0pt,
	frame hidden,
	coltitle = black,
	colback = blue!7,
	% colback = gBlue!30,
	left = 0.5em,
	sharp corners = all,
}

\newtheorem{corollary}{Corollary}[section]{\bfseries}{}
\tcolorboxenvironment{corollary}{
	breakable,
	enhanced,
	boxrule = 0pt,
	frame hidden,
	coltitle = black,
	colback = white!0,
	left = 0.5em,
	sharp corners = all,
}

\newtheorem{lemma}{Lemma}[section]{\bfseries}{}
\tcolorboxenvironment{lemma}{
	breakable,
	enhanced,
	boxrule = 0pt,
	frame hidden,
	coltitle = black,
	colback = green!7,
	left = 0.5em,
	sharp corners = all,
}

\newtheorem{definition}{Definition}[section]{\bfseries}{}
\tcolorboxenvironment{definition}{
	breakable,
	coltitle = black,
	colback = white,
	frame hidden,
	boxsep = 0pt,
	boxrule = 0pt,
	borderline west = {3pt}{0pt}{orange},
	% borderline west = {3pt}{0pt}{gOrange},
	sharp corners = all,
	enhanced,
}

\newtheorem{example}{Example}[section]{\bfseries}{}
\tcolorboxenvironment{example}{
	% title = \textbf{Example},
	% detach title,
	% before upper = {\tcbtitle\quad},
	breakable,
	coltitle = black,
	colback = white,
	frame hidden,
	boxrule = 0pt,
	boxsep = 0pt,
	borderline west={3pt}{0pt}{green!70!black},
	% borderline west={3pt}{0pt}{gGreen},
	sharp corners = all,
	enhanced,
}

\newtheoremstyle{remark}{0pt}{4pt}{}{}{\bfseries\itshape}{\normalfont.}{0.5em}{}
\theoremstyle{remark}
\newtheorem*{remark}{Remark}


% TColorBoxes
\newtcolorbox{week}{
	colback = black,
	coltext = white,
	fontupper = {\large\bfseries},
	width = 1.2\paperwidth,
	size = fbox,
	halign upper = center,
	center
}

\newcommand{\banner}[2]{
    \pagebreak
    \begin{week}
   		\section*{#1}
    \end{week}
    \addcontentsline{toc}{section}{#1}
    \addtocounter{section}{1}
    \setcounter{subsection}{0}
}

% Hyperref
\usepackage{hyperref}
\hypersetup{
	colorlinks=true,
	linktoc=all,
	linkcolor=links,
	bookmarksopen=true
}

% Error Handling
\PackageWarningNoLine{ExtSizes}{It is better to use one of the extsizes 
                          classes,^^J if you can}


\def\homeworknumber{1}
\fancyhead[R]{\textbf{Math 140A: Homework \#\homeworknumber}}
\fancyhead[L]{Eli Griffiths}
\renewcommand{\headrulewidth}{1pt}
\setlength\parindent{0pt}


\begin{document}

% §18:  8, 10, 12, 18, 19, 24, 25, 37, 40, 48, 50;

\subsection*{8}
It does not form a ring because $\langle \Z_+, + \rangle$ doesn't have an identity and therefore cannot be a group.

\subsection*{10}
It does form a ring since $n \Z$ is a ring for $n \geq 1$ and the direct product of rings is also a ring. $n \Z$ is commutative for $n \geq 1$, meaning $2 \Z \times \Z$ is also commutative. It does not have unity since $2 \Z$ doesn't have unity. Since $2 \Z \times \Z$ doesn't have unity, it cannot be a field.

\subsection*{12} % TODO
Let $\mathcal{R} = \qty{a + b \sqrt{2} : a,b \in \Q}$. First, check that the binary operations are closed. Let $a,b,c,d \in \Q$. Then
\[
    \qty(a + b \sqrt{2}) + \qty(c + d \sqrt{2} ) = (a + c) + (b + d)\sqrt{2}
\]
and
\[
    \qty(a + b \sqrt{2}) \qty(c + d \sqrt{2} ) = ac + ad \sqrt{2} + bc \sqrt{2} + 2 bd = (ac + 2bd) + (ad + bc) \sqrt{2}
.\]
Since the resulting components are also rational numbers, both operations are closed. Next, check if $\langle \mathcal{R}, + \rangle$ is an abelian group.

\begin{enumerate}[leftmargin=2cm]
    \item[$\mathcal{G}_1)$]
        The given addition operation is associative and hence is also associative on $\mathcal{R}$
    \item[$\mathcal{G}_2)$]
        The additive identity $0 = 0 + 0 \sqrt{2}$ is in $\mathcal{R}$

    \item[$\mathcal{G}_3)$]
        Since for any $a + b \sqrt{2}$ $-a,-b \in \Q$, every element has an inverse
    \item[Abelian)]
        The given addition is commutative and therefore the group is abelian
\end{enumerate}

Since multiplication of real numbers is associative and $\mathcal{R} \subset \R$, multiplication is also associative. The given multiplication is also commutative and satisfies the distributive laws. $\mathcal{R}$ has unity since $1 + 0 \sqrt{2} \in \mathcal{R}$. Let $a,b \in \Q$ such that $a + b \sqrt{2} \neq 0$. Note that
\[
    \frac{1}{a + b \sqrt{2}} = \frac{1}{a + b \sqrt{2}} \cdot \frac{a - b \sqrt{2}}{a - b \sqrt{2}} = \frac{a - b \sqrt{2}}{a^2 - 2b^2} = \frac{a}{a - 2b^2} - \frac{b}{a^2 -2b^2} \sqrt{2}
.\]
Therefore every element is a unit. Therefore, in total $\mathcal{R}$ is a commutative division ring with unity and also a field.

\subsection*{18}
$1$ and $-1$ are the only units of $\Z$, and every non zero element in $\Q$ is a unit meaning the units of $\Z \times \Q \times \Z$ are of the form $(\pm 1, q, \pm 1)$ with $q \in \Q^*$.

\subsection*{19}
The units are $1$ and $3$ since $1\cdot 1 = 1$ and $3 \cdot 3 = 1$

\subsection*{24}
If $\phi : \Z \to \Z \times \Z$ is a ring homomorphism, then $\phi(1)^2 = \phi(1) \phi(1) = \phi(1^2) = \phi(1)$. Therefore $\phi(1)$ must be an element in $\Z \times \Z$ where its square is itself. Since $(0,0)$, $(1, 0)$, $(0,1)$ and $(1,1)$ are the only elements with this property in $\Z \times \Z$, the possible ring homomorphisms are
\begin{align*}
    \phi_{(0,0)} (n) = (0,0) \\
    \phi_{(1,0)} (n) = (n,0) \\
    \phi_{(0,1)} (n) = (0,n) \\
    \phi_{(1,1)} (n) = (n,n)
\end{align*}

$\phi_{(0,0)}$ is trivially a ring homorphism. For $\phi_{(1,0)}$,
\begin{alignat*}{5}
    \phi_{(1,0)}(a + b) &= (a + b&&, 0) &&= \phi_{(1,0)}(a) + \phi_{(1,0)}(b)) \\
    \phi_{(1,0)}(ab) &= (ab&&, 0) &&= \phi_{(1,0)}(a) \phi_{(1,0)}(b)
\end{alignat*}
Therefore $\phi_{(1,0)}$ is a homorphism and by a similar argument so is $\phi_{(0,1)}$. In the case of $\phi_{(1,1)}$
\begin{alignat*}{5}
    \phi_{(1,1)}(a + b) &= (a + b&&, a + b) &&= \phi_{(1,1)}(a) + \phi_{(1,1)}(b) \\
    \phi_{(1,1)}(ab) &= (ab&&, ab) &&= \phi_{(1,1)}(a) \phi_{(1,1)}(b)
\end{alignat*}

\subsection*{25}
If $\phi : \Z \times \Z \to \Z$ is a ring homomorphism, then the elements $(1,0)$ and $(0,1)$ (which are squares of themselves) must map to an element in $\Z$ whose square is itself. Since $0$ and $1$ satisfy this condition, there are 4 candidate homomorphisms.

\begin{align*}
    \phi_1((1,0)) = 1, \phi_1((0,1)) = 1 &\implies \phi_1((a,b)) = a+b \\ 
    \phi_2((1,0)) = 1, \phi_2((0,1)) = 0 &\implies \phi_2((a,b)) = a \\ 
    \phi_3((1,0)) = 0, \phi_3((0,1)) = 1 &\implies \phi_3((a,b)) = b \\ 
    \phi_4((1,0)) = 0, \phi_4((0,1)) = 0 &\implies \phi_4((a,b)) = 0
\end{align*}

$\phi_4$ is trivially a ring homomorphism. For $\phi_2$,
\[
    \phi_2((a,b) \cdot (c,d)) = \phi_2((ac, bd)) = ac = \phi((a,b)) \phi((c,d))
\]
therefore $\phi_2$ is also a ring homomorphism. By the same argument, $\phi_3$ is as well. However, note that
\[
    \phi_1((1,2) \cdot (2,3)) = \phi_1((2,6)) = 8 \neq 10 = \phi((1,2)) \phi((2,3))
\]
therefore $\phi_1$ is not a ring homomorphism. Hence the only ring homomorphisms from $\Z \times \Z$ to $\Z$ are $\phi_2, \phi_3$ and $\phi_4$.

\subsection*{37}
\begin{proof}
    Let $x,y \in U$. Assume towards contradiction that $xy \notin U$. Then $xy$ is not a unit and therefore has no multiplicative inverse. However, since $x$ and $y$ are units, they have multiplicative inverses. Therefore $(xy)(y^{-1}x^{-1}) = 1$. However this is a contradiction meaning $xy \in U$, hence $U$ is closed under $\cdot$. Examining the group axioms
    \begin{enumerate}
        \item[$\mathcal{G}_1.)$]
            Since $R$ is a ring, the operator $\cdot$ is associative and therefore is associative over $U$.
        \item[$\mathcal{G}_2.)$]
            Since $R$ is a ring with unity, it has a multiplicative identity $1$ meaning that $U$ will have an identity (specifically $1$).
        \item[$\mathcal{G}_3.)$]
            Let $x \in U$. Since it is a unit, it has a multiplicative inverse $x^{-1}$. But since $x$ is the inverse of $x^{-1}$, $x^{-1}$ is also a unit and so $x^{-1} \in U$. Therefore every element in $U$ has an inverse.
    \end{enumerate}
    Since $U$ is closed under $\cdot$ and satisfies the group axioms, $\langle G, \cdot \rangle$ is a group.
\end{proof}

\subsection*{40}
Take $\phi : 2 \Z \to 3 \Z$ to be a ring isomorphism. Since $\phi$ must be a group homorphism over addition, $\phi(2)$ must be equal to either $3$ or $-3$. Therefore $\phi(2n) = \phi(3n)$ or $\phi(2n) = \phi(-3n)$. Consider
\[
    \phi(2 \cdot 2) = \phi(4) = \pm 6 \neq 9 = \phi(2) \cdot \phi(2)
.\]
Therefore $\phi$ cannot be a ring homomorphism and hence cannot be a ring isomorphism. $\R$ and $\C$ are not isomorphic since every element in $\C$ can be written as the square of another element in $\C$, but the same does not hold in $\R$.

\subsection*{48}
\begin{proof}
    \hfill\begin{enumerate}
        \item[$\Rightarrow)$]
            Assume $S$ is a subring of $R$. Then the $S$ is an abelian group under $+$ meaning it must have the additive identity $0$. Furthermore, every element has an additive inverse and $S$ is closed under $+$ meaning $(a - b) \in S$ for all $a,b \in S$. $S$ must be closed under $\cdot$ by the assumption and therefore it follows $ab \in S$ for all $a,b \in S$.
        \item[$\Leftarrow)$]
            Examine the condtions for $S$ to be a subring of $R$
            \begin{enumerate}
                \item[Closure)]
                    Multiplication is closed since $ab \in S$ for all $a,b \in S$. Since $0 \in S$ and $a - b \in S$, $-b \in S$ for all $b \in S$ meaning $S$ contains its additive inverses. Therefore $a - (-b) \in S$ and hence $a + b \in S$ for all $a,b \in S$.
                \item[$\mathcal{R}_1)$]
                    Since addition from $R$ is associative and commutative and $S$ contains an additive identity and inverses, $\langle S, + \rangle$ is an abelian group.
                \item[$\mathcal{R}_2)$]
                    Multiplication from $R$ is associative meaning it is associative on $S$
                \item[$\mathcal{R}_3)$]
                    The left and right distributive laws hold for the multiplication operator and hence they hold on $S$.
            \end{enumerate}
    \end{enumerate}
\end{proof}

\subsection*{50}
\begin{proof}
    Since $a\cdot 0 = 0$ for any $a$, $0 \in I_a$. Let $x,y \in I_a$. Then $ax = 0$ and $ay = 0$. Therefore $ax - ay = 0 \implies a(x-y) = 0$ meaning $x-y \in I_a$. Furthermore, $a(xy) = (ax)y = 0y = 0$, meaning $xy \in I_a$. Therefore by Exercise $48$, $I_a$ is a subring of $R$.
\end{proof}

\end{document}
