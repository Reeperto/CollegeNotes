\documentclass{article}

% Document Layout and Font
\usepackage{subfiles}
\usepackage[margin=2cm, headheight=15pt]{geometry}
\usepackage{fancyhdr}
\usepackage{enumitem}	
\usepackage{wrapfig}
\usepackage{float}
\usepackage{multicol}

\usepackage[p,osf]{scholax}

\renewcommand*\contentsname{Table of Contents}
% \renewcommand{\headrulewidth}{0pt}
\pagestyle{fancy}
\fancyhf{}
\fancyfoot[R]{$\thepage$}
\setlength{\parindent}{0cm}
\setlength{\headheight}{17pt}
\hfuzz=9pt

% Figures
\usepackage{svg}

% Utility Management
\usepackage{color}
\usepackage{colortbl}
\usepackage{xcolor}
\usepackage{xpatch}
\usepackage{xparse}

\definecolor{gBlue}{HTML}{7daea3}
\definecolor{gOrange}{HTML}{e78a4e}
\definecolor{gGreen}{HTML}{a9b665}
\definecolor{gPurple}{HTML}{d3869b}

\definecolor{links}{HTML}{1c73a5}
\definecolor{bar}{HTML}{584AA8}

% Math Packages
\usepackage{mathtools, amsmath, amsthm, thmtools, amssymb, physics}
\usepackage[scaled=1.075,ncf,vvarbb]{newtxmath}

\newcommand\B{\mathbb{C}}
\newcommand\C{\mathbb{C}}
\newcommand\R{\mathbb{R}}
\newcommand\Q{\mathbb{Q}}
\newcommand\N{\mathbb{N}}
\newcommand\Z{\mathbb{Z}}

\DeclareMathOperator{\lcm}{lcm}

% Probability Theory
\newcommand\Prob[1]{\mathbb{P}\qty(#1)}
\newcommand\Var[1]{\text{Var}\qty(#1)}
\newcommand\Exp[1]{\mathbb{E}\qty[#1]}

% Analysis
\newcommand\ball[1]{\B\qty(#1)}
\newcommand\conj[1]{\overline{#1}}
\DeclareMathOperator{\Arg}{Arg}
\DeclareMathOperator{\cis}{cis}

% Linear Algebra
\DeclareMathOperator{\dom}{dom}
\DeclareMathOperator{\spann}{span}
\DeclareMathOperator{\nullity}{nullity}

% TIKZ
\usepackage{tikz}
\usepackage{pgfplots}
\usetikzlibrary{arrows.meta}
\usetikzlibrary{math}
\usetikzlibrary{cd}

% Boxes and Theorems
\usepackage[most]{tcolorbox}
\tcbuselibrary{skins}
\tcbuselibrary{breakable}
\tcbuselibrary{theorems}

\newtheoremstyle{default}{0pt}{0pt}{}{}{\bfseries}{\normalfont.}{0.5em}{}
\theoremstyle{default}

\renewcommand*{\proofname}{\textit{\textbf{Proof.}}}
\renewcommand*{\qedsymbol}{$\blacksquare$}
\tcolorboxenvironment{proof}{
	breakable,
	coltitle = black,
	colback = white,
	frame hidden,
	boxrule = 0pt,
	boxsep = 0pt,
	borderline west={3pt}{0pt}{bar},
	% borderline west={3pt}{0pt}{gPurple},
	sharp corners = all,
	enhanced,
}

\newtheorem{theorem}{Theorem}[section]{\bfseries}{}
\tcolorboxenvironment{theorem}{
	breakable,
	enhanced,
	boxrule = 0pt,
	frame hidden,
	coltitle = black,
	colback = blue!7,
	% colback = gBlue!30,
	left = 0.5em,
	sharp corners = all,
}

\newtheorem{corollary}{Corollary}[section]{\bfseries}{}
\tcolorboxenvironment{corollary}{
	breakable,
	enhanced,
	boxrule = 0pt,
	frame hidden,
	coltitle = black,
	colback = white!0,
	left = 0.5em,
	sharp corners = all,
}

\newtheorem{lemma}{Lemma}[section]{\bfseries}{}
\tcolorboxenvironment{lemma}{
	breakable,
	enhanced,
	boxrule = 0pt,
	frame hidden,
	coltitle = black,
	colback = green!7,
	left = 0.5em,
	sharp corners = all,
}

\newtheorem{definition}{Definition}[section]{\bfseries}{}
\tcolorboxenvironment{definition}{
	breakable,
	coltitle = black,
	colback = white,
	frame hidden,
	boxsep = 0pt,
	boxrule = 0pt,
	borderline west = {3pt}{0pt}{orange},
	% borderline west = {3pt}{0pt}{gOrange},
	sharp corners = all,
	enhanced,
}

\newtheorem{example}{Example}[section]{\bfseries}{}
\tcolorboxenvironment{example}{
	% title = \textbf{Example},
	% detach title,
	% before upper = {\tcbtitle\quad},
	breakable,
	coltitle = black,
	colback = white,
	frame hidden,
	boxrule = 0pt,
	boxsep = 0pt,
	borderline west={3pt}{0pt}{green!70!black},
	% borderline west={3pt}{0pt}{gGreen},
	sharp corners = all,
	enhanced,
}

\newtheoremstyle{remark}{0pt}{4pt}{}{}{\bfseries\itshape}{\normalfont.}{0.5em}{}
\theoremstyle{remark}
\newtheorem*{remark}{Remark}


% TColorBoxes
\newtcolorbox{week}{
	colback = black,
	coltext = white,
	fontupper = {\large\bfseries},
	width = 1.2\paperwidth,
	size = fbox,
	halign upper = center,
	center
}

\newcommand{\banner}[2]{
    \pagebreak
    \begin{week}
   		\section*{#1}
    \end{week}
    \addcontentsline{toc}{section}{#1}
    \addtocounter{section}{1}
    \setcounter{subsection}{0}
}

% Hyperref
\usepackage{hyperref}
\hypersetup{
	colorlinks=true,
	linktoc=all,
	linkcolor=links,
	bookmarksopen=true
}

% Error Handling
\PackageWarningNoLine{ExtSizes}{It is better to use one of the extsizes 
                          classes,^^J if you can}


\usepackage{fancyhdr}
\pagestyle{fancy}
\fancyhead[lcr]{}
\fancyhead[l]{Eli Griffiths}
\fancyhead[c]{MATH $121$B}
\fancyhead[r]{HW \#$1$}

\newcommand{\iprd}[1]{\left\langle #1 \right\rangle}

\begin{document}

\section*{Problem 1}
\begin{proof}
    Let $n$ denote the size of $C$. We proceed with induction on $n$. Consider the base case $n = 1$. Then
    \[
        \det(C - t I) = \mqty| -a_0 - t | = (-1)^n (t + a_0)
    \]
    hence the base case holds. Assume that for a matrix of size $m \geq 1$ the given equation for the characteristic polynomial is correct. Consider the $(m+1) \times (m+1)$ matrix
    \[
        C = \mqty(
        0 & 0 & \cdots & 0 & -a_0 \\
        1 & 0 & \cdots & 0 & -a_1 \\
        0 & 1 & \cdots & 0 & -a_2 \\
        \vdots & \vdots & & \vdots & \vdots \\
        0 & 0 & \cdots & 0 & -a_{m-1} \\
        0 & 0 & \cdots & 1 & -a_{m} \\
        )
    .\]
    Then finding the characteristic polynomial and expanding along the first row gives
    \begin{align*}
        \det(C - tI) &= \mqty|
        -t & 0 & \cdots & 0 & -a_0 \\
        1 & -t & \cdots & 0 & -a_1 \\
        0 & 1 & \cdots & 0 & -a_2 \\
        \vdots & \vdots & & \vdots & \vdots \\
        0 & 0 & \cdots & -t & -a_{m-1} \\
        0 & 0 & \cdots & 1 & -t - a_{m} \\
        | \\
        &= (-t)\mqty|
        -t & 0 & \cdots & -a_1 \\
        1 & -t & \cdots & -a_2 \\
        \vdots & \ddots & \ddots & \vdots \\
        0 & \cdots & 1 & -a_m - t
        | + (-1)^{m} (-a_0) \det \mathcal{I}_{-t} \\
        \intertext{By the induction hypothesis,}
          &= (-t)(-1)^{m} (t^m + a_{m} t^{m-1} + \ldots + a_2 t + a_1) + (-1)^{m+1} a_0 \det \mathcal{I}_{-t} \\
          &= (-1)^{m+1} (t^{m+1} + a_{m} t^{m} + \ldots + a_2 t^2 + a_1 t) + (-1)^{m+1} a_0 \det \mathcal{I}_{-t}
    \end{align*}
    where $\mathcal{I}_{-t}$ is the $m\times m$ identitiy matrix with the next upper diagonal as all $-t$. That is
    \[
        \mathcal{I}_{-t} = \mqty(
        1 & -t & 0 & \cdots & 0 \\
        0 & 1 & -t & \cdots & 0 \\
        0 & 0 & 1 & \ddots & \vdots \\
        \vdots & \vdots &  & \ddots  & -t \\
         0 & 0 & \cdots & 0 & 1 \\
        )
    .\]
    Since $\mathcal{I}_{-t}$ is an upper triangular matrix, its determinant is equal to the product of its diagonal which is simply $1$. Therefore
    \begin{align*}
        \det(C - tI) &= (-1)^{m+1} (t^{m+1} + a_{m} t^{m} + \ldots + a_2 t^2 + a_1 t) + (-1)^{m+1} a_0 \\
        &= (-1)^{m+1} (t^{m+1} + a_{m} t^m + \ldots + a_1 t + a_0)
    \end{align*}
    which was to be shown.
\end{proof}
\newpage

\section*{Problem 2}
It is true for all matrices $A$. 

\begin{proof}
    Assume that $a_0 \neq 0$. Note that
    \[
        p_{A}(0) = (-1)^n (0^n + a_{n-1} \cdot 0^{n-1} + \ldots + a_0) = (-1)^n a_0 \neq 0
    .\]
    By the definition of the characteristic polynomial, we know $p_A(t) = \det(A - t I)$. Therefore
    \[
        p_A(0) = \det(A - 0(I)) = \det A = (-1)^n a_0 \neq 0
    .\]
    Since the determinant of $A$ is non-zero, it must be invertible.
\end{proof}

\section*{Problem 3}
\subsection*{Part A}
\begin{proof}
    We will show $\iprd{\cdot,\cdot}_{F}$ satisfies the requirements of being an inner product. Let $A,B \in M_{n\times n}(\R)$ and $s \in \R$. Some useful tools are
    \begin{itemize}
        \item The trace and transpose are linear operators. 
        \item $(A^T A)_{ij} = \displaystyle\sum_{k=1}^n A_{ki} A_{kj} \implies \tr(A^T A) = \sum_{i=1}^n \sum_{j=1}^n A_{ij}^2$
    \end{itemize}
    We proceed with the 4 requirements of being an inner product.
    \begin{enumerate}
        \item 
        We want to show linearity of the inner product. Take $C \in M_{n\times n}(\R)$. Note that
        \begin{align*}
            \iprd{A + C, B}_F &= \tr((A+C)^T B) \\
            &= \tr((A^T + C^T) B) \\
            &= \tr(A^T B + C^T B) \\
            &= \tr(A^T B) + \tr(C^T B) = \iprd{A,B}_F + \iprd{C,B}_F
        \end{align*}
        as a consequence of the linearity of the trace and transpose. Therefore $\iprd{\cdot, \cdot}_F$ satisfies linearity.
        \item
            We want to show $\iprd{sA,B}_{F} = s \iprd{A,B}_F$. This is the case since
            \[
                \iprd{sA,B}_F = \tr((sA)^T B) = \tr(sA^T B) = s \tr(A^T B) = s \iprd{A,B}_F
            \]
            using the fact the trace and transpose are linear operators.
        \item
            We want to show $\iprd{A, B}_F = \iprd{B, A}_F$. That is,
            \[
                \tr(A^T B) = \tr(B^T A)
            .\]
            Note that the trace of a matrix is the same as the trace of its transpose as the entries on the diagonal do not change. Therefore
            \[
                \tr(A^T B) = \tr(\qty(A^T B)^T) = \tr(B^T A)
            \]
            which was to be shown.
        \item 
            Assume that $A \neq 0$. Then there must be some entry of $A$ that is non zero. Therefore
            \[
                \iprd{A, A}_F = \tr(A^T A) = \sum_{i = 0}^n \sum_{j=0}^n A_{ij}^2 > 0
            \]
            since some $A_{ij}$ is non zero meaning its square must be larger than $0$ and every other term is greater than or equal to $0$.
    \end{enumerate}
    $\iprd{\cdot, \cdot}_F$ satisifies the required conditions meaning it is an inner product.
\end{proof}

\subsection*{Part B}
\begin{proof}
    Let $A \in M_{n \times n}(\R)$ and assume that $A$ is diagonalizable. Then $A = P D P^{-1}$ where $P$ is unitary and $D$ is a diagonal matrix with its entries being the eigenvalues of $A$. Note that
    \begin{align*}
        A^T A &= \qty(P D P^{-1})^T \qty(P D P^{-1}) \\
              &= \qty(P^{-1})^T D^{T} P^T P D P^{-1}
              \intertext{Since $P$ is unitary, its transpose is its inverse giving}
              &= P D^T I D P^{-1} \\
              &= P D^2 P^{-1}.
    \end{align*}
    Therefore $A^T A$ is also diagonalizable. Note that $D^2$ will be a diagonal matrix as well with entries $\lambda_i^2$ where $\lambda_i$ are the original entries from $D$. Since $D^2$ is the diagonal matrix in the decomposition of $A^T A$, its entries $\lambda_i^2$ are the eigenvalues of $A^T A$. Since the trace of a matrix is equal to the sum of its eigenvalues, it follows
    \[
        \tr (A^T A) = \sum_{i=1}^n \lambda_i^2 \implies \norm{A}_{F} = \sqrt{\sum_{i=1}^n \lambda_i^2} \qedhere
    \]
\end{proof}

\end{document}
