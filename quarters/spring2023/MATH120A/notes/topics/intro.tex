\documentclass[../notes.tex]{subfiles}
\graphicspath{
    {"../figures"}
}

\begin{document}

\banner{Introduction}

\subsection{Understanding Relations and Sets}

\begin{definition}[Relations]
	A relation between two sets $A$ and $B$ is denoted by $\mathcal{R}$. $\mathcal{R}$ is a subset of $A \times B$ where $(a,b) \in \mathcal{R}$ is read as "$a$ is related to $b$".
\end{definition}

Analysts are initimately familiar with the concept of functions. A function is a relation (which for these purposes will be denoted by $\phi$) on a domain $X$ and codomain $Y$ such that
\[
	(x, y) \in \phi \Longleftrightarrow \phi(x) = y
.\]

Some often thought of functions are $x \mapsto x^2$ and $x \mapsto e^x$. Consider the following function:
\[
	+ : \mathbb{R} \times \mathbb{R} \rightarrow \mathbb{R} : (a,b) \mapsto a + b
.\]

While it may not seem like a function in the same vain as $f(x) = x^2$, it is just as valid. Hence operators on sets can be thought as a function and hence a relation on the set (in this instance on $\mathbb{R}$).

\begin{definition}[Equivalence Relation]
	A relation $\mathcal{R}$ is called an \textit{Equivalence Relation} if it satisfies the following three properties:
	\begin{align*}
		&\textbf{Reflexive} &\forall &a \in A, (a,a) \in \mathcal{R} \\
		&\textbf{Symmetric} &\forall &a,b \in A, (a,b) \in \mathcal{R} \implies (b,a) \in \mathcal{R} \\
		&\textbf{Transitive} &\forall &a,b,c \in A, (a,b), (b,c) \in \mathcal{R} \implies (a,c) \in \mathcal{R}
	.\end{align*}
\end{definition}


Consider the relation $\mathcal{R}$ where
\[
	x \mathcal{R} y \Longleftrightarrow |x| = |y|
.\]

Checking the three properties reveals that $\mathcal{R}$ is indeed an equivalence relation.
\begin{align*}
		&\textbf{Reflexivity} &|x| &= |x| \\
		&\textbf{Symmetry} &|x| &= |y| \implies |y|=|x| \\
		&\textbf{Transitivity} &|x| &= |y|, |y|=|z| \implies |x| = |z|
.\end{align*}

\begin{theorem}[Partitions from Equivalence Relations]
	Let $A$ be a non empty set, and $\sim$ be an equivalence relation on A. Then for each $a \in A$ it follows that $\bar{a} := \qty{x \in A : x \sim a} \subseteq A$. These subsets produce a partition of $A$.
\end{theorem}

\begin{proof}
	First show that if $\bar{a} \cap \bar{b} \neq \varnothing$, then $\bar{a} = \bar{b}$. Let $w \in \bar{a} \cap \bar{b}$. Consider an element $x \in \bar{a}$. Then $x \sim a$. Since $w \in \bar{a}$ it follows that $w \sim a$ hence $x \sim w$. Since $w \in \bar{b}$, its true that $w \sim b$, meaning $x \sim b$. Therefore $x \in \bar{b}$, meaning $\bar{a} \subseteq \bar{b}$. Other direction follows in the same manner.
\end{proof}

\begin{definition}[Powerset]
	The powerset of a set $A$, denoted by $\mathcal{P}(A)$, is the set containing all subsets of $A$. Equivalently,
	\[
		\mathcal{P}(A) = \qty{ S : S \subseteq A }
	.\]
\end{definition}

\begin{theorem}
	Given a finite, non-empty set $A$, it follows that
	$
		\mathcal{P}(A) = 2^{\qty|A|}
	$.
\end{theorem}

\begin{proof}
	Let $A$ be a finite, non-empty set. Define a binary string structure in the following manner. Given a binary string such as 
	\begin{align*}
		\qty(0,0,0,\ldots, 0) &\implies \varnothing \\
		\qty(1,0,0,\ldots, 0) &\implies \qty{a_1} \\
		\qty(1,1,0,\ldots, 0) &\implies \qty{a_1, a_2} \\
													&\vdots
	\end{align*}
	where $a_1, a_2, \ldots a_n$ denote all the elements in $A$. Therefore, if a set $A$ has $n$ elements, the cardinality of $\mathcal{P}(A)$ is equivalent to the question "How many binary strings are there with length $n$". Each entry of the string provides 2 choices, $\qty{0,1}$. Therefore since there are $n$ choices in the entirety of the string, the number of binary strings of length $n$ is equal to $2^n$. Therefore $\mathcal{P}(A) = 2^n = 2^{|A|}$.
\end{proof}

\end{document}
