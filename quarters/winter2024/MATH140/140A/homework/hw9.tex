\documentclass[12pt,titlepage]{extarticle}
% Document Layout and Font
\usepackage{subfiles}
\usepackage[margin=2cm, headheight=15pt]{geometry}
\usepackage{fancyhdr}
\usepackage{enumitem}	
\usepackage{wrapfig}
\usepackage{float}
\usepackage{multicol}

\usepackage[p,osf]{scholax}

\renewcommand*\contentsname{Table of Contents}
\renewcommand{\headrulewidth}{0pt}
\pagestyle{fancy}
\fancyhf{}
\fancyfoot[R]{$\thepage$}
\setlength{\parindent}{0cm}
\setlength{\headheight}{17pt}
\hfuzz=9pt

% Figures
\usepackage{svg}

% Utility Management
\usepackage{color}
\usepackage{colortbl}
\usepackage{xcolor}
\usepackage{xpatch}
\usepackage{xparse}

\definecolor{gBlue}{HTML}{7daea3}
\definecolor{gOrange}{HTML}{e78a4e}
\definecolor{gGreen}{HTML}{a9b665}
\definecolor{gPurple}{HTML}{d3869b}

\definecolor{links}{HTML}{1c73a5}
\definecolor{bar}{HTML}{584AA8}

% Math Packages
\usepackage{mathtools, amsmath, amsthm, thmtools, amssymb, physics}
\usepackage[scaled=1.075,ncf,vvarbb]{newtxmath}

\newcommand\B{\mathbb{C}}
\newcommand\C{\mathbb{C}}
\newcommand\R{\mathbb{R}}
\newcommand\Q{\mathbb{Q}}
\newcommand\N{\mathbb{N}}
\newcommand\Z{\mathbb{Z}}

\DeclareMathOperator{\lcm}{lcm}

% Probability Theory
\newcommand\Prob[1]{\mathbb{P}\qty(#1)}
\newcommand\Var[1]{\text{Var}\qty(#1)}
\newcommand\Exp[1]{\mathbb{E}\qty[#1]}

% Analysis
\newcommand\ball[1]{\B\qty(#1)}
\newcommand\conj[1]{\overline{#1}}
\DeclareMathOperator{\Arg}{Arg}
\DeclareMathOperator{\cis}{cis}

% Linear Algebra
\DeclareMathOperator{\dom}{dom}
\DeclareMathOperator{\range}{range}
\DeclareMathOperator{\spann}{span}
\DeclareMathOperator{\nullity}{nullity}

% TIKZ
\usepackage{tikz}
\usepackage{pgfplots}
\usetikzlibrary{arrows.meta}
\usetikzlibrary{math}
\usetikzlibrary{cd}

% Boxes and Theorems
\usepackage[most]{tcolorbox}
\tcbuselibrary{skins}
\tcbuselibrary{breakable}
\tcbuselibrary{theorems}

\newtheoremstyle{default}{0pt}{0pt}{}{}{\bfseries}{\normalfont.}{0.5em}{}
\theoremstyle{default}

\renewcommand*{\proofname}{\textit{\textbf{Proof.}}}
\renewcommand*{\qedsymbol}{$\blacksquare$}
\tcolorboxenvironment{proof}{
	breakable,
	coltitle = black,
	colback = white,
	frame hidden,
	boxrule = 0pt,
	boxsep = 0pt,
	borderline west={3pt}{0pt}{bar},
	% borderline west={3pt}{0pt}{gPurple},
	sharp corners = all,
	enhanced,
}

\newtheorem{theorem}{Theorem}[section]{\bfseries}{}
\tcolorboxenvironment{theorem}{
	breakable,
	enhanced,
	boxrule = 0pt,
	frame hidden,
	coltitle = black,
	colback = blue!7,
	% colback = gBlue!30,
	left = 0.5em,
	sharp corners = all,
}

\newtheorem{corollary}{Corollary}[section]{\bfseries}{}
\tcolorboxenvironment{corollary}{
	breakable,
	enhanced,
	boxrule = 0pt,
	frame hidden,
	coltitle = black,
	colback = white!0,
	left = 0.5em,
	sharp corners = all,
}

\newtheorem{lemma}{Lemma}[section]{\bfseries}{}
\tcolorboxenvironment{lemma}{
	breakable,
	enhanced,
	boxrule = 0pt,
	frame hidden,
	coltitle = black,
	colback = green!7,
	left = 0.5em,
	sharp corners = all,
}

\newtheorem{definition}{Definition}[section]{\bfseries}{}
\tcolorboxenvironment{definition}{
	breakable,
	coltitle = black,
	colback = white,
	frame hidden,
	boxsep = 0pt,
	boxrule = 0pt,
	borderline west = {3pt}{0pt}{orange},
	% borderline west = {3pt}{0pt}{gOrange},
	sharp corners = all,
	enhanced,
}

\newtheorem{example}{Example}[section]{\bfseries}{}
\tcolorboxenvironment{example}{
	% title = \textbf{Example},
	% detach title,
	% before upper = {\tcbtitle\quad},
	breakable,
	coltitle = black,
	colback = white,
	frame hidden,
	boxrule = 0pt,
	boxsep = 0pt,
	borderline west={3pt}{0pt}{green!70!black},
	% borderline west={3pt}{0pt}{gGreen},
	sharp corners = all,
	enhanced,
}

\newtheoremstyle{remark}{0pt}{4pt}{}{}{\bfseries\itshape}{\normalfont.}{0.5em}{}
\theoremstyle{remark}
\newtheorem*{remark}{Remark}


% TColorBoxes
\newtcolorbox{week}{
	colback = black,
	coltext = white,
	fontupper = {\large\bfseries},
	width = 1.2\paperwidth,
	size = fbox,
	halign upper = center,
	center
}

\newcommand{\banner}[2]{
    \pagebreak
    \begin{week}
   		\section*{#1}
    \end{week}
    \addcontentsline{toc}{section}{#1}
    \addtocounter{section}{1}
    \setcounter{subsection}{0}
}

% Hyperref
\usepackage{hyperref}
\hypersetup{
	colorlinks=true,
	linktoc=all,
	linkcolor=links,
	bookmarksopen=true
}

% Error Handling
\PackageWarningNoLine{ExtSizes}{It is better to use one of the extsizes 
                          classes,^^J if you can}


\def\homeworknumber{9}
\fancyhead[R]{\textbf{Math 140A: Homework \#\homeworknumber}}
\fancyhead[L]{Eli Griffiths}
\renewcommand{\headrulewidth}{1pt}
\setlength\parindent{0pt}


% 17.9, 17.14
% 18.1, 18.2, 18.4, 18.6, 18.9

\begin{document}

\subsection*{17.9}
\subsubsection*{Part A}
\begin{proof}
    Let $f : \mathbb{R} \to \mathbb{R} : x \mapsto x^2$ and $x_0 = 2$. We want to show that for all $\epsilon > 0$ that $\exists \delta > 0$ such that $|x - x_0| < \delta \implies |f(x) - f(x_0)| < \epsilon$. Note that
    \[
        |f(x) - f(x_0)| = |x^2 - 4| = |x+2||x-2|
    .\]
    If we let $\delta < 1$, then $|x - x_0| < \delta < 1$ meaning $x_0 - 1 < x < x_0 + 1 \implies |x+2| < x_0 + 3$. Therefore
    \[
        |x+2||x-2| < |x-2| \cdot (x_0 + 3) < \epsilon \implies |x-2| < \frac{\epsilon}{x_0+3}
    .\]
    So then formally, take $\epsilon > 0$ and let $\delta = \min\qty{1, \frac{\epsilon}{x_0 + 3}}$. Then
    \[
        |x-x_0| = |x-2| < \delta \implies |f(x) - f(x_0)| = |x^2 - 4| < \epsilon
    .\]
    Hence $f$ is continuous at $x_0 = 2$.
\end{proof}

\subsubsection*{Part B}
\begin{proof}
    Note that
    \[
        |f(x) - f(0)| = |\sqrt{x}| = \sqrt{x} < \epsilon \implies x < \epsilon^2 \implies |x| < \epsilon^2
    .\]
    Therefore take $\epsilon > 0$ and let $\delta = \epsilon^2$. Then
    \[
        |x-0| = |x| < \delta \implies |x| < \epsilon^2 \implies x < \epsilon^2 \implies |\sqrt{x}| < \epsilon
    .\]
    Hence $f$ is continuous at $x_0 = 0$.
\end{proof}

\subsubsection*{Part C}
\begin{proof}
    Take $\epsilon > 0$ and let $\delta = \epsilon$. Note then that
    \[
        |x| < \delta \implies |x| < \epsilon
    .\]
    Since $|\sin x| \leq 1$ for all $x \in \mathbb{R}$, $|\sin \frac{1}{x}| \leq 1$ for $x \neq 0$ and therefore
    \[
        |x||\sin \frac{1}{x}| \leq |x| < \epsilon
    .\]
    Therefore
    \[
        |x \sin \frac{1}{x}| < \epsilon
    .\]
    Hence $f$ is continuous at $x = 0$.
\end{proof}

\subsubsection*{Part D}
\begin{proof}
    Take $\epsilon > 0$. Let $\delta = \min\qty{1, \frac{\varepsilon}{3|x_0|^2 + 3|x_0| + 1}}$. Assume that $|x-x_0| < \delta$. Then
    \begin{align*}
        |x-x_0| < \delta \implies &|x-x_0| < \frac{\varepsilon}{3|x_0|^2 + 3|x_0| + 1} \\
                                &|x-x_0| (3|x_0|^2 + 3|x_0| + 1) < \varepsilon \\
                                &|x-x_0| ((1 + |x_0|)^2 + |x_0|(1+|x_0|) + |x_0|^2) < \epsilon
    \intertext{Since $|x-x_0| < 1$, $|x| = |x - x_0 + x_0| \leq 1 + |x_0|$. Therefore}
                                &|x-x_0| (|x|^2 + |x_0 x| + |x_0^2|) < \varepsilon \\
                                &|x^3 - x_0^3| < \epsilon
    \end{align*}
    Therefore $|x-x_0| < \delta \implies |x^3 - x_0^3| < \epsilon$, meaning $f(x) = x^3$ is continuous over all of $\mathbb{R}$.
\end{proof}

\subsection*{17.14}
\begin{proof}
    First show that $f$ is discontinuous at any point in $\mathbb{Q}$. Let $x \in \mathbb{Q}$ with $x = \frac{p}{q}$. Let $\alpha \in \mathbb{R} \setminus \mathbb{Q}$. Construct the sequence 
    \[
(x_n)_{n\in \mathbb{N}}, x_n = x + \frac{\alpha}{n}, \forall n
    .\] 
    Note that $\lim x_n = x$ and $x_n \notin \mathbb{Q}$ for all $n$. Assume towards contradiction that $f$ is continuous at any point in $\mathbb{Q}$. Then since $x \in \mathbb{Q}$ and $(x_n)$ is a sequence that converges to $x$, $\lim f(x_n) = f(x) = \frac{1}{q} > 0$. However, since $x_n \notin \mathbb{Q}$ for all $n$, $f(x_n) = 0$ for all $n$. However, this means that $\lim f(x_n) = 0$ which contradicts the assumption. Therefore $f$ is not continuous at points in $\mathbb{Q}$. 
    Let $x_0 \in \mathbb{R}\setminus \mathbb{Q}$. Then for every $q \in \mathbb{N}$, there exists some $\delta_q$ such that $|x_0 - y| \geq \delta_q$ for all $y \in \qty{\frac{p}{q} : p \in \mathbb{Z}}$. Take $\epsilon > 0$ and choose $N \in \mathbb{N}$ such that $\frac{1}{N} < \epsilon$. Let $\delta = \min\qty{\delta_1, \ldots, \delta_N}$. Then
    \[
        |x - x_0| < \delta \implies x = \frac{p}{q'}, q' > N \implies |f(x) - f(x_0)| = |f(x)| < \frac{1}{N} < \epsilon
    .\]
    Therefore $f$ is continuous on all points in $\mathbb{R}\setminus \mathbb{Q}$.
\end{proof}

\subsection*{18.1}
\begin{proof}
    Let $f$ be the function outlined. Assume that $-f$ assumes its maximum at $x_0 \in [a,b]$. That is,
    \[
        -f(x) \leq -f(x_0), \forall x\in [a,b] \implies f(x) \geq f(x_0), \forall x \in [a,b]
    .\]
    Therefore $x_0$ is the minimum of $f$.
\end{proof}

\subsection*{18.2}
The proof breaks down when the interval is open because the sequence $(x_n)$ that is constructed can converge to some $x_0$ that is the boundary point of the interval, which if open are not a part of the functions domain of continuity.

\subsection*{18.4}
\begin{proof}
    Let $S \subset \mathbb{R}$ and assume that there is a sequence $(x_n)$ in $S$ that converges to a number $x_0$ not in $S$. Let $f : S \to \mathbb{R} : x \mapsto \frac{1}{x - x_0}$. Note that $f$ is continuous on $S$ since $x_0 \notin S$ and is unbounded.
\end{proof}

\subsection*{18.6}
\begin{proof}
    Let $g : [0, \frac{\pi}{2}] \to \mathbb{R} : x\mapsto x - \cos(x)$. Note that
    \[
        g(0) = -1 < 0, g(\frac{\pi}{2}) = \frac{\pi}{2} > 0
    .\]
    Therefore by the IVT, there exists some $x_0 \in (0, \frac{\pi}{2})$ such that $g(x_0) = 0$ and therefore $x_0 - \cos x_0 = 0 \implies x_0 = \cos x_0$.
\end{proof}

\subsection*{18.9}
\begin{proof}
    Let $f$ be a polynomial of odd degree $n$. WLOG let the sign of the highest order odd term be positive. Note then that $\lim_{x\to \infty} f(x) = +\infty$ and $\lim{x\to -\infty} f(x) = -\infty$ since the highest order term is odd. This means that there is some $x_1, x_2 \in \mathbb{R}$ such that $f(x_1) < 0$ and $f(x_2) > 0$. By the intermediate value theorem, there must be some $x \in (x_1, x_2)$ such that $f(x) = 0$. Therefore $f$ has at least a single real root.
\end{proof}

\end{document}
