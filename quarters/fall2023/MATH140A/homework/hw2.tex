\documentclass[12pt,titlepage]{extarticle}
% Document Layout and Font
\usepackage{subfiles}
\usepackage[margin=2cm, headheight=15pt]{geometry}
\usepackage{fancyhdr}
\usepackage{enumitem}	
\usepackage{wrapfig}
\usepackage{float}
\usepackage{multicol}

\usepackage[p,osf]{scholax}

\renewcommand*\contentsname{Table of Contents}
\renewcommand{\headrulewidth}{0pt}
\pagestyle{fancy}
\fancyhf{}
\fancyfoot[R]{$\thepage$}
\setlength{\parindent}{0cm}
\setlength{\headheight}{17pt}
\hfuzz=9pt

% Figures
\usepackage{svg}

% Utility Management
\usepackage{color}
\usepackage{colortbl}
\usepackage{xcolor}
\usepackage{xpatch}
\usepackage{xparse}

\definecolor{gBlue}{HTML}{7daea3}
\definecolor{gOrange}{HTML}{e78a4e}
\definecolor{gGreen}{HTML}{a9b665}
\definecolor{gPurple}{HTML}{d3869b}

\definecolor{links}{HTML}{1c73a5}
\definecolor{bar}{HTML}{584AA8}

% Math Packages
\usepackage{mathtools, amsmath, amsthm, thmtools, amssymb, physics}
\usepackage[scaled=1.075,ncf,vvarbb]{newtxmath}

\newcommand\B{\mathbb{C}}
\newcommand\C{\mathbb{C}}
\newcommand\R{\mathbb{R}}
\newcommand\Q{\mathbb{Q}}
\newcommand\N{\mathbb{N}}
\newcommand\Z{\mathbb{Z}}

\DeclareMathOperator{\lcm}{lcm}

% Probability Theory
\newcommand\Prob[1]{\mathbb{P}\qty(#1)}
\newcommand\Var[1]{\text{Var}\qty(#1)}
\newcommand\Exp[1]{\mathbb{E}\qty[#1]}

% Analysis
\newcommand\ball[1]{\B\qty(#1)}
\newcommand\conj[1]{\overline{#1}}
\DeclareMathOperator{\Arg}{Arg}
\DeclareMathOperator{\cis}{cis}

% Linear Algebra
\DeclareMathOperator{\dom}{dom}
\DeclareMathOperator{\range}{range}
\DeclareMathOperator{\spann}{span}
\DeclareMathOperator{\nullity}{nullity}

% TIKZ
\usepackage{tikz}
\usepackage{pgfplots}
\usetikzlibrary{arrows.meta}
\usetikzlibrary{math}
\usetikzlibrary{cd}

% Boxes and Theorems
\usepackage[most]{tcolorbox}
\tcbuselibrary{skins}
\tcbuselibrary{breakable}
\tcbuselibrary{theorems}

\newtheoremstyle{default}{0pt}{0pt}{}{}{\bfseries}{\normalfont.}{0.5em}{}
\theoremstyle{default}

\renewcommand*{\proofname}{\textit{\textbf{Proof.}}}
\renewcommand*{\qedsymbol}{$\blacksquare$}
\tcolorboxenvironment{proof}{
	breakable,
	coltitle = black,
	colback = white,
	frame hidden,
	boxrule = 0pt,
	boxsep = 0pt,
	borderline west={3pt}{0pt}{bar},
	% borderline west={3pt}{0pt}{gPurple},
	sharp corners = all,
	enhanced,
}

\newtheorem{theorem}{Theorem}[section]{\bfseries}{}
\tcolorboxenvironment{theorem}{
	breakable,
	enhanced,
	boxrule = 0pt,
	frame hidden,
	coltitle = black,
	colback = blue!7,
	% colback = gBlue!30,
	left = 0.5em,
	sharp corners = all,
}

\newtheorem{corollary}{Corollary}[section]{\bfseries}{}
\tcolorboxenvironment{corollary}{
	breakable,
	enhanced,
	boxrule = 0pt,
	frame hidden,
	coltitle = black,
	colback = white!0,
	left = 0.5em,
	sharp corners = all,
}

\newtheorem{lemma}{Lemma}[section]{\bfseries}{}
\tcolorboxenvironment{lemma}{
	breakable,
	enhanced,
	boxrule = 0pt,
	frame hidden,
	coltitle = black,
	colback = green!7,
	left = 0.5em,
	sharp corners = all,
}

\newtheorem{definition}{Definition}[section]{\bfseries}{}
\tcolorboxenvironment{definition}{
	breakable,
	coltitle = black,
	colback = white,
	frame hidden,
	boxsep = 0pt,
	boxrule = 0pt,
	borderline west = {3pt}{0pt}{orange},
	% borderline west = {3pt}{0pt}{gOrange},
	sharp corners = all,
	enhanced,
}

\newtheorem{example}{Example}[section]{\bfseries}{}
\tcolorboxenvironment{example}{
	% title = \textbf{Example},
	% detach title,
	% before upper = {\tcbtitle\quad},
	breakable,
	coltitle = black,
	colback = white,
	frame hidden,
	boxrule = 0pt,
	boxsep = 0pt,
	borderline west={3pt}{0pt}{green!70!black},
	% borderline west={3pt}{0pt}{gGreen},
	sharp corners = all,
	enhanced,
}

\newtheoremstyle{remark}{0pt}{4pt}{}{}{\bfseries\itshape}{\normalfont.}{0.5em}{}
\theoremstyle{remark}
\newtheorem*{remark}{Remark}


% TColorBoxes
\newtcolorbox{week}{
	colback = black,
	coltext = white,
	fontupper = {\large\bfseries},
	width = 1.2\paperwidth,
	size = fbox,
	halign upper = center,
	center
}

\newcommand{\banner}[2]{
    \pagebreak
    \begin{week}
   		\section*{#1}
    \end{week}
    \addcontentsline{toc}{section}{#1}
    \addtocounter{section}{1}
    \setcounter{subsection}{0}
}

% Hyperref
\usepackage{hyperref}
\hypersetup{
	colorlinks=true,
	linktoc=all,
	linkcolor=links,
	bookmarksopen=true
}

% Error Handling
\PackageWarningNoLine{ExtSizes}{It is better to use one of the extsizes 
                          classes,^^J if you can}


\def\homeworknumber{2}
\fancyhead[R]{\textbf{Math 140A: Homework \#\homeworknumber}}
\fancyhead[L]{Eli Griffiths}
\renewcommand{\headrulewidth}{1pt}
\setlength\parindent{0pt}


% 3.5, 4.1 
% 4.2 for a,b,e,f,h,i,k,n,t,u (find lower/upper bounds and inf/sup, if they exist), 
% 4.5, 4.7, 4.10-4.12
% 5.1, 5.2, 5.5,

\begin{document}

\subsection*{3.5}
\subsubsection*{Part A}
\begin{proof}
	Let $a,b \in \mathbb{R}$. Consider both directions.
	\begin{enumerate}
		\item[$\Rightarrow)$]
			Assume that $|b| \leq a$. If $b \geq 0$, then $|b| = b \leq a$. If $b < 0$, then $|b| = -b \leq a$ or equivalently $b \geq -a$. Therefore for any $b$, $-a \leq b \leq a$.
		\item[$\Leftarrow)$]
			Assume that $-a \leq b \leq a$. Note this implies that $b \leq a$ and that $-b \leq a$ by using both sides of the inequality. Since both $-b$ and $b$ are less than $a$, $|b| \leq a$.
	\end{enumerate}
\end{proof}

\subsubsection*{Part B}
\begin{proof}
	Let $a,b \in \mathbb{R}$. Note that $|a| = |(a-b) + b| \leq |a-b| + |b|$ by the triangle inequality, giving $|a| - |b| \leq |a-b|$. Equivalently, $|b| - |a| \leq |b - a|$, meaning $|a| - |b| \geq -|a-b|$. Since $-|a-b| \leq |a| - |b| \leq |a-b|$, by 3.5,
	\[
		||a| - |b|| \leq |a-b|
	\]
\end{proof}

\subsection*{4.1 / 4.2}

\begin{center}
	\def\arraystretch{1.5}
	\begin{tabular}{r|c|c|c|c|c|c|c|c|c|c}
			   & A  & B  & E  & F  & H   & I  & K   & N           & T   & U   \\\hline
		LB     & -1 & -1 & -1 & -1 & -1  & -2 & 0   & -2          & DNE & -3  \\\hline
		UB     & 5  & 4  & 2  & 1  & DNE & 3  & DNE & 2           & 10  & DNE \\\hline
		$\inf$ & 0  & 0  & 0  & 0  & 2   & 0  & 0   & $-\sqrt{2}$ & DNE & 0   \\\hline
		$\sup$ & 1  & 1  & 1  & 0  & DNE & 1  & DNE & $\sqrt{2}$  & 2   & DNE
	\end{tabular}
\end{center}

\subsection*{4.5}
\begin{proof}
	Let $S \subset \mathbb{R}$ be non-empty that is bounded above. Assume that $\sup S = M \in S$. Since $M$ is the supremum, $s \leq M$ for all $s \in S$. Since $s \leq M$ for all $s \in S$ and $M \in S$, $M$ is by definition the max of $S$, hence $\sup S = \max S$.
\end{proof}

\subsection*{4.7}
\subsubsection*{Part A}
\begin{proof}
	Let $S, T \subset \mathbb{R}$ be non empty and bounded. Assume that $S \subset T$. Let $x \in S$. Note that $\inf T \leq s$ for all $s \in S$ since every member of $S$ is a member of $T$. This means that $\inf T$ is a lower bound of $S$ implying $\inf T \leq \inf S$. Equivalently, $\sup T \geq s$ for all $s \in S$, meaning $\sup T$ is an upper bound of $S$ and therefore $\sup T \geq \sup S$. Since $\inf S \leq \sup S$, these inequalities combine to
	\[
		\inf T \leq \inf S \leq \sup S \leq \sup T
	\]
\end{proof}

\subsubsection*{Part B}
\begin{proof}
	Let $S, T \subset \mathbb{R}$ be non empty and bounded. Note that $S, T \subset S \cup T$. Therefore from (A) it follows that $\sup S, \sup T \leq \sup(S \cup T)$ meaning $\max\qty{\sup S, \sup T} \leq \sup(S\cup T)$. Let $s \in S$. Then $s \leq \sup S \leq \max\qty{\sup S, \sup T}$. Let $t \in T$. Then $t \leq \sup T \leq \max\qty{\sup S, \sup T}$. Therefore for an element $x \in S \cup T$, $x \leq \max\qty{\sup S, \sup T}$. Therefore $\max\qty{\sup S, \sup T}$ admits an upper bound on $S \cup T$ meaning $\sup(S \cup T) \leq \max\qty{\sup S, \sup T}$. Since $\sup(S \cup T) \leq \max\qty{\sup S, \sup T}$ and $\max\qty{\sup S, \sup T} \leq \sup(S \cup T)$, $\sup(S \cup T) = \max\qty{\sup S, \sup T}$.
\end{proof}

\subsection*{4.10}
\begin{proof}
	Let $a > 0$. By applying the archimedean property twice, $\exists N_1, N_2$ such that $a > \frac{1}{N_1}$ and $a < N_2$. Let $N = \max\qty{N_1, N_2}$. Note that $\frac{1}{N_1} \leq \frac{1}{N}$ and $N_2 \leq N$, therefore $\frac{1}{N} < a < N$.
\end{proof}

\subsection*{4.11}
Let $a,b \in \mathbb{R}$ with $a < b$. By the denseness of $\mathbb{Q}$, there is a rational $r_1$ such that $a < r_1 < b$. Since $r_1 \in \mathbb{R}$, the denseness of $\mathbb{Q}$ can be applied to the range $a < r_1$ to give an $r_2$ such that $a < r_2 < r_1$. This leads to an infinite sequence of $r_n$'s that are between $a$ and $b$. Hence there are infinitely many rationals between $a$ and $b$.

\subsection*{4.12}
\begin{proof}
	Let $a,b \in \mathbb{R}$ such that $a < b$. Consider $r + \sqrt{2}$ where $r = \frac{p}{q} \in \mathbb{Q}$. Assume towards contradiction that $r + \sqrt{2}$ is rational. That is there is an $m \in \mathbb{Z}$ and $n \in \mathbb{N}$ such that
	\[
		\frac{p}{q} + \sqrt{2} = \frac{m}{n}
	\]
	However, this implies that $\sqrt{2} = \frac{mq - np}{nq}$ which is rational. Therefore $r + \sqrt{2}$ is irrational for all $r \in \mathbb{Q}$. Since $a < b$, $a - \sqrt{2} < b - \sqrt{2}$. By the denseness of $\mathbb{Q}$, there exists $h \in \mathbb{Q}$ such that $a - \sqrt{2} < h < b - \sqrt{2}$, or equivalently $a < h + \sqrt{2} < b$. Note that $h + \sqrt{2}$ is irrational and hence there is an irrational between $a$ and $b$.
\end{proof}

\subsection*{5.1}
\begin{align*}
	\qty{x \in \mathbb{R} : x < 0} &= (-\infty, 0) \\
	\qty{x \in \mathbb{R} : x^3 \leq 8} &= (-\infty, 2] \\
	\qty{x^2 : x \in \mathbb{R}} &= [0, \infty) \\
	\qty{x \in \mathbb{R} : x^2 < 8} &= (-\sqrt{8}, \sqrt{8})
\end{align*}

\subsection*{5.2}
\begin{align*}
	& \inf\qty{x \in \mathbb{R} : x < 0} = -\infty & \sup\qty{x \in \mathbb{R} : x < 0} = 0 \\
	& \inf\qty{x \in \mathbb{R} : x^3 \leq 8} = -\infty & \sup\qty{x \in \mathbb{R} : x^3 \leq 8} = 2 \\
	& \inf\qty{x^2 : x \in \mathbb{R}} = 0 & \sup\qty{x^2 : x \in \mathbb{R}} = \infty \\
	& \inf\qty{x \in \mathbb{R} : x^2 < 8} = -\sqrt{8} & \sup\qty{x \in \mathbb{R} : x^2 < 8} = \sqrt{8} &
\end{align*}

\subsection*{5.5}
\begin{proof}
	Let $S \subset \mathbb{R}$ be non-empty. If $S$ is unbounded, $\inf S = -\infty$ and $\sup S = \infty$ meaning that $\inf S \leq \sup S$. If $S$ is bounded below but unbounded above, $\inf S = m \leq \infty = \sup S$ meaning $\inf S \leq \sup S$. Alternatively, if $S$ is bounded above but unbouneded below, $\sup S = M \geq -\infty = \inf S$ meaning $\inf S \leq \sup S$. If $S$ is bounded, $\inf S = m$ and $\sup S = M$. For all $s \in S$, $m \leq s \leq M$, meaning $m \leq M$ and therefore $\inf S \leq \inf S$.
\end{proof}

\end{document}
