\documentclass[12pt,titlepage]{extarticle}
% Document Layout and Font
\usepackage{subfiles}
\usepackage[margin=2cm, headheight=15pt]{geometry}
\usepackage{fancyhdr}
\usepackage{enumitem}	
\usepackage{wrapfig}
\usepackage{float}
\usepackage{multicol}

\usepackage[p,osf]{scholax}

\renewcommand*\contentsname{Table of Contents}
\renewcommand{\headrulewidth}{0pt}
\pagestyle{fancy}
\fancyhf{}
\fancyfoot[R]{$\thepage$}
\setlength{\parindent}{0cm}
\setlength{\headheight}{17pt}
\hfuzz=9pt

% Figures
\usepackage{svg}

% Utility Management
\usepackage{color}
\usepackage{colortbl}
\usepackage{xcolor}
\usepackage{xpatch}
\usepackage{xparse}

\definecolor{gBlue}{HTML}{7daea3}
\definecolor{gOrange}{HTML}{e78a4e}
\definecolor{gGreen}{HTML}{a9b665}
\definecolor{gPurple}{HTML}{d3869b}

\definecolor{links}{HTML}{1c73a5}
\definecolor{bar}{HTML}{584AA8}

% Math Packages
\usepackage{mathtools, amsmath, amsthm, thmtools, amssymb, physics}
\usepackage[scaled=1.075,ncf,vvarbb]{newtxmath}

\newcommand\B{\mathbb{C}}
\newcommand\C{\mathbb{C}}
\newcommand\R{\mathbb{R}}
\newcommand\Q{\mathbb{Q}}
\newcommand\N{\mathbb{N}}
\newcommand\Z{\mathbb{Z}}

\DeclareMathOperator{\lcm}{lcm}

% Probability Theory
\newcommand\Prob[1]{\mathbb{P}\qty(#1)}
\newcommand\Var[1]{\text{Var}\qty(#1)}
\newcommand\Exp[1]{\mathbb{E}\qty[#1]}

% Analysis
\newcommand\ball[1]{\B\qty(#1)}
\newcommand\conj[1]{\overline{#1}}
\DeclareMathOperator{\Arg}{Arg}
\DeclareMathOperator{\cis}{cis}

% Linear Algebra
\DeclareMathOperator{\dom}{dom}
\DeclareMathOperator{\range}{range}
\DeclareMathOperator{\spann}{span}
\DeclareMathOperator{\nullity}{nullity}

% TIKZ
\usepackage{tikz}
\usepackage{pgfplots}
\usetikzlibrary{arrows.meta}
\usetikzlibrary{math}
\usetikzlibrary{cd}

% Boxes and Theorems
\usepackage[most]{tcolorbox}
\tcbuselibrary{skins}
\tcbuselibrary{breakable}
\tcbuselibrary{theorems}

\newtheoremstyle{default}{0pt}{0pt}{}{}{\bfseries}{\normalfont.}{0.5em}{}
\theoremstyle{default}

\renewcommand*{\proofname}{\textit{\textbf{Proof.}}}
\renewcommand*{\qedsymbol}{$\blacksquare$}
\tcolorboxenvironment{proof}{
	breakable,
	coltitle = black,
	colback = white,
	frame hidden,
	boxrule = 0pt,
	boxsep = 0pt,
	borderline west={3pt}{0pt}{bar},
	% borderline west={3pt}{0pt}{gPurple},
	sharp corners = all,
	enhanced,
}

\newtheorem{theorem}{Theorem}[section]{\bfseries}{}
\tcolorboxenvironment{theorem}{
	breakable,
	enhanced,
	boxrule = 0pt,
	frame hidden,
	coltitle = black,
	colback = blue!7,
	% colback = gBlue!30,
	left = 0.5em,
	sharp corners = all,
}

\newtheorem{corollary}{Corollary}[section]{\bfseries}{}
\tcolorboxenvironment{corollary}{
	breakable,
	enhanced,
	boxrule = 0pt,
	frame hidden,
	coltitle = black,
	colback = white!0,
	left = 0.5em,
	sharp corners = all,
}

\newtheorem{lemma}{Lemma}[section]{\bfseries}{}
\tcolorboxenvironment{lemma}{
	breakable,
	enhanced,
	boxrule = 0pt,
	frame hidden,
	coltitle = black,
	colback = green!7,
	left = 0.5em,
	sharp corners = all,
}

\newtheorem{definition}{Definition}[section]{\bfseries}{}
\tcolorboxenvironment{definition}{
	breakable,
	coltitle = black,
	colback = white,
	frame hidden,
	boxsep = 0pt,
	boxrule = 0pt,
	borderline west = {3pt}{0pt}{orange},
	% borderline west = {3pt}{0pt}{gOrange},
	sharp corners = all,
	enhanced,
}

\newtheorem{example}{Example}[section]{\bfseries}{}
\tcolorboxenvironment{example}{
	% title = \textbf{Example},
	% detach title,
	% before upper = {\tcbtitle\quad},
	breakable,
	coltitle = black,
	colback = white,
	frame hidden,
	boxrule = 0pt,
	boxsep = 0pt,
	borderline west={3pt}{0pt}{green!70!black},
	% borderline west={3pt}{0pt}{gGreen},
	sharp corners = all,
	enhanced,
}

\newtheoremstyle{remark}{0pt}{4pt}{}{}{\bfseries\itshape}{\normalfont.}{0.5em}{}
\theoremstyle{remark}
\newtheorem*{remark}{Remark}


% TColorBoxes
\newtcolorbox{week}{
	colback = black,
	coltext = white,
	fontupper = {\large\bfseries},
	width = 1.2\paperwidth,
	size = fbox,
	halign upper = center,
	center
}

\newcommand{\banner}[2]{
    \pagebreak
    \begin{week}
   		\section*{#1}
    \end{week}
    \addcontentsline{toc}{section}{#1}
    \addtocounter{section}{1}
    \setcounter{subsection}{0}
}

% Hyperref
\usepackage{hyperref}
\hypersetup{
	colorlinks=true,
	linktoc=all,
	linkcolor=links,
	bookmarksopen=true
}

% Error Handling
\PackageWarningNoLine{ExtSizes}{It is better to use one of the extsizes 
                          classes,^^J if you can}


\def\homeworknumber{5}
\fancyhead[R]{\textbf{Math 140A: Homework \#\homeworknumber}}
\fancyhead[L]{Eli Griffiths}
\renewcommand{\headrulewidth}{1pt}
\setlength\parindent{0pt}


% 11.1, 11.2, 11.5, 11.8, 11.9,
% 12.1, 12.3, 12.4, 12.10

\begin{document}

\subsection*{11.1}
\subsubsection*{Part A}
\[
    1, 5, 1, 5, 1, 5, 1, 5
\]
\subsubsection*{Part B}
\[
    \sigma(k) = 2k, (s_{n_k}) = 5, \forall k \in \mathbb{N}
\]


\subsection*{11.2}
\begin{table}[h!]
    \def\arraystretch{1.5}
    \centering
    \begin{tabular}{rllll}
                & \multicolumn{1}{c}{$a_n$} & \multicolumn{1}{c}{$b_n$} & \multicolumn{1}{c}{$c_n$} & \multicolumn{1}{c}{$d_n$} \\
    Monotone    & $\sigma(k) = 2k$          & $\sigma(k) = 2k$          & $\sigma(k) = 2k - 1$      & $\sigma(k) = 3k$          \\
    Sub. Limits & $\qty{1, -1}$             & $\qty{0}$                 & $\qty{+\infty}$           & $\qty{\frac{6}{7}}$       \\
    Liminf  & $-1$         & $0$          & $+\infty$ & $\frac{6}{7}$ \\
    Limsup  & $1$          & $0$          & $+\infty$ & $\frac{6}{7}$ \\
    Bounded & $\checkmark$ & $\checkmark$ &           & $\checkmark$  \\
    Limit   & DNE          & $0$          & $+\infty$ & $\frac{6}{7}$
    \end{tabular}
\end{table}

\subsection*{11.5}
The set of subsequential limits is $[0,1] \subset \mathbb{R}$.
\begin{align*}
    \limsup_{n \to \infty} q_n &= 1 \\
    \liminf_{n \to \infty} q_n &= 0 \\
\end{align*}

\subsection*{11.8}
\begin{align*}
    \liminf s_n &= \lim_{N \to \infty} \inf \qty{s_n : n > N} \\
    &= -\lim_{N \to \infty} \sup \qty{-s_n : n > N} \\
    &= -\limsup (-s_n)
\end{align*}

\subsection*{11.9}
\subsubsection*{Part A}
\begin{proof}
    Let $(s_n)$ be a sequence of reals in $[a,b]$ with $\lim s_n = s$. Since $a \leq s_n \leq b$ for all $n$, it follows that $a \leq s \leq b$, hence $[a,b]$ is closed.
\end{proof}

\subsubsection*{Part B}
No since $(0,1)$ is not closed.


\subsection*{12.1}
\begin{proof}
    Let $a_N = \inf\qty{s_n : n > N}$ and $b_N = \inf\qty{t_n : n > N}$. For $n > N > N_0$, $a_N \leq s_n \leq t_n$ hence $a_N \leq b_N$ for all $N > N_0$. Therefore by excercise 9.9, $\lim a_N \leq \lim b_N$ or equivalently $\liminf s_n \leq \liminf t_n$. The same argument works for $\sup$.
\end{proof}

\subsection*{12.3}
\begin{enumerate}[label=\alph*)]
    \item $0$
    \item $1$
    \item $2$
    \item $3$
    \item $4$
    \item $0$
    \item $2$
\end{enumerate}


\subsection*{12.4}
\begin{proof}
    Since $s_n$ and $t_n$ are bounded, their sups exist and note that $s_n + t_n \leq \sup\qty{s_n : n > N} + \sup\qty{t_n : n > N}$ for all $n > N \in \mathbb{N}$. Therefore $\sup\qty{s_n : n > N} + \sup\qty{t_n : n > N}$ is an upper bound for $s_n + t_n$ meaning
    \[
        \sup\qty{s_n + t_n : n > N} \leq \sup\qty{s_n : n > N} + \sup\qty{t_n : n > N}, \forall n > N
    \]
    Since $N$ is arbitrary, it holds for all $N \in \mathbb{N}$ meaning along with the results from 9.9,
    $
        \limsup (s_n + t_n) \leq \limsup(s_n) + \limsup(t_n)
    $
\end{proof}

\subsection*{12.10}
\begin{proof}
    Let $(s_n)$ be a sequence.
    \begin{enumerate}
        \item[$\Rightarrow)$]
            Assume that $s_n$ is bounded. That is $\exists M \in \mathbb{R}$ such that $|s_n| \leq M, \forall n \in \mathbb{N}$. Then $\sup\qty{|s_n| : n > N} \leq M$ for all $N \in \mathbb{N}$, hence $\limsup |s_n| \leq M < +\infty$.
        \item[$\Leftarrow)$]
            Proof by contrapositive. Assume that $s_n$ is not bounded. That is, for all $M \in \mathbb{R}$, $\exists N \in \mathbb{N}$ such that $|s_n| > M$ for all $n > N$. Therefore $\sup{|s_n| : n > N} > M$. That means that the supremum is larger than any real number, hence $\limsup s_n = +\infty$.
    \end{enumerate}
\end{proof}

\end{document}
