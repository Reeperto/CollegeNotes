\documentclass[../notes.tex]{subfiles}
\graphicspath{
    {'../figures'}
}

\begin{document}

\banner{Matrices}

\subsection{Elementary Matrices}

\begin{definition}[Elementary Matrix]
    An elementary matrix is a matrix obtained by doing on the types of operations on a $n \times n$ identity matrix
    \begin{enumerate}
        \item Exchange two rows (or columns) $\implies$ Type I
        \item Multiply a row (or column) $\implies$ Type II
        \item Add a row to a row (or column to column) $\implies$ Type III
    \end{enumerate}
\end{definition}

It will be useful to have notation to refer to each type of elementary matrix. For Type I, $E_{ij}$ will indicate swapping row $i$ and $j$. For Type II, $E_{\lambda}$ represents a row multiplication and Type III will be represented by $E(\lambda)$.

\begin{tcolorbox}[sharp corners = all, colback = white,]
    The following are the common inverses and relationships between the elementary matrices
    \begin{align*}
        & E_{ij}^{-1} = E_{ij} & E_{ij}^2 = I \\
        & E_{\lambda}^{-1} = E_{\lambda^{-1}} & E_{\lambda} E_{\lambda^{-1}} = I
    \end{align*}
\end{tcolorbox}

\begin{theorem}[Application of Elementary Operations]
    % \label{thm:}
    Let $A \in M_{n\times n}(\mathbb{F})$. Let $E$ be an elementary operation/matrix. Then
    \begin{align*}
        A &\xrightarrow{\text{Row operation } E} EA \\
        A &\xrightarrow{\text{Col. operation } E} AE
    \end{align*}
\end{theorem}

\subsection{Rank of Matrices}

Recall that the rank of a linear transformation $T : V \to W$ is $\rank T = \dim T[V]$. Therefore it would make sense to extend this to matrices as they represent linear transformations.

\begin{definition}[Rank of a Matrix]
    Let $A \in M_{m \times n}(\mathbb{F})$. Then the rank of $A$ is defined as the rank of the linear transformation $L_A : \mathbb{F}^n \to \mathbb{F}^m : x \mapsto Ax$.
\end{definition}

It is also the case that rank of a linear transformation is preserved in any matrix representation. This can be stated as a theorem:

\begin{theorem}[Rank Across Representation]
    \label{thm:linearrankpreservation}
    Let $T: V \to W$ be linear and let $\beta$ be a basis of $V$ and $\gamma$ a basis of $W$. Then
    \[
        \rank T = \rank [T]_{\beta}^\gamma
    \]
\end{theorem}
\begin{proof}
    % TODO: Draw an isomorphism diagram :)
    % \[\begin{tikzcd}[column sep=small]
    %     {v \in V} & V &&& W & {w \in W} \\
    %     \\
    %     \\
    %     {[v]_\beta} & {\mathbb{F}^n} &&& {\mathbb{F}^m} & {[w]_\gamma}
    %     \arrow[from=1-2, to=1-5]
    %     \arrow["\simeq"', tail reversed, from=1-2, to=4-2]
    %     \arrow[from=4-2, to=4-5]
    %     \arrow["\simeq"', tail reversed, from=4-5, to=1-5]
    %     \arrow[tail reversed, from=1-1, to=4-1]
    %     \arrow[tail reversed, from=4-6, to=1-6]
    % \end{tikzcd}\]
\end{proof}

\begin{theorem}[Rank and General Linear Group]
    Let $A \in M_{m\times n}(\mathbb{F})$ and $P \in GL_m(\mathbb{F})$, $Q \in GL_n(\mathbb{F})$. Then
    \begin{enumerate}
        \item $\rank(PA) = \rank(A)$
        \item $\rank(AQ) = \rank(A)$
        \item $\rank(PAQ) = \rank(A)$
    \end{enumerate}
\end{theorem}

\begin{theorem}[Rank of Matrix]
    \label{thm:columnsequalrank}
    For any matrix $A$,
    \[
        \rank A = \dim(C(A)) = \text{Max number of linearly independent columns}
    \]
\end{theorem}

\end{document}
