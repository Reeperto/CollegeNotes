\documentclass[../notes.tex]{subfiles}
\graphicspath{
    {'../figures'}
}

\begin{document}

\banner{Complex Regions}

\begin{definition}[$\epsilon$-Neighborhood]
    An $\epsilon$-neighborhood of a point $z_0 \in \C$ is the set of points given by
    \[
        |z - z_0| < \epsilon
    .\]
    This is often denoted by $B_\epsilon (z_0)$ or $B(z_0, \epsilon)$.
\end{definition}

\begin{definition}[Interior, Exterior, and Boundary Points]
    Given a set $S \subset \C$ and a point $z_0 \in \C$, there are 3 possibilities in how it sits in relation to $S$.

    \begin{enumerate}
        \item There is an $\epsilon$-neighborhood of $z_0$ that is contained entirely in $S$. In this case, $z_0$ is an \textbf{interior point}
        \item There is an $\epsilon$-neighborhood of $z_0$ that is disjoint from $S$. In this case, $z_0$ is an \textbf{exterior point}
        \item For all $\epsilon$-neighborhood's of $z_0$, there are points that are in $S$ and not in $S$. In this case, $z_0$ is a \textbf{boundary point}
    \end{enumerate}
\end{definition}

\begin{definition}[Open and Closed Sets]
    Let $S \subset \C$. $S$ is \textbf{open} if all its points are interior points. That is
    \[
        \forall z \in S, \exists \epsilon > 0 \text{ s.t. } B_\epsilon(z) \subset S
    .\]
    $S$ is \textbf{closed} if it contains its boundary points.
\end{definition}

\begin{theorem}[Closure and Complement]
    A set $S \subset \C$ is open iff $\C \setminus S$ is closed.
\end{theorem}

\begin{proof}
    \hfill\begin{enumerate}
        \item[$\Rightarrow)$] %-------------------------------------------------
        Suppose $S$ is open. Let $z_0$ be a boundary point of $\C \setminus S$. This means that for every $\epsilon$-neighborhood of $z_0$, there is a point in $\C \setminus S$ and a point outside of $\C \setminus S$. This means that there is a point always in $S$ and a point outside of $S$, hence $z_0$ is also a boundary point of $S$. Since $S$ is open, $z_0$ is not in $S$ and therefore it is in $\C \setminus S$ and therefore $\C \setminus S$ contains it's boundary. Hence it is closed.
        \item[$\Leftarrow)$] %-------------------------------------------------
        Suppose that $\C \setminus S$ is closed. Let $z_0 \in S$. Since $z_0$ is always in any $\epsilon$-neighborhood around itself, it cant be an exterior point. Assume towards contradiction that $z_0$ is a boundary point of $S$. Then by the previous direction, it is also a boundary point of $\C \setminus S$. Since $\C \setminus S$ is closed, it contains $z_0$ and hence a contradiction. Therefore $z_0$ is neither an exterior or boundary point and must be an interior point of $S$.
    \end{enumerate}
\end{proof}

Something important to note is that sets are not in a binary of open or closed. Sets can fall into 4 different categories

\[
    \setlength{\tabcolsep}{30pt}
    \renewcommand\arraystretch{1.5}
    \begin{array}{c|c|c}
               & \text{Closed} & \text{Not Closed} \\\hline
        \text{Open}   & \varnothing, \C & B_\epsilon(z_0) \\\hline
        \text{Not Open} & \conj{B}_\epsilon(z_0) & \qty{z \in \C : r < |z| \leq R}
    \end{array}
.\]

\begin{definition}[Closure]
    Let $S \subset \C$. Then the closure of $S$ is $\conj{S} = S \cup \partial S$
\end{definition}


\begin{definition}[Connectedness]
    An open set $S \subset \C$ is connected if given $u,v \in S$ there exists a finite set of points $u = w_1, w_2, \ldots, w_n = v$ such that $\conj{w_i w_{i+1}} \subset S$ for $i = 1,2,\ldots,n-1$. That is there exists a path of finite line segments between the two points contained in $S$.
\end{definition}

\begin{definition}[Domain]
    A set $S \subset \C$ is a domain if it is a connected open set.
\end{definition}

\begin{definition}[Region]
    $S \subset \C$ is a region if it is a domain unioned with a subset of its boundary.
\end{definition}

\begin{definition}[Boundedness]
    A set $S \subset \C$ is bounded if there is an $R \in \R$ such that $S \subset B_R(0)$.
\end{definition}

\begin{example}
    Consider the set $S = \qty{z \in \C : \frac{\pi}{4} < \arg z < \frac{\pi}{2}}$
\end{example}

\begin{definition}[Accumulation Point]
    Let $S \subset \C$. $z_0$ is an accumulation point of $S$ if
    \[
        (B_\epsilon(z_0) \setminus z_0) \cap S \neq \varnothing, \forall \epsilon > 0
    .\]
    That is, $z_0$ is an accumulation point if every neighborhood contains a point in $S$ that isnt $z_0$.
\end{definition}

An accumulation point can be thought of as a point that can be continually well approximated by points inside some set $S$. This idea also applies to things such as the suprememum on $\R$ or the limit of a sequence over a toplogy.

\end{document}
