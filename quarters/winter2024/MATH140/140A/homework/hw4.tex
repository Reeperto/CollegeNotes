\documentclass[12pt,titlepage]{extarticle}
% Document Layout and Font
\usepackage{subfiles}
\usepackage[margin=2cm, headheight=15pt]{geometry}
\usepackage{fancyhdr}
\usepackage{enumitem}	
\usepackage{wrapfig}
\usepackage{float}
\usepackage{multicol}

\usepackage[p,osf]{scholax}

\renewcommand*\contentsname{Table of Contents}
\renewcommand{\headrulewidth}{0pt}
\pagestyle{fancy}
\fancyhf{}
\fancyfoot[R]{$\thepage$}
\setlength{\parindent}{0cm}
\setlength{\headheight}{17pt}
\hfuzz=9pt

% Figures
\usepackage{svg}

% Utility Management
\usepackage{color}
\usepackage{colortbl}
\usepackage{xcolor}
\usepackage{xpatch}
\usepackage{xparse}

\definecolor{gBlue}{HTML}{7daea3}
\definecolor{gOrange}{HTML}{e78a4e}
\definecolor{gGreen}{HTML}{a9b665}
\definecolor{gPurple}{HTML}{d3869b}

\definecolor{links}{HTML}{1c73a5}
\definecolor{bar}{HTML}{584AA8}

% Math Packages
\usepackage{mathtools, amsmath, amsthm, thmtools, amssymb, physics}
\usepackage[scaled=1.075,ncf,vvarbb]{newtxmath}

\newcommand\B{\mathbb{C}}
\newcommand\C{\mathbb{C}}
\newcommand\R{\mathbb{R}}
\newcommand\Q{\mathbb{Q}}
\newcommand\N{\mathbb{N}}
\newcommand\Z{\mathbb{Z}}

\DeclareMathOperator{\lcm}{lcm}

% Probability Theory
\newcommand\Prob[1]{\mathbb{P}\qty(#1)}
\newcommand\Var[1]{\text{Var}\qty(#1)}
\newcommand\Exp[1]{\mathbb{E}\qty[#1]}

% Analysis
\newcommand\ball[1]{\B\qty(#1)}
\newcommand\conj[1]{\overline{#1}}
\DeclareMathOperator{\Arg}{Arg}
\DeclareMathOperator{\cis}{cis}

% Linear Algebra
\DeclareMathOperator{\dom}{dom}
\DeclareMathOperator{\range}{range}
\DeclareMathOperator{\spann}{span}
\DeclareMathOperator{\nullity}{nullity}

% TIKZ
\usepackage{tikz}
\usepackage{pgfplots}
\usetikzlibrary{arrows.meta}
\usetikzlibrary{math}
\usetikzlibrary{cd}

% Boxes and Theorems
\usepackage[most]{tcolorbox}
\tcbuselibrary{skins}
\tcbuselibrary{breakable}
\tcbuselibrary{theorems}

\newtheoremstyle{default}{0pt}{0pt}{}{}{\bfseries}{\normalfont.}{0.5em}{}
\theoremstyle{default}

\renewcommand*{\proofname}{\textit{\textbf{Proof.}}}
\renewcommand*{\qedsymbol}{$\blacksquare$}
\tcolorboxenvironment{proof}{
	breakable,
	coltitle = black,
	colback = white,
	frame hidden,
	boxrule = 0pt,
	boxsep = 0pt,
	borderline west={3pt}{0pt}{bar},
	% borderline west={3pt}{0pt}{gPurple},
	sharp corners = all,
	enhanced,
}

\newtheorem{theorem}{Theorem}[section]{\bfseries}{}
\tcolorboxenvironment{theorem}{
	breakable,
	enhanced,
	boxrule = 0pt,
	frame hidden,
	coltitle = black,
	colback = blue!7,
	% colback = gBlue!30,
	left = 0.5em,
	sharp corners = all,
}

\newtheorem{corollary}{Corollary}[section]{\bfseries}{}
\tcolorboxenvironment{corollary}{
	breakable,
	enhanced,
	boxrule = 0pt,
	frame hidden,
	coltitle = black,
	colback = white!0,
	left = 0.5em,
	sharp corners = all,
}

\newtheorem{lemma}{Lemma}[section]{\bfseries}{}
\tcolorboxenvironment{lemma}{
	breakable,
	enhanced,
	boxrule = 0pt,
	frame hidden,
	coltitle = black,
	colback = green!7,
	left = 0.5em,
	sharp corners = all,
}

\newtheorem{definition}{Definition}[section]{\bfseries}{}
\tcolorboxenvironment{definition}{
	breakable,
	coltitle = black,
	colback = white,
	frame hidden,
	boxsep = 0pt,
	boxrule = 0pt,
	borderline west = {3pt}{0pt}{orange},
	% borderline west = {3pt}{0pt}{gOrange},
	sharp corners = all,
	enhanced,
}

\newtheorem{example}{Example}[section]{\bfseries}{}
\tcolorboxenvironment{example}{
	% title = \textbf{Example},
	% detach title,
	% before upper = {\tcbtitle\quad},
	breakable,
	coltitle = black,
	colback = white,
	frame hidden,
	boxrule = 0pt,
	boxsep = 0pt,
	borderline west={3pt}{0pt}{green!70!black},
	% borderline west={3pt}{0pt}{gGreen},
	sharp corners = all,
	enhanced,
}

\newtheoremstyle{remark}{0pt}{4pt}{}{}{\bfseries\itshape}{\normalfont.}{0.5em}{}
\theoremstyle{remark}
\newtheorem*{remark}{Remark}


% TColorBoxes
\newtcolorbox{week}{
	colback = black,
	coltext = white,
	fontupper = {\large\bfseries},
	width = 1.2\paperwidth,
	size = fbox,
	halign upper = center,
	center
}

\newcommand{\banner}[2]{
    \pagebreak
    \begin{week}
   		\section*{#1}
    \end{week}
    \addcontentsline{toc}{section}{#1}
    \addtocounter{section}{1}
    \setcounter{subsection}{0}
}

% Hyperref
\usepackage{hyperref}
\hypersetup{
	colorlinks=true,
	linktoc=all,
	linkcolor=links,
	bookmarksopen=true
}

% Error Handling
\PackageWarningNoLine{ExtSizes}{It is better to use one of the extsizes 
                          classes,^^J if you can}


\def\homeworknumber{4}
\fancyhead[R]{\textbf{Math 140A: Homework \#\homeworknumber}}
\fancyhead[L]{Eli Griffiths}
\renewcommand{\headrulewidth}{1pt}
\setlength\parindent{0pt}


% 9.1b, 9.1c, 9.3, 9.4, 9.9, 9.12
% 10.1, 10.3, 10.4, 10.6, 10.8, 10.11

\begin{document}

\subsection*{9.1}
\subsubsection*{Part B}
\begin{align*}
    \lim \frac{3n+7}{6n-5} &= \lim \frac{3 + \frac{7}{n}}{6 - \frac{5}{n}} \\
    &= \frac{\lim\qty(3 + \frac{7}{n})}{\lim\qty(6 - \frac{5}{n})} \\
    &= \frac{3 + 7 \lim \frac{1}{n}}{6 - 5 \lim \frac{1}{n}} \\
    &= \frac{3 + 7(0)}{6 - 5(0)} \\
    &= \frac{3}{6} = \frac{1}{2}.
\end{align*}

\subsubsection*{Part C}
\begin{align*}
    \lim \frac{17n^5 + 73n^4 -18n^2 + 3}{23n^5 + 13n^3} &= \lim\frac{17 + \frac{73}{n} - \frac{18}{n^3} + \frac{3}{n^5}}{23 + \frac{13}{n^2}} \\
    &= \frac{17 + 73 \cdot \lim \frac{1}{n} - 18\cdot \lim \frac{1}{n^3} + 3\cdot \frac{1}{n^5}}{23 + 13\cdot\lim \frac{1}{n^2}} \\
    &= \frac{17 + 73(0) - 18(0) + 3(0)}{23 + 13(0)} \\
    &= \frac{17}{23}
\end{align*}

\subsection*{9.3}
Since $b_n^2 + 1 > 0$ for all $n \in \mathbb{N}$,
\begin{align*}
    \lim s_n = \frac{\lim a_n^3 + 4 a_n}{\lim b_n^2 + 1} &= \frac{\lim a_n^3 + 4 \lim a_n}{b^2 + 1} \\
    &= \frac{(\lim a_n)^3 + 4a}{b^2 + 1} = \frac{a^3 + 4a}{b^2 + 1}
\end{align*}

\subsection*{9.4}
\subsubsection*{Part A}
\begin{align*}
    s_1 &= 1 \\
    s_2 &= \sqrt{2} \\
    s_3 &= \sqrt{\sqrt{2} + 1} \\
    s_4 &= \sqrt{\sqrt{\sqrt{2} + 1} + 1} \\
\end{align*}

\subsubsection*{Part B}
Since $s_n$ converges, let $\lim_{n\to \infty} s_n = s$. Then
\[
    \lim_{n\to \infty} s_{n+1} = \lim_{n \to \infty} \sqrt{s_n + 1}
\]
meaning
\begin{align*}
    s = \sqrt{s + 1} &\implies s^2 - s - 1 = 0 \\
                     &\implies s = \frac{1 \pm \sqrt{5}}{2}
\end{align*}
Since $s_n > 0$ for all $n \in \mathbb{N}$, it follows that $s = \frac{1 + \sqrt{5}}{2}$.

\subsection*{9.9}
Suppose that $\exists N_0 \in \mathbb{N}$ such that $s_n \leq t_n$ for all $n > N_0$.

\subsubsection*{Part A}
\begin{proof}
    Assume that $\lim s_n = +\infty$. That is
    \[
        \forall M > 0, \exists N_1 \in \mathbb{N} \text{ such that } s_n > M, \forall n > N_1
    \]
    Take $N = \max\qty{N_0, N_1}$. If $n > N$, then $t_n \geq s_n > M$. Therefore $\lim t_n = + \infty$.
\end{proof}

\subsubsection*{Part B}
\begin{proof}
    Assume that $\lim t_n = -\infty$. That is
    \[
        \forall M < 0, \exists N_1 \in \mathbb{N} \text{ such that } t_n < M, \forall n > N_1
    \]
    Take $N = \min\qty{N_0, N_1}$. If $n > N$, then $s_n \leq t_n < M$. Therefore $\lim s_n = - \infty$.
\end{proof}

\subsubsection*{Part C}
\begin{proof}
    Assume that $\lim s_n = s$ and $\lim t_n = t$ exist. Consider the case that $s$ and $t$ are finite. Let $\epsilon > 0$. Then there exists $N_1, N_2 \in \mathbb{N}$ such that
    \begin{align*}
        |s_n - s| < \epsilon, \forall n > N_1 \\
        |t_n - t| < \epsilon, \forall n > N_2
    \end{align*}
    Therefore considering $N = \max\qty{N_0, N_1, N_2}$,
    \[
        s - \epsilon < s_n \leq t_n < t + \epsilon \implies s < t + 2\epsilon
    \]
    Since $\epsilon$ is arbitrary, it follows then that $s \leq t$. If $s = -\infty$, then $t \geq s$ no matter what $t$ is. If $s = \infty$, then by part A it follows that $t = \infty \geq \infty$.
\end{proof}

\subsection*{9.12}
\subsubsection*{Part A}
\begin{proof}
    Assume that $L < 1$. Let $a \in (L, 1) > 0$ such that $L < a < 1$. Take $\epsilon = a - L > 0$. Since $\qty|\frac{s_{n+1}}{s_n}|$ converges to $L$, there exists $N \in \mathbb{N}$ such that
    \[
        L - \epsilon < \qty|\frac{s_{n+1}}{s_n}| < L + \epsilon, \forall n > N
    \]
    meaning
    \[
        -a + 2L < \qty|\frac{s_{n+1}}{s_n}| < a \implies \qty|\frac{s_{n+1}}{s_n}| < a, \forall n > N.
    \]
    Therefore $|s_{n+1}| < a |s_n|$ for all $n > N$. Proceed with induction to show that $|s_n| < a^{n - N} |s_N|$ for $n > N$. Consider the base case $n = N + 1$. By the previous result, $|s_{N + 1}| < a |s_N| = a^{N + 1 - N} |s_N|$, hence the base case holds. Assume for some fixed $n > N$ that $|s_n| < a^{n - N} |s_N|$. Since $a > 0$,
    \[
        a|s_n| < a^{(n+1) - N} |s_N|
    \]
    And since $|s_{n+1}| < a |s_n|$ for all $n > N$,
    \[
        |s_{n+1}| < a |s_n| < a^{(n+1) - N} |s_N| \implies |s_{n+1}| < a^{(n+1) - N} |s_N|
    \]
    Therefore the statement holds for all $n > N$. Note then that
    \[
        0 \leq |s_n| \leq a^{n - N} |s_N|, \forall n > N
    \]
    Since $0 < a < 1$, $a^{n-N} |s_N|$ converges to $0$. By the squeeze theorem, $\lim |s_n| = 0$ hence $\lim s_n = 0$.
\end{proof}

\subsubsection*{Part B}
\begin{proof}
    Assume that $L > 1$ and let $t_n = \frac{1}{|s_n|}$. Note that then $\lim\qty|\frac{t_{n+1}}{t_n}| = \frac{1}{L}$ when $L < \infty$ and $0$ when $L = +\infty$. Therefore $\lim\qty|\frac{t_{n+1}}{t_n}| < 1$, which by part A means $\lim t_n = 0$. By Theorem 9.10, $\lim s_n = + \infty$.
\end{proof}

\subsection*{10.1}
Empty means false.
\begin{table}[h!]
    \centering
    \def\arraystretch{1.5}
    \begin{tabular}{r|l|l|l|l|l|l}
     & A & B & C & D & E & F \\\hline
    Increasing &             &              & $\checkmark$ &              &  &              \\\hline
    Decreasing & $\checkmark$ &              &              &              &  & $\checkmark$ \\\hline
    Bounded    & $\checkmark$ & $\checkmark$ &              & $\checkmark$ &  & $\checkmark$
    \end{tabular}
\end{table}

\subsection*{10.3}
\begin{proof}
    Let $K.d_1 d_2 d_3 \ldots$ be a decimal expansion of a real number. Note that for all $n \in \mathbb{N}$,
    \[
       \frac{d_1}{10} + \frac{d_2}{10^2} + \ldots + \frac{d_n}{10^n} \leq \frac{9}{10} + \frac{9}{10^2} + \ldots + \frac{9}{10^n} = 1 - \frac{1}{10^n} < 1
    \]
    hence
    \[
       \frac{d_1}{10} + \frac{d_2}{10^2} + \ldots + \frac{d_n}{10^n} < 1 \implies K + \frac{d_1}{10} + \frac{d_2}{10^2} + \ldots + \frac{d_n}{10^n} < K + 1
    \]
    for all $n$. By the definition $(s_n)$, it follows that
    $
        s_n < K + 1, \forall n \in \mathbb{N}
    $
\end{proof}

\subsection*{10.4}
Both theorems rely on the completeness axiom to ensure the existence of a supremum which does not hold for $\mathbb{Q}$.

\subsection*{10.6}
\subsubsection*{Part A}

\begin{proof}
    Let $(s_n)$ be a sequence and assume that $|s_{n+1} - s_n| < 2^{-n}$ for all $n \in \mathbb{N}$. 
    Let $m,w \in \mathbb{N}$ and WLOG assume $m \geq w$. Note that
    \[
        |s_m - s_w| = |s_m - s_{m-1} + s_{m-1} - s_{m-2} + \ldots + s_{w + 1} - s_w|
    \]
    which by triangle inequality
    \begin{align*}
        |s_m - s_w| \leq |s_{m+1} - s_m| + |s_{m} - s_{m-1}| + \ldots + |s_{w+1} - s_w| &< 2^{-m} + 2^{1-m} + \ldots + 2^{-w} \\
        &= \frac{1}{2^m} + \ldots + \frac{1}{2^w} \\
        &= \frac{1}{2^m}\qty(1 + \frac{1}{2} + \ldots \frac{1}{2^{w-m}}) \\
    \end{align*}
    Since $1 + \frac{1}{2} + \ldots + \frac{1}{2^{w-m}} < 2$,
    \[
        |s_m - s_w| < \frac{1}{2^m}\qty(1 + \frac{1}{2} + \ldots \frac{1}{2^{w-m}}) < \frac{1}{2^{w-1}}
    \]

    Take $\epsilon > 0$. Note that for any $n \in \mathbb{N}$, $n < 2^n$ or equivalently $2^{-n} < \frac{1}{n}$ f`or all $n$. By the archimedean property, $\exists N_0$ such that $\frac{1}{N} < \epsilon$. Then $2^{-N} < \epsilon$. If $m,w > N$, then $\frac{1}{2^{m-1}} \leq \frac{1}{2^{-N}}$. Therefore
    \[
        |s_m - s_w| < \frac{1}{2^{m-1}} \leq \frac{1}{2^{-N}} < \epsilon
    \]
    Therefore $s_n$ is Cauchy and converges since all Cauchy sequences converge.
\end{proof}

\subsubsection*{Part B}
The result would not be true if we assume it is less than $\frac{1}{n}$. A key part of the proof is that the series $1 + \frac{1}{2} + \frac{1}{4} + \frac{1}{8} + \ldots$ has an upper bound. In the case of $\frac{1}{n}$, the series $1 + \frac{1}{2} + \frac{1}{3} + \frac{1}{4}$ is not bounded and therefore it is not possible to obtain an $N$ large enough that for any distance between $m$ and $n$ the difference $|s_m - s_n|$ stays below a fixed upper bound $\epsilon$.

\subsection*{10.8}
\begin{proof}
    Note that
    \begin{align*}
        \sigma_{n+1} - \sigma_n &= \frac{s_1 + \ldots + s_{n+1}}{n+1} - \frac{s_1 + \ldots + s_n}{n} \\
        &= \frac{(n+1)(s_1 + \ldots + s_{n+1})}{n(n+1)} - \frac{n(s_1 + \ldots + s_n)}{n(n+1)} \\
        &= \frac{(s_{n+1} - s_n) + (s_{n+1} - s_{n-1}) + \ldots + (s_{n+1} - s_1)}{n(n+1)} \\
        \intertext{Since $s_n$ is increasing, $s_n \geq s_m \implies s_n - s_m \geq 0$ for all $n \geq m$, meaning}
        &\geq \frac{0 + \ldots + 0}{n(n+1)} = 0 \\
    \end{align*}
    Therefore $\sigma_{n+1} - \sigma_n \geq 0$ meaning $\sigma_{n+1} \geq \sigma_n$, hence $\sigma_n$ is an increasing sequence.
\end{proof}

\subsection*{10.11}
\subsubsection*{Part A}
\begin{proof}
    First note that $t_n$ is a decreasing sequence since $t_{n+1}$ is $t_n$ multiplied by a number between $0$ and $1$. Additionally, it is bounded above by $1$ since it is decreasing and below by $0$ since each successive term is positive and $t_1 > 0$. Therefore since $t_n$ is a bounded monotonic sequence, it converges.
\end{proof}

\subsubsection*{Part B}
Intuitively, $t_n > 0.5$. By creating a desmos simulation, $t_{574} \approx 0.636$, and using Mathematica to solve the recurrence relation gives $\lim t_n = \frac{2}{\pi}$.

\[
    t_n = \frac{\qty(\frac{1}{2})_{n-1} \cdot \qty(\frac{3}{2})_{n-1}}{(1)_{n-1}^2}
\]

\end{document}
