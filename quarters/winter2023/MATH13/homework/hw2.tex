\documentclass[12pt]{extarticle}

% Document Layout and Font
\usepackage{subfiles}
\usepackage[margin=2cm, headheight=15pt]{geometry}
\usepackage{fancyhdr}
\usepackage{enumitem}	
\usepackage{wrapfig}
\usepackage{float}
\usepackage{multicol}

\usepackage[p,osf]{scholax}

\renewcommand*\contentsname{Table of Contents}
\renewcommand{\headrulewidth}{0pt}
\pagestyle{fancy}
\fancyhf{}
\fancyfoot[R]{$\thepage$}
\setlength{\parindent}{0cm}
\setlength{\headheight}{17pt}
\hfuzz=9pt

% Figures
\usepackage{svg}

% Utility Management
\usepackage{color}
\usepackage{colortbl}
\usepackage{xcolor}
\usepackage{xpatch}
\usepackage{xparse}

\definecolor{gBlue}{HTML}{7daea3}
\definecolor{gOrange}{HTML}{e78a4e}
\definecolor{gGreen}{HTML}{a9b665}
\definecolor{gPurple}{HTML}{d3869b}

\definecolor{links}{HTML}{1c73a5}
\definecolor{bar}{HTML}{584AA8}

% Math Packages
\usepackage{mathtools, amsmath, amsthm, thmtools, amssymb, physics}
\usepackage[scaled=1.075,ncf,vvarbb]{newtxmath}

\newcommand\B{\mathbb{C}}
\newcommand\C{\mathbb{C}}
\newcommand\R{\mathbb{R}}
\newcommand\Q{\mathbb{Q}}
\newcommand\N{\mathbb{N}}
\newcommand\Z{\mathbb{Z}}

\DeclareMathOperator{\lcm}{lcm}

% Probability Theory
\newcommand\Prob[1]{\mathbb{P}\qty(#1)}
\newcommand\Var[1]{\text{Var}\qty(#1)}
\newcommand\Exp[1]{\mathbb{E}\qty[#1]}

% Analysis
\newcommand\ball[1]{\B\qty(#1)}
\newcommand\conj[1]{\overline{#1}}
\DeclareMathOperator{\Arg}{Arg}
\DeclareMathOperator{\cis}{cis}

% Linear Algebra
\DeclareMathOperator{\dom}{dom}
\DeclareMathOperator{\range}{range}
\DeclareMathOperator{\spann}{span}
\DeclareMathOperator{\nullity}{nullity}

% TIKZ
\usepackage{tikz}
\usepackage{pgfplots}
\usetikzlibrary{arrows.meta}
\usetikzlibrary{math}
\usetikzlibrary{cd}

% Boxes and Theorems
\usepackage[most]{tcolorbox}
\tcbuselibrary{skins}
\tcbuselibrary{breakable}
\tcbuselibrary{theorems}

\newtheoremstyle{default}{0pt}{0pt}{}{}{\bfseries}{\normalfont.}{0.5em}{}
\theoremstyle{default}

\renewcommand*{\proofname}{\textit{\textbf{Proof.}}}
\renewcommand*{\qedsymbol}{$\blacksquare$}
\tcolorboxenvironment{proof}{
	breakable,
	coltitle = black,
	colback = white,
	frame hidden,
	boxrule = 0pt,
	boxsep = 0pt,
	borderline west={3pt}{0pt}{bar},
	% borderline west={3pt}{0pt}{gPurple},
	sharp corners = all,
	enhanced,
}

\newtheorem{theorem}{Theorem}[section]{\bfseries}{}
\tcolorboxenvironment{theorem}{
	breakable,
	enhanced,
	boxrule = 0pt,
	frame hidden,
	coltitle = black,
	colback = blue!7,
	% colback = gBlue!30,
	left = 0.5em,
	sharp corners = all,
}

\newtheorem{corollary}{Corollary}[section]{\bfseries}{}
\tcolorboxenvironment{corollary}{
	breakable,
	enhanced,
	boxrule = 0pt,
	frame hidden,
	coltitle = black,
	colback = white!0,
	left = 0.5em,
	sharp corners = all,
}

\newtheorem{lemma}{Lemma}[section]{\bfseries}{}
\tcolorboxenvironment{lemma}{
	breakable,
	enhanced,
	boxrule = 0pt,
	frame hidden,
	coltitle = black,
	colback = green!7,
	left = 0.5em,
	sharp corners = all,
}

\newtheorem{definition}{Definition}[section]{\bfseries}{}
\tcolorboxenvironment{definition}{
	breakable,
	coltitle = black,
	colback = white,
	frame hidden,
	boxsep = 0pt,
	boxrule = 0pt,
	borderline west = {3pt}{0pt}{orange},
	% borderline west = {3pt}{0pt}{gOrange},
	sharp corners = all,
	enhanced,
}

\newtheorem{example}{Example}[section]{\bfseries}{}
\tcolorboxenvironment{example}{
	% title = \textbf{Example},
	% detach title,
	% before upper = {\tcbtitle\quad},
	breakable,
	coltitle = black,
	colback = white,
	frame hidden,
	boxrule = 0pt,
	boxsep = 0pt,
	borderline west={3pt}{0pt}{green!70!black},
	% borderline west={3pt}{0pt}{gGreen},
	sharp corners = all,
	enhanced,
}

\newtheoremstyle{remark}{0pt}{4pt}{}{}{\bfseries\itshape}{\normalfont.}{0.5em}{}
\theoremstyle{remark}
\newtheorem*{remark}{Remark}


% TColorBoxes
\newtcolorbox{week}{
	colback = black,
	coltext = white,
	fontupper = {\large\bfseries},
	width = 1.2\paperwidth,
	size = fbox,
	halign upper = center,
	center
}

\newcommand{\banner}[2]{
    \pagebreak
    \begin{week}
   		\section*{#1}
    \end{week}
    \addcontentsline{toc}{section}{#1}
    \addtocounter{section}{1}
    \setcounter{subsection}{0}
}

% Hyperref
\usepackage{hyperref}
\hypersetup{
	colorlinks=true,
	linktoc=all,
	linkcolor=links,
	bookmarksopen=true
}

% Error Handling
\PackageWarningNoLine{ExtSizes}{It is better to use one of the extsizes 
                          classes,^^J if you can}


\fancyhead[R]{\textbf{Homework \#2}}
\setlength\parindent{0pt}

\begin{document}

\section*{Problem 1}

Show that for any given integers $a, b, c$, if $a$ is even and $b$ is odd, then $7a - ab + 12c + b^2 + 4$ is odd.

\subsection*{Solution}

A direct proof that for any given integers $a, b, c$, if $a$ is even and $b$ is odd, then $7a - ab + 12c + b^2 + 4$ is odd.

\begin{proof}
	Let $a, b, c \in \mathbb{Z}$. Suppose $a$ is an even integer and $b$ is an odd integer. There exists $m,k \in \mathbb{Z}$ such that $a = 2m$ and $b = 2k + 1$. Then
  \begin{align*}
  	7a - ab + 12c + b^2 + 4 &= 7(2m) - (2m)(2k+1) + 12c + (2k+1)^2 + 4 \\
  													&= 14m - 4mk - 2m + 12c + 4k^2 + 4k + 1 + 4 \\
  													&= 2(6m - 2mk + 6c + 2k^2 + 2k + 2) + 1 
  .\end{align*}
	is by definition an odd integer because $6m - 2mk + 6c + 2k^2 + 2k + 2 \in \mathbb{Z}$ .\qedhere
\end{proof}

\section*{Problem 3}

Prove or disprove the following conjectures: 

\begin{enumerate}[label=(\alph*)]
	\item The sum of any 3 consecutive integers is divisible by 3. 
	\item The sum of any 4 consecutive integers is divisible by 4.
\end{enumerate}

\subsection*{Solution}

\subsubsection*{Part A}

A direct proof that the sum of any 3 consecutive integers is divisible by 3.
\begin{proof}
	Let $a,b,c \in \mathbb{Z}$. Suppose that $a$, $b$ and $c$ are consecutive. Without loss of generality they can be expressed as $a = a$, $b = a+1$, and $c = a+2$ by the definition consecutive integers. It then follows
	\begin{align*}
		a + b + c &= (a) + (a+1) + (a+2) \\
		&= 3a + 3 \\
		&= 3(a+1).
	\end{align*}
	is a multiple of 3.
\end{proof}

\subsubsection*{Part B}

A direct proof that the sum of any 4 consecutive integers is not divisible by 4, disproving the second conjecture.
\begin{proof}
	Let $a,b,c,d \in \mathbb{Z}$. Suppose that $a,b,c$ and $d$ are consecutive. Without loss of generality they can be written as $a = a$, $b = a+1$, $c = a+2$, and $c = a+3$. Then
	\begin{align*}
		a + b + c + d &= (a) + (a+1) + (a+2) + (a+3) \\
									&= 4a + 6.
	\end{align*}
	is not a multiple of 4.
\end{proof}

\section*{Problem 5}

Prove that if $n$ is a natural number greater than 1, then $n! + 2$ is even.

\subsection*{Solution}

A direct proof that if $n$ is a natural number greater than 1, then $n! + 2$ is even.
\begin{proof}
	Let $n \in \mathbb{N}$ such that $n > 1$. By definition of the factorial, $n! = n \cdot (n-1) \cdot (n-2)\ldots 3 \cdot 2 \cdot 1$. Then
	\begin{align*}
		n! + 2 &= n \cdot (n-1) \cdot (n-2)\ldots 3 \cdot 2 \cdot 1 + 2 \\
					 &= 2 \qty(n \cdot (n-1) \cdot (n-2)\ldots 3 \cdot 1 + 1)
	\end{align*}
	Since $n \cdot (n-1) \cdot (n-2)\ldots 3 \cdot 1 + 1$ is an integer, $n! + 2$	 is an even number.
\end{proof}

\section*{Problem 7}

\begin{enumerate}[label=(\alph*)]
	\item Let $x \in \mathbb{Z}$. Prove that $5x + 3$ is even if and only if $7x - 2$ is odd.
	\item Can you conclude anything about $7x - 2$ if $5x + 3$ is odd?
\end{enumerate}

\subsection*{Solution}

\subsubsection*{Part A}

A direct proof of both directions.

\begin{proof}
	Let $x \in \mathbb{Z}$. Suppose that $5x+3$ is even. By definition there exists $k \in \mathbb{Z}$ such that $5x + 3 = 2k$. It follows that
	\begin{align*}
		5x + 3 &= 2k \\
					 &\big\Updownarrow \\
		5x - 2 &= 2k - 5 \\
					 &\big\Updownarrow \\
		7x - 2 &= 2k + 2x - 5 \\
					 &= 2k + 2x - 6 + 1 \\
					 &= 2(k + x - 3) + 1 \\
	\end{align*}
	Since $k + x - 3 \in \mathbb{Z}$, $7x-2$ is by definition an odd integer.
\end{proof}

\subsubsection*{Part B}

Yes. One can conclude that if $5x + 3$ is odd then $7x-2$ is even. Consider the backwards direction of Part A. That is: "If $7x-2$ is odd, then $5x + 3$ is even". This is a true statement as established in Part A. Therefore its contrapositive is also true. Therefore the statement: "If $5x + 3$ is odd, then $7x - 2$ is even" is true.

\section*{Problem 10}

\begin{definition}
	A real number $x$ is rational if it may be written in the form $x = \frac{p}{q}$ where $p$ is an integer and $q$ is a positive integer. $x$ is irrational if it is not rational.
\end{definition}

Prove or disprove the following conjecture. 
\begin{conjecture}
  If $x$ and $y$ are real numbers such that $3x + 5y$ is irrational, then at least one of $x$ and $y$ is irrational.
\end{conjecture}

\subsection*{Solution}

A proof by contrapositive that if $x$ and $y$ are real numbers such that $3x + 5y$ is irrational, then at least one of $x$ and $y$ is irrational.
\begin{proof}
	Let $x,y \in \mathbb{R}$. Suppose both are rational. Therefore both can be written in the form $x = \frac{p}{q}$ and $y = \frac{m}{n}$ where $p,q,m$, and $n$ are integers with $q$ and $n$ being positive. Then
	\begin{align*}
		3x + 5y &= 3\qty(\frac{p}{q}) + 5 \qty(\frac{m}{n}) \\
						&= \frac{3p}{q} + \frac{5m}{n} \\
						&= \frac{3pn + 5mq}{qn}
	\end{align*}
	is by definition a rational number since the top is an integer and the bottom is a product of positive integers and therefore also a positive integer. This proves the contrapositive and therefore the original proposition.
\end{proof}

\section*{Problem 11}

Let $x$ and $y$ be integers. Prove: For $x^2 + y^2$ to be even, it is necessary that $x$ and $y$ have the same parity (i.e. both even or both odd).

\subsection*{Solution}

% x and y having the same parity is neccessary for x^2 + y^2 to be even.
% If x^2 + y^2 is even, then x and y have the same parity

% Contrapositive: If x and y have a different parity, then x^2 + y^2 is odd

Proof by contrapositive that if $x^2 + y^2$ is even then $x$ and $y$ have the same parity.
\begin{proof}
	Suppose there are two integers $x$ and $y$ with different parity. That is, one of $x$ or $y$ is even with the other being odd. Without loss of generality, assume that $x$ is an even integer and $y$ is an odd integer. Therefore there exists integers $m$ and $n$ such that $x = 2m$ and $y=2n+1$. Then
	\begin{align*}
		x^2 + y^2 &= (2m)^2 + (2n+1)^2 \\
							&= 4m^2 + 4n^2 + 4n + 1 \\
							&= 2(2m^2 + 2n^2 + 2n) + 1.
	\end{align*}
	is an odd integer. This proves the contrapositive and therefore the original proposition.
\end{proof}

\section*{Problem 12}

Prove that if x and y are positive real numbers, then $\sqrt{x+y} \neq \sqrt{x} + \sqrt{y}$. \textit{Argue by contradiction.}

\subsection*{Solution}

Proof by contradiction that if $x$ and $y$ are positive real numbers, then $\sqrt{x+y} \neq \sqrt{x} + \sqrt{y}$.

\begin{proof}
	Let $x$ and $y$ be  real numbers. Assume towards contradiction that $x$ and $y$ are positive and that $\sqrt{x + y} = \sqrt{x} + \sqrt{y}$. It follows that
	\begin{align*}
		\sqrt{x+y} &= \sqrt{x} + \sqrt{y} \\
		\qty(\sqrt{x+y})^2 &= \qty(\sqrt{x} + \sqrt{y})^2 \\
		x+y &= x + \sqrt{xy} + y \\
		\sqrt{xy} &= 0 \\
		xy &= 0.
	\end{align*}
	which implies that either $x$ or $y$ are 0. However since it was assumed both $x$ and $y$ are positive, they are both stricly greater than 0 and hence a contradiction.
\end{proof}


\end{document}
