\documentclass[../notes.tex]{subfiles}
\graphicspath{
    {"../figures"}
}

\begin{document}

\banner{Set Theory I}

\subsection{Sets}

\begin{definition}[Sets]
	\label{def:sets}
	A set can be defined as a 'well-defined' collection of objects. Objects themselves can also be other sets. The set containing no elements is the empty set, denoted as $\varnothing$.
\end{definition}

Sets are denoted using curly brackets in some form or another. There are many ways to notate a set.

\begin{table}[h!]
	\centering
	\begin{tabular}{|c | c | p{0.4\linewidth}|}\hline
		\textbf{Type} & \textbf{Representation} & \textbf{Meaning} \\\hline
		Explicit & $\{1,2,3,4,5\}$ & Sometimes called roster notation, a set can be defined via an exhaustive list of its elements \\\hline
		Implicit & $\{1,2,3\ldots\}$ & Similar to roster notation. Outlines a pattern and implies its continuation \\\hline
		Set-Builder & \{ Elements \::\: Condition \} & Set-builder notation outlines what the elements of the set look like, under the given conditions outlined on the right \\\hline
		Open Interval & $(a,b)$ & $(a,b) = \{ x \in \mathbb{R} : a < x < b \}$ \\\hline
		Closed Interval & $[a,b]$ & $[a,b] = \{ x \in \mathbb{R} : a \geq x \leq b \}$ \\\hline
	\end{tabular}
\end{table}

\begin{definition}[Cardinality]
	\label{def:cardinality}
	A set $A$ is \textit{finite} if the set has finitely many elements. The number of elements within $A$, denoted as $|A|$, is the cardinality of $A$. If $A$ has infinitely many elements, then its cardinality is infinite and is said to be an \textit{infinite set}.
\end{definition}

Consider the set $A = \{1, 10 -5 ,4\}$. We say that the cardinality of $A$ is finite and therefore $|A| = 4$. Alternatively, the set $B = (1,2)$ has an infinite cardinality and is therefore an infinite set.

\subsection{Subsets}
\begin{definition}[Subset]
	\label{def:subset}
	If $A$ and $B$ are sets, $A$ is a \textit{subset} of $B$ if all the elements of $A$ are in $B$. This is denoted as
	\[
	  A \subseteq B \Longleftrightarrow \qty(\forall x \in A \Rightarrow x \in B)
	.\]
	Additionally, if $A \neq B$ and $A \neq \varnothing$, then $A$ is a \textit{proper subset} of $B$. This is denoted as
	\[
		A \subset B \Longleftrightarrow \forall x \in A, x \in B \text{ and } \exists y \in B \text{ s.t. } y \notin A
	.\]
\end{definition}

Now with all of these definitions, useful tools can be constructed.

\begin{theorem}[Set Equality]
	Two sets are equal if and only if they are subsets of each other
\end{theorem}
\begin{proof}
	Let $A$ and $B$ be sets and suppose that $A = B$. $A$ and $B$ have the same elements so $x \in A$ if and only if $x \in B$. Now, $A \subseteq B$ since $x \in A \implies x \in B$ and $B \subseteq A$ since $x \in B \implies x \in A$. Therefore, $A = B$ if and only if $A$ and $B$ are subsets of each other.
\end{proof}

% PROVE TRANSITIVITY OF SUBSETS

\end{document}
